% %%%%%%%%%%%%%%%%%%%%%%%%%%%%%%%%%%%%%%%%%%%%%%%%%%%%%%%%%%%%%%%%%%%%%%% %
%                                                                         %
% Project Gutenberg's An Investigation of the Laws of Thought, by George Boole
%                                                                         %
% This eBook is for the use of anyone anywhere in the United States and most
% other parts of the world at no cost and with almost no restrictions     %
% whatsoever.  You may copy it, give it away or re-use it under the terms of
% the Project Gutenberg License included with this eBook or online at     %
% www.gutenberg.org.  If you are not located in the United States, you'll have
% to check the laws of the country where you are located before using this ebook.
%                                                                         %
%                                                                         %
%                                                                         %
% Title: An Investigation of the Laws of Thought                          %
%                                                                         %
% Author: George Boole                                                    %
%                                                                         %
% Release Date: July 19, 2017 [EBook #15114]                              %
%                                                                         %
% Language: English                                                       %
%                                                                         %
% Character set encoding: ASCII                                           %
%                                                                         %
% *** START OF THIS PROJECT GUTENBERG EBOOK LAWS OF THOUGHT ***           %
%                                                                         %
% %%%%%%%%%%%%%%%%%%%%%%%%%%%%%%%%%%%%%%%%%%%%%%%%%%%%%%%%%%%%%%%%%%%%%%% %

\def\ebook{15114}
%%%%%%%%%%%%%%%%%%%%%%%%%%%%%%%%%%%%%%%%%%%%%%%%%%%%%%%%%%%%%%%%%%%%%%
%%                                                                  %%
%% Packages and substitutions:                                      %%
%%                                                                  %%
%% book:     Required.                                              %%
%%                                                                  %%
%% babel:    Greek language processing; required                    %%
%%                                                                  %%
%% amsmath:  AMS mathematics enhancements. Required.                %%
%% amssymb:  Additional mathematical symbols. Required.             %%
%%                                                                  %%
%%                                                                  %%
%% PDF pages: 344                                                   %%
%% PDF page size: US Letter (8.5 x 11in)                            %%
%%                                                                  %%
%% Summary of log file:                                             %%
%% * Eleven underfull hboxes                                        %%
%% * Seven overfull hboxes (up to 37.5pt)                           %%
%%                                                                  %%
%% Compile History:                                                 %%
%%                                                                  %%
%% July, 2017:     adhere (Andrew D. Hwang)                         %%
%%                 texlive2007, GNU/Linux                           %%
%%                                                                  %%
%% Command block:                                                   %%
%%                                                                  %%
%%     pdflatex                                                     %%
%%                                                                  %%
%%                                                                  %%
%% July 2017: pglatex.                                              %%
%%   Compile this project with:                                     %%
%%   pdflatex 15114-t.tex                                           %%
%%                                                                  %%
%%   pdfTeX, Version 3.1415926-2.5-1.40.14 (TeX Live 2013/Debian)   %%
%%                                                                  %%
%%%%%%%%%%%%%%%%%%%%%%%%%%%%%%%%%%%%%%%%%%%%%%%%%%%%%%%%%%%%%%%%%%%%%%
\listfiles
\documentclass[oneside]{book}
\usepackage[greek,english]{babel}
\usepackage{amssymb,amsmath}
\selectlanguage{english}

\setlength{\emergencystretch}{1.125em}

%%%%%%%%%%%%%%%%%%%%%%%% START OF DOCUMENT %%%%%%%%%%%%%%%%%%%%%%%%%%
\begin{document}
\def\thechapter{\Roman{chapter}}
\def\theequation{\arabic{equation}}
\thispagestyle{empty}
\small
\begin{verbatim}
Project Gutenberg's An Investigation of the Laws of Thought, by George Boole

This eBook is for the use of anyone anywhere in the United States and most
other parts of the world at no cost and with almost no restrictions
whatsoever.  You may copy it, give it away or re-use it under the terms of
the Project Gutenberg License included with this eBook or online at
www.gutenberg.org.  If you are not located in the United States, you'll have
to check the laws of the country where you are located before using this ebook.



Title: An Investigation of the Laws of Thought

Author: George Boole

Release Date: July 19, 2017 [EBook #15114]

Language: English

Character set encoding: ASCII

*** START OF THIS PROJECT GUTENBERG EBOOK LAWS OF THOUGHT ***
\end{verbatim}
\normalsize
\clearpage

\frontmatter

\begin{center}
AN INVESTIGATION \\
\bigskip
\bigskip
OF \\
\bigskip
\bigskip
\Huge
THE LAWS OF THOUGHT,\\
\bigskip
\normalsize
ON WHICH ARE FOUNDED \\
\bigskip
\large
THE MATHEMATICAL THEORIES OF LOGIC
AND PROBABILITIES. \\
\bigskip
\bigskip
\bigskip
\bigskip
\normalsize
BY \\
\bigskip
\large
GEORGE BOOLE, LL. D.\\
\normalsize
PROFESSOR OF MATHEMATICS IN QUEEN'S COLLEGE, CORK.
\newpage

TO\\
\bigskip
\large
JOHN RYALL, LL.D.\\
\bigskip
\normalsize
VICE-PRESIDENT AND PROFESSOR OF GREEK\\
\bigskip
IN QUEEN'S COLLEGE, CORK,\\
\bigskip
THIS WORK IS INSCRIBED\\
\bigskip
IN TESTIMONY OF FRIENDSHIP AND ESTEEM\\
\end{center}
\chapter[PREFACE.]{}
\begin{center}\Huge PREFACE.\\
\bigskip
\normalsize
---$\diamond$---\\
\end{center}
\bigskip
The following work is not a republication of a former treatise
by the Author, entitled, ``The Mathematical Analysis
of Logic.'' Its earlier portion is indeed devoted to the same
object, and it begins by establishing the same system of fundamental
laws, but its methods are more general, and its range of
applications far wider. It exhibits the results, matured by some
years of study and reflection, of a principle of investigation relating
to the intellectual operations, the previous exposition of
which was written within a few weeks after its idea had been
conceived.

That portion of this work which relates to Logic presupposes
in its reader a knowledge of the most important terms of the
science, as usually treated, and of its general object. On these
points there is no better guide than Archbishop Whately's
``Elements of Logic,'' or Mr. Thomson's ``Outlines of the Laws
of Thought.'' To the former of these treatises, the present revival
of attention to this class of studies seems in a great measure
due. Some acquaintance with the principles of Algebra is also
requisite, but it is not necessary that this application should have
been carried beyond the solution of simple equations. For the
study of those chapters which relate to the theory of probabilities,
a somewhat larger knowledge of Algebra is required, and especially
of the doctrine of Elimination, and of the solution of Equations
containing more than one unknown quantity. Preliminary
information upon the subject-matter will be found in the special
treatises on Probabilities in ``Lardner's Cabinet Cyclop{\ae}dia,''
and the ``Library of Useful Knowledge,'' the former of these by
Professor De~Morgan, the latter by Sir John Lubbock; and in
an interesting series of Letters translated from the French of
M.~Quetelet. Other references will be given in the work. On a
first perusal the reader may omit at his discretion, Chapters \textsc{x}.,
\textsc{xiv}., and \textsc{xix}., together with any of the applications which he
may deem uninviting or irrelevant.

In different parts of the work, and especially in the notes to
the concluding chapter, will be found references to various writers,
ancient and modern, chiefly designed to illustrate a certain view of
the history of philosophy. With respect to these, the Author
thinks it proper to add, that he has in no instance given a citation
which he has not believed upon careful examination to be
supported either by parallel authorities, or by the general tenor
of the work from which it was taken. While he would gladly
have avoided the introduction of anything which might by possibility
be construed into the parade of learning, he felt it to be
due both to his subject and to the truth, that the statements in
the text should be accompanied by the means of verification.
And if now, in bringing to its close a labour, of the extent of
which few persons will be able to judge from its apparent fruits,
he may be permitted to speak for a single moment of the feelings
with which he has pursued, and with which he now lays aside,
his task, he would say, that he never doubted that it was worthy of
his best efforts; that he felt that whatever of truth it might bring
to light was not a private or arbitrary thing, not dependent, as to
its essence, upon any human opinion. He was fully aware that
learned and able men maintained opinions upon the subject of
Logic directly opposed to the views upon which the entire argument
and procedure of his work rested. While he believed those
opinions to be erroneous, he was conscious that his own views
might insensibly be warped by an influence of another kind. He
felt in an especial manner the danger of that intellectual bias which
long attention to a particular aspect of truth tends to produce.
But he trusts that out of this conflict of opinions the same truth
will but emerge the more free from any personal admixture; that
its different parts will be seen in their just proportion; and that
none of them will eventually be too highly valued or too lightly
regarded because of the prejudices which may attach to the
mere form of its exposition.

To his valued friend, the Rev. George Stephens Dickson,
of Lincoln, the Author desires to record his obligations for much
kind assistance in the revision of this work, and for some important
suggestions.

5, \textsc{Grenville-place, Cork},

\textit{Nov}. 30\textit{th}. 1853.
%\tableofcontents
\chapter[CONTENTS.]{}
\begin{center}\Huge CONTENTS.\\
\bigskip
\normalsize
---$\diamond$---\\
\bigskip
\large
CHAPTER I.
\end{center}
\normalsize
\textsc{Nature and Design of this Work,}\dotfill 1

\large
\begin{center}
CHAPTER II.
\end{center}
\normalsize
\textsc{Signs and their Laws,}\dotfill 17

\large
\begin{center}
CHAPTER III.
\end{center}
\normalsize
\textsc{Derivation of the Laws,}\dotfill 28

\large
\begin{center}
CHAPTER IV.
\end{center}
\normalsize
\textsc{Division of Propositions,}\dotfill 37

\large
\begin{center}
CHAPTER V.
\end{center}
\normalsize
\textsc{Principles of Symbolic Reasoning,}\dotfill 48

\large
\begin{center}
CHAPTER VI.
\end{center}
\normalsize
\textsc{Of Interpretation,}\dotfill 59

\large
\begin{center}
CHAPTER VII.
\end{center}
\normalsize
\textsc{Of Elimination,}\dotfill 74

\large
\begin{center}
CHAPTER VIII.
\end{center}
\normalsize
\textsc{Of Reduction,}\dotfill 87
\newpage
\large
\begin{center}
CHAPTER IX.
\end{center}
\normalsize
\textsc{Methods of Abbreviation,}\dotfill 100

\large
\begin{center}
CHAPTER X.
\end{center}
\normalsize
\textsc{Conditions of a Perfect Method,}\dotfill 117

\large
\begin{center}
CHAPTER XI.
\end{center}
\normalsize
\textsc{Of Secondary Propositions,}\dotfill 124

\large
\begin{center}
CHAPTER XII.
\end{center}
\normalsize
\textsc{Methods in Secondary Propositions,}\dotfill 137

\large
\begin{center}
CHAPTER XIII.
\end{center}
\normalsize
\textsc{Clarke and Spinoza,}\dotfill 143

\large
\begin{center}
CHAPTER XIV.
\end{center}
\normalsize
\textsc{Example of Analysis,}\dotfill 169

\large
\begin{center}
CHAPTER XV.
\end{center}
\normalsize
\textsc{Of the Aristotelian Logic,}\dotfill 174

\large
\begin{center}
CHAPTER XVI.
\end{center}
\normalsize
\textsc{Of the Theory of Probabilities,}\dotfill 187

\large
\begin{center}
CHAPTER XVII.
\end{center}
\normalsize
\textsc{General Method in Probabilities,}\dotfill 194

\large
\begin{center}
CHAPTER XVIII.
\end{center}
\normalsize
\textsc{Elementary Illustrations,}\dotfill 211

\large
\begin{center}
CHAPTER XIX.
\end{center}
\normalsize
\textsc{Of Statistical Conditions,}\dotfill 227

\large
\begin{center}
CHAPTER XX.
\end{center}
\normalsize
\textsc{Problems on Causes,}\dotfill 247

\large
\begin{center}
CHAPTER XXI.
\end{center}
\normalsize
\textsc{Probability of Judgments,}\dotfill 293

\large
\begin{center}
CHAPTER XXII.
\end{center}
\normalsize
\textsc{Constitution of the Intellect,}\dotfill 311


\chapter[]{}
\begin{center}
\Huge
NOTE.\normalsize \bigskip
\end{center}
In Prop. II., p. 261, by the ``absolute probabilities'' of the events $x, y, z ..$  is
meant simply what the probabilities of those events ought to be, in order that,
regarding them as independent, and their probabilities as our only data, the calculated probabilities of the same events under the condition $V$ should be $p, g, r ..$
The statement of the appended problem of the urn must be modified in a similar
way. The true solution of that problem, as actually stated, is $p' = cp, q' = cq$,
in which $c$ is the arbitrary probability of the condition that the ball drawn shall
be either white, or of marble, or both at once.--See p. 270, CASE II.*

Accordingly, since by the logical reduction the solution of all questions in
the theory of probabilities is brought to a form in which, from the probabilities
of simple events, $s$, $t$, \&c. under a given condition, $V$, it is required to determine
the probability of some combination, $A$, of those events under the same condition,
the principle of the demonstration in Prop. IV. is really the following:--``The
probability of such combination $A$ under the condition $V$ must be calculated
as if the events $s$, $t$, \&c. were independent, and possessed of such probabilities
as would cause the derived probabilities of the said events under the same
condition $V$ to be such as are assigned to them in the data.'' This principle I
regard as axiomatic. At the same time it admits of indefinite verification, as
well directly as through the results of the method of which it forms the basis.
I think it right to add, that it was in the above form that the principle first presented
itself to my mind, and that it is thus that I have always understood it,
the error in the particular problem referred to having arisen from inadvertence
in the choice of a material illustration.



\mainmatter

\chapter[NATURE AND DESIGN OF THIS WORK]{\large NATURE AND DESIGN OF THIS WORK.}




1. The design of the following treatise is to investigate the
fundamental laws of those operations of the mind by which
reasoning is performed; to give expression to them in the symbolical
language of a Calculus, and upon this foundation to establish the
science of Logic and construct its method; to make that method
itself the basis of a general method for the application of the mathematical
doctrine of Probabilities; and, finally, to collect from
the various elements of truth brought to view in the course of
these inquiries some probable intimations concerning the nature
and constitution of the human mind.

2. That this design is not altogether a novel one it is almost
needless to remark, and it is well known that to its two main
practical divisions of Logic and Probabilities a very considerable
share of the attention of philosophers has been directed. In its
ancient and scholastic form, indeed, the subject of Logic stands
almost exclusively associated with the great name of Aristotle.
As it was presented to ancient Greece in the partly technical,
partly metaphysical disquisitions of the Organon, such, with
scarcely any essential change, it has continued to the present
day. The stream of original inquiry has rather been directed
towards questions of general philosophy, which, though they
have arisen among the disputes of the logicians, have outgrown
their origin, and given to successive ages of speculation their peculiar
bent and character. The eras of Porphyry and Proclus,
of Anselm and Abelard, of Ramus, and of Descartes, together
with the final protests of Bacon and Locke, rise up before the
mind as examples of the remoter influences of the study upon the
course of human thought, partly in suggesting topics fertile of
discussion, partly in provoking remonstrance against its own undue
pretensions. The history of the theory of Probabilities, on
the other hand, has presented far more of that character of steady
growth which belongs to science. In its origin the early genius
of Pascal,--in its maturer stages of development the most recondite
of all the mathematical speculations of Laplace,--were directed
to its improvement; to omit here the mention of other names
scarcely less distinguished than these. As the study of Logic has
been remarkable for the kindred questions of Metaphysics to
which it has given occasion, so that of Probabilities also has been
remarkable for the impulse which it has bestowed upon the
higher departments of mathematical science. Each of these subjects
has, moreover, been justly regarded as having relation to a
speculative as well as to a practical end. To enable us to deduce
correct inferences from given premises is not the only object of
Logic; nor is it the sole claim of the theory of Probabilities that
it teaches us how to establish the business of life assurance on a
secure basis; and how to condense whatever is valuable in the
records of innumerable observations in astronomy, in physics, or
in that field of social inquiry which is fast assuming a character
of great importance. Both these studies have also an interest
of another kind, derived from the light which they shed upon
the intellectual powers. They instruct us concerning the mode
in which language and number serve as instrumental aids to the
processes of reasoning; they reveal to us in some degree the
connexion between different powers of our common intellect;
they set before us what, in the two domains of demonstrative and
of probable knowledge, are the essential standards of truth and
correctness,--standards not derived from without, but deeply
founded in the constitution of the human faculties. These ends
of speculation yield neither in interest nor in dignity, nor yet, it
may be added, in importance, to the practical objects, with the
pursuit of which they have been historically associated. To unfold
the secret laws and relations of those high faculties of
thought by which all beyond the merely perceptive knowledge
of the world and of ourselves is attained or matured, is an object
which does not stand in need of commendation to a rational
mind.

3. But although certain parts of the design of this work have
been entertained by others, its general conception, its method,
and, to a considerable extent, its results, are believed to be original.
For this reason I shall offer, in the present chapter, some
preparatory statements and explanations, in order that the real
aim of this treatise may be understood, and the treatment of its
subject facilitated.

It is designed, in the first place, to investigate the fundamental
laws of those operations of the mind by which reasoning is
performed. It is unnecessary to enter here into any argument to
prove that the operations of the mind are in a certain real sense
subject to laws, and that a science of the mind is therefore {\it possible}.
If these are questions which admit of doubt, that doubt is not
to be met by an endeavour to settle the point of dispute \textit{\`{a} priori},
but by directing the attention of the objector to the evidence of
actual laws, by referring him to an actual science. And thus the
solution of that doubt would belong not to the introduction to
this treatise, but to the treatise itself. Let the assumption be
granted, that a science of the intellectual powers is possible, and
let us for a moment consider how the knowledge of it is to be
obtained.

4. Like all other sciences, that of the intellectual operations
must primarily rest upon observation,--the subject of such observation
being the very operations and processes of which we
desire to determine the laws. But while the necessity of a foundation
in experience is thus a condition common to all sciences,
there are some special differences between the modes in which
this principle becomes available for the determination of general
truths when the subject of inquiry is the mind, and when the
subject is external nature. To these it is necessary to direct
attention.

The general laws of Nature are not, for the most part, immediate
objects of perception. They are either inductive inferences
from a large body of facts, the common truth in which they express,
or, in their origin at least, physical hypotheses of a causal
nature serving to explain ph{\ae}nomena with undeviating precision,
and to enable us to predict new combinations of them. They
are in all cases, and in the strictest sense of the term, \textit{probable}
conclusions, approaching, indeed, ever and ever nearer to certainty,
as they receive more and more of the confirmation of experience.
But of the character of probability, in the strict and
proper sense of that term, they are never wholly divested. On the
other hand, the knowledge of the laws of the mind does not require
as its basis any extensive collection of observations. The general
truth is seen in the particular instance, and it is not confirmed
by the repetition of instances. We may illustrate this position
by an obvious example. It may be a question whether that formula
of reasoning, which is called the \textit{dictum} of Aristotle, \textit{de omni et nullo},
expresses a primary law of human reasoning or not; but
it is no question that it expresses a general truth in Logic. Now
that truth is made manifest in all its generality by reflection
upon a single instance of its application. And this is both an
evidence that the particular principle or formula in question is
founded upon some general law or laws of the mind, and an illustration
of the doctrine that the perception of such general truths
is not derived from an induction from many instances, but is involved
in the clear apprehension of a single instance. In connexion
with this truth is seen the not less important one that
our knowledge of the laws upon which the science of the intellectual
powers rests, whatever may be its extent or its deficiency, is
not probable knowledge. For we not only see in the particular
example the general truth, but we see it also as a certain truth,--a
truth, our confidence in which will not continue to increase
with increasing experience of its practical verifications.

5. But if the general truths of Logic are of such a nature that
when presented to the mind they at once command assent,
wherein consists the difficulty of constructing the Science of
Logic? Not, it may be answered, in collecting the materials of
knowledge, but in discriminating their nature, and determining
their mutual place and relation. All sciences consist of general
truths, but of those truths some only are primary and fundamental,
others are secondary and derived. The laws of elliptic motion,
discovered by Kepler, are general truths in astronomy, but
they are not its fundamental truths. And it is so also in the
purely mathematical sciences. An almost boundless diversity of
theorems, which are known, and an infinite possibility of others,
as yet unknown, rest together upon the foundation of a few simple
axioms; and yet these are all \textit{general} truths. It may be
added, that they are truths which to an intelligence sufficiently
refined would shine forth in their own unborrowed light, without
the need of those connecting links of thought, those steps
of wearisome and often painful deduction, by which the knowledge
of them is actually acquired. Let us define as fundamental
those laws and principles from which all other general truths of
science may be deduced, and into which they may all be again
resolved. Shall we then err in regarding that as the true science
of Logic which, laying down certain elementary laws, confirmed
by the very testimony of the mind, permits us thence to deduce,
by uniform processes, the entire chain of its secondary consequences,
and furnishes, for its practical applications, methods of
perfect generality? Let it be considered whether in any science,
viewed either as a system of truth or as the foundation of a practical
art, there can properly be any other test of the completeness
and the fundamental character of its laws, than the completeness
of its system of derived truths, and the generality of the methods
which it serves to establish. Other questions may indeed present
themselves. Convenience, prescription, individual preference,
may urge their claims and deserve attention. But as
respects the question of what constitutes science in its abstract
integrity, I apprehend that no other considerations than the
above are properly of any value.

6. It is designed, in the next place, to give expression in this
treatise to the fundamental laws of reasoning in the symbolical
language of a Calculus. Upon this head it will suffice to say, that
those laws are such as to suggest this mode of expression, and
to give to it a peculiar and exclusive fitness for the ends in view.
There is not only a close analogy between the operations of the
mind in general reasoning and its operations in the particular
science of Algebra, but there is to a considerable extent an exact
agreement in the laws by which the two classes of operations are
conducted. Of course the laws must in both cases be determined
independently; any formal agreement between them can only be
established \textit{\`{a} posteriori} by actual comparison. To borrow the
notation of the science of Number, and then assume that in its
new application the laws by which its use is governed will remain
unchanged, would be mere hypothesis. There exist, indeed,
certain general principles founded in the very nature of language,
by which the use of symbols, which are but the elements of
scientific language, is determined. To a certain extent these
elements are arbitrary. Their interpretation is purely conventional:
we are permitted to employ them in whatever sense we
please. But this permission is limited by two indispensable conditions,--first,
that from the sense once conventionally established
we never, in the same process of reasoning, depart; secondly,
that the laws by which the process is conducted be founded exclusively
upon the above fixed sense or meaning of the symbols
employed. In accordance with these principles, any agreement
which may be established between the laws of the symbols of
Logic and those of Algebra can but issue in an agreement of processes.
The two provinces of interpretation remain apart and
independent, each subject to its own laws and conditions.

Now the actual investigations of the following pages exhibit
Logic, in its practical aspect, as a system of processes carried on
by the aid of symbols having a definite interpretation, and subject
to laws founded upon that interpretation alone. But at the
same time they exhibit those laws as identical in form with the
laws of the general symbols of algebra, with this single addition,
viz., that the symbols of Logic are further subject to a special
law (Chap, II.), to which the symbols of quantity, as such, are
not subject. Upon the nature and the evidence of this law it is not
purposed here to dwell. These questions will be fully discussed
in a future page. But as constituting the essential ground of
difference between those forms of inference with which Logic is
conversant, and those which present themselves in the particular
science of Number, the law in question is deserving of more
than a passing notice. It may be said that it lies at the very
foundation of general reasoning,--that it governs those intellectual
acts of conception or of imagination which are preliminary to
the processes of logical deduction, and that it gives to the processes
themselves much of their actual form and expression. It
may hence be affirmed that this law constitutes the germ or seminal
principle, of which every approximation to a general method
in Logic is the more or less perfect development.

7. The principle has already been laid down (5) that the
sufficiency and truly fundamental character of any assumed system
of laws in the science of Logic must partly be seen in the
perfection of the methods to which they conduct us. It remains,
then, to consider what the requirements of a general method in
Logic are, and how far they are fulfilled in the system of the present
work.

Logic is conversant with two kinds of relations,--relations
among things, and relations among facts. But as facts are expressed
by propositions, the latter species of relation may, at
least for the purposes of Logic, be resolved into a relation among
propositions. The assertion that the fact or event $A$ is an invariable
consequent of the fact or event $B$ may, to this extent at
least, be regarded as equivalent to the assertion, that the truth
of the proposition affirming the occurrence of the event $B$ always
implies the truth of the proposition affirming the occurrence of
the event $A$. Instead, then, of saying that Logic is conversant
with relations among things and relations among facts, we are
permitted to say that it is concerned with relations among things
and relations among propositions. Of the former kind of relations
we have an example in the proposition--``All men are mortal;''
of the latter kind in the proposition--``If the sun is totally
eclipsed, the stars will become visible.'' The one expresses a relation
between ``men'' and ``mortal beings,'' the other between
the elementary propositions--``The sun is totally eclipsed;''
``The stars will become visible.'' Among such relations I suppose
to be included those which affirm or deny existence with
respect to things, and those which affirm or deny truth with respect
to propositions. Now let those things or those propositions
among which relation is expressed be termed the elements of
the propositions by which such relation is expressed. Proceeding
from this definition, we may then say that the \textit{premises} of any
logical argument express \textit{given} relations among certain elements,
and that the conclusion must express an \textit{implied} relation among
those elements, or among a part of them, i.e. a relation implied
by or inferentially involved in the premises.

8. Now this being premised, the requirements of a general
method in Logic seem to be the following:--

1st. As the conclusion must express a relation among the
whole or among a part of the elements involved in the premises,
it is requisite that we should possess the means of eliminating
those elements which we desire not to appear in the conclusion,
and of determining the whole amount of relation implied by the
premises among the elements which we wish to retain. Those
elements which do not present themselves in the conclusion are,
in the language of the common Logic, called middle terms; and
the species of elimination exemplified in treatises on Logic consists
in deducing from two propositions, containing a common element
or middle term, a conclusion connecting the two remaining terms.
But the problem of elimination, as contemplated in this work,
possesses a much wider scope. It proposes not merely the elimination
of one middle term from two propositions, but the elimination
generally of middle terms from propositions, without
regard to the number of either of them, or to the nature of their
connexion. To this object neither the processes of Logic nor
those of Algebra, in their actual state, present any strict parallel.
In the latter science the problem of elimination is known to be
limited in the following manner:--From two equations we can
eliminate one symbol of quantity; from three equations two
symbols; and, generally, from $n$ equations $n-1$ symbols. But
though this condition, necessary in Algebra, seems to prevail in
the existing Logic also, it has no essential place in Logic as a
science. There, no relation whatever can be proved to prevail
between the number of terms to be eliminated and the number
of propositions from which the elimination is to be effected.
From the equation representing a single proposition, any number
of symbols representing terms or elements in Logic may be
eliminated; and from any number of equations representing propositions,
one or any other number of symbols of this kind may
be eliminated in a similar manner. For such elimination there
exists one general process applicable to all cases. This is one of
the many remarkable consequences of that distinguishing law of
the symbols of Logic, to which attention has been already
directed.

2ndly. It should be within the province of a general method
in Logic to express the final relation among the elements of the
conclusion by any admissible \textit{kind} of proposition, or in any selected
\textit{order} of terms. Among varieties of kind we may reckon
those which logicians have designated by the terms categorical,
hypothetical, disjunctive, \&c. To a choice or selection in the
order of the terms, we may refer whatsoever is dependent upon
the appearance of particular elements in the subject or in the
predicate, in the antecedent or in the consequent, of that proposition
which forms the ``conclusion.'' But waiving the language
of the schools, let us consider what really distinct species of
problems may present themselves to our notice. We have seen
that the elements of the final or inferred relation may either be
\textit{things} or \textit{propositions}. Suppose the former case; then it might
be required to deduce from the premises a definition or description
of some one thing, or class of things, constituting an element of
the conclusion in terms of the other things involved in it. Or
we might form the conception of some thing or class of things,
involving more than one of the elements of the conclusion, and
require its expression in terms of the other elements. Again,
suppose the elements retained in the conclusion to be propositions,
we might desire to ascertain such points as the following,
viz., Whether, in virtue of the premises, any of those propositions,
taken singly, are true or false?--Whether particular
combinations of them are true or false?--Whether, assuming a
particular proposition to be true, any consequences will follow,
and if so, what consequences, with respect to the other
propositions?--Whether any particular condition being assumed with
reference to certain of the propositions, any consequences, and
what consequences, will follow with respect to the others? and
so on. I say that these are general questions, which it should
fall within the scope or province of a general method in Logic to
solve. Perhaps we might include them all under this one statement
of the final problem of practical Logic. Given a set of
premises expressing relations among certain elements, whether
things or propositions: required explicitly the whole relation
consequent among \textit{any} of those elements under any proposed
conditions, and in any proposed form. That this problem, under
all its aspects, is resolvable, will hereafter appear. But it is not
for the sake of noticing this fact, that the above inquiry into the
nature and the functions of a general method in Logic has been
introduced. It is necessary that the reader should apprehend
what are the specific ends of the investigation upon which we
are entering, as well as the principles which are to guide us to
the attainment of them.

9. Possibly it may here be said that the Logic of Aristotle,
in its rules of syllogism and conversion, sets forth the elementary
processes of which all reasoning consists, and that beyond these
there is neither scope nor occasion for a general method. I have
no desire to point out the defects of the common Logic, nor do I
wish to refer to it any further than is necessary, in order to place
in its true light the nature of the present treatise. With this
end alone in view, I would remark:--1st. That syllogism, conversion,
\&c., are not the ultimate processes of Logic. It will
be shown in this treatise that they are founded upon, and are resolvable
into, ulterior and more simple processes which constitute
the real elements of method in Logic. Nor is it true in fact that
all inference is reducible to the particular forms of syllogism and
conversion.--\textit{Vide} Chap. xv. 2ndly. If all inference were reducible
to these two processes (and it has been maintained that
it is reducible to syllogism alone), there would still exist the
same necessity for a general method. For it would still be requisite
to determine in what order the processes should succeed
each other, as well as their particular nature, in order that the
desired relation should be obtained. By the desired relation I
mean that full relation which, in virtue of the premises, connects
any elements selected out of the premises at will, and which,
moreover, expresses that relation in any desired form and order.
If we may judge from the mathematical sciences, which are the
most perfect examples of method known, this \textit{directive} function
of Method constitutes its chief office and distinction. The fundamental
processes of arithmetic, for instance, are in themselves
but the elements of a possible science. To assign their nature is
the first business of its method, but to arrange their succession
is its subsequent and higher function. In the more complex
examples of logical deduction, and especially in those which form
a basis for the solution of difficult questions in the theory of
Probabilities, the aid of a directive method, such as a Calculus
alone can supply, is indispensable.

10. Whence it is that the ultimate laws of Logic are mathematical
in their form; why they are, except in a single point,
identical with the general laws of Number; and why in that particular
point they differ;--are questions upon which it might not
be very remote from presumption to endeavour to pronounce a
positive judgment. Probably they lie beyond the reach of our
limited faculties. It may, perhaps, be permitted to the mind to
attain a knowledge of the laws to which it is itself subject, without
its being also given to it to understand their ground and
origin, or even, except in a very limited degree, to comprehend
their fitness for their end, as compared with other and conceivable
systems of law. Such knowledge is, indeed, unnecessary for the
ends of science, which properly concerns itself with what is, and
seeks not for grounds of preference or reasons of appointment.
These considerations furnish a sufficient answer to all protests
against the exhibition of Logic in the form of a Calculus. It is
not because we choose to assign to it such a mode of manifestation,
but because the ultimate laws of thought render that mode
possible, and prescribe its character, and forbid, as it would
seem, the perfect manifestation of the science in any other form,
that such a mode demands adoption. It is to be remembered
that it is the business of science not to create laws, but to discover
them. We do not originate the constitution of our own minds,
greatly as it may be in our power to modify their character.
And as the laws of the human intellect do not depend upon our
will, so the forms of the science, of which they constitute the basis,
are in all essential regards independent of individual choice.

11. Beside the general statement of the principles of the
above method, this treatise will exhibit its application to the
analysis of a considerable variety of propositions, and of trains of
propositions constituting the premises of demonstrative arguments.
These examples have been selected from various writers,
they differ greatly in complexity, and they embrace a wide range
of subjects. Though in this particular respect it may appear to
some that too great a latitude of choice has been exercised, I do
not deem it necessary to offer any apology upon this account.
That Logic, as a science, is susceptible of very wide applications
is admitted; but it is equally certain that its ultimate forms and
processes are mathematical. Any objection \textit{\`{a} priori} which may
therefore be supposed to lie against the adoption of such forms
and processes in the discussion of a problem of morals or of general
philosophy must be founded upon misapprehension or false
analogy. It is not of the essence of mathematics to be conversant
with the ideas of number and quantity. Whether as a general
habit of mind it would be desirable to apply symbolical processes
to moral argument, is another question. Possibly, as I have
elsewhere observed,\footnote{Mathematical Analysis of Logic. London : G. Bell. 1847.}
the perfection of the method of Logic may
be chiefly valuable as an evidence of the speculative truth of its
principles. To supersede the employment of common reasoning,
or to subject it to the rigour of technical forms, would be the last
desire of one who knows the value of that intellectual toil and
warfare which imparts to the mind an athletic vigour, and teaches
it to contend with difficulties, and to rely upon itself in emergencies.
Nevertheless, cases may arise in which the value of a
scientific procedure, even in those things which fall confessedly
under the ordinary dominion of the reason, may be felt and acknowledged.
Some examples of this kind will be found in the
present work.

12. The general doctrine and method of Logic above explained
form also the basis of a theory and corresponding method
of Probabilities. Accordingly, the development of such a theory
and method, upon the above principles, will constitute a distinct
object of the present treatise. Of the nature of this application
it may be desirable to give here some account, more especially as
regards the character of the solutions to which it leads. In connexion
with this object some further detail will be requisite concerning
the forms in which the results of the logical analysis are
presented.

The ground of this necessity of a prior method in Logic, as
the basis of a theory of Probabilities, may be stated in a few
words. Before we can determine the mode in which the expected
frequency of occurrence of a particular event is dependent upon
the known frequency of occurrence of any other events, we must be
acquainted with the mutual dependence of the events themselves.
Speaking technically, we must be able to express the event
whose probability is sought, as a function of the events whose
probabilities are given. Now this explicit determination belongs
in all instances to the department of Logic. Probability, however,
in its mathematical acceptation, admits of numerical measurement.
Hence the subject of Probabilities belongs equally to
the science of Number and to that of Logic. In recognising the
co-ordinate existence of both these elements, the present treatise
differs from all previous ones; and as this difference not only
affects the question of the possibility of the solution of problems
in a large number of instances, but also introduces new and important
elements into the solutions obtained, I deem it necessary
to state here, at some length, the peculiar consequences of the
theory developed in the following pages.

13. The measure of the probability of an event is usually
defined as a fraction, of which the numerator represents the number
of cases favourable to the event, and the denominator the
whole number of cases favourable and unfavourable; all cases
being supposed equally likely to happen. That definition is
adopted in the present work. At the same time it is shown that
there is another aspect of the subject (shortly to be referred to)
which might equally be regarded as fundamental, and which
would actually lead to the same system of methods and conclusions.
It may be added, that so far as the received conclusions
of the theory of Probabilities extend, and so far as they are consequences
of its fundamental definitions, they do not differ from
the results (supposed to be equally correct in inference) of the
method of this work.


Again, although questions in the theory of Probabilities
present themselves under various aspects, and may be variously
modified by algebraical and other conditions, there seems to be
one general type to which all such questions, or so much of each
of them as truly belongs to the theory of Probabilities, may be
referred. Considered with reference to the \textit{data} and the \textit{qu\ae{}situm},
that type may be described as follows:---1st. The data are
the probabilities of one or more given events, each probability
being either that of the absolute fulfilment of the event to which
it relates, or the probability of its fulfilment under given supposed
conditions. 2ndly. The \textit{qu\ae{}situm}, or object sought, is the
probability of the fulfilment, absolutely or conditionally, of some
other event differing in expression from those in the data, but
more or less involving the same elements. As concerns the data,
they are either \textit{causally given},---as when the probability of a particular
throw of a die is deduced from a knowledge of the constitution
of the piece,---or they are derived from observation of
repeated instances of the success or failure of events. In the
latter case the probability of an event may be defined as the
limit toward which the ratio of the favourable to the whole number
of observed cases approaches (the uniformity of nature being
presupposed) as the observations are indefinitely continued.
Lastly, as concerns the nature or relation of the events in question,
an important distinction remains. Those events are either
\textit{simple} or \textit{compound}. By a compound event is meant one of
which the expression in language, or the conception in thought,
depends upon the expression or the conception of other events,
which, in relation to it, may be regarded as \textit{simple} events. To
say ``it rains,'' or to say ``it thunders,'' is to express the occurrence
of a simple event; but to say ``it rains and thunders,'' or
to say ``it either rains or thunders,'' is to express that of a compound
event. For the expression of that event depends upon
the elementary expressions, ``it rains,'' ``it thunders.'' The criterion
of simple events is not, therefore, any supposed simplicity
in their nature. It is founded solely on the mode of their expression
in language or conception in thought.

14. Now one general problem, which the existing theory of
Probabilities enables us to solve, is the following, viz.:---Given
the probabilities of any simple events: required the probability of
a given compound event, i.e. of an event compounded in a given
manner out of the given simple events. The problem can also
be solved when the compound event, whose probability is required,
is subjected to given conditions, i.e. to conditions dependent
also in a given manner on the given simple events.
Beside this general problem, there exist also particular problems
of which the principle of solution is known. Various questions
relating to \textit{causes} and \textit{effects} can be solved by known methods
under the particular hypothesis that the causes are mutually exclusive,
but apparently not otherwise. Beyond this it is not
clear that any advance has been made toward the solution of
what may be regarded as the general problem of the science, viz.:
Given the probabilities of any events, simple or compound, conditioned
or unconditioned: required the probability of any other
event equally arbitrary in expression and conception. In the
statement of this question it is not even postulated that the
events whose probabilities are given, and the one whose probability
is sought, should involve some common elements, because
it is the office of a method to determine whether the data of a
problem are sufficient for the end in view, and to indicate, when
they are not so, wherein the deficiency consists.

This problem, in the most unrestricted form of its statement,
is resolvable by the method of the present treatise; or, to speak
more precisely, its theoretical solution is completely given, and
its practical solution is brought to depend only upon processes
purely mathematical, such as the resolution and analysis of equations.
The order and character of the general solution may be
thus described.

15. In the first place it is always possible, by the preliminary
method of the Calculus of Logic, to express the event whose
probability is sought as a logical function of the events whose
probabilities are given. The result is of the following character:
Suppose that $X$ represents the event whose probability is sought,
$A$, $B$, $C$, \&c. the events whose probabilities are given, those
events being either simple or compound. Then the \textit{whole} relation
of the event $X$ to the events $A$, $B$, $C$, \&c. is deduced in the
form of what mathematicians term a \textit{development}, consisting, in
the most general case, of four distinct classes of terms. By the
first class are expressed those combinations of the events $A$, $B$, $C$,
which both necessarily accompany and necessarily indicate the
occurrence of the event $X$; by the second class, those combinations
which necessarily accompany, but do not necessarily imply,
the occurrence of the event $X$; by the third class, those combinations
whose occurrence in connexion with the event $X$ is impossible,
but not otherwise impossible; by the fourth class,
those combinations whose occurrence is impossible under any circumstances.
I shall not dwell upon this statement of the result
of the logical analysis of the problem, further than to remark
that the elements which it presents are precisely those by which
the expectation of the event $X$, as dependent upon our knowledge
of the events $A$, $B$, $C$, is, or alone can be, affected. General
reasoning would verify this conclusion; but general reasoning
would not usually avail to disentangle the complicated web
events and circumstances from which the solution above described
must be evolved. The attainment of this object constitutes
the first step towards the complete solution of the question I
proposed. It is to be noted that thus far the process of solution
is logical, i. e. conducted by symbols of logical significance, and
resulting in an equation interpretable into a \textit{proposition}. Let this
result be termed the \textit{final logical equation}.

The second step of the process deserves attentive remark.
From the final logical equation to which the previous step has
conducted us, are deduced, by inspection, a series of algebraic
equations implicitly involving the complete solution of the problem
proposed. Of the mode in which this transition is effected
let it suffice to say, that there exists a definite relation between
the laws by which the probabilities of events are expressed as
algebraic functions of the probabilities of other events upon which
they depend, and the laws by which the logical connexion of
the events is itself expressed. This relation, like the other coincidences
of formal law which have been referred to, is not
founded upon hypothesis, but is made known to us by observation
(I.4), and reflection. If, however, its reality were assumed \textit{\`{a} priori}
as the basis of the very definition of Probability, strict deduction
would thence lead us to the received numerical definition as a
necessary consequence. The Theory of Probabilities stands, as
it has already been remarked (I.12), in equally close relation to
Logic and to Arithmetic; and it is indifferent, so far as results
are concerned, whether we regard it as springing out of the latter
of these sciences, or as founded in the mutual relations which
connect the two together.

16. There are some circumstances, interesting perhaps to the
mathematician, attending the general solutions deduced by the
above method, which it may be desirable to notice.

1st. As the method is independent of the number and the
nature of the data, it continues to be applicable when the latter
are insufficient to render determinate the value sought. When
such is the case, the final expression of the solution will contain
terms with arbitrary constant coefficients. To such terms there
will correspond terms in the final logical equation (I. 15), the
interpretation of which will inform us what new data are requisite
in order to determine the values of those constants, and
thus render the numerical solution complete. If such data are
not to be obtained, we can still, by giving to the constants their
limiting values $0$ and $1$, determine the limits within which the
probability sought must lie independently of all further experience.
When the event whose probability is sought is \textit{quite} independent
of those whose probabilities are given, the limits thus
obtained for its value will be $0$ and $1$, as it is evident that they
ought to be, and the interpretation of the constants will only
lead to a re-statement of the original problem.

2ndly. The expression of the final solution will in all cases
involve a particular element of quantity, determinable by the solution
of an algebraic equation. Now when that equation is of
an elevated degree, a difficulty may seem to arise as to the selection
of the proper root. There are, indeed, cases in which
both the elements given and the element sought are so obviously
restricted by necessary conditions that no choice remains. But
in complex instances the discovery of such conditions, by unassisted
force of reasoning, would be hopeless. A distinct method
is requisite for this end,---a method which might not
appropriately be termed the Calculus of Statistical Conditions,
into the nature of this method I shall not here further enter
than to say, that, like the previous method, it is based upon the
employment of the ``final logical equation,'' and that it definitely
assigns, 1st, the conditions which must be fulfilled among the
numerical elements of the data, in order that the problem may
be real, i.e. derived from a \textit{possible experience}; 2ndly, the numerical
limits, within which the probability sought must have
been confined, if, instead of being determined by theory, it had
been deduced directly by observation from the same system of
ph{\ae}nomena from which the data were derived. It is clear that
these limits will be actual limits of the probability sought.
Now, on supposing the data subject to the conditions above assigned
to them, it appears in every instance which I have examined
that there exists one root, and only one root, of the final
algebraic equation which is subject to the required limitations.
Every source of ambiguity is thus removed. It would even seem
that new truths relating to the theory of algebraic equations
are thus incidentally brought to light. It is remarkable that
the special element of quantity, to which the previous discussion
relates, depends only upon the \textit{data}, and not at all upon the
\textit{qu\ae{}situm} of the problem proposed. Hence the solution of each
particular problem unties the knot of difficulty for a system of
problems, viz., for that system of problems which is marked by
the possession of common data, independently of the nature of
their \textit{qu\ae{}sita}. This circumstance is important whenever from a
particular system of data it is required to deduce a series of connected
conclusions. And it further gives to the solutions of
particular problems that character of relationship, derived from
their dependence upon a central and fundamental unity, which
not unfrequently marks the application of general methods.

17. But though the above considerations, with others of a
like nature, justify the assertion that the method of this treatise,
for the solution of questions in the theory of Probabilities, is a
general method, it does not thence follow that we are relieved in
all cases from the necessity of recourse to hypothetical grounds.
It has been observed that a solution may consist entirely of terms
affected by arbitrary constant coefficients,---may, in fact, be
wholly indefinite. The application of the method of this work to
some of the most important questions within its range would--were
the data of experience alone employed--present results of
this character. To obtain a \textit{definite} solution it is necessary, in
such cases, to have recourse to hypotheses possessing more or less
of independent probability, but incapable of exact verification.
Generally speaking, such hypotheses will differ from the immediate
results of experience in partaking of a logical rather than of a
numerical character; in prescribing the conditions under which
ph{\ae}nomena occur, rather than assigning the relative frequency
of their occurrence. This circumstance is, however, unimportant.
Whatever their nature may be, the hypotheses assumed must
thenceforth be regarded as belonging to the actual data, although
tending, as is obvious, to give to the solution itself somewhat of
a hypothetical character. With this understanding as to the
possible sources of the data actually employed, the method is
perfectly general, but for the correctness of the hypothetical elements
introduced it is of course no more responsible than for the
correctness of the numerical data derived from experience.

In illustration of these remarks we may observe that the
theory of the reduction of astronomical observations\footnote{
The author designs to treat this subject either in a separate work or in a
future Appendix. In the present treatise he avoids the use of the integral
calculus.}
rests, in
part, upon hypothetical grounds. It assumes certain positions
as to the nature of error, the equal probabilities of its occurrence
in the form of excess or defect, \&c., without which it would be
impossible to obtain any \textit{definite} conclusions from a system of
conflicting observations. But granting such positions as the
above, the residue of the investigation falls strictly within the
province of the theory of Probabilities. Similar observations
apply to the important problem which proposes to deduce from
the records of the majorities of a deliberative assembly the mean
probability of correct judgment in one of its members. If the
method of this treatise be applied to the mere numerical data,
the solution obtained is of that wholly indefinite kind above described.
And to show in a more eminent degree the insufficiency
of those data by themselves, the interpretation of the arbitrary
constants (I. 16) which appear in the solution, merely produces
a re-statement of the original problem. Admitting, however,
the hypothesis of the independent formation of opinion in the
individual mind, either absolutely, as in the speculations of
Laplace and Poisson, or under limitations imposed by the actual
data, as will be seen in this treatise, Chap. XXI., the problem assumes
a far more definite character. It will be manifest that the
ulterior value of the theory of Probabilities must depend very
much upon the correct formation of such mediate hypotheses,
where the purely experimental data are insufficient for \textit{definite}
solution, and where that further experience indicated by the interpretation
of the final logical equation is unattainable. Upon
the other hand, an undue readiness to form hypotheses in subjects
which from their very nature are placed beyond human
ken, must re-act upon the credit of the theory of Probabilities,
and tend to throw doubt in the general mind over its most legitimate
conclusions.

18. It would, perhaps, be premature to speculate here upon
the question whether the methods of abstract science are likely at
any future day to render service in the investigation of social
problems at all commensurate with those which they have rendered
in various departments of physical inquiry. An attempt
to resolve this question upon pure \textit{\`{a} priori} grounds of reasoning
would be very likely to mislead us. For example, the consideration
of human free-agency would seem at first sight to preclude
the idea that the movements of the social system should ever manifest
that character of orderly evolution which we are prepared
to expect under the reign of a physical necessity. Yet already
do the researches of the statist reveal to us facts at variance with
such an anticipation. Thus the records of crime and pauperism
present a degree of regularity unknown in regions in which the
disturbing influence of human wants and passions is unfelt. On
the other hand, the distemperature of seasons, the eruption of
volcanoes, the spread of blight in the vegetable, or of epidemic
maladies in the animal kingdom, things apparently or chiefly the
product of natural causes, refuse to be submitted to regular and
apprehensible laws. ``Fickle as the wind,'' is a proverbial expression.
Reflection upon these points teaches us in some degree
to correct our earlier judgments. We learn that we are not to
expect, under the dominion of necessity, an order perceptible to
human observation, unless the play of its producing causes is
sufficiently simple; nor, on the other hand, to deem that free
agency in the individual is inconsistent with regularity in the
motions of the system of which he forms a component unit.
Human freedom stands out as an apparent fact of our consciousness,
while it is also, I conceive, a highly probable deduction of
analogy (Chap, XXII.) from the nature of that portion of the
mind whose scientific constitution we are able to investigate.
But whether accepted as a fact reposing on consciousness, or as
a conclusion sanctioned by the reason, it must be so interpreted
as not to conflict with an established result of observation, viz.:
that ph\ae{}nomena, in the production of which large masses of men
are concerned, do actually exhibit a very remarkable degree of
regularity, enabling us to collect in each succeeding age the elements
upon which the estimate of its state and progress, so far
as manifested in outward results, must depend. There is thus no
sound objection \textit{\`{a} priori} against the possibility of that species of
data which is requisite for the experimental foundation of a
science of social statistics. Again, whatever other object this
treatise may accomplish, it is presumed that it will leave no
doubt as to the existence of a system of abstract principles and of
methods founded upon those principles, by which any collective
body of social data may be made to yield, in an explicit form,
whatever information they implicitly involve. There may, where
the data are exceedingly complex, be very great difficulty in obtaining
this information,---difficulty due not to any imperfection
of the theory, but to the laborious character of the analytical
processes to which it points. It is quite conceivable that in many
instances that difficulty may be such as only united effort could
overcome. But that we possess theoretically in all cases, and
practically, so far as the requisite labour of calculation may be
supplied, the means of evolving from statistical records the seeds
of general truths which lie buried amid the mass of figures, is a
position which may, I conceive, with perfect safety be affirmed.

19. But beyond these general positions I do not venture to
speak in terms of assurance. Whether the results which might
be expected from the application of scientific methods to statistical
records, over and above those the discovery of which requires
no such aid, would so far compensate for the labour involved
as to render it worth while to institute such investigations
upon a proper scale of magnitude, is a point which could, perhaps,
only be determined by experience. It is to be desired,
and it might without great presumption be expected, that in
this, as in other instances, the abstract doctrines of science should
minister to more than intellectual gratification. Nor, viewing
the apparent order in which the sciences have been evolved, and
have successively contributed their aid to the service of mankind,
does it seem very improbable that a day may arrive in which similar
aid may accrue from departments of the field of knowledge
yet more intimately allied with the elements of human welfare.
Let the speculations of this treatise, however, rest at present
simply upon their claim to be regarded as true.

20. I design, in the last place, to endeavour to educe from
the scientific results of the previous inquiries some general intimations
respecting the nature and constitution of the human
mind. Into the grounds of the possibility of this species of inference
it is not necessary to enter here. One or two general
observations may serve to indicate the track which I shall endeavour
to follow. It cannot but be admitted that our views of
the science of Logic must materially influence, perhaps mainly
determine, our opinions upon the nature of the intellectual faculties.
For example, the question whether reasoning consists
merely in the application of certain first or necessary truths,
with which the mind has been originally imprinted, or whether
the mind is itself a seat of law, whose operation is as manifest
and as conclusive in the particular as in the general formula, or
whether, as some not undistinguished writers seem to maintain,
all reasoning is of particulars; this question, I say, is one which
not merely affects the science of Logic, but also concerns the formation
of just views of the constitution of the intellectual faculties.
Again, if it is concluded that the mind is by original
constitution a seat of law, the question of the nature of its subjection
to this law,---whether, for instance, it is an obedience
founded upon necessity, like that which sustains the revolutions
of the heavens, and preserves the order of Nature,---or whether
it is a subjection of some quite distinct kind, is also a matter of
deep speculative interest. Further, if the mind is truly determined
to be a subject of law, and if its laws also are truly assigned,
the question of their probable or necessary influence upon the
course of human thought in different ages is one invested with
great importance, and well deserving a patient investigation, as
matter both of philosophy and of history. These and other
questions I propose, however imperfectly, to discuss in the concluding
portion of the present work. They belong, perhaps, to
the domain of probable or conjectural, rather than to that of positive,
knowledge. But it may happen that where there is not
sufficient warrant for the certainties of science, there may be
grounds of analogy adequate for the suggestion of highly probable
opinions. It has seemed to me better that this discussion
should be entirely reserved for the sequel of the main business of
this treatise,---which is the investigation of scientific truths and
laws. Experience sufficiently instructs us that the proper order
of advancement in all inquiries after truth is to proceed from the
known to the unknown. There are parts, even of the philosophy
and constitution of the human mind, which have been placed
fully within the reach of our investigation. To make a due acquaintance
with those portions of our nature the basis of all endeavours
to penetrate amid the shadows and uncertainties of that
conjectural realm which lies beyond and above them, is the
course most accordant with the limitations of our present condition.




\chapter[SIGNS AND THEIR LAWS]{\large OF SIGNS IN GENERAL, AND OF THE SIGNS APPROPRIATE TO THE
SCIENCE OF LOGIC IN PARTICULAR; ALSO OF THE LAWS TO WHICH
THAT CLASS OF SIGNS ARE SUBJECT.}

1. That Language is an instrument of human reason, and
not merely a medium for the expression of thought, is a
truth generally admitted. It is proposed in this chapter to inquire
what it is that renders Language thus subservient to the
most important of our intellectual faculties. In the various
steps of this inquiry we shall be led to consider the constitution
of Language, considered as a system adapted to an end or purpose;
to investigate its elements; to seek to determine their mutual
relation and dependence; and to inquire in what manner they
contribute to the attainment of the end to which, as co-ordinate
parts of a system, they have respect.

In proceeding to these inquiries, it will not be necessary to
enter into the discussion of that famous question of the schools,
whether Language is to be regarded as an \textit{essential} instrument
of reasoning, or whether, on the other hand, it is possible for us
to reason without its aid. I suppose this question to be beside
the design of the present treatise, for the following reason, viz.,
that it is the business of Science to investigate laws; and that,
whether we regard signs as the representatives of things and of
their relations, or as the representatives of the conceptions and
operations of the human intellect, in studying the laws of signs,
we are in effect studying the manifested laws of reasoning. If
there exists a difference between the two inquiries, it is one which
does not affect the scientific expressions of formal law, which are
the object of investigation in the present stage of this work, but
relates only to the mode in which those results are presented to
the mental regard. For though in investigating the laws of signs,
\textit{\`{a} posteriori}, the immediate subject of examination is Language,
with the rules which govern its use; while in making the internal
processes of thought the direct object of inquiry, we appeal in a
more immediate way to our personal consciousness,---it will be
found that in both cases the results obtained are formally equivalent.
Nor could we easily conceive, that the unnumbered
tongues and dialects of the earth should have preserved through
a long succession of ages so much that is common and universal,
were we not assured of the existence of some deep foundation of
their agreement in the laws of the mind itself.

2. The elements of which all language consists are signs or
symbols. Words are signs. Sometimes they are said to represent
things; sometimes the operations by which the mind combines
together the simple notions of things into complex conceptions;
sometimes they express the relations of action, passion, or
mere quality, which we perceive to exist among the objects of our
experience; sometimes the emotions of the perceiving mind. But
words, although in this and in other ways they fulfil the office of
signs, or representative symbols, are not the only signs which we
are capable of employing. Arbitrary marks, which speak only to
the eye, and arbitrary sounds or actions, which address themselves
to some other sense, are equally of the nature of signs, provided
that their representative office is defined and understood. In the
mathematical sciences, letters, and the symbols $+$, $-$, $=$, \&c., are
used as signs, although the term ``sign'' is applied to the latter
class of symbols, which represent operations or relations, rather
than to the former, which represent the elements of number and
quantity. As the real import of a sign does not in any way depend
upon its particular form or expression, so neither do the
laws which determine its use. In the present treatise, however,
it is with written signs that we have to do, and it is with reference
to these exclusively that the term ``sign'' will be employed. The
essential properties of signs are enumerated in the following definition.

\textit{Definition.}---A sign is an arbitrary mark, having a fixed interpretation,
and susceptible of combination with other signs in
subjection to fixed laws dependent upon their mutual interpretation.

3. Let us consider the particulars involved in the above definition
separately.

(1.) In the first place, a sign is an \textit{arbitrary} mark. It is
clearly indifferent what particular word or token we associate
with a given idea, provided that the association once made is
permanent. The Romans expressed by the word ``civitas'' what
we designate by the word ``state.'' But both they and we
might equally well have employed any other word to represent
the same conception. Nothing, indeed, in the nature of Language
would prevent us from using a mere letter in the same sense.
Were this done, the laws according to which that letter would
require to be used would be essentially the same with the laws
which govern the use of ``civitas'' in the Latin, and of ``state''
in the English language, so far at least as the use of those words
is regulated by any general principles common to all languages
alike.

(2.) In the second place, it is necessary that each sign should
possess, within the limits of the same discourse or process of
reasoning, a fixed interpretation. The necessity of this condition
is obvious, and seems to be founded in the very nature of the
subject. There exists, however, a dispute as to the precise nature
of the representative office of words or symbols used as names in
the processes of reasoning. By some it is maintained, that they
represent the conceptions of the mind alone; by others, that they
represent things. The question is not of great importance here,
as its decision cannot affect the laws according to which signs
are employed. I apprehend, however, that the general answer
to this and such like questions is, that in the processes of reasoning,
signs stand in the place and fulfil the office of the conceptions
and operations of the mind; but that as those conceptions
and operations represent things, and the connexions and relations
of things, so signs represent things with their connexions and relations;
and lastly, that as signs stand in the place of the conceptions
and operations of the mind, they are subject to the laws
of those conceptions and operations. This view will be more
fully elucidated in the next chapter; but it here serves to explain
the third of those particulars involved in the definition of a sign,
viz., its subjection to fixed laws of combination depending upon
the nature of its interpretation.

4. The analysis and classification of those signs by which the
operations of reasoning are conducted will be considered in the
following Proposition:

\begin{center}
\textsc{Proposition I.}
\end{center}

\textit{All the operations of Language, as an instrument of reasoning,
may be conducted by a system of signs composed of the following elements,
viz.:}

1st. \textit{Literal symbols, as $x$, $y$, \&c., representing things as subjects
of our conceptions.}

2nd. \textit{Signs of operation, as $+$, $-$, $\times$, standing for those operations
of the mind by which the conceptions of things are combined or resolved
so as to form new conceptions involving the same elements.}

3rd. \textit{The sign of identity, $=$.}

\textit{And these symbols of Logic are in their use subject to definite
laws, partly agreeing with and partly differing from the laws of the
corresponding symbols in the science of Algebra.}

Let it be assumed as a criterion of the true elements of rational
discourse, that they should be susceptible of combination
in the simplest forms and by the simplest laws, and thus combining
should generate all other known and conceivable forms of
language; and adopting this principle, let the following classification
be considered.

\begin{center}
\textsc{class i}.
\end{center}
5. \textit{Appellative or descriptive signs, expressing either the name
of a thing, or some quality or circumstance belonging to it.}

To this class we may obviously refer the substantive proper
or common, and the adjective. These may indeed be regarded as
differing only in this respect, that the former expresses the substantive
existence of the individual thing or things to which it
refers; the latter implies that existence. If we attach to the
adjective the universally understood subject ``being'' or ``thing,''
it becomes virtually a substantive, and may for all the essential
purposes of reasoning be replaced by the substantive. Whether
or not, in every particular of the mental regard, it is the same
thing to say, ``Water is a fluid thing,'' as to say, ``Water is
fluid;'' it is at least equivalent in the expression of the processes
of reasoning.

It is clear also, that to the above class we must refer any sign
which may conventionally be used to express some circumstance
or relation, the detailed exposition of which would involve the
use of many signs. The epithets of poetic diction are very frequently
of this kind. They are usually compounded adjectives,
singly fulfilling the office of a many-worded description. Homer's
``deep-eddying ocean'' embodies a virtual description in the single
word \textgreek{bajud'inhc}. %$\beta\alpha\theta\upsilon\delta\acute{\iota}\nu\eta\varsigma$.
And conventionally any other description addressed
either to the imagination or to the intellect might equally
be represented by a single sign, the use of which would in all essential
points be subject to the same laws as the use of the adjective
``good'' or ``great.'' Combined with the subject ``thing,''
such a sign would virtually become a substantive; and by a single
substantive the combined meaning both of thing and quality
might be expressed.

6. Now, as it has been defined that a sign is an arbitrary
mark, it is permissible to replace all signs of the species above
described by letters. Let us then agree to represent the class of
individuals to which a particular name or description is applicable,
by a single letter, as $x$. If the name is ``men,'' for instance,
let $x$ represent ``all men,'' or the class ``men.'' By a class is
usually meant a collection of individuals, to each of which a
particular name or description may be applied; but in this work
the meaning of the term will be extended so as to include the
case in which but a single individual exists, answering to the
required name or description, as well as the cases denoted by
the terms ``nothing'' and ``universe,'' which as ``classes''
should be understood to comprise respectively ``no beings,''
``all beings.'' Again, if an adjective, as ``good,'' is employed
as a term of description, let us represent by a letter, as $y$, all
things to which the description ``good'' is applicable, i.e. ``all
good things,'' or the class ``good things.'' Let it further be
agreed, that by the combination $xy$ shall be represented that
class of things to which the names or descriptions represented by
$x$ and $y$ are simultaneously applicable. Thus, if $x$ alone stands
for ``white things,'' and $y$ for ``sheep,'' let $xy$ stand for ``white
sheep;'' and in like manner, if $z$ stand for ``horned things,'' and
$x$ and $y$ retain their previous interpretations, let $zxy$ represent
``horned white sheep,'' i.e. that collection of things to which
the name ``sheep,'' and the descriptions ``white'' and ``horned''
are together applicable.

Let us now consider the laws to which the symbols $x$, $y$, \&c.,
used in the above sense, are subject.

7. First, it is evident, that according to the above combinations,
the order in which two symbols are written is indifferent.
The expressions $xy$ and $yx$ equally represent that class of things
to the several members of which the names or descriptions $x$ and
$y$ are together applicable. Hence we have,
\begin{equation}
xy=yx.
\end{equation}


In the case of $x$ representing white things, and $y$ sheep, either
of the members of this equation will represent the class of ``white
sheep.'' There may be a difference as to the order in which the
conception is formed, but there is none as to the individual things
which are comprehended under it. In like manner, if $x$ represent
``estuaries,'' and $y$ ``rivers,'' the expressions $xy$ and $yx$ will indifferently
represent ``rivers that are estuaries,'' or ``estuaries
that are rivers,'' the combination in this case being in ordinary
language that of two substantives, instead of that of a substantive
and an adjective as in the previous instance. Let there be a
third symbol, as $z$, representing that class of things to which the
term ``navigable'' is applicable, and any one of the following
expressions,
\[
zxy,\; zyx,\; xyz,\; \textrm{\&c.},
\]
will represent the class of ``navigable rivers that are estuaries.''

If one of the descriptive terms should have some implied reference
to another, it is only necessary to include that reference
expressly in its stated meaning, in order to render the above
remarks still applicable. Thus, if $x$ represent ``wise'' and $y$
``counsellor,'' we shall have to define whether $x$ implies wisdom
in the absolute sense, or only the wisdom of counsel. With such
definition the law $xy = yx$ continues to be valid.

\textit{We are permitted, therefore, to employ the symbols $x$, $y$, $z$, \&c., in
the place of the substantives, adjectives, and descriptive phrases subject
to the rule of interpretation, that any expression in which several of
these symbols are written together shall represent all the objects or individuals
to which their several meanings are together applicable, and
to the law that the order in which the symbols succeed each other is
indifferent.}

As the rule of interpretation has been sufficiently exemplified,
I shall deem it unnecessary always to express the subject
``things'' in defining the interpretation of a symbol used for an
adjective. When I say, let $x$ represent ``good,'' it will be understood
that $x$ only represents ``good'' when a subject for that
quality is supplied by another symbol, and that, used alone, its interpretation
will be ``good things.''

8. Concerning the law above determined, the following observations,
which will also be more or less appropriate to certain
other laws to be deduced hereafter, may be added.

First, I would remark, that this law is a law of thought, and
not, properly speaking, a law of things. Difference in the order
of the qualities or attributes of an object, apart from all questions
of causation, is a difference in conception merely. The law
(1) expresses as a general truth, that the same thing may be conceived
in different ways, and states the nature of that difference;
and it does no more than this.

Secondly, As a law of thought, it is actually developed in a
law of Language, the product and the instrument of thought.
Though the tendency of prose writing is toward uniformity,
yet even there the order of sequence of adjectives absolute in
their meaning, and applied to the same subject, is indifferent,
but poetic diction borrows much of its rich diversity from the
extension of the same lawful freedom to the substantive also.
The language of Milton is peculiarly distinguished by this species
of variety. Not only does the substantive often precede the
adjectives by which it is qualified, but it is frequently placed in
their midst. In the first few lines of the invocation to Light,
we meet with such examples as the following:
\begin{verse}
``\textit{Offspring of heaven first-born.}''\\
``The rising world of \textit{waters dark and deep}.''\\
``Bright effluence of \textit{bright essence increate}.''
\end{verse}

Now these inverted forms are not simply the fruits of a poetic
license. They are the natural expressions of a freedom sanctioned
by the intimate laws of thought, but for reasons of convenience
not exercised in the ordinary use of language.

Thirdly, The law expressed by (1) may be characterized by
saying that the literal symbols $x$, $y$, $z$, are \textit{commutative, like the
symbols of Algebra}. In saying this, it is not affirmed that the
process of multiplication in Algebra, of which the fundamental
law is expressed by the equation
\[
xy = yx,
\]
possesses in itself any analogy with that process of logical combination
which $xy$ has been made to represent above; but only
that if the arithmetical and the logical process are expressed in
the same manner, their symbolical expressions will be subject to
the same formal law. The evidence of that subjection is in the
two cases quite distinct.

9. As the combination of two literal symbols in the form $xy$
expresses the whole of that class of objects to which the names
or qualities represented by $x$ and $y$ are together applicable, it
follows that if the two symbols have exactly the same signification,
their combination expresses no more than either of the
symbols taken alone would do. In such case we should therefore
have
\[
xy = x.
\]
As $y$ is, however, supposed to have the same meaning as $x$, we
may replace it in the above equation by $x$, and we thus get
\[
xx = x.
\]
Now in common Algebra the combination $xx$ is more briefly represented
by $x^2$. Let us adopt the same principle of notation
here; for the mode of expressing a particular succession of mental
operations is a thing in itself quite as arbitrary as the mode of
expressing a single idea or operation (II. 3). In accordance with
this notation, then, the above equation assumes the form
\begin{equation}
x^2 = x,
\end{equation}
and is, in fact, the expression of a second general law of those
symbols by which names, qualities, or descriptions, are symbolically
represented.

The reader must bear in mind that although the symbols $x$
and $y$ in the examples previously formed received significations
distinct from each other, nothing prevents us from attributing to
them precisely the same signification. It is evident that the
more nearly their actual significations approach to each other,
the more nearly does the class of things denoted by the combination
$xy$ approach to identity with the class denoted by $x$, as
well as with that denoted by $y$. The case supposed in the demonstration
of the equation (2) is that of $absolute$ identity of
meaning. The law which it expresses is practically exemplified
in language. To say ``good, good,'' in relation to any subject,
though a cumbrous and useless pleonasm, is the same as to say
``good.'' Thus ``good, good'' men, is equivalent to ``good''
men. Such repetitions of words are indeed sometimes employed
to heighten a quality or strengthen an affirmation. But this
effect is merely secondary and conventional; it is not founded in
the intrinsic relations of language and thought. Most of the
operations which we observe in nature, or perform ourselves, are
of such a kind that their effect is augmented by repetition, and
this circumstance prepares us to expect the same thing in language,
and even to use repetition when we design to speak with
emphasis. But neither in strict reasoning nor in exact discourse
is there any just ground for such a practice.

10. We pass now to the consideration of another class of the
signs of speech, and of the laws connected with their use.

\begin{center}
\textsc {class ii}.
\end{center}

11. \textit{Signs of those mental operations whereby we collect parts
into a whole, or separate a whole into its parts.}

We are not only capable of entertaining the conceptions of
objects, as characterized by names, qualities, or circumstances,
applicable to each individual of the group under consideration,
but also of forming the aggregate conception of a group of objects
consisting of partial groups, each of which is separately named
or described. For this purpose we use the conjunctions ``and,''
``or,'' \&c. ``Trees and minerals,'' ``barren mountains, or fertile
vales,'' are examples of this kind. In strictness, the words
``and,'' ``or,'' interposed between the terms descriptive of two or
more classes of objects, imply that those classes are quite distinct,
so that no member of one is found in another. In this and in
all other respects the words ``and'' ``or'' are analogous with the
sign $+$ in algebra, and their laws are identical. Thus the expression
``men and women'' is, conventional meanings set aside,
equivalent with the expression ``women and men.'' Let $x$ represent
``men,'' $y$, ``women;'' and let $+$ stand for ``\textit{and}'' and ``\textit{or},''
then we have
\begin{equation}
x + y = y + x,
\end{equation}
an equation which would equally hold true if $x$ and $y$ represented
\textit{numbers}, and $+$ were the sign of arithmetical addition.

Let the symbol $z$ stand for the adjective ``European,'' then
since it is, in effect, the same thing to say ``European men and
women,'' as to say ``European men and European women,'' we
have
\begin{equation}
z\left(x + y\right) = zx + zy.
\end{equation}
And this equation also would be equally true were $x$, $y$, and $z$
symbols of number, and were the juxtaposition of two literal
symbols to represent their algebraic product, just as in the logical
signification previously given, it represents the class of objects to
which both the epithets conjoined belong.

The above are the laws which govern the use of the sign
$+$, here used to denote the positive operation of aggregating
parts into a whole. But the very idea of an operation effecting
some positive change seems to suggest to us the idea of an opposite
or negative operation, having the effect of undoing what the
former one has done. Thus we cannot conceive it possible to
collect parts into a whole, and not conceive it also possible to
separate a part from a whole. This operation we express in
common language by the sign except, as, ``All men \textit{except}
Asiatics,'' ``All states \textit{except} those which are monarchical.''
Here it is implied that the things excepted form a part of the
things from which they are excepted. As we have expressed
the operation of aggregation by the sign $+$, so we may express
the negative operation above described by $-$ minus. Thus if $x$
be taken to represent men, and $y$, Asiatics, i. e. Asiatic men,
then the conception of ``All men except Asiatics'' will be expressed
by $x - y$. And if we represent by $x$, ``states,'' and by
$y$ the descriptive property ``having a monarchical form,'' then
the conception of ``All states except those which are monarchical''
will be expressed by $x - xy$.

As it is indifferent for all the \textit{essential} purposes of reasoning
whether we express excepted cases first or last in the order of
speech, it is also indifferent in what order we write any series of
terms, some of which are affected by the sign $-$. Thus we have,
as in the common algebra,
\begin{equation}
x-y = -y + x.
\end{equation}
Still representing by $x$ the class ``men,'' and by $y$ ``Asiatics,''
let $z$ represent the adjective ``white.'' Now to apply the adjective
``white'' to the collection of men expressed by the phrase
``Men except Asiatics,'' is the same as to say, ``White men,
except white Asiatics.'' Hence we have
\begin{equation}
z\left(x-y\right) = zx-zy.
\end{equation}
This is also in accordance with the laws of ordinary algebra.

The equations (4) and (6) may be considered as exemplification
of a single general law, which may be stated by saying, \textit{that
the literal symbols, $x$, $y$, $z$, \&c. are distributive in their operation.}
The general fact which that law expresses is this, viz.:---If any
quality or circumstance is ascribed to all the members of a group,
formed either by aggregation or exclusion of partial groups, the
resulting conception is the same as if the quality or circumstance
were first ascribed to each member of the partial groups, and the
aggregation or exclusion effected afterwards. That which is
ascribed to the members of the whole is ascribed to the members
of all its parts, howsoever those parts are connected together.

\begin{center}
\textsc{class iii}.
\end{center}

12. \textit{Signs by which relation is expressed, and by which we
form propositions.}

Though all verbs may with propriety be referred to this class,
it is sufficient for the purposes of Logic to consider it as including
only the substantive verb \textit{is} or \textit{are}, since every other verb
may be resolved into this element, and one of the signs included
under Class I. For as those signs are used to express quality or
circumstance of every kind, they may be employed to express
the active or passive relation of the subject of the verb, considered
with reference either to past, to present, or to future time.
Thus the Proposition, ``C\ae{}sar conquered the Gauls,'' may be
resolved into ``C\ae{}sar is he who conquered the Gauls.'' The
ground of this analysis I conceive to be the following:---Unless
we understand what is meant by having conquered the Gauls,
i.e. by the expression ``One who conquered the Gauls,'' we
cannot understand the sentence in question. It is, therefore,
truly an element of that sentence; another element is ``C\ae{}sar,''
and there is yet another required, the copula \textit{is} to show the
connexion of these two. I do not, however, affirm that there is
no other mode than the above of contemplating the relation expressed
by the proposition, ``C\ae{}sar conquered the Gauls;'' but
only that the analysis here given is a correct one for the particular
point of view which has been taken, and that it suffices for
the purposes of logical deduction. It may be remarked that the
passive and future participles of the Greek language imply the
existence of the principle which has been asserted, viz.: that the
sign \textit{is} or \textit{are} may be regarded as an element of every personal
verb.

13. The above sign, \textit{is}  or \textit{are} may be expressed by the symbol
$=$. The laws, or as would usually be said, the axioms which
the symbol introduces, are next to be considered.

Let us take the Proposition, ``The stars are the suns and the
planets,'' and let us represent stars by $x$, suns by $y$, and planets
by $z$; we have then
\begin{equation}
x=y + z.
\end{equation}
Now if it be true that the stars are the suns and the planets, it
will follow that the stars, except the planets, are suns. This
would give the equation
\begin{equation}
x - z = y,
\end{equation}
which must therefore be a deduction from (7). Thus a term $z$
has been removed from one side of an equation to the other by
changing its sign. This is in accordance with the algebraic rule
of transposition.

But instead of dwelling upon particular cases, we may at once
affirm the general axioms:---

1st. If equal things are added to equal things, the wholes are
equal.

2nd. If equal things are taken from equal things, the remainders
are equal.

And it hence appears that we may add or subtract equations,
and employ the rule of transposition above given just as in common
algebra.

Again: If two classes of things, $x$ and $y$, be identical, that is,
if all the members of the one are members of the other, then
those members of the one class which possess a given property $z$
will be identical with those members of the other which possess
the same property $z$. Hence if we have the equation
\[
x = y;
\]
then whatever class or property $z$ may represent, we have also
\[
zx = zy.
\]
This is formally the same as the algebraic law:---If both members
of an equation are multiplied by the same quantity, the
products are equal.

In like manner it may be shown that if the corresponding
members of two equations are multiplied together, the resulting
equation is true.

14. Here, however, the analogy of the present system with
that of algebra, as commonly stated, appears to stop. Suppose it
true that those members of a class $x$ which possess a certain property
$z$ are identical with those members of a class $y$ which possess
the same property $z$, it does not follow that the members of
the class $x$ universally are identical with the members of the
class $y$. Hence it cannot be inferred from the equation
\[
zx = zy,
\]
that the equation
\[
x = y
\]
is also true. In other words, the axiom of algebraists, that both
sides of an equation may be divided by the same quantity, has no
formal equivalent here. I say no \textit{formal equivalent}, because, in
accordance with the general spirit of these inquiries, it is not
even sought to determine whether the mental operation which is
represented by removing a logical symbol, $z$, from a combination
$zx$, is in itself analogous with the operation of division in Arithmetic.
That mental operation is indeed identical with what is
commonly termed Abstraction, and it will hereafter appear that
its laws are dependent upon the laws already deduced in this
chapter. What has now been shown is, that there does not
exist among those laws anything analogous in \textit{form} with a commonly
received axiom of Algebra.

But a little consideration will show that even in common
algebra that axiom does not possess the generality of those other
axioms which have been considered. The deduction of the
equation $x = y$ from the equation $zx = zy$ is only valid when it
is known that $z$ is not equal to $0$. If then the value $z = 0$ is
supposed to be admissible in the algebraic system, the axiom
above stated ceases to be applicable, and the analogy before exemplified
remains at least unbroken.

15. However, it is not with the symbols of quantity generally
that it is of any importance, except as a matter of speculation, to
trace such affinities. We have seen (II. 9) that the symbols of
Logic are subject to the special law,
\[
x^2 = x.
\]
Now of the symbols of Number there are but two, viz. $0$ and $1$,
which are subject to the same formal law. We know that $0^2 = 0$,
and that $1^2 = 1$; and the equation $x^2 = x$, considered as algebraic,
has no other roots than $0$ and $1$. Hence, instead of determining
the measure of formal agreement of the symbols of Logic with
those of Number generally, it is more immediately suggested to
us to compare them with symbols of quantity \textit{admitting only of
the values $0$ and $1$}. Let us conceive, then, of an Algebra in
which the symbols $x$, $y$, $z$, etc. admit indifferently of the values
$0$ and $1$, and of these values alone. The laws, the axioms, and
the processes, of such an Algebra will be identical in their whole
extent with the laws, the axioms, and the processes of an Algebra
of Logic. Difference of interpretation will alone divide
them. Upon this principle the method of the following work is
established.

16. It now remains to show that those constituent parts of
ordinary language which have not been considered in the previous
sections of this chapter are either resolvable into the same
elements as those which have been considered, or are subsidiary
to those elements by contributing to their more precise definition.

The substantive, the adjective, and the verb, together with
the particles \textit{and}, \textit{except}, we have already considered. The pronoun
may be regarded as a particular form of the substantive or
the adjective. The adverb modifies the meaning of the verb, but
does not affect its nature. Prepositions contribute to the expression
of circumstance or relation, and thus tend to give precision
and detail to the meaning of the literal symbols. The
conjunctions \textit{if}, \textit{either}, \textit{or}, are used chiefly in the expression of
relation among propositions, and it will hereafter be shown that
the same relations can be completely expressed by elementary
symbols analogous in interpretation, and identical in form and
law with the symbols whose use and meaning have been explained
in this Chapter. As to any remaining elements of
speech, it will, upon examination, be found that they are used
either to give a more definite significance to the terms of discourse,
and thus enter into the interpretation of the literal symbols
already considered, or to express some emotion or state of
feeling accompanying the utterance of a proposition, and thus do
not belong to the province of the understanding, with which
alone our present concern lies. Experience of its use will testify
to the sufficiency of the classification which has been adopted.


\chapter[DERIVATION OF THE LAWS]{\large DERIVATION OF THE LAWS OF THE SYMBOLS OF LOGIC FROM THE
LAWS OF THE OPERATIONS OF THE HUMAN MIND.}


1. The object of science, properly so called, is the knowledge
of laws and relations. To be able to distinguish what
is essential to this end, from what is only accidentally associated
with it, is one of the most important conditions of scientific progress.
I say, to \textit{distinguish} between these elements, because a consistent
devotion to science does not require that the attention
should be altogether withdrawn from other speculations, often of a
metaphysical nature, with which it is not unfrequently connected.
Such questions, for instance, as the existence of a sustaining
ground of ph\ae nomena, the reality of cause, the propriety of forms
of speech implying that the successive states of things are connected
by \textit{operations}, and others of a like nature, may possess
a deep interest and significance in relation to science, without
being essentially scientific. It is indeed scarcely possible to
express the conclusions of natural science without borrowing
the language of these conceptions. Nor is there necessarily
any practical inconvenience arising from this source. They who
believe, and they who refuse to believe, that there is more in the
relation of cause and effect than an invariable order of succession,
agree in their interpretation of the conclusions of physical astronomy.
But they only agree because they recognise a common element
of scientific truth, which is independent of their particular
views of the nature of causation.

2. If this distinction is important in physical science, much
more does it deserve attention in connexion with the science of
the intellectual powers. For the questions which this science
presents become, in expression at least, almost necessarily mixed
up with modes of thought and language, which betray a metaphysical
origin. The idealist would give to the laws of reasoning
one form of expression; the sceptic, if true to his principles, another.
They who regard the ph\ae nomena with which we are concerned
in this inquiry as the mere successive \textit{states} of the thinking
subject devoid of any causal connexion, and they who refer them
to the \textit{operations} of an active intelligence, would, if consistent,
equally differ in their modes of statement. Like difference would
also result from a difference of classification of the mental faculties.
Now the principle which I would here assert, as affording us the
only ground of confidence and stability amid so much of seeming
and of real diversity, is the following, viz., that if the laws in question
are really deduced from observation, they have a real existence
as laws of the human mind, independently of any metaphysical
theory which may seem to be involved in the mode of their statement.
They contain an element of truth which no ulterior criticism
upon the nature, or even upon the reality, of the mind's
operations, can essentially affect. Let it even be granted that
the mind is but a succession of states of consciousness, a series
of fleeting impressions uncaused from without or from within,
emerging out of nothing, and returning into nothing again,---the
last refinement of the sceptic intellect,---still, as laws of succession,
or at least of a past succession, the results to which observation
had led would remain true. They would require to be
interpreted into a language from whose vocabulary all such terms
as cause and effect, operation and subject, substance and attribute,
had been banished; but they would still be valid as scientific
truths.

Moreover, as any statement of the laws of thought, founded
upon actual observation, must thus contain scientific elements
which are independent of metaphysical theories of the nature of
the mind, the practical application of such elements to the construction
of a system or method of reasoning must also be independent
of metaphysical distinctions. For it is upon the scientific
elements involved in the statement of the laws, that any
practical application will rest, just as the practical conclusions of
physical astronomy are independent of any theory of the cause
of gravitation, but rest only on the knowledge of its ph\ae{}nomenal
effects. And, therefore, as respects both the determination
of the laws of thought, and the practical use of them
when discovered, we are, for all really scientific ends, unconcerned
with the truth or falsehood of any metaphysical speculations
whatever.

3. The course which it appears to me to be expedient, under
these circumstances, to adopt, is to avail myself as far as possible
of the language of common discourse, without regard to any
theory of the nature and powers of the mind which it may be
thought to embody. For instance, it is agreeable to common
usage to say that we converse with each other by the communication
of ideas, or conceptions, such communication being the
office of words; and that with reference to any particular ideas or
conceptions presented to it, the mind possesses certain powers or
faculties by which the mental regard maybe fixed upon some ideas,
to the exclusion of others, or by which the given conceptions or
ideas may, in various ways, be combined together. To those
faculties or powers different names, as Attention, Simple Apprehension,
Conception or Imagination, Abstraction, \&c., have been
given,---names which have not only furnished the titles of distinct
divisions of the philosophy of the human mind, but passed into
the common language of men. Whenever, then, occasion shall
occur to use these terms, I shall do so without implying thereby
that I accept the theory that the mind possesses such and such
powers and faculties as distinct elements of its activity. Nor is
it indeed necessary to inquire whether such powers of the understanding
have a distinct existence or not. We may merge these
different titles under the one generic name of \textit{Operations} of the
human mind, define these operations so far as is necessary for the
purposes of this work, and then seek to express their ultimate laws.
Such will be the general order of the course which I shall pursue,
though reference will occasionally be made to the names which
common agreement has assigned to the particular states or operations
of the mind which may fall under our notice.

It will be most convenient to distribute the more definite results
of the following investigation into distinct Propositions.

\begin{center}
\textsc{Proposition I.}
\end{center}

4. \textit{To deduce the laws of the symbols of Logic from a consideration
of those operations of the mind which are implied in the strict
use of language as an instrument of reasoning.}

In every discourse, whether of the mind conversing with its
own thoughts, or of the individual in his intercourse with others,
there is an assumed or expressed limit within which the subjects of
its operation are confined. The most unfettered discourse is that
in which the words we use are understood in the widest possible
application, and for them the limits of discourse are co-extensive
with those of the universe itself. But more usually we confine ourselves
to a less spacious field. Sometimes, in discoursing of men
we imply (without expressing the limitation) that it is of men
only under certain circumstances and conditions that we speak,
as of civilized men, or of men in the vigour of life, or of men
under some other condition or relation. Now, whatever may be
the extent of the field within which all the objects of our discourse
are found, that field may properly be termed the universe
of discourse.

5. Furthermore, this universe of discourse is in the strictest
sense the ultimate \textit{subject} of the discourse. The office of any name
or descriptive term employed under the limitations supposed is not
to raise in the mind the conception of all the beings or objects to
which that name or description is applicable, but only of those
which exist within the supposed universe of discourse. If that
universe of discourse is the actual universe of things, which it
always is when our words are taken in their real and literal sense,
then by men we mean \textit{all men that exist}; but if the universe of
discourse is limited by any antecedent implied understanding,
then it is of men under the limitation thus introduced that we
speak. It is in both cases the business of the word \textit{men} to direct
a certain operation of the mind, by which, from the proper universe
of discourse, we select or fix upon the individuals signified.

6. Exactly of the same kind is the mental operation implied
by the use of an adjective. Let, for instance, the universe of discourse
be the actual Universe. Then, as the word \textit{men} directs
us to select mentally from that Universe all the beings to which
the term ``men'' is applicable; so the adjective ``good,'' in the
combination ``good men,'' directs us still further to select mentally
from the class of \textit{men} all those who possess the further
quality ``good;'' and if another adjective were prefixed to the
combination ``good men,'' it would direct a further operation of
the same nature, having reference to that further quality which
it might be chosen to express.

It is important to notice carefully the real nature of the operation
here described, for it is conceivable, that it might have
been different from what it is. Were the adjective simply \textit{attributive}
in its character, it would seem, that when a particular set
of beings is designated by \textit{men}, the prefixing of the adjective
\textit{good} would direct us to attach mentally to all those beings the
quality of goodness. But this is not the real office of the adjective.
The operation which we really perform is one of \textit{selection
according to a prescribed principle or idea}. To what faculties
of the mind such an operation would be referred, according
to the received classification of its powers, it is not important to
inquire, but I suppose that it would be considered as dependent
upon the two faculties of Conception or Imagination, and Attention.
To the one of these faculties might be referred the formation
of the general conception; to the other the fixing of the
mental regard upon those individuals within the prescribed universe
of discourse which answer to the conception. If, however,
as seems not improbable, the power of Attention is nothing more
than the power of continuing the exercise of any other faculty of the
mind, we might properly regard the whole of the mental process
above described as referrible to the mental faculty of Imagination
or Conception, the first step of the process being the conception
of the Universe itself, and each succeeding step limiting in a definite
manner the conception thus formed. Adopting this view, I
shall describe each such step, or any definite combination of such
steps, as a \textit{definite act of conception}. And the use of this term I
shall extend so as to include in its meaning not only the conception
of classes of objects represented by particular names or simple
attributes of quality, but also the combination of such conceptions
in any manner consistent with the powers and limitations
of the human mind; indeed, any intellectual operation short
of that which is involved in the structure of a sentence or proposition.
The general laws to which such operations of the mind
are subject are now to be considered.

7. Now it will be shown that the laws which in the preceding
chapter have been determined \textit{\`{a} posteriori} from the constitution
of language, for the use of the literal symbols of Logic,
are in reality the laws of that definite mental operation which
has just been described. We commence our discourse with a
certain understanding as to the limits of its subject, i.e. as to
the limits of its Universe. Every name, every term of description
that we employ, directs him whom we address to the performance
of a certain mental operation upon that subject. And
thus is thought communicated. But as each name or descriptive
term is in this view but the representative of an intellectual operation,
that operation being also prior in the order of nature, it
is clear that the laws of the name or symbol must be of a derivative
character,---must, in fact, originate in those of the operation
which they represent. That the laws of the symbol and of the
mental process are identical in expression will now be shown.

8. Let us then suppose that the universe of our discourse is
the actual universe, so that words are to be used in the full extent
of their meaning, and let us consider the two mental operations
implied by the words ``white'' and ``men.'' The word
``men'' implies the operation of selecting in thought from its
subject, the universe, all men; and the resulting conception,
\textit{men}, becomes the subject of the next operation. The operation
implied by the word ``white'' is that of selecting from its subject,
``men,'' all of that class which are white. The final resulting
conception is that of ``white men.'' Now it is perfectly apparent
that if the operations above described had been performed
in a converse order, the result would have been the same. Whether
we begin by forming the conception of ``\textit{men},'' and then
by a second intellectual act limit that conception to ``white
men,'' or whether we begin by forming the conception of ``white
objects,'' and then limit it to such of that class as are ``men,'' is
perfectly indifferent so far as the result is concerned. It is obvious
that the order of the mental processes would be equally
indifferent if for the words ``white'' and ``men'' we substituted
any other descriptive or appellative terms whatever, provided
only that their meaning was fixed and absolute. And thus the
indifference of the order of two successive acts of the faculty of
Conception, the one of which furnishes the subject upon which
the other is supposed to operate, is a general condition of the
exercise of that faculty. It is a law of the mind, and it is the
real origin of that law of the literal symbols of Logic which constitutes
its formal expression (1) Chap. II.

9. It is equally clear that the mental operation above described
is of such a nature that its effect is not altered by repetition.
Suppose that by a definite act of conception the attention
has been fixed upon men, and that by another exercise of the
same faculty we limit it to those of the race who are white.
Then any further repetition of the latter mental act, by which
the attention is limited to white objects, does not in any way
modify the conception arrived at, viz., that of white men. This
is also an example of a general law of the mind, and it has its
formal expression in the law ((2) Chap, II.) of the literal symbols.

10. Again, it is manifest that from the conceptions of two
distinct classes of things we can form the conception of that collection
of things which the two classes taken together compose;
and it is obviously indifferent in what order of position or of
priority those classes are presented to the mental view. This is
another general law of the mind, and its expression is found in
(3) Chap. II.

11. It is not necessary to pursue this course of inquiry and
comparison. Sufficient illustration has been given to render manifest
the two following positions, viz.:

First, That the operations of the mind, by which, in the
exercise of its power of imagination or conception, it combines
and modifies the simple ideas of things or qualities, not less than
those operations of the reason which are exercised upon truths
and propositions, are subject to general laws.

Secondly, That those laws are mathematical in their form,
and that they are actually developed in the essential laws of
human language. Wherefore the laws of the symbols of Logic
are deducible from a consideration of the operations of the mind
in reasoning.

12. The remainder of this chapter will be occupied with
questions relating to that law of thought whose expression is
$x^2 = x$ (II. 9), a law which, as has been implied (II. 15), forms
the characteristic distinction of the operations of the mind in its
ordinary discourse and reasoning, as compared with its operations
when occupied with the general algebra of quantity. An important
part of the following inquiry will consist in proving that
the symbols $0$ and $1$ occupy a place, and are susceptible of an
interpretation, among the symbols of Logic; and it may first be
necessary to show how particular symbols, such as the above,
may with propriety and advantage be employed in the representation
of distinct systems of thought.

The ground of this propriety cannot consist in any community
of interpretation. For in systems of thought so truly
distinct as those of Logic and Arithmetic (I use the latter term
in its widest sense as the science of Number), there is, properly
speaking, no community of subject. The one of them is conversant
with the very conceptions of things, the other takes account
solely of their numerical relations. But inasmuch as the forms
and methods of any system of reasoning depend immediately upon
the laws to which the symbols are subject, and only mediately,
through the above link of connexion, upon their interpretation,
there may be both propriety and advantage in employing the
same symbols in different systems of thought, provided that such
interpretations can be assigned to them as shall render their formal
laws identical, and their use consistent. The ground of that
employment will not then be community of interpretation, but
the community of the formal laws, to which in their respective
systems they are subject. Nor must that community of formal
laws be established upon any other ground than that of a careful
observation and comparison of those results which are seen to
flow independently from the interpretations of the systems under
consideration.

These observations will explain the process of inquiry adopted
in the following Proposition. The literal symbols of Logic are
universally subject to the law whose expression is $x^2 = x$. Of
the symbols of Number there are two only, $0$ and $1$, which satisfy
this law. But each of these symbols is also subject to a law
peculiar to itself in the system of numerical magnitude, and this
suggests the inquiry, what interpretations must be given to the
literal symbols of Logic, in order that the same peculiar and
formal laws may be realized in the logical system also.

\begin{center}
\textsc{Proposition II}
\end{center}
%*** Need to reset equation numbering here ***%
\setcounter{equation}{0}
13. \textit{To determine the logical value and significance of the
symbols $0$ and $1$.}

The symbol $0$, as used in Algebra, satisfies the following formal
law,
\begin{equation}
0 \times y = 0, \textrm{ or } 0y = 0,
\end{equation}
whatever \textit{number} $y$ may represent. That this formal law may be
obeyed in the system of Logic, we must assign to the symbol $0$
such an interpretation that the class represented by $0y$ may be
identical with the class represented by $0$, whatever the class $y$
may be. A little consideration will show that this condition is
satisfied if the symbol $0$ represent Nothing. In accordance with
a previous definition, we may term Nothing a class. In fact,
Nothing and Universe are the two limits of class extension, for
they are the limits of the possible interpretations of general
names, none of which can relate to fewer individuals than are
comprised in Nothing, or to more than are comprised in the
Universe. Now whatever the class $y$ may be, the individuals
which are common to it and to the class ``Nothing'' are identical
with those comprised in the class ``Nothing,'' for they are
none. And thus by assigning to $0$ the interpretation Nothing,
the law (1) is satisfied; and it is not otherwise satisfied
consistently with the perfectly general character of the class $y$.

Secondly, The symbol $1$ satisfies in the system of Number
the following law, viz.,
\[
1 \times y = y, \textrm{ or } 1y = y,
\]
whatever number $y$ may represent. And this formal equation
being assumed as equally valid in the system of this work, in
which $1$ and $y$ represent classes, it appears that the symbol $1$
must represent such a class that all the individuals which are
found in \textit{any} proposed class $y$ are also all the individuals $1y$ that
are common to that class $y$ and the class represented by $1$. A
little consideration will here show that the class represented by $1$
must be ``the Universe,'' since this is the only class in which
are found \textit{all} the individuals that exist in \textit{any} class. Hence the
respective interpretations of the symbols $0$ and $1$ in the system
of Logic are \textit{Nothing} and \textit{Universe}.

14. As with the idea of any class of objects as ``men,'' there
is suggested to the mind the idea of the contrary class of beings
which are not men; and as the whole Universe is made up of
these two classes together, since of every individual which it
comprehends we may affirm either that it is a man, or that it is
not a man, it becomes important to inquire how such contrary
names are to be expressed. Such is the object of the following
Proposition.

\begin{center}
\textsc{Proposition III.}
\end{center}

\textit{If $x$ represent any class of objects, then will $1 - x$ represent the
contrary or supplementary class of objects., i.e. the class including
all objects which are not comprehended in the class $x$.}
%[*i. e. originally NOT italicized.
%this is how I think it should be done,
%but you might be of another opinion.]

For greater distinctness of conception let $x$ represent the class
men, and let us express, according to the last Proposition, the
Universe by $1$; now if from the conception of the Universe, as
consisting of ``men'' and ``not-men,'' we exclude the conception
of ``men,'' the resulting conception is that of the contrary class,
``not-men.'' Hence the class ``not-men'' will be represented by
$1 - x$. And, in general, whatever class of objects is represented
by the symbol $x$, the contrary class will be expressed by $1 - x$.

15. Although the following Proposition belongs in strictness
to a future chapter of this work, devoted to the subject of
\textit{maxims} or \textit{necessary truths}, yet, on account of the great importance
of that law of thought to which it relates, it has been
thought proper to introduce it here.

\begin{center}
\textsc{Proposition IV.}
\end{center}

\textit{That axiom of metaphysicians which is termed the principle of
contradiction, and which affirms that it is impossible for any being to
possess a quality, and at the same time not to possess it, is a consequence
of the fundamental law of thought, whose expression is $x^2 = x$.}

Let us write this equation in the form
\[
x - x^2 = 0,
\]
whence we have
\setcounter{equation}{0}
\begin{equation}
x\left(1-x\right) = 0;
\end{equation}
both these transformations being justified by the axiomatic laws
of combination and transposition (II. 13). Let us, for simplicity
of conception, give to the symbol $x$ the particular interpretation
of \textit{men}, then $1 - x$ will represent the class: of ``not-men''
(Prop. III.) Now the formal product of the expressions of two
classes represents that class of individuals which is common to
them both (II. 6). Hence $x\left(1 - x\right)$ will represent the class
whose members are at once ``men,'' and ``not men,'' and the
equation (1) thus express the principle, \textit{that a class whose members
are at the same time men and not men does not exist}. In
other words, that \textit{it is impossible for the same individual to be at
the same time a man and not a man}. Now let the meaning of
the symbol $x$ be extended from the representing of ``men,'' to
that of any class of beings characterized by the possession of any
quality whatever; and the equation (1) will then express that it
is impossible for a being to possess a quality and not to possess
that quality at the same time. But this is identically that
``principle of contradiction'' which Aristotle has described as the
fundamental axiom of all philosophy. ``It is impossible that the
same quality should both belong and not belong to the same
thing.\dots This is the most certain of all principles.\dots Wherefore
they who demonstrate refer to this as an ultimate opinion. For
it is by nature the source of all the other axioms.''\footnote{
\textgreek{T`o g`ar a>ut`o >'ama <up'arqein te ka`i m`h <up'arqein >ad'unaton t\~y a>ut\~y ka`i kat`a
t`o a>ut'o\dots A<'uth d`h pasv\~wn  >est`i bebaiot'ath t\~wn >arq\~wn\dots Di`o p'antes o<i >apodeikn'untes
e>is ta'uthn >an'agousin >esq'athn d'oxan' f'usei g`ar >arx`h ka`i t\~wn >'allwn
>axewm'atwn a<'uth p'antwn.}---\textit{Metaphysica}, III, 3.}
%**first round proofer notes below
%[Footnote: \textit{[Greek: To gar auto ama uparchein te kai mae uparchein adunaton tps[*!] autps[*!] kai kata
%to auto.... Autae dae pas*n esti bebaiotatae t*v arch*v.... Dio *ante oi apodeik-*nunte
%ei tautaen anagousin eschataen doxan phusei gar arch* kai t*n all*n
%axi*mar*n aurae panr*n].--Metaphysica, III. 3.}]
%[*didn't care about diacritical marks, since I can't distinguish them from non-diacriticals (and no mark can be both, right?).]
%[*this really needs a thorough 2nd round. I don't feel fluent with the Greek at all]

The above interpretation has been introduced not on account
of its immediate value in the present system, but as an illustration
of a significant fact in the philosophy of the intellectual powers,
viz., that what has been commonly regarded as the fundamental
axiom of metaphysics is but the consequence of a law of thought,
mathematical in its form. I desire to direct attention also to the
circumstance that the equation (1) in which that fundamental
law of thought is expressed is an equation of the second degree.\footnote{%
%*really long footnote
Should it here be said that the existence of the equation $x^2=x$ necessitates
also the existence of the equation $x^3=x$, which is of the third degree, and then
inquired whether that equation does not indicate a process of \textit{trichotomy}; the
answer is, that the equation $x^3=x$ is not interpretable in the system of logic.
For writing it in either of the forms
\begin{eqnarray}
x\left(1-x\right)\left(1+x\right)&=&0,\\
x\left(1-x\right)\left(-1-x\right)&=&0,
\end{eqnarray}
we see that its interpretation, if possible at all, must involve that of the factor
$1+x$, or of the factor $-1-x$. The former is not interpretable, because we
cannot conceive of the addition of any class $x$ to the universe $1$; the latter is not
interpretable, because the symbol $-1$ is not subject to the law $x(1-x)=0$, to
which all class symbols are subject. Hence the equation $x^3=x$ admits of no interpretation
analogous to that of the equation $x^2=x$. Were the former equation,
however, true independently of the latter, i.e. were that act of the mind which
is denoted by the symbol $x$, such that its second repetition should reproduce the
result of a single operation, but not its first or mere repetition, it is presumable
that we should be able to interpret one of the forms (1), (2), which under the
actual conditions of thought we cannot do. There exist operations, known to
the mathematician, the law of which may be adequately expressed by the equation
$x^3=x$. But they are of a nature altogether foreign to the province of
general reasoning.

In saying that it is conceivable that the law of thought might have been different
from what it is, I mean only that we can frame such an hypothesis, and
study its consequences. The possibility of doing this involves no such doctrine
  as that the actual law of human reason is the product either of chance or of arbitrary
will.}
Without speculating at all in this chapter upon the question,
whether that circumstance is necessary in its own nature, we
may venture to assert that if it had not existed, the whole procedure
of the understanding would have been different from what
it is. Thus it is a consequence of the fact that the fundamental
equation of thought is of the second degree, that we perform the
operation of analysis and classification, by division into pairs of
opposites, or, as it is technically said, by \textit{dichotomy.} Now if the
equation in question had been of the third degree, still admitting
of interpretation as such, the mental division must have been
threefold in character, and we must have proceeded by a species
of \textit{trichotomy}, the real nature of which it is impossible for us,
with our existing faculties, adequately to conceive, but the laws
of which we might still investigate as an object of intellectual
speculation.

16. The law of thought expressed by the equation (1) will,
for reasons which are made apparent by the above discussion, be
occasionally referred to as the ``law of duality.''

\chapter[DIVISION OF PROPOSITIONS]
{\large OF THE DIVISION OF PROPOSITIONS INTO THE TWO CLASSES OF
``PRIMARY'' AND ``SECONDARY;'' OF THE CHARACTERISTIC PROPERTIES
OF THOSE CLASSES, AND OF THE LAWS OF THE EXPRESSION
OF PRIMARY PROPOSITIONS.}

1. The laws of those mental operations which are concerned
in the processes of Conception or Imagination having
been investigated, and the corresponding laws of the symbols
by which they are represented explained, we are led to consider
the practical application of the results obtained: first, in the
expression of the complex terms of propositions; secondly, in
the expression of propositions; and lastly, in the construction of
a general method of deductive analysis. In the present chapter
we shall be chiefly concerned with the first of these objects, as
an introduction to which it is necessary to establish the following
Proposition:

\begin{center}
\textsc{Proposition I.}
\end{center}

\textit{All logical propositions may be considered as belonging to one
or the other of two great classes, to which the respective names of
``Primary'' or ``Concrete Propositions,'' and ``Secondary'' or ``Abstract
Propositions,'' may be given.}

Every assertion that we make may be referred to one or the
other of the two following kinds. Either it expresses a relation
among \textit{things}, or it expresses, or is equivalent to the expression of,
a relation among \textit{propositions}. An assertion respecting the properties
of things, or the ph\ae{}nomena which they manifest, or the
circumstances in which they are placed, is, properly speaking, the
assertion of a relation among things. To say that ``snow is
white,'' is for the ends of logic equivalent to saying, that ``snow
is a white thing.'' An assertion respecting facts or events, their
mutual connexion and dependence, is, for the same ends, generally
equivalent to the assertion, that such and such propositions concerning
those events have a certain relation to each other as
respects their mutual truth or falsehood. The former class of
propositions, relating to \textit{things}, I call ``Primary;'' the latter class,
relating to \textit{propositions}, I call ``Secondary.'' The distinction is
in practice nearly but not quite co-extensive with the common
logical distinction of propositions as categorical or hypothetical.

For instance, the propositions, ``The sun shines,'' ``The earth
is warmed,'' are primary; the proposition, ``If the sun shines
the earth is warmed,'' is secondary. To say, ``The sun shines,''
is to say, ``The sun is that which shines,'' and it expresses a relation between two classes of things, viz., ``the sun'' and ``things
which shine.'' The secondary proposition, however, given above,
expresses a relation of dependence between the two primary propositions,
``The sun shines,'' and ``The earth is warmed.'' I do not
hereby affirm that the relation between these propositions is, like
that which exists between the facts which they express, a relation
of causality, but only that the relation among the propositions
so implies, and is so implied by, the relation among the
facts, that it may for the ends of logic be used as a fit representative
of that relation.

2. If instead of the proposition, ``The sun shines,'' we say,
``It is true that the sun shines,'' we then speak not directly of
things, but of a proposition concerning things, viz., of the proposition,
``The sun shines.'' And, therefore, the proposition in
which we thus speak is a secondary one. Every primary proposition
may thus give rise to a secondary proposition, viz., to
that secondary proposition which asserts its truth, or declares its
falsehood.

It will usually happen, that the particles \textit{if}, \textit{either}, \textit{or}, will
indicate that a proposition is secondary; but they do not necessarily
imply that such is the case. The proposition, ``Animals
are either rational or irrational,'' is primary. It cannot be resolved
into ``Either animals are rational or animals are irrational,''
and it does not therefore express a relation of dependence
between the two propositions connected together in the latter
disjunctive sentence. The particles, \textit{either}, \textit{or}, are in fact no
\textit{criterion} of the nature of propositions, although it happens that
they are more frequently found in secondary propositions. Even
the conjunction \textit{if} may be found in primary propositions. ``Men
are, if wise, then temperate,'' is an example of the kind. It
cannot be resolved into ``If all men are wise, then all men are
temperate.''

3. As it is not my design to discuss the merits or defects of
the ordinary division of propositions, I shall simply remark here,
that the principle upon which the present classification is founded
is clear and definite in its application, that it involves a real
and fundamental distinction in propositions, and that it is of
essential importance to the development of a general method of
reasoning. Nor does the fact that a primary proposition may
be put into a form in which it becomes secondary at all conflict
with the views here maintained. For in the case thus supposed,
it is not of the things connected together in the primary proposition
that any direct account is taken, but only of the proposition
itself considered as \textit{true} or as \textit{false}.

4. In the expression both of primary and of secondary propositions,
the same symbols, subject, as it will appear, to the same
laws, will be employed in this work. The difference between
the two cases is a difference not of form but of interpretation.
In both cases the actual relation which it is the object of the
proposition to express will be denoted by the sign $=$. In the
expression of primary propositions, the members thus connected
will usually represent the ``terms'' of a proposition, or, as they
are more particularly designated, its subject and predicate.

\begin{center}
\textsc{Proposition II}.
\end{center}

5. \textit{To deduce a general method, founded upon the enumeration of
possible varieties, for the expression of any class or collection of things,
which may constitute a ``term'' of a Primary Proposition.}

First, If the class or collection of things to be expressed is
defined only by names or qualities common to all the individuals
of which it consists, its expression will consist of a single term,
in which the symbols expressive of those names or qualities will
be combined without any connecting sign, as if by the algebraic
process of multiplication. Thus, if $x$ represent opaque
substances, $y$ polished substances, $z$ stones, we shall have,

\begin{itemize}
\item[] $xyz = $ opaque polished stones;
\item[] $xy (1 - z) = $ opaque polished substances which are not stones;
\item[] $x (1 - y)(1 - z) = $ opaque substances which are not polished,
and are not stones;
\end{itemize}

and so on for any other combination. Let it be observed, that
each of these expressions satisfies the same law of duality, as the
individual symbols which it contains. Thus,
\begin{eqnarray*}
&&xyz \times xyz = xyz;\\
&&xy (1 - z) \times xy (1 - z) = xy (1 - z);
\end{eqnarray*}
and so on. Any such term as the above we shall designate as
a ``class term,'' because it expresses a class of things by means
of the common properties or names of the individual members of
such class.

Secondly, If we speak of a collection of things, different
portions of which are defined by different properties, names, or
attributes, the expressions for those different portions must be
separately formed, and then connected by the sign $+$. But if
the collection of which we desire to speak has been formed by
excluding from some wider collection a defined portion of its
members, the sign $-$ must be prefixed to the symbolical expression
of the excluded portion. Respecting the use of these symbols
some further observations may be added.

6. Speaking generally, the symbol $+$ is the equivalent of the
conjunctions ``and,'' ``or,'' and the symbol $-$, the equivalent of
the preposition ``except.'' Of the conjunctions ``and'' and ``or,''
the former is usually employed when the collection to be described
forms the subject, the latter when it forms the predicate,
of a proposition. ``The scholar \textit{and} the man of the world desire
happiness,'' may be taken as an illustration of one of these
cases. ``Things possessing utility are \textit{either} productive of pleasure
\textit{or} preventive of pain,'' may exemplify the other. Now
whenever an expression involving these particles presents itself
in a primary proposition, it becomes very important to know
whether the groups or classes separated in thought by them are
intended to be quite distinct from each other and mutually exclusive,
or not. Does the expression, ``Scholars and men of the
world,'' include or exclude those who are both? Does the ex-pression,
``Either productive of pleasure or preventive of pain,''
include or exclude things which possess both these qualities? I
apprehend that in strictness of meaning the conjunctions ``and,''
``or,'' do possess the power of separation or exclusion here referred
to; that the formula, ``All $x$'s are either $y$'s or $z$'s,''
rigorously interpreted, means, ``All $x$'s are either $y$'s, but not $z$'s,''
or, ``$z$'s but not $y$'s.'' But it must at the same time be admitted,
that the ``jus et norma loquendi'' seems rather to favour an opposite
interpretation. The expression, ``Either $y$'s or $z$'s,'' would
generally be understood to include things that are $y$'s and $z$'s at
the same time, together with things which come under the one,
but not the other. Remembering, however, that the symbol $+$
does possess the separating power which has been the subject of
discussion, we must resolve any disjunctive expression which may
come before us into elements really separated in thought, and
then connect their respective expressions by the symbol $+$.

And thus, according to the meaning implied, the expression,
``Things which are either $x$'s or $y$'s,'' will have two different symbolical
equivalents. If we mean, ``Things which are $x$'s, but
not $y$'s, or $y$'s, but not $x$'s,'' the expression will be
\[ x (1 - y) + y ( 1 - x); \]
the symbol $x$ standing for $x$'s, $y$ for $y$'s. If, however, we mean,
``Things which are either $x$'s, or, if not $x$'s, then $y$'s,'' the expression
will be
\[ x + y(1 - x). \]
This expression supposes the admissibility of things which are
both $x$'s and $y$'s at the same time. It might more fully be expressed
in the form
\[ xy + x (1 - y) + y ( 1 - x); \]
but this expression, on addition of the two first terms, only reproduces
the former one.

Let it be observed that the expressions above given satisfy
the fundamental law of duality (III. 16). Thus we have
\begin{eqnarray*}
\{x (1 - y) + y (1 - x)\}^2 &=& x (1 -y) + y (1 - x),\\
\{x + (1-x)\}^2 &=& x+y(1-x).\\
\end{eqnarray*}
It will be seen hereafter, that this is but a particular manifestation
of a general law of expressions representing ``classes or
collections of things.''

7. The results of these investigations may be embodied in
the following rule of expression.

\textsc{Rule}.---\textit{Express simple names or qualities by the symbols $x$, $y$, $z$,
\&c., their contraries by $1 - x$, $1 - y$, $1 - z$, \&c.; classes of things
defined by common names or qualities, by connecting the corresponding
symbols as in multiplication; collections of things, consisting of
portions different from each other, by connecting the expressions of
those portions by the sign $+$. In particular, let the expression, ``Either
$x$'s or $y$'s,'' be expressed by $x (1 - y) + y (1 - x)$, when the classes denoted
by $x$ and $y$ are exclusive, by $x + y (1 - x)$ when they are not
exclusive. Similarly let the expression, ``Either $x$'s, or $y$'s, or $z$'s,'' be
expressed by $x(1-y)(1-z) + y(1-x)(1-z) + z(1-x)(1-y)$,
when the classes denoted by $x$, $y$, and $z$, are designed to be mutually
exclusive, by $x + y (1 - x) + z (1 - x) (1-y)$, when they are not meant
to be exclusive, and so on.}

8. On this rule of expression is founded the converse rule of
interpretation. Both these will be exemplified with, perhaps,
sufficient fulness in the following instances. Omitting for brevity
the universal subject ``things,'' or ``beings,'' let us assume
\[ x = \textrm{hard}, y = \textrm{elastic}, z = \textrm{metals}; \]
and we shall have the following results:

\begin{center}
``Non-elastic metals,'' will be expressed by $z (1 - y)$;

``Elastic substances with non-elastic metals,'' by $y + z (1 - y)$;

``Hard substances, except metals,'' by $x-z$;

``Metallic substances, except those which are neither hard nor
elastic,'' by $z-z(1 - x)(1 - y)$, or by $z \{1 - (1 - x) (1 - y)\}$,
\textit{vide}~(6), Chap. II.
\end{center}

In the last example, what we had really to express was ``Metals,
except not hard, not elastic, metals.'' Conjunctions used between
\textit{adjectives} are usually superfluous, and, therefore, must
not be expressed symbolically.

Thus, ``Metals hard and elastic,'' is equivalent to ``Hard
elastic metals,'' and expressed by $xyz$.

Take next the expression, ``Hard substances, except those
which are metallic and non-elastic, and those which are elastic
and non-metallic.'' Here the word \textit{those} means hard substances,
so that the expression really means, \textit{Hard substances except hard
substances, metallic, non-elastic, and hard substances non-metallic,
elastic}; the word \textit{except} extending to both the classes which
follow it. The complete expression is

\begin{eqnarray*}
x - \{xz (1 - y) &+& xy (1 - z)\};\\
\textrm{or, } x &-& xz (1 - y) - xy (1 - z).\\
\end{eqnarray*}

9. The preceding Proposition, with the different illustrations
which have been given of it, is a necessary preliminary to the
following one, which will complete the design of the present
chapter.

\begin{center}
\textsc{Proposition III}.
\end{center}

\textit{To deduce from an examination of their possible varieties a general
method for the expression of Primary or Concrete Propositions.}

A primary proposition, in the most general sense, consists of
two terms, between which a relation is asserted to exist. These
terms are not necessarily single-worded names, but may represent
any collection of objects, such as we have been engaged in considering
in the previous sections. The mode of expressing those
terms is, therefore, comprehended in the general precepts above
given, and it only remains to discover how the relations between
the terms are to be expressed. This will evidently depend upon
the nature of the relation, and more particularly upon the question
whether, in that relation, the terms are understood to be
universal or particular, i.e. whether we speak of the whole of
that collection of objects to which a term refers, or indefinitely of
the whole or of a part of it, the usual signification of the prefix,
``some.''

Suppose that we wish to express a relation of identity between
the two classes, ``Fixed Stars'' and ``Suns,'' i.e. to
express that ``All fixed stars are suns,'' and ``All suns are fixed
stars.'' Here, if $x$ stand for fixed stars, and $y$ for suns, we shall
have
\[
x = y
\]
for the equation required.


In the proposition, ``All fixed stars are suns,'' the term ``all
fixed stars'' would be called the \textit{subject}, and `` suns'' the \textit{predicate}.
Suppose that we extend the meaning of the terms \textit{subject}
and \textit{predicate} in the following manner. By subject let us mean
the first term of any affirmative proposition, i. e. the term which
precedes the copula \textit{is} or \textit{are}; and by predicate let us agree to
mean the second term, i.e. the one which follows the copula;
and let us admit the assumption that either of these may be universal
or particular, so that, in either case, the whole class may
be implied, or only a part of it. Then we shall have the following
Rule for cases such as the one in the last example:--

10. \textsc{Rule.}---\textit{When both Subject and Predicate of a Proposition
are universal, form the separate expressions for them, and connect them
by the sign =.}

This case will usually present itself in the expression of the
definitions of science, or of subjects treated after the manner of
pure science. Mr. Senior's definition of wealth affords a good
example of this kind, viz.:

``Wealth consists of things transferable, limited in supply,
and either productive of pleasure or preventive of pain.''

Before proceeding to express this definition symbolically, it
must be remarked that the conjunction \textit{and} is superfluous.
Wealth is really defined by its possession of three properties or
qualities, not by its composition out of three classes or collections
of objects. Omitting then the conjunction \textit{and}, let us make
\begin{eqnarray*}
w &=& \textrm{wealth.} \\
t &=& \textrm{things transferable.} \\
s &=& \textrm{limited in supply.} \\
p &=& \textrm{productive of pleasure.} \\
r &=& \textrm{preventive of pain.}
\end{eqnarray*}

Now it is plain from the nature of the subject, that the expression,
``Either productive of pleasure or preventive of pain,''
in the above definition, is meant to be equivalent to ``Either productive
of pleasure; or, if not productive of pleasure, preventive
of pain.'' Thus the class of things which the above expression,
taken alone, would define, would consist of all things productive
of pleasure, together with all things not productive of pleasure,
but preventive of pain, and its symbolical expression would be

$$p+(1-p)r.$$

If then we attach to this expression placed in brackets to denote
that both its terms are referred to, the symbols $s$ and $t$ limiting
its application to things ``transferable'' and ``limited in supply,''
we obtain the following symbolical equivalent for the original
definition, viz.:

\begin{equation}
w = st\{p + r(1-p)\}.
\end{equation}

If the expression, ``Either productive of pleasure or preventive of
pain,'' were intended to point out merely those things which are
productive of pleasure without being preventive of pain, $p (1 - r)$,
or preventive of pain, without being productive of pleasure,
$r (1 - p)$ (exclusion being made of those things which are both
productive of pleasure and preventive of pain), the expression in
symbols of the definition would be

\begin{equation}
w = st\{p(1-r) + r(1-p)\}.
\end{equation}

All this agrees with what has before been more generally stated.
The reader may be curious to inquire what effect would be
produced if we literally translated the expression, ``Things productive
of pleasure or preventive of pain,'' by $p + r$, making the
symbolical equation of the definition to be

\begin{equation}
w = st(p + r).
\end{equation}

The answer is, that this expression would be equivalent to (2),
with the additional implication that the classes of things denoted
by $stp$ and $str$ are quite distinct, so that of things transferable
and limited in supply there exist none in the universe which are
at the same time both productive of pleasure and preventive of
pain. How the full import of any equation may be determined
will be explained hereafter. What has been said may show that before
attempting to translate our data into the rigorous language
of symbols, it is above all things necessary to ascertain the \textit{intended}
import of the words we are using. But this necessity
cannot be regarded as an evil by those who value correctness of
thought, and regard the right employment of language as both
its instrument and its safeguard.

11. Let us consider next the case in which the predicate of
the proposition is particular, e.g. ``All men are mortal.''

In this case it is clear that our meaning is, ``All men are
some mortal beings,'' and we must seek the expression of the
predicate, ``some mortal beings.'' Represent then by $v$, a class
indefinite in every respect but this, viz., that some of its members
are mortal beings, and let $x$ stand for ``mortal beings,'' then will
$vx$ represent ``some mortal beings.'' Hence if $y$ represent men,
the equation sought will be
\[
y = vx.
\]

From such considerations we derive the following Rule, for
expressing an affirmative universal proposition whose predicate
is particular:

\textsc{Rule}.---\textit{Express as before the subject and the predicate, attach
to the latter the indefinite symbol $v$, and equate the expressions.}

It is obvious that $v$ is a symbol of the same kind as $x$, $y$, \&c.,
and that it is subject to the general law,
\[
v^2 = v, or v (1 - v) = 0.
\]

Thus, to express the proposition, ``The planets are either
primary or secondary,'' we should, according to the rule, proceed
thus:

Let x represent planets (the subject);
\begin{eqnarray*}
y &=& \textrm{primary bodies;}\\
z &=& \textrm{secondary bodies;}
\end{eqnarray*}
then, assuming the conjunction ``or'' to separate absolutely the
class of ``primary'' from that of ``secondary'' bodies, so far as
they enter into our consideration in the proposition given, we
find for the equation of the proposition
\begin{equation}
x = v \left\{ y \left( 1 - z \right) + z \left( 1 - y \right) \right\}.
\end{equation}
It may be worth while to notice, that in this case the \textit{literal}
translation of the premises into the form
\begin{equation}
x = v (y + z)
\end{equation}

would be exactly equivalent, $v$ being an indefinite class symbol.
The form (4) is, however, the better, as the expression

\[
y \left( 1 - z \right) + z \left( 1 - y \right)
\]

consists of terms representing classes quite distinct from each
other, and satisfies the fundamental law of duality.

If we take the proposition, ``The heavenly bodies are either
suns, or planets, or comets,'' representing these classes of things
by $w$, $x$, $y$, $z$, respectively, its expression, on the supposition that
none of the heavenly bodies belong at once to two of the divisions
above mentioned, will be

\[
w = v \left\{ x \left( 1 - y \right) \left( 1 - z \right)%
  + y \left( 1 - x \right) \left( 1 - z \right)%
  + z \left( 1 - x \right) \left( 1 - y \right) \right\}
\]

If, however, it were meant to be implied that the heavenly
bodies were either suns, or, if not suns, planets, or, if neither, suns
nor planets, fixed stars, a meaning which does not exclude the
supposition of some of them belonging at once to two or to all
three of the divisions of suns, planets, and fixed stars,---the expression
required would be

\begin{equation}
w = v \left\{ x + y \left( 1 - x \right)%
  + z \left( 1 - x \right) \left(1 -y \right) \right\}.
\end{equation}

The above examples belong to the class of descriptions, not
definitions. Indeed the predicates of propositions are usually
particular. When this is not the case, either the predicate is a
singular term, or we employ, instead of the copula ``is'' or ``are,''
some form of connexion, which implies that the predicate is to be
taken universally.

12. Consider next the case of universal negative propositions,
e.g. ``No men are perfect beings.''

Now it is manifest that in this case we do not speak of a class
termed ``no men,'' and assert of this class that all its members
are ``perfect beings.'' But we virtually make an assertion about
``\textit{all men}'' to the effect that they are ``\textit{not perfect beings}.'' Thus
the true meaning of the proposition is this:

``All men (subject) are (copula) not perfect (predicate);''
whence, if $y$ represent ``men,'' and $x$ ``perfect beings,'' we shall
have

\[
y = v \left( 1 - x \right),
\]
and similarly in any other case. Thus we have the following
Rule:

\textsc{Rule}.---\textit{To express any proposition of the form ``No $x$'s are
$y$'s,'' convert it into the form ``All $x$'s are not $y$'s,'' and then proceed
as in the previous case.}

13. Consider, lastly, the case in which the subject of the
proposition is particular, e.g. ``Some men are not wise.'' Here,
as has been remarked, the negative \textit{not} may properly be referred,
certainly, at least, for the ends of Logic, to the predicate \textit{wise};
for we do not mean to say that it is not true that ``Some men
are wise,'' but we intend to predicate of ``some men'' a want of
wisdom. The requisite form of the given proposition is, therefore,
``Some men are not-wise.'' Putting, then, $y$ for ``men,''
$x$ for ``wise,'' i. e. ``wise beings,'' and introducing $v$ as the symbol
of a class indefinite in all respects but this, that it contains
some individuals of the class to whose expression it is prefixed,
we have
\[
v y = v \left( 1 - x \right).
\]

14. We may comprise all that we have determined in the
following general Rule:

\begin{center}
\textsc{general rule for the symbolical expression of primary propositions}.
\end{center}

1st. \textit{If the proposition is affirmative, form the expression of the
subject and that of the predicate. Should either of them be particular,
attach to it the indefinite symbol $v$, and then equate the resulting expressions.}

2ndly. \textit{If the proposition is negative, express first its true meaning
by attaching the negative particle to the predicate, then proceed as
above.}

One or two additional examples may suffice for illustration.

Ex.---``No men are placed in exalted stations, and free from
envious regards.''

Let $y$ represent ``men,'' $x$, ``placed in exalted stations,'' $z$,
``free from envious regards.''

Now the expression of the class described as ``placed in
exalted station,'' and ``free from envious regards,'' is $xz$. Hence
the contrary class, i.e. they to whom this description does not
apply, will be represented by $1 - xz$, and to this class all men
are referred. Hence we have
\[
y = v \left( 1 - xz \right).
\]
If the proposition thus expressed had been placed in the equiva-
lent form, ``Men in exalted stations are not free from envious
regards,'' its expression would have been
\[
yx = v \left( 1 - z \right).
\]
It will hereafter appear that this expression is really equivalent
to the previous one, on the particular hypothesis involved, viz.,
that $v$ is an indefinite class symbol.

Ex.---``No men are heroes but those who unite self-denial to
courage.''

Let $x$ = ``men,'' $y$ = ``heroes,'' $z$ = ``those who practise self-denial,''
$w$, ``those who possess courage.''

The assertion really is, that ``men who do not possess courage
and practise self-denial are not heroes.''

Hence we have
\[
x \left( 1 - zw \right) = v \left( 1 - y \right)
\]
for the equation required.

15. In closing this Chapter it may be interesting to compare
together the great leading types of propositions symbolically expressed.
If we agree to represent by $X$ and $Y$ the symbolical
expressions of the ``terms,'' or things related, those types will
be
\begin{eqnarray*}
X &=& vY, \\
X &=& Y, \\
vX &=& vY.
\end{eqnarray*}
In the first, the predicate only is particular; in the second, both
terms are universal; in the third, both are particular. Some minor
forms are really included under these. Thus, if $Y = 0$, the
second form becomes
\[
X=0;
\]
and if $Y = 1$ it becomes
\[
X = 1;
\]

both which forms admit of interpretation. It is further to be
noticed, that the expressions $X$ and $Y$, if founded upon a sufficiently
careful analysis of the meaning of the ``terms'' of the
proposition, will satisfy the fundamental law of duality which
requires that we have
\begin{eqnarray*}
X^2 = X &\textrm{or}& X \left( 1 - X \right) = 0,\\
Y^2 = Y &\textrm{or}& Y \left( 1 - Y \right) = 0.
\end{eqnarray*}

\chapter[PRINCIPLES OF SYMBOLIC REASONING]
{\large OF THE FUNDAMENTAL PRINCIPLES OF SYMBOLICAL REASONING, AND
OF THE EXPANSION OR DEVELOPMENT OF EXPRESSIONS INVOLVING
LOGICAL SYMBOLS.}

1. The previous chapters of this work have been devoted to
the investigation of the fundamental laws of the operations
of the mind in reasoning; of their development in the
laws of the symbols of Logic; and of the principles of expression,
by which that species of propositions called primary may be represented
in the language of symbols. These inquiries have been
in the strictest sense preliminary. They form an indispensable
introduction to one of the chief objects of this treatise---the construction
of a system or method of Logic upon the basis of an
exact summary of the fundamental laws of thought. There are
certain considerations touching the nature of this end, and the
means of its attainment, to which I deem it necessary here to
direct attention.

2. I would remark in the first place that the generality of a
method in Logic must very much depend upon the generality of
its elementary processes and laws. We have, for instance, in the
previous sections of this work investigated, among other things,
the laws of that logical process of \textit{addition} which is symbolized
by the sign $+$. Now those laws have been determined from the
study of instances, in all of which it has been a necessary condition,
that the classes or things added together in thought should
be mutually exclusive. The expression $x + y$ seems indeed uninterpretable,
unless it be assumed that the things represented
by $x$ and the things represented by $y$ are entirely separate;
that they embrace no individuals in common. And conditions
analogous to this have been involved in those acts of conception
from the study of which the laws of the other symbolical operations
have been ascertained. The question then arises, whether
it is necessary to restrict the application of these symbolical laws
and processes by the same conditions of interpretability under
which the knowledge of them was obtained. If such restriction
is necessary, it is manifest that no such thing as a general
method in Logic is possible. On the other hand, if such restriction is unnecessary, in what light are we to contemplate processes
which appear to be uninterpretable in that sphere of thought
which they are designed to aid? These questions do not belong
to the science of Logic alone. They are equally pertinent to every
developed form of human reasoning which is based upon the
employment of a symbolical language.

3. I would observe in the second place, that this apparent
failure of correspondency between process and interpretation does
not manifest itself in the \emph{ordinary} applications of human reason.
For no operations are there performed of which the meaning
and the application are not seen; and to most minds it does
not suffice that merely formal reasoning should connect their
premises and their conclusions; but every step of the connecting
train, every mediate result which is established in the course of
demonstration, must be intelligible also. And without doubt,
this is both an actual condition and an important safeguard, in
the reasonings and discourses of common life.

There are perhaps many who would be disposed to extend
the same principle to the general use of symbolical language as
an instrument of reasoning. It might be argued, that as the
laws or axioms which govern the use of symbols are established
upon an investigation of those cases only in which interpretation
is possible, we have no right to extend their application to other
cases in which interpretation is impossible or doubtful, even
though (as should be admitted) such application is employed in
the intermediate steps of demonstration only. Were this objection conclusive, it must be acknowledged that slight advantage
would accrue from the use of a symbolical method in
Logic. Perhaps that advantage would be confined to the mechanical gain of employing short and convenient symbols in the
place of more cumbrous ones. But the objection itself is fallacious. Whatever our \textit{\`{a} priori} anticipations might be, it is an
unquestionable fact that the validity of a conclusion arrived at
by any symbolical process of reasoning, does not depend upon
our ability to interpret the formal results which have presented
themselves in the different stages of the investigation. There
exist, in fact, certain general principles relating to the use of
symbolical methods, which, as pertaining to the particular subject
of Logic, I shall first state, and I shall then offer some remarks
upon the nature and upon the grounds of their claim to
acceptance.

4. The conditions of valid reasoning, by the aid of symbols,
are---

1st, That a fixed interpretation be assigned to the symbols
employed in the expression of the data; and that the laws of the
combination of those symbols be correctly determined from that
interpretation.

2nd, That the formal processes of solution or demonstration
be conducted throughout in obedience to all the laws determined
as above, without regard to the question of the interpretability
of the particular results obtained.

3rd, That the final result be interpretable in form, and that
it be actually interpreted in accordance with that system of interpretation
which has been employed in the expression of the
data. Concerning these principles, the following observations
may be made.

5. The necessity of a fixed interpretation of the symbols has
already been sufficiently dwelt upon (II. 3). The necessity that
the fixed result should be in such a form as to admit of that interpretation
being applied, is founded on the obvious principle,
that the use of symbols is a means towards an end, that end
being the knowledge of some intelligible fact or truth. And
that this end may be attained, the final result which expresses
the symbolical conclusion must be in an interpretable form. It
is, however, in connexion with the second of the above general
principles or conditions (V. 4), that the greatest difficulty is
likely to be felt, and upon this point a few additional words are
necessary.

I would then remark, that the principle in question may be
considered as resting upon a general law of the mind, the knowledge
of which is not given to us \textit{\`{a} priori}, i.e. antecedently to
experience, but is derived, like the knowledge of the other laws
of the mind, from the clear manifestation of the general principle
in the particular instance. A single example of reasoning, in
which symbols are employed in obedience to laws founded upon
their interpretation, but without any sustained reference to that
interpretation, the chain of demonstration conducting us through
intermediate steps which are not interpretable, to a final result
which is interpretable, seems not only to establish the validity of
the particular application, but to make known to us the general
law manifested therein. No accumulation of instances can properly
add weight to such evidence. It may furnish us with clearer
conceptions of that common element of truth upon which the application
of the principle depends, and so prepare the way for its
reception. It may, where the immediate force of the evidence is
not felt, serve as a verification, \textit{\`{a} posteriori}, of the practical validity
of the principle in question. But this does not affect the position
affirmed, viz., that the general principle must be seen in the
particular instance,---seen to be general in application as well as
true in the special example. The employment of the uninterpretable
symbol $\sqrt{-1}$, in the intermediate processes of trigonometry,
furnishes an illustration of what has been said. I apprehend that
there is no mode of explaining that application which does not
covertly assume the very principle in question. But that principle,
though not, as I conceive, warranted by formal reasoning
based upon other grounds, seems to deserve a place among those
axiomatic truths which constitute, in some sense, the foundation
of the possibility of general knowledge, and which may properly
be regarded as expressions of the mind's own laws and constitution.

6. The following is the mode in which the principle above
stated will be applied in the present work. It has been seen,
that any system of propositions may be expressed by equations
involving symbols $x$, $y$, $z$, which, whenever interpretation is possible,
are subject to laws identical in form with the laws of a system
of quantitative symbols, susceptible only of the values $0$ and
$1$ (II. 15). But as the formal processes of reasoning depend only
upon the laws of the symbols, and not upon the nature of their
interpretation, we are permitted to treat the above symbols,
$x$, $y$, $z$, as if they were quantitative symbols of the kind above
described. \textit{We may in fact lay aside the logical interpretation of
the symbols in the given equation; convert them into quantitative symbols,
susceptible only of the values $0$ and $1$; perform upon them as such
all the requisite processes of solution; and finally restore to them their
logical interpretation.} And this is the mode of procedure which
will actually be adopted, though it will be deemed unnecessary
to restate in every instance the nature of the transformation employed.
The processes to which the symbols $x$, $y$, $z$, regarded
as quantitative and of the species above described, are subject, are
not limited by those conditions of thought to which they would,
if performed upon purely logical symbols, be subject, and a freedom of operation is given to us in the use of them, without
which, the inquiry after a general method in Logic would be a
hopeless quest.

Now the above system of processes would conduct us to no
intelligible result, unless the final equations resulting therefrom
were in a form which should render their interpretation, after
restoring to the symbols their logical significance, possible.
There exists, however, a general method of reducing equations
to such a form, and the remainder of this chapter will be devoted
to its consideration. I shall say little concerning the way in
which the method renders interpretation possible,---this point
being reserved for the next chapter,---but shall chiefly confine
myself here to the mere process employed, which may be characterized as a process of ``development.'' As introductory to
the nature of this process, it may be proper first to make a few
observations.

7. Suppose that we are considering any class of things with
reference to this question, viz., the relation in which its members
stand as to the possession or the want of a certain property
$x$. As
every individual in the proposed class either possesses or does
not possess the property in question, we may divide the class
into two portions, the former consisting of those individuals
which possess, the latter of those which do not possess, the property.
This possibility of dividing in thought the whole class
into two constituent portions, is antecedent to all knowledge of
the constitution of the class derived from any other source; of
which knowledge the effect can only be to inform us, more or
less precisely, to what further conditions the portions of the class
which possess and which do not possess the given property are
subject. Suppose, then, such knowledge is to the following effect,
viz., that the members of that portion which possess the property
$x$, possess also a certain property $u$, and that these conditions
united are a sufficient definition of them. We may then represent
that portion of the original class by the expression $ux$ (II. 6).
If, further, we obtain information that the members of the original
class which do not possess the property $x$, are subject to a
condition $v$, and are thus defined, it is clear, that those members
will be represented by the expression $v \left( 1 -x \right)$. Hence the class
in its totality will be represented by
\[
ux + v \left( 1 - x \right);
\]
which may be considered as a general developed form for the
expression of any class of objects considered with reference to
the possession or the want of a given property $x$.

The general form thus established upon purely logical
grounds may also be deduced from distinct considerations of
formal law, applicable to the symbols $x$, $y$, $z$, equally in their
logical and in their quantitative interpretation already referred to
(V. 6).

8. \textit{Definition.}---Any algebraic expression involving a symbol
$x$ is termed a function of $x$, and may be represented under
the abbreviated general form $f\left(x\right)$. Any expression involving
two symbols, $x$ and $y$, is similarly termed a function of $x$ and $y$,
and may be represented under the general form $f\left(x, y\right)$, and so
on for any other case.

Thus the form $f\left(x\right)$ would indifferently represent any of the
following functions, viz., $x$, $1-x$, $\frac{1+x}{1-x}$, \&c.; and $f\left(x,y\right)$ would
equally represent any of the forms $x + y$, $x - 2y$, $\frac{x + y}{x - 2y}$, \&c.

On the same principles of notation, if in any function $f\left(x\right)$
we change $x$ into $1$, the result will be expressed by the form
$f\left(1\right)$; if in the same function we change $x$ into $0$, the result will
be expressed by the form $f\left(0\right)$. Thus, if $f\left(x\right)$ represent the
function $\frac{a + x}{a - 2x}$, $f\left(1\right)$ will represent $\frac{a + 1}{a - 2}$, and $f\left(0\right)$ will represent
$\frac{a}{a}$.

9. \textit{Definition.}---Any function $f\left(x\right)$, in which $x$ is a logical
symbol, or a symbol of quantity susceptible only of the values
$0$ and $1$, is said to be developed, when it is reduced to the form
$ax + b\left(1-x\right)$, $a$ and $b$ being so determined as to make the result
equivalent to the function from which it was derived.

This definition assumes, that it is possible to represent any
function $f\left(x\right)$ in the form supposed. The assumption is vindicated
in the following Proposition.

\begin{center}
\textsc{Proposition I.}
\end{center}

10. \textit{To develop any function $f\left(x\right)$ in which $x$ is a logical symbol.}

By the principle which has been asserted in this chapter, it
is lawful to treat $x$ as a quantitative symbol, susceptible only of
the values $0$ and $1$.

Assume then,
\[
f\left(x\right) = ax + b \left(1 - x\right),
\]
and making $x = 1$, we have
\[
f\left(1\right) = a.
\]
Again, in the same equation making $x = 0$, we have
\[
f\left(0\right) = b.
\]
Hence the values of a and b are determined, and substituting
them in the first equation, we have
\begin{equation}
f\left(x\right) = f\left(1\right) x + f\left(0\right)\left(1-x\right);
\end{equation}
as the development sought.\footnote{
To some it may be interesting to remark, that the development of $f\left(x\right)$
obtained in this chapter, strictly holds, in the logical system, the place of the
expansion of $f\left(x\right)$ in ascending powers of $x$ in the system of ordinary algebra.
Thus it may be obtained by introducing into the expression of Taylor's well-known
theorem, viz.: %*** Need to reset equation numbering here for footnote only ***%
\setcounter{equation}{0}
\begin{equation}
f\left(x\right) = f\left(0\right)%
                + f^{\prime}\left(0\right)x%
                + f^{\prime\prime}\left(0\right)\frac{x^2}{1\cdot{}2}%
                + f^{\prime\prime\prime}\left(0\right)\frac{x^3}{1\cdot2\cdot3}, \textrm{ \&c.}
\end{equation}
the condition $x (1 - x) = 0$, whence we find $x^2 = x$, $x^3 = x$, \&c., and
\begin{equation}
f\left(x\right) = f\left(0\right) + \left\{f' \left(0\right) + \frac{f'' \left(0\right)}{1 \cdot 2} +
\frac{f''' \left(0\right)}{1 \cdot 2 \cdot 3} + \textrm{ \&c.}\right\} x.
\end{equation}

But making in (1), $x=1$, we get
\[
f(1) = f(0) + f'(0) + \frac{f''(0)}{1 \cdot 2} +
\frac{f'''(0)}{1 \cdot 2 \cdot 3} + \textrm{ \&c.}
\]
whence
\[
f'(0) + \frac{f''(0)}{1 \cdot 2} + \textrm{ \&c.} = f(1) - f(0),
\]
and (2) becomes, on substitution,
\begin{eqnarray*}
f(x) = f(0) + \{f(1)-f(0)\} x,\\
= f(1)x + f(0)(1-x),
\end{eqnarray*}
the form in question. This demonstration in supposing $f\left(x\right)$ to be developable in
a series of ascending powers of $x$ is less general than the one in the text.}
The second member of the equation
adequately represents the function $f\left(x\right)$, whatever the form
of that function may be. For $x$ regarded as a quantitative symbol admits
only of the values $0$ and $1$, and for each of these
values the development
\[
f\left(1\right)x + f\left(0\right) \left(1-x\right),
\]
assumes the same value as the function $f\left(x\right)$.

As an illustration, let it be required to develop the function $\frac{1+x}{1+2x}$. Here, when $x=1$, we find $f\left(1\right) = \frac{2}{3}$, and when $x = 0$,
we find $f\left(0\right) = \frac{1}{1}$, or $1$. Hence the expression required is
\[
\frac{1+x}{1+2x} = \frac{2}{3}x + 1 - x ~;
\]
and this equation is satisfied for each of the values of which the
symbol $x$ is susceptible.

\begin{center}
\textsc{Proposition II.}
\end{center}

\textit{To expand or develop a function involving any number of logical
symbols.}

Let us begin with the case in which there are two symbols,
$x$ and $y$, and let us represent the function to be developed by
$f\left(x, y\right)$.

First, considering $f\left(x, y\right)$ as a function of $x$ alone, and expanding
it by the general theorem (1), we have
\setcounter{equation}{1}
\begin{equation}
f\left(x,y\right) = f\left(1,y\right)x + f\left(0,y\right)\left(1-x\right);
\end{equation}

wherein $f\left(1, y\right)$ represents what the proposed function becomes,
when in it for $x$; we write $1$, and $f\left(0,y\right)$ what the said function
becomes, when in it for $x$ we write $0$.

Now, taking the coefficient $f\left(1, y\right)$, and regarding it as a function
of $y$, and expanding it accordingly, we have
\begin{equation}
f\left(1, y\right) = f\left(1, 1\right)y + f\left(1, 0\right)\left(1 - y\right),
\end{equation}
wherein $f\left(1, 1\right)$ represents what $f\left(1, y\right)$ becomes when $y$ is made
equal to $1$, and $f\left(1, 0\right)$ what $f\left(1, y\right)$ becomes when $y$ is made
equal to $0$.

In like manner, the coefficient $f\left(0, y\right)$ gives by expansion,
\begin{equation}
f\left(0, y\right) = f\left(0, 1\right)y + f\left(0, 0\right)\left(1 - y\right).
\end{equation}
Substitute in (2) for $f\left(1, y\right)$, $f\left(0, y\right)$, their values given in (3)
and (4), and we have
\begin{eqnarray}
f\left(x, y\right) = f\left(1, 1\right)xy
                   + f\left(1, 0\right)x\left(1 - y \right)
                   + f\left(0, 1\right)\left(1 - x \right)y\\
                   + f\left(0, 0\right)\left(1 - x \right)\left(1 - y \right),
\end{eqnarray}
for the expansion required. Here $f\left(1, 1\right)$ represents what $f\left(x, y\right)$
becomes when we make therein $x=1$, $y=1$; $f\left(1, 0\right)$ represents
what $f\left(x, y\right)$ becomes when we make therein $x=1$, $y=0$, and
so on for the rest.

Thus, if $f\left(x, y\right)$ represent the function $\frac{1-x}{1-y}$, we find
\[
f\left(1, 1\right) = \frac{0}{0},\qquad%
f\left(1, 0\right) = \frac{0}{1},\qquad%
f\left(0, 1\right) = \frac{1}{0},\qquad%
f\left(0, 0\right) = 1
\]
whence the expansion of the given function is
\[
\frac{0}{0}xy + 0x\left(1-y\right) + \frac{1}{0}\left(1-x\right)y + \left(1-x\right)\left(1-y\right).
\]
It will in the next chapter be seen that the forms $\frac{0}{0}$ and $\frac{1}{0}$, the
former of which is known to mathematicians as the symbol of indeterminate
quantity, admit, in such expressions as the above, of
a very important logical interpretation.

Suppose, in the next place, that we have three symbols in
the function to be expanded, which we may represent under the
general form $f\left(x, y, z\right)$. Proceeding as before, we get

\begin{eqnarray*}
f\left(x,y,z\right) &=& f\left(1,1,1\right)xyz + f\left(1,1,0\right)xy\left(1-z\right) + f\left(1, 0, 1\right)x\left(1-y\right)z\\
         &+& f\left(1,0,0\right)x\left(1-y\right)\left(1-z\right) + f\left(0,1,1\right)\left(1-x\right)yz\\
         &+& f\left(0,1,0\right)\left(1-x\right)y\left(1-z\right) + f\left(0,0,1\right)\left(1-x\right)\left(1-y\right)z\\
         &+& f\left(0,0,0\right)\left(1-x\right)\left(1-y\right)\left(1-z\right)
\end{eqnarray*}
in which $f\left(1, 1, 1\right)$ represents what the function $f\left(x, y, z\right)$ becomes
when we make therein $x = 1$, $y = 1$, $z = 1$, and so on for
the rest.

11. It is now easy to see the general law which determines
the expansion of any proposed function, and to reduce the method
of effecting the expansion to a rule. But before proceeding
to the expression of such a rule, it will be convenient to premise
the following observations:---

Each form of expansion that we have obtained consists of certain
terms, into which the symbols $x$, $y$, \&c. enter, multiplied by
coefficients, into which those symbols do not enter. Thus the
expansion of $f\left(x\right)$ consists of two terms, $x$ and $1 - x$, multiplied
by the coefficients $f\left(1\right)$ and $f\left(0\right)$ respectively. And the expansion
of $f\left(x, y\right)$ consists of the four terms $xy$, $x\left(1 - y\right)$, $\left(1 - x\right)y$,
and $\left(1 - x\right)$, $\left(1 - y\right)$, multiplied by the coefficients $f\left(1, 1\right)$, $f\left(1, 0\right)$,
$f\left(0, 1\right)$, $f\left(0, 0\right)$, respectively. The terms $x$, $1 - x$, in the former
case, and the terms $xy$, $x\left(1 - y\right)$, \&c., in the latter, we shall call
the \textit{constituents} of the expansion. It is evident that they are in
form independent of the form of the function to be expanded.
Of the constituent $xy$, $x$ and $y$ are termed the \textit{factors}.

The general rule of development will therefore consist of two
parts, the first of which will relate to the formation of the \textit{constituents}
of the expansion, the second to the determination of their
respective coefficients. It is as follows:

1st. \textit{To expand any function of the symbols $x$, $y$, $z$}.---Form a
series of constituents in the following manner: Let the first constituent
be the product of the symbols; change in this product
any symbol $z$ into $1 - z$, for the second constituent. Then in
both these change any other symbol $y$ into $1 - y$, for two more
constituents. Then in the four constituents thus obtained change
any other symbol $x$ into $1 - x$, for four new constituents, and so
on until the number of possible changes is exhausted.

2ndly. \textit{To find the coefficient of any constituent}.---If that constituent
involves $x$ as a factor, change in the original function $x$
into $1$; but if it involves $1 - x$ as a factor, change in the original
function $x$ into $0$. Apply the same rule with reference to the
symbols $y$, $z$, \&c.: the final calculated value of the function thus
transformed will be the coefficient sought.

The sum of the constituents, multiplied each by its respective
coefficient, will be the expansion required.

12. It is worthy of observation, that a function may be developed
with reference to symbols which it does not explicitly
contain. Thus if, proceeding according to the rule, we seek to
develop the function $1-x$, with reference to the symbols $x$ and
$y$, we have,

\begin{tabular}{c l c l c l}
\textrm{When } &x = 1 &\textrm{ and } &y = 1  &\textrm{ the given function }  &= 0.\\
             &x = 1  &"  &y = 0   &"  "  &= 0.\\
             &x = 0  &"  &y = 1   &"  "  &= 1.\\
             &x = 0  &"  &y = 0   &"  "  &= 1.
\end{tabular}\\
Whence the development is
\[
1-x = 0xy + 0x\left(1-y\right)+\left(1-x\right)y+\left(1-x\right)\left(1-y\right);
\]
and this is a true development. The addition of the terms $\left(1-x\right)y$
and $\left(1-x\right)\left(1-y\right)$ produces the function $1-x$.

The symbol $1$ thus developed according to the rule, with respect
to the symbol $x$, gives
\[
x + 1 - x.
\]
Developed with respect to $x$ and $y$, it gives
\[
xy + x\left(1-y\right) + \left(1-x\right)y + \left(1-x\right) \left(1-y\right).
\]
Similarly developed with respect to any set of symbols, it produces
a series consisting of all possible constituents of those
symbols.

13. A few additional remarks concerning the nature of the
general expansions may with propriety be added. Let us take,
for illustration, the general theorem (5), which presents the type
of development for functions of two logical symbols.

In the first place, that theorem is perfectly true and intelligible
when $x$ and $y$ are quantitative symbols of the species considered
in this chapter, whatever algebraic form may be assigned
to the function $f\left(x, y\right)$, and it may therefore be intelligibly employed
in any stage of the process of analysis intermediate between
the change of interpretation of the symbols from the
logical to the quantitative system above referred to, and the final
restoration of the logical interpretation.

Secondly. The theorem is perfectly true and intelligible when
$x$ and $y$ are logical symbols, provided that the form of the function
$f\left(x, y\right)$ is such as to represent a \textit{class or collection of things,}
in which case the second member is always logically interpretable.
For instance, if $f\left(x, y\right)$ represent the function $1 - x + xy$, we obtain
on applying the theorem
\begin{eqnarray*}
1-x+xy &=& xy+0x\left(1-y\right)+\left(1-x\right)y+\left(1-x\right)\left(1-y\right),\\
       &=& xy+\left(1-x\right)y+\left(1-x\right)\left(1-y\right),
\end{eqnarray*}
and this result is intelligible and true.

Thus we may regard the theorem as true and intelligible for
quantitative symbols of the species above described, \textit{always}; for
logical symbols, \textit{always when interpretable}. Whensoever therefore
it is employed in this work it must be understood that the
symbols $x$, $y$ are quantitative and of the particular species referred
to, if the expansion obtained is not interpretable.

But though the expansion is not always immediately interpretable,
it always conducts us at once to results which are interpretable.
Thus the expression $x-y$ gives on development
the form
\[
x\left(1-y\right)-y\left(1-x\right),
\]
which is not generally interpretable. We cannot take, in thought,
from the class of things which are $x$'s and not $y$'s, the class of
things which are $y$'s and not $x$'s, because the latter class is not
contained in the former. But if the form $x - y$ presented itself
as the first member of an equation, of which the second member
was $0$, we should have on development
\[
x\left(1-y\right)-y\left(1-x\right)=0.
\]
Now it will be shown in the next chapter that the above equation,
$x$ and $y$ being regarded as quantitative and of the species
described, is resolvable at once into the two equations
\[
x\left(1-y\right)=0, \quad y\left(1-x\right)=0,
\]
and these equations are directly interpretable in Logic when logical
interpretations are assigned to the symbols $x$ and $y$. And
it may be remarked, that though \textit{functions} do not necessarily become
interpretable upon development, yet \textit{equations} are always
reducible by this process to interpretable forms.

14. The following Proposition establishes some important
properties of constituents. In its enunciation the symbol $t$ is
employed to represent indifferently any constituent of an expansion.
Thus if the expansion is that of a function of two symbols
$x$ and $y$, $t$ represents any of the four forms $xy$, $x\left(1 - y\right)$, $\left(1 - x\right)y$,
and $\left(1 - x\right)\left(1 - y\right)$. Where it is necessary to represent the constituents
of an expansion by single symbols, and yet to distinguish
them from each other, the distinction will be marked by suffixes.
Thus $t_1$ might be employed to represent $xy$, $t_2$ to represent $x \left(1 - y\right)$,
and so on.

\begin{center}
\textsc{Proposition III.}
\end{center}

\textit{Any single constituent $t$ of an expansion satisfies the law of duality
whose expression is
\[
t\left(1-t\right) = 0.
\]
The product of any two distinct constituents of an expansion is equal
to $0$, and the sum of all the constituents is equal to $1$.}

1st. Consider the particular constituent $xy$. We have
\[
xy \times xy = x^2y^2.
\]
But $x^2 = x$, $y^2 = y$, by the fundamental law of class symbols;
hence
\[
xy \times xy = xy.
\]
Or representing $xy$ by $t$,
\[
t \times t = t,
\]
or
\[
t\left(1 -t\right) = 0.
\]
Similarly the constituent $x \left(1 - y\right)$ satisfies the same law. For we
have
\begin{eqnarray*}
x^2 = x, \left(1 - y\right)^2 = 1 - y, \\
\therefore \left\{x\left(1-y\right)\right\}^2 = x\left(1-y\right), \textrm{ or } t\left(1-t\right) = 0.
\end{eqnarray*}
Now every factor of every constituent is either of the form $x$ or
of the form $1 - x$. Hence the square of each factor is equal to that
factor, and therefore the square of the product of the factors, i.e.
of the constituent, is equal to the constituent; wherefore $t$ representing
any constituent, we have
\[
t^2 = t, \textrm{ or } t\left(1 -t\right) = 0.
\]

2ndly. The product of any two constituents is $0$. This is
evident from the general law of the symbols expressed by the
equation $x \left(1 - x\right) = 0$; for whatever constituents in the same expansion
we take, there will be at least one factor $x$ in the one, to
which will correspond a factor $1 - x$ in the other.

3rdly. The sum of all the constituents of an expansion is
unity. This is evident from addition of the two constituents $x$
and $1 - x$, or of the four constituents, $xy$, $x(1-y)$, $(1-x)y$,
$(1-x)(1-y)$. But it is also, and more generally, proved by
expanding $1$ in terms of any set of symbols (V. 12). The constituents
in this case are formed as usual, and all the coefficients
are unity.

15. With the above Proposition we may connect the following.

\begin{center}
\textsc{Proposition IV.}
\end{center}

\textit{If $V$ represent the sum of any series of constituents, the separate
coefficients of which are $1$, then is the condition satisfied,
\[
V\left(1-V\right) = 0
\]}

Let $t_1$, $t_2$ \dots $t_n$ be the constituents in question, then
\[
V=t_1 + t_2 \dots + t_n.
\]
Squaring both sides, and observing that $t_1^2 = t_1$, $t_1 t_2, = 0$, \&c., we
have
\[
V^2 = t_1 + t_2 \dots + t_n;
\]
whence
\[
V = V^2.
\]
Therefore
\[
V\left(1-V\right) = 0.
\]

\chapter[OF INTERPRETATION]
{\large OF THE GENERAL INTERPRETATION OF LOGICAL EQUATIONS, AND
THE RESULTING ANALYSIS OF PROPOSITIONS. ALSO, OF THE
CONDITION OF INTERPRETABILITY OF LOGICAL FUNCTIONS.}

1. It has been observed that the complete expansion of any
function by the general rule demonstrated in the last
chapter, involves two distinct sets of elements, viz., the constituents
of the expansion, and their coefficients. I propose in
the present chapter to inquire, first, into the interpretation of
constituents, and afterwards into the mode in which that interpretation
is modified by the coefficients with which they are
connected.

The terms ``logical equation,'' ``logical function,'' \&c., will
be employed generally to denote any equation or function involving
the symbols $x$, $y$, \&c., which may present itself either
in the expression of a system of premises, or in the train of symbolical
results which intervenes between the premises and the
conclusion. If that function or equation is in a form not immediately
interpretable in Logic, the symbols $x$, $y$, \&c., must be regarded
as quantitative symbols of the species described in previous
chapters (II. 15), (V. 6), as satisfying the law,
\[
x\left(1-x\right) = 0.
\]

By the problem, then, of the interpretation of any such logical
function or equation, is meant the reduction of it to a form in
which, when logical values are assigned to the symbols $x$, $y$, \&c.,
it shall become interpretable, together with the resulting interpretation.
These conventional definitions are in accordance with
the general principles for the conducting of the method of this
treatise, laid down in the previous chapter.


\begin{center}
\textsc{Proposition I.}
\end{center}

\textit{2. The constituents of the expansion of any function of the logical
symbols $x$, $y$, \&c., are interpretable, and represent the several
exclusive divisions of the universe of discourse, formed by the predication
and denial in every possible way of the qualities denoted by the
symbols $x$, $y$, \&c.}

For greater distinctness of conception, let it be supposed that
the function expanded involves two symbols $x$ and $y$, with reference
to which the expansion has been effected. We have then
the following constituents, viz.:
\[
xy,\; x\left(1-y\right),\; \left(1-x\right)y,\; \left(1-x\right)\; \left(1-y\right).
\]
Of these it is evident, that the first $xy$ represents that class
of objects which at the same time possess both the elementary
qualities expressed by $x$ and $y$, and that the second $x\left(1-y\right)$ represents
the class possessing the property $x$, but not the property
$y$. In like manner the third constituent represents the class of
objects which possess the property represented by $y$, but not
that represented by $x$; and the fourth constituent $\left(1-x\right)\left(1-y\right)$,
represents that class of objects, the members of which possess neither
of the qualities in question.

Thus the constituents in the case just considered represent
all the four classes of objects which can be described by affirmation
and denial of the properties expressed by $x$ and $y$. Those
classes are distinct from each other. No member of one is a member
of another, for each class possesses some property or quality
contrary to a property or quality possessed by any other class.
Again, these classes together make up the universe, for there is
no object which may not be described by the presence or the
absence of a proposed quality, and thus each individual thing in
the universe may be referred to some one or other of the four
classes made by the possible combination of the two given
classes $x$ and $y$, and their contraries.

The remarks which have here been made with reference to the
constituents of $f\left(x,y\right)$ are perfectly general in character. The
constituents of any expansion represent classes---those classes
are mutually distinct, through the possession of contrary qualities,
and they together make up the universe of discourse.

3. These properties of constituents have their expression in
the theorems demonstrated in the conclusion of the last chapter,
and might thence have been deduced. From the fact that every
constituent satisfies the fundamental law of the individual symbols,
it might have been conjectured that each constituent would
represent a class. From the fact that the product of any two
constituents of an expansion vanishes, it might have been concluded
that the classes they represent are mutually exclusive.
Lastly, from the fact that the sum of the constituents of an expansion
is unity, it might have been inferred, that the classes
which they represent, together make up the universe.

4. Upon the laws of constituents and the mode of their interpretation
above determined, are founded the analysis and the
interpretation of logical equations. That all such equations admit
of interpretation by the theorem of development has already
been stated. I propose here to investigate the forms of possible
solution which thus present themselves in the conclusion of a
train of reasoning, and to show how those forms arise. Although,
properly speaking, they are but manifestations of a single fundamental
type or principle of expression, it will conduce to clearness
of apprehension if the minor varieties which they exhibit are
presented separately to the mind.

The forms, which are three in number, are as follows:

\begin{center}
\textsc{form i}.
\end{center}

5. The form we shall first consider arises when any logical
equation $V=0$ is developed, and the result, after resolution into
its component equations, is to be interpreted. The function is supposed
to involve the logical symbols $x$, $y$, \&c., in combinations which
are not fractional. Fractional combinations indeed only arise in
the class of problems which will be considered when we come to
speak of the third of the forms of solution above referred to.

\begin{center}
\textsc{Proposition II.}
\end{center}

\textit{To interpret the logical equation $V = 0$.}

For simplicity let us suppose that $V$ involves but two symbols,
$x$ and $y$, and let us represent the development of the given
equation by
\begin{equation}
axy + bx\left(1-y\right) + c\left(1-x\right)y + d\left(1-x\right)\left(1-y\right)=0;
\end{equation}
$a$, $b$, $c$, and $d$ being definite numerical constants.

Now, suppose that any coefficient, as $a$, does not vanish.
Then multiplying each side of the equation by the constituent $xy$,
to which that coefficient is attached, we have
\[
axy = 0,
\]
whence, as $a$ does not vanish,
\[
xy = 0,
\]
and this result is quite independent of the nature of the other coefficients
of the expansion. Its interpretation, on assigning to
$x$ and $y$ their logical significance, is ``No individuals belonging at
once to the class represented by $x$, and the class represented by $y$,
exist.''

But if the coefficient $a$ \textit{does} vanish, the term $axy$ does not
appear in the development (1), and, therefore, the equation $xy = 0$
cannot thence be deduced.

In like manner, if the coefficient $b$ does not vanish, we have
\[
x\left(1-y\right) = 0,
\]
which admits of the interpretation, ``There are no individuals
which at the same time belong to the class $x$, and do not belong
to the class $y$.''

Either of the above interpretations may, however, as will subsequently
be shown, be exhibited in a different form.

The sum of the distinct interpretations thus obtained from
the several terms of the expansion whose coefficients do not
vanish, will constitute the complete interpretation of the equation
$V = 0$. The analysis is essentially independent of the number
of logical symbols involved in the function $V$, and the object of
the proposition will, therefore, in all instances, be attained by the
following Rule: --

\textsc{Rule.}---\textit{Develop the function $V$, and equate to $0$ every constituent
whose coefficient does not vanish. The interpretation of these
results collectively will constitute the interpretation of the given
equation.}

6. Let us take as an example the definition of ``clean beasts,''
laid down in the Jewish law, viz., ``Clean beasts are those
which both divide the hoof and chew the cud,'' and let us assume\\
\begin{tabular}{c l}
&$x$ = clean beasts;\\
&$y$ = beasts dividing the hoof;\\
&$z$ = beasts chewing the cud;
\end{tabular}
Then the given proposition will be represented by the equation
\[
x = yz
\]
which we shall reduce to the form
\[
x - yz = 0,
\]
and seek that form of interpretation to which the present method
leads. Fully developing the first member, we have
\begin{eqnarray*}
0xyz + xy\left(1-z\right) + x\left(1-y\right)z + x\left(1-y\right)\left(1-z\right)\\
- \left(1-x\right)yz + 0\left(1-x\right)y\left(1-z\right) + 0\left(1-x\right)\left(1-y\right)z + 0\left(1-x\right)\left(1-y\right)\left(1-z\right).
\end{eqnarray*}
Whence the terms, whose coefficients do not vanish, give
\[
xy\left(1-z\right) = 0,\; xz\left(1-y\right) = 0,\; x\left(1-y\right)\left(1-z\right) = 0,\; \left(1-x\right)yz = 0.
\]
These equations express a denial of the existence of certain classes
of objects, viz.:

1st. Of beasts which are clean, and divide the hoof, but do
not chew the cud.

2nd. Of beasts which are clean, and chew the cud, but do not
divide the hoof.

3rd. Of beasts which are clean, and neither divide the hoof
nor chew the cud.

4th. Of beasts which divide the hoof, and chew the cud, and
are not clean.

Now all these several denials are really involved in the original
proposition. And conversely, if these denials be granted,
the original proposition will follow as a necessary consequence.
They are, in fact, the separate elements of that proposition.
Every primary proposition can thus be resolved into a series of
denials of the existence of certain defined classes of things, and
may, from that system of denials, be itself reconstructed. It
might here be asked, how it is possible to make an assertive proposition
out of a series of denials or negations? From what
source is the positive element derived? I answer, that the mind
assumes the existence of a universe not  \textit{\`{a} priori} as a fact independent
of experience, but either \textit{\`{a} posteriori} as a deduction
from experience, or \textit{hypothetically} as a foundation of the possibility
of assertive reasoning. Thus from the Proposition, ``There
are no men who are not fallible,'' which is a negation or denial of
the existence of ``infallible men,'' it may be inferred either hypothetically,
``All men (if men exist) are fallible,'' or absolutely,
(experience having assured us of the existence of the race), ``All
men are fallible.''

The form in which conclusions are exhibited by the method
of this Proposition may be termed the form of ``Single or Conjoint
Denial.''

\begin{center}
\textsc{form ii}.
\end{center}

7. As the previous form was derived from the development
and interpretation of an equation whose second member is $0$, the
present form, which is supplementary to it, will be derived from
the development and interpretation of an equation whose second
member is $1$. It is, however, readily suggested by the analysis
of the previous Proposition.

Thus in the example last discussed we deduced from the
equation
\[
x - yz = 0
\]
the conjoint denial of the existence of the classes represented by
the constituents
\[
xy\left(1-z\right),\quad  xz\left(1-y\right), \quad x\left(1-y\right)\left(1-z\right), \quad \left(1-x\right)yz,
\]
whose coefficients were not equal to 0. It follows hence that
the remaining constituents represent classes which make up the
universe. Hence we shall have
\[
xyz+\left(1-x\right)y\left(1-z\right)+\left(1-x\right)\left(1-y\right)z+\left(1-x\right)\left(1-y\right)\left(1-z\right)=1.
\]
This is equivalent to the affirmation that all existing things belong
to some one or other of the following classes, viz.:

1st. Clean beasts both dividing the hoof and chewing the
cud.

2nd. Unclean beasts dividing the hoof, but not chewing the
cud.

3rd. Unclean beasts chewing the cud, but not dividing the
hoof.

4th. Things which are neither clean beasts, nor chewers of
the cud, nor dividers of the hoof.

This form of conclusion may be termed the form of ``Single
or Disjunctive Affirmation,''---single when but one constituent
appears in the final equation; disjunctive when, as above, more
constituents than one are there found.

Any equation, $V=0$, wherein $V$ satisfies the law of duality,
may also be made to yield this form of interpretation by reducing
it to the form $1-V=1$, and developing the first member. The
case, however, is really included in the next general form. Both
the previous forms are of slight importance compared with the
following one.

\begin{center}
\textsc{form iii}.
\end{center}

8. In the two preceding cases the functions to be developed
were equated to $0$ and to $1$ respectively. In the present case I
shall suppose the corresponding function equated to any logical
symbol $w$. We are then to endeavour to interpret the equation
$V = w$, $V$ being a function of the logical symbols $x$, $y$, $z$, \&c. In
the first place, however, I deem it necessary to show how the
equation $V=w$, or, as it will usually present itself, $w=V$, arises.

Let us resume the definition of ``clean beasts,'' employed in
the previous examples, viz., ``Clean beasts are those which both
divide the hoof and chew the cud,'' and suppose it required to determine
the relation in which ``beasts chewing the cud'' stand to
``clean beasts'' and ``beasts dividing the hoof.'' The equation
expressing the given proposition is
\[
x=yz,
\]
and our object will be accomplished if we can determine $z$ as an
interpretable function of $x$ and $y$.

Now treating $x$, $y$, $z$ as symbols of quantity subject to a peculiar
law, we may deduce from the above equation, by solution,
\[
z=\frac{x}{y}.
\]

But this equation is not at present in an interpretable form. If
we can reduce it to such a form it will furnish the relation
required.

On developing the second member of the above equation, we
have
\[
z = xy + \frac{1}{0}x\left(1 -y\right) + 0\left(1 -x\right)y + \frac{0}{0}\left(1-x\right)\left(1-y\right),
\]
and it will be shown hereafter (Prop. 3) that this admits of the
following interpretation:

``Beasts which chew the cud consist of all clean beasts
(which also divide the hoof), together with an indefinite remainder
(some, none, or all) of unclean beasts which do not divide
the hoof.''

9. Now the above is a particular example of a problem of the
utmost generality in Logic, and which may thus be stated:---``Given
any logical equation connecting the symbols $x$, $y$, $z$, $w$,
required an interpretable expression for the relation of the class
represented by $w$ to the classes represented by the other symbols
$x$, $y$, $z$, \&c.''

The solution of this problem consists in all cases in determining,
from the equation \textit{given}, the expression of the above
symbol $w$, in terms of the other symbols, and rendering that expression
interpretable by development. Now the equation given
is always of the first degree with respect to each of the symbols
involved. The required expression for $w$ can therefore always
be found. In fact, if we develop the given equation, whatever
its form may be with respect to $w$, we obtain an equation of the
form
\setcounter{equation}{0}
\begin{equation}
Ew + E^\prime\left(1-w\right) = 0,
\end{equation}
$E$ and $E^\prime$ being functions of the remaining symbols. From the
above we have
\[
E^\prime=\left(E^\prime - E\right)w.
\]
Therefore
\begin{equation}
w = \frac{E^\prime}{E^\prime-E}
\end{equation}
and expanding the second member by the rule of development, it
will only remain to interpret the result in logic by the next
proposition.

If the fraction $\frac{E^\prime}{E^\prime-E}$ has common factors in its numerator
and denominator, we are not permitted to reject them, unless they
are mere numerical constants. For the symbols $x$, $y$, \&c., regarded
as quantitative, may admit of such values $0$ and $1$ as to
cause the common factors to become equal to $0$, in which case
the algebraic rule of reduction fails. This is the case contemplated
in our remarks on the failure of the algebraic axiom of
division (II. 14). To \textit{express} the solution in the form (2), and
without attempting to perform any unauthorized reductions, to
interpret the result by the theorem of development, is a course
strictly in accordance with the general principles of this treatise.

If the relation of the class expressed by $1 - w$ to the other
classes, $x$, $y$, \&c. is required, we deduce from (1), in like manner
as above,
\[
1-w=\frac{E}{E-E^\prime},
\]
to the interpretation of which also the method of the following
Proposition is applicable:

\begin{center}
\textsc{Proposition III.}
\end{center}

10. \textit{To determine the interpretation of any logical equation of
the form $w=V$, in which $w$ is a class symbol, and $V$ a function of
other class symbols quite unlimited in its form.}

Let the second member of the above equation be fully expanded.
Each coefficient of the result will belong to some one
of the four classes, which, with their respective interpretations,
we proceed to discuss.

1st. Let the coefficient be $1$. As this is the symbol of the
universe, and as the product of any two class symbols represents
those individuals which are found in both classes, any constituent
which has unity for its coefficient must be interpreted without
limitation, i.e. the whole of the class which it represents is
implied.

2nd. Let the coefficient be $0$. As in Logic, equally with
Arithmetic, this is the symbol of Nothing, no part of the class
represented by the constituent to which it is prefixed must be
taken.

3rd. Let the coefficient be of the form $\frac{0}{0}$. Now, as in Arithmetic,
the symbol $\frac{0}{0}$ represents an \textit{indefinite number}, except when
otherwise determined by some special circumstance, analogy
would suggest that in the system of this work the same symbol
should represent an \textit{indefinite class}. That this is its true
meaning will be made clear from the following example:

Let us take the Proposition, ``Men not mortal do not exist;''
represent this Proposition by symbols; and seek, in obedience to
the laws to which those symbols have been proved to be subject,
a reverse definition of ``mortal beings,'' in terms of ``men.''

Now if we represent ``men'' by $y$, and ``mortal beings'' by $x$,
the Proposition, ``Men who are not mortals do not exist,'' will
be expressed by the equation
\[
y \left(1 - x\right) = 0,
\]
from which we are to seek the value of $x$. Now the above
equation gives
\[
y - yx = 0,\textrm{ or }yx = y.
\]
Were this an ordinary algebraic equation, we should, in the next
place, divide both sides of it by $y$. But it has been remarked in
Chap. 11. that the operation of division cannot be \textit{performed} with
the symbols with which we are now engaged. Our resource, then,
is to \textit{express} the operation, and develop the result by the method
of the preceding chapter. We have, then, first,
\[
x = \frac{y}{y},
\]
and, expanding the second member as directed,
\[
x = y + \frac{0}{0}\left(1 - y\right).
\]
This implies that mortals ($x$) consist of all men ($y$), together
with such a remainder of beings which are not men $\left(1 - y\right)$, as
be indicated by the coefficient $\frac{0}{0}$. Now let us inquire what
remainder of ``not men'' is implied by the premiss. It might
happen that the remainder included all the beings who are not
men, or it might include only some of them, and not others, or it
might include none, and any one of these assumptions would be
in perfect accordance with our premiss. In other words, whether
those beings which are not men are \textit{all}, or \textit{some}, or \textit{none}, of them
\textit{mortal}, the truth of the premiss which virtually asserts that all
men are mortal, will be equally unaffected, and therefore the
expression $\frac{0}{0}$ here indicates that \textit{all}, \textit{some}, or \textit{none} of the class to
whose expression it is affixed must be taken.

Although the above determination of the significance of the
symbol $\frac{0}{0}$ is founded only upon the examination of a particular
case, yet the principle involved in the demonstration is general,
and there are no circumstances under which the symbol can present
itself to which the same mode of analysis is inapplicable.
We may properly term $\frac{0}{0}$ an \textit{indefinite class symbol}, and may, if
convenience should require, replace it by an uncompounded symbol
$v$, subject to the fundamental law, $v (1 - v) = 0$.

4th. It may happen that the coefficient of a constituent in an
expansion does not belong to any of the previous cases. To ascertain
its true interpretation when this happens, it will be necessary
to premise the following theorem:

11. \textsc{Theorem.}---\textit{If a function $V$, intended to represent any
class or collection of objects, $w$, be expanded, and if the numerical
coefficient, $a$, of any constituent in its development, do not satisfy
the law.
\[
a\left(1-a\right) = 0,
\]
then the constituent in question must be made equal to 0.}

To prove the theorem generally, let us represent the expansion
given, under the form
\setcounter{equation}{0}
\begin{equation}
w = a_1 t_1 + a_2 t_2  + a_3 t_3 + \&c.,
\end{equation}
in which $t_1$, $t_2$, $t_3$, \&c. represent the constituents, and $a_1$, $a_2$, $a_3$, \&c.
the coefficients; let us also suppose that $a_1$ and $a_2$ do not satisfy
the law
\[
a_1 \left(1 - a_1\right) = 0, \quad a_2 \left(1 - a_2\right) = 0;
\]

but that the other coefficients are subject to the law in question,
so that we have
\[
{a_3}^2 = a_3, \textrm{ \&c.}
\]
Now multiply each side of the equation (1) by itself. The result
will be
\begin{equation}
w = {a_1}^2 t_1 + {a_2}^2 t_2 + \textrm{ \&c.}
\end{equation}
This is evident from the fact that it must represent the development
of the equation
\[
w = V^2,
\]
but it may also be proved by actually squaring (1), and observing
that we have
\[
{t_1}^2 = t_1, \quad {t_2}^2 = t_2, \quad t_1 t_2 = 0, \textrm{ \&c.}
\]
by the properties of constituents. Now subtracting (2) from (1),
we have
\[
\left(a_1 - {a_1}^2\right) t_1 + \left(a_2 - {a_2}^2\right) t_2 = 0.
\]
Or,
\[
a_1\left(1 - a_1\right) t_1 + a_2 \left(1 - a_2\right) t_2 = 0.
\]
Multiply the last equation by $t_1$; then since $t_1 t_2 = 0$, we have
\[
a_1 \left(1 - a_1\right) t_1 = 0,\textrm{ whence } t_2 = 0.
\]
In like manner multiplying the same equation by $t_2$, we have
\[
a_2 \left(1 - a_2\right) t_2 = 0,\textrm{ whence } t_2 = 0.
\]

Thus it may be shown generally that any constituent whose
coefficient is not subject to the same fundamental law as the symbols
themselves must be separately equated to $0$. The usual
form under which such coefficients occur is $\frac{1}{0}$. This is the algebraic
symbol of infinity. Now the nearer any number approaches
to infinity (allowing such an expression), the more does it depart
from the condition of satisfying the fundamental law above referred
to.

The symbol $\frac{0}{0}$, whose interpretation was previously discussed,
does not necessarily disobey the law we are here considering,
for it admits of the numerical values $0$ and $1$ indifferently.
Its actual interpretation, however, as an indefinite class symbol,
cannot, I conceive, except upon the ground of analogy, be deduced
from its arithmetical properties, but must be established
experimentally.

12. We may now collect the results to which we have been
led, into the following summary:

1st. The symbol 1, as the coefficient of a term in a development,
indicates that the whole of the class which that constituent
represents, is to be taken.

2nd. The coefficient 0 indicates that none of the class are to
be taken.

3rd. The symbol $\frac{0}{0}$ indicates that a perfectly \textit{indefinite} portion
of the class, i.e. \textit{some}, \textit{none}, or \textit{all} of its members are to be
taken.

4th. Any other symbol as a coefficient indicates that the
constituent to which it is prefixed must be equated to 0.

It follows hence that if the solution of a problem, obtained
by development, be of the form
\[
w = A+0B+\frac{0}{0}C+\frac{1}{0}D,
\]
that solution may be resolved into the two following equations,
viz.,
\begin{eqnarray}
w = A + vC,\\
D = O,
\end{eqnarray}
$v$ being an indefinite class symbol. The interpretation of (3)
shows what elements enter, or may enter, into the composition
of $w$, the class of things whose definition is required; and the
interpretation of (4) shows what relations exist among the elements
of the original problem, in perfect independence of $w$.

Such are the canons of interpretation. It may be added, that
they are universal in their application, and that their use is
always unembarrassed by exception or failure.

13. \textit{Corollary}.--If $V$ be an independently interpretable logical
function, it will satisfy the symbolical law, $V(1-V) = 0$.

By an independently interpretable logical function, I mean
one which is interpretable, without presupposing any relation
among the things represented by the symbols which it involves.
Thus $x(1-y)$ is independently interpretable, but $x-y$ is not so.
The latter function presupposes, as a condition of its interpretation,
that the class represented by $y$ is wholly contained in the class
represented by $x$; the former function does not imply any such requirement.

Now if $V$ be independently interpretable, and if $w$ represent
the collection of individuals which it contains, the equation
$w = V$ will hold true without entailing as a consequence the vanishing
of any of the constituents in the development of $V$;
since such vanishing of constituents would imply relations among
the classes of things denoted by the symbols in $V$. Hence the
development of $V$ will be of the form
\[
a_1t_1+a_2t_2+\textrm{\&c}.
\]
the coefficients $a_1$, $a_2$, \&c. all satisfying the condition
\[
a_1(1-a_1) = 0, a_2 (1-a_2) = 0, \&c.
\]

Hence by the reasoning of Prop. 4, Chap. v. the function $V$ will
be subject to the law
\[
V(1-V) = 0.
\]

This result, though evident \textit{\`{a} priori} from the fact that $V$ is supposed
to represent a class or collection of things, is thus seen to follow also from the properties of the constituents of which it is
composed. The condition $V(1-V) = 0$ may be termed ``the
condition of interpretability of logical functions.''

14. The general form of solutions, or logical conclusions developed
in the last Proposition, may be designated as a ``Relation
between terms.'' I use, as before, the word ``terms'' to denote
the parts of a proposition, whether simple or complex, which are
connected by the copula ``is'' or ``are.'' The classes of things represented
by the individual symbols may be called the elements
of the proposition.

15. Ex. 1.--Resuming the definition of ``clean beasts,''
(VI.6), required a description of ``unclean beasts.''

Here, as before, $x$ standing for `` clean beasts,'' $y$ for ``beasts
dividing the hoof,'' $z$ for ``beasts chewing the cud,'' we have
\begin{equation}
x = yz;
\end{equation}
whence
\[
1-x = 1-yz ;
\]
and developing the second member,

\[
     1-x = y(1-z)+z(1-y)+(1-y)(1-z);
\]
which is interpretable into the following Proposition: \textit{Unclean
beasts are all which divide the hoof without chewing the cud, all
which chew the cud without dividing the hoof, and all which neither
divide the hoof nor chew the cud.}

Ex. 2.--The same definition being given, required a description
of beasts which do not divide the hoof.

From the equation $x = yz$ we have
\[
     y = \frac{x}{z};
\]
therefore,
\[
     1-y = \frac{z-x}{z};
\]
and developing the second member,
\[
     1-y = 0\:xz+\frac{-1}{0}x(1-z)+(1-x)z+\frac{0}{0}(1-x)(1-z).
\]s
Here, according to the Rule, the term whose coefficients is $\frac{-1}{0}$,
must be separately equated to $0$, whence we have
\begin{eqnarray*}
     1-y = (1-x)z+v(1-x)(1-z),\\
     x(1-z)=0;
\end{eqnarray*}
whereof the first equation gives by interpretation the Proposition:
\textit{Beasts which do not divide the hoof consist of all unclean beasts which
chew the cud, and an indefinite remainder (some, none, or all) of unclean
beasts which do not chew the cud.}

The second equation gives the Proposition: \textit{There are no clean
beasts which do not chew the cud.} This is one of the independent
relations above referred to. We sought the direct relation of
``Beasts not dividing the hoof,'' to ``Clean beasts and beasts
which chew the cud.'' It happens, however, that independently
of any relation to beasts not dividing the hoof, there exists, in
virtue of the premiss, a separate relation between clean beasts
and beasts which chew the cud. This relation is also necessarily
given by the process.

Ex. 3.--Let us take the following definition, viz.: ``Responsible
beings are all rational beings who are either free to act, or
have voluntarily sacrificed their freedom,'' and apply to it the
preceding analysis.

\begin{tabular}{c l c l}
Let &$x$ &stand for &responsible beings.\\
    &$y$     &"     &rational beings. \\
    &$z$     &"     &those who are free to act,\\
    &$w$     &"     &those who have voluntarily sacrificed their \\
    &        &      &freedom of action.
\end{tabular}\\

In the expression of this definition I shall assume, that the
two alternatives which it presents, viz.: ``Rational beings free
to act,'' and ``Rational beings whose freedom of action has been
voluntarily sacrificed,'' are mutually exclusive, so that no individuals
are found at once in both these divisions. This will permit
us to interpret the proposition literally into the language of
symbols, as follows:

\begin{equation}
x = yz + yw.
\end{equation}

Let us first determine hence the relation of ``rational beings'' to
responsible beings, beings free to act, and beings whose freedom
of action has been voluntarily abjured. Perhaps this object will
be better stated by saying, that we desire to express the relation
among the elements of the premiss in such a form as will enable
us to determine how far rationality may be inferred from responsibility,
freedom of action, a voluntary sacrifice of freedom, and
their contraries.

From (6) we have

\[
y = \frac{x}{(z + w)},
\]

and developing the second member, but rejecting terms whose
coefficients are 0,

\begin{eqnarray*}
y = \frac{1}{2}xzw + xz(1-w) + x(1-z)w + \frac{1}{0}x(1-z)(1-w)\\
                                       + \frac{0}{0}(1-x)(1-z)(1-w),
\end{eqnarray*}

whence, equating to 0 the terms whose coefficients are $\frac{1}{2}$ and $\frac{1}{0}$,
we have

\begin{equation}
y = xz(1-w) + xw(1-z) + v(1-x)(1-z)(1-w);
\end{equation}

\begin{equation}
xzw = 0;
\end{equation}

\begin{equation}
x(1-z)(1-w)= 0;
\end{equation}
whence by interpretation---

\textsc{Direct Conclusion}.---\textit{Rational beings are all responsible beings
who are either free to act, not having voluntarily sacrificed their freedom, or not free to act, having voluntarily sacrificed their freedom,
together with an indefinite remainder (some, none, or all) of beings
not responsible, not free, and not having voluntarily sacrificed their
freedom.}

\textsc{First Independent Relation}.---\textit{No responsible beings are at
the same time free to act, and in the condition of having voluntarily
sacrificed their freedom.}

\textsc{Second}.--\textit{No responsible beings are not free to act, and at the
same time in the condition of not having sacrificed their freedom.}

The independent relations above determined may, however,
be put in another and more convenient form. Thus (8) gives
\[
xw=\frac{0}{z}=0z+\frac{0}{0}(1-z), \textrm{ on development;}
\]
or,
\begin{equation}
xw=v(1-z);
\end{equation}
and in like manner (9) gives
\[
x(1-w)=\frac{0}{1-z}=\frac{0}{0}z+0(1-z);
\]
or,
\begin{equation}
x (1 - w) = vz;
\end{equation}
and (10) and (11) interpreted give the following Propositions:

1st. \textit{Responsible beings who have voluntarily sacrificed their freedom
are not free.}

2nd. \textit{Responsible beings who have not voluntarily sacrificed their
freedom are free.}

These, however, are merely different forms of the relations
before determined.

16. In examining, these results, the reader must bear in mind,
that the sole province of a method of inference or analysis, is to
determine those relations which are necessitated by the \textit{connexion}
of the terms in the original proposition. Accordingly, in estimating the completeness with which this object is effected, we
have nothing whatever to do with those other relations which
may be suggested to our minds by the \textit{meaning} of the terms
employed, as distinct from their expressed connexion. Thus it
seems obvious to remark, that ``They who have voluntarily sacrificed
their freedom are not free,'' this being a relation implied
in the very meaning of the terms. And hence it might appear,
that the first of the two independent relations assigned by the method
is on the one hand needlessly limited, and on the other hand
superfluous. However, if regard be had merely to the connexion
of the terms in the original premiss, it will be seen that the relation
in question is not liable to either of these charges. The
solution, as expressed in the direct conclusion and the independent
relations, conjointly, is perfectly complete, without being
in any way superfluous.

If we wish to take into account the implicit relation above
referred to, viz., ``They who have voluntarily sacrificed their
freedom are not free,'' we can do so by making this a distinct
proposition, the proper expression of which would be
\[
w = v (1 - z).
\]

This equation we should have to employ together with that
expressive of the original premiss. The mode in which such an
examination must be conducted will appear when we enter upon
the theory of systems of propositions in a future chapter. The
sole difference of result to which the analysis leads is, that the
first of the independent relations deduced above is superseded.

17. Ex. 4. -- Assuming the same definition as in Example 2,
let it be required to obtain a description of irrational persons.

We have
\begin{eqnarray*}
1-y &=& l - \frac{x}{z+w}\\
&=& \frac{z + w - x}{z + w}
\end{eqnarray*}
\begin{eqnarray*}
&=& \frac{1}{2} xzw + 0 xz (1 - w) + 0 x (1 - z) w - \frac{1}{0} x (1 - z) (1 - w)\\
&+& (1-x)zw + (1-x)z(1-w) + (1-x)(1-z)w + \frac{0}{0}(1-x)(1-z)(1-w)\\
&=& (1-x)zw + (1-x)z(1-w) + (1-x)(1-z)w + v(1-x)(1-z)(1-w)\\
&=& (1-x)z + (1-x)(1-z)w + v(1-x)(1-z)(1-w),
\end{eqnarray*}
with $xzw = 0, \quad x(1-z)(1-w) = 0$.

The independent relations here given are the same as we
before arrived at, as they evidently ought to be, since whatever
relations prevail independently of the existence of a given class
of objects $y$, prevail independently also of the existence of the
contrary class $1 - y$.

The direct solution afforded by the first equation is:--\textit{Irrational
persons consist of all irresponsible beings who are either free to
act, or have voluntarily sacrificed their liberty, and are not free to
act; together with an indefinite remainder of irresponsible beings
who have not sacrificed their liberty, and are not free to act}.

18. The propositions analyzed in this chapter have been of
that species called definitions. I have discussed none of which
the second or predicate term is particular, and of which the general
type is $Y= vX$, $Y$ and $X$ being functions of the logical
symbols $x$, $y$, $z$, \&c., and $v$ an indefinite class symbol. The analysis
of such propositions is greatly facilitated (though the step
is not an essential one) by the elimination of the symbol $v$, and
this process depends upon the method of the next chapter. I
postpone also the consideration of another important problem
necessary to complete the theory of single propositions, but of
which the analysis really depends upon the method of the reduction
of systems of propositions to be developed in a future page
of this work.


\chapter[OF ELIMINATION]
{\large ON ELIMINATION.}

1. In the examples discussed in the last chapter, all the elements
of the original premiss re-appeared in the conclusion,
only in a different order, and with a different connexion. But it
more usually happens in common reasoning, and especially when
we have more than one premiss, that some of the elements are
required not to appear in the conclusion. Such elements, or, as
they are commonly called, ``middle terms,'' may be considered
as introduced into the original propositions only for the sake of
that connexion which they assist to establish among the other
elements, which are alone designed to enter into the expression of
the conclusion.

2. Respecting such intermediate elements, or middle terms,
some erroneous notions prevail. It is a general opinion, to which,
however, the examples contained in the last chapter furnish a contradiction,
that inference consists peculiarly in the elimination of
such terms, and that the elementary type of this process is exhibited
in the elimination of one middle term from two premises, so as
to produce a single resulting conclusion into which that term does
not enter. Hence it is commonly held, that \textit{syllogism} is the basis,
or else the common type, of all inference, which may thus, however
complex its form and structure, be resolved into a series of
syllogisms. The propriety of this view will be considered in a
subsequent chapter. At present I wish to direct attention to an
important, but hitherto unnoticed, point of difference between
the system of Logic, as expressed by symbols, and that of common
algebra, with reference to the subject of elimination. In
the algebraic system we are able to eliminate one symbol from
two equations, two symbols from three equations, and generally
$n - 1$ symbols from $n$ equations. There thus exists a definite
connexion between the number of independent equations given,
and the number of symbols of quantity which it is possible to
eliminate from them. But it is otherwise with the system of
Logic. No fixed connexion there prevails between the number
of equations given representing propositions or premises,
and the number of typical symbols of which the elimination
can be effected. From a single equation an indefinite number
of such symbols may be eliminated. On the other hand,
from an indefinite number of equations, a single class symbol
only may be eliminated. We may affirm, that in this peculiar
system, the problem of elimination is resolvable under all circumstances
alike. This is a consequence of that remarkable law of
duality to which the symbols of Logic are subject. To the equations
furnished by the premises given, there is added another
equation or system of equations drawn from the fundamental
laws of thought itself, and supplying the necessary means for the
solution of the problem in question. Of the many consequences
which flow from the law of duality, this is perhaps the most
deserving of attention.

3. As in Algebra it often happens, that the elimination of
symbols from a given system of equations conducts to a mere
identity in the form $0 = 0$, no independent relations connecting
the symbols which remain; so in the system of Logic, a like result,
admitting of a similar interpretation, may present itself.
Such a circumstance does not detract from the generality of
the principle before stated. The object of the method upon
which we are about to enter is to eliminate any number of symbols
from any number of logical equations, and to exhibit in the
result the actual relations which remain. Now it may be, that
no such residual relations exist. In such a case the truth of the
method is shown by its leading us to a merely identical proposition.

4. The notation adopted in the following Propositions is
similar to that of the last chapter. By $f(x)$ is meant any expression
involving the logical symbol $x$, with or without other
logical symbols. By $f(1)$ is meant what $f(x)$ becomes when $x$
is therein changed into 1; by $f(0)$ what the same function becomes
when $x$ is changed into 0.

\begin{center}
\textsc{Proposition I}.
\end{center}

5. If $f(x)=0$ be any logical equation involving the class symbol
$x$, with or without other class symbols, then will the equation

\[
f(1)f(0)=0
\]

be true, independently of the interpretation of $x$; and it will be the complete result of the elimination of $x$ from the above equation.

In other words, the elimination of $x$ from any given equation,
$f(x)=0$, will be effected by successively changing in that equation $x$ into 1, and $x$ into 0, and multiplying the two resulting equations together.

Similarly the complete result of the elimination of any class symbols, $x$, $y$, etc.,from any equation of the form $V=0$, will be obtained
by completely expanding the first member of that equation in constituents of the given symbols, and multiplying together all the coefficients of those constituents, and equating the product to 0.

Developing the first member of the equation $f(x)=0$, we
have (V. 10),

\begin{eqnarray}
f(1)x+f(0)(1-x)=0;\nonumber \\
\textrm{or, }
[f(1)-f(0)]x+f(0)=0.
\therefore x=\frac{f(0)}{f(0)-f(1)}; \nonumber\\
\textrm{and }
1-x=-\frac{f(1)}{f(0)-f(1)}.\nonumber
\end{eqnarray}

Substitute these expressions for $x$ and $1-x$ in the fundamental
equation
\[
x(1-x) = 0,
\]

and there results
\begin{eqnarray}
-\frac{f(0)f(1)}{[f(0)-f(1)]^2}=0; \nonumber\\
\textrm{or, }f(1)f(0)=0,
\end{eqnarray}

the form required.

6. It is seen in this process, that the elimination is really effected
between the given equation $f(x)=0$ and the universally true
equation $x(1-x)=0$, expressing the fundamental law of logical
symbols, \textit{qua} logical. There exists, therefore, no need of more
than one premiss or equation, in order to render possible the elimination
of a term, the necessary law of thought virtually supplying
the other premiss or equation. And though the demonstration
of this conclusion may be exhibited in other forms, yet
the same element furnished by the mind itself will still be virtually
present. Thus we might proceed as follows:

Multiply (1) by $x$, and we have
\begin{equation}
f(1)x=0,
\end{equation}
and let us seek by the forms of ordinary algebra to eliminate $x$
from this equation and (1).

Now if we have two algebraic equations of the form
\[
ax + b = 0,
\]
\[
a'x + b'= 0;
\]
it is well known that the result of the elimination of $x$ is
\begin{equation}
ab'-a'b=0
\end{equation}
But comparing the above pair of equations with (1) and (3)
respectively, we find
\[
a=f(1)-f(0),\quad b=f(0);
\]
\[
a'=f(1) \quad b'=0;
\]
which, substituted in (4), give
\[
f(1)f(0)=0,
\]
as before. In this form of the demonstration, the fundamental
equation $x(1 - x) = 0$, makes its appearance in the derivation of
(3) from (1).

7. I shall add yet another form of the demonstration, partaking
of a half logical character, and which may set the demonstration
of this important theorem in a clearer light.

We have as before
\[
f(1)x+f(0)(1-x)=0.
\]
Multiply this equation first by $x$, and secondly by $1 - x$, we get
\[
f(1)x=0 \quad f(0)(1-x)=0.
\]
From these we have by solution and development,

\begin{eqnarray*}
     f(1) = \frac{0}{x}
          = \frac{0}{0}(1-x),\:\textrm{ on development},\\
     f(0) = \frac{0}{1-x}
          = \frac{0}{0}x.
\end{eqnarray*}
The direct interpretation of these equations is--

1st. Whatever individuals are included in the class represented by $f(1)$, are not $x$'s.

2nd. Whatever individuals are included in the class represented
by $f(0)$, are $x$'s.

Whence by common logic, there are no individuals at once
in the class $f(1)$ and in the class $f(0)$, i.e. there are no individuals
in the class $f(1)f(0)$. Hence,
\begin{equation}
     f(1)f(0) = 0.
\end{equation}
Or it would suffice to multiply together the developed equations,
whence the result would immediately follow.

8. The theorem (5) furnishes us with the following Rule :

TO ELIMINATE ANY SYMBOL FROM A PROPOSED EQUATION.

RULE.--\textit{The terms of the equation having been brought, by
transposition if necessary, to the first side, give to the symbol
successively the values 1 and 0, and multiply the resulting equations
together.}

The first part of the Proposition is now proved.

9. Consider in the next place the general equation
\[
     f(x,y) = 0;
\]
the first member of which represents any function of $x$, $y$, and
other symbols.

By what has been shown, the result of the elimination of $y$
from this equation will be
\[
     f(x,1)f(x,0) = 0;
\]
for such is the form to which we are conducted by successively
changing in the given equation $y$ into 1, and $y$ into 0, and
multiplying the results together.

Again, if in the result obtained we change successively $x$ into
1, and $x$ into 0, and multiply the results together, we have
\begin{equation}
     f(1,1)f(1,0)f(0,1)f(0,0) = 0;
\end{equation}
as the final result of elimination.
But the four factors of the first member of this equation are
the four coefficients of the complete expansion of $f(x, y)$, the
first member of the original equation; whence the second part of
the Proposition is manifest.

\begin{center}
\textsc{examples}.
\end{center}

10. Ex. 1. -- Having given the Proposition, ``All men are
mortal,'' and its symbolical expression, in the equation,
\[
y = vx,
\]

in which $y$ represents ``men,'' and $x$ ``mortals,'' it is required to
eliminate the indefinite class symbol $v$, and to interpret the
result.

Here bringing the terms to the first side, we have
\[
y - vx = 0.
\]
When $v = 1$ this becomes
\[
y - x = 0;
\]
and when $v = 0$ it becomes
\[
y = 0;
\]
and these two equations multiplied together, give
\[
y - yx = 0,
\]
or $y(1 - x) = 0$, \\
it being observed that $y^2 = y$.

The above equation is the required result of elimination, and
its interpretation is, \textit{Men who are not mortal do not exist}, -- an obvious conclusion.

If from the equation last obtained we seek a description of
beings who are not mortal, we have
\[
x = \frac{y}{y},
\]
\[
\therefore 1 - x = \frac{0}{y}.
\]

Whence, by expansion, $1 - x = \frac{0}{0}(1 - y)$, which interpreted gives,
\textit{They who are not mortal are not men}. This is an example of
what in the common logic is called conversion by contraposition,
or negative conversion.
\footnote{Whately's Logic, Book II. chap. II. sec. 4.}

Ex. 2.--Taking the Proposition, ``No men are perfect,'' as
represented by the equation
\[
     y = v(1-x),
\]
wherein $y$ represents ``men,'' and $x$ ``perfect beings,'' it is required
to eliminate $v$, and find from the result a description both
of \textit{perfect beings} and of \textit{imperfect beings}. We have
\[
     y-v(1-x) = 0.
\]
Whence, by the rule of elimination,
\[
     \{y-(1-x)\}\times y = 0,
\]
or
\[
     y-y(1-x) = 0,
\]
or
\[
     yx = 0;
\]
which is interpreted by the Proposition, \textit{Perfect men do not exist}.
From the above equation we have
\[
     x = \frac{0}{y} = \frac{0}{0}(1-y)\:\textrm{by development};
\]
whence, by interpretation, \textit{No perfect beings are men.} Similarly,
\[
     1-x = 1-\frac{0}{y} = \frac{y}{y} = y+\frac{0}{0}(1-y),
\]
which, on interpretation, gives, \textit{Imperfect beings are all men
with an indefinite remainder of beings, which are not men}.

11. It will generally be the most convenient course, in the
treatment of propositions, to eliminate first the indefinite class
symbol $v$, wherever it occurs in the corresponding equations.
This will only modify their form, without impairing their significance.
Let us apply this process to one of the examples of
Chap. IV. For the Proposition, ``No men are placed in exalted
stations and free from envious regards,'' we found the expression
\[
     y = v(1-xz),
\]
and for the equivalent Proposition, ``Men in exalted stations are
not free from envious regards,'' the expression
\[
     yx = v(1-z);
\]

and it was observed that these equations, $v$ being an indefinite
class symbol, were themselves equivalent. To prove this, it is
only necessary to eliminate from each the symbol $v$. The first
equation is
\[
y - v (1 - xz) = 0,
\]
whence, first making $v = 1$, and then $v = 0$, and multiplying the
results, we have

\begin{eqnarray*}
(y - 1 + xz) y = 0,\\
\textrm{or } yxz = 0.
\end{eqnarray*}

Now the second of the given equations becomes on transposition

\begin{eqnarray*}
yx - v(1 - z) - 0; \\
\textrm{whence } (x - 1 + z) yx = 0,\\
\textrm{or } yxz = 0,
\end{eqnarray*}

as before. The reader will easily interpret the result.

12. Ex. 3.--As a subject for the general method of this
chapter, we will resume Mr. Senior's definition of wealth, viz.:
``Wealth consists of things transferable, limited in supply, and
either productive of pleasure or preventive of pain.'' We shall
consider this definition, agreeably to a former remark, as including
all things which possess at once both the qualities expressed in
the last part of the definition, upon which assumption we have,
as our representative equation,

\begin{eqnarray*}
w = st \{pr + p (1 - r) + r(1 - p)\}, \\
\textrm{or } w = st \{p + r(1 -p)\},
\end{eqnarray*}

wherein

\begin{tabular}{c l l}
$w$&stands for &wealth.\\
$s$&''&things limited in supply.\\
$t$&''&things transferable.\\
$p$&''&things productive of pleasure.\\
$r$&''&things preventive of pain.\\
\end{tabular}\\

From the above equation we can eliminate any symbols that
we do not desire to take into account, and express the result by
solution and development, according to any proposed arrangement
of subject and predicate.

Let us first consider what the expression for $w$, wealth, would
be if the element $r$, referring to prevention of pain, were
eliminated. Now bringing the terms of the equation to the first side,
we get

\[
w - st(p+r-rp) = 0.
\]

Making $r = 1$, the first member becomes $w - st$, and making
$r = 0$ it becomes $w - stp$; whence we have by the Rule,

\begin{equation}
(w-st)(w-stp)=0
\end{equation}

or

\begin{equation}
w-wstp - wst + stp = 0;
\end{equation}

whence

\[
w = \frac{stp}{st + stp - 1};
\]

the development of the second member of which equation gives

\begin{equation}
w = stp + \frac{0}{0}st (1 - p).
\end{equation}

Whence we have the conclusion,--\textit{Wealth consists of all things
limited in supply, transferable, and productive of pleasure, and an
indefinite remainder of things limited in supply, transferable, and
not productive of pleasure.} This is sufficiently obvious.

Let it be remarked that it is not necessary to perform the
multiplication indicated in (7), and reduce that equation to the
form (8), in order to determine the expression of $w$ in terms of
the other symbols. The process of development may in all cases
be made to supersede that of multiplication. Thus if we
develop (7) in terms of $w$, we find

\[
(1 - sf) (1 - stp)w + stp(1 - w) = 0,
\]

whence

\[
w=\frac{stp}{stp -(1-st)(1-stp)};
\]

and this equation developed will give, as before,

\[
w = stp + \frac{0}{0} st(1 - p).
\]

13. Suppose next that we seek a description of things limited
in supply, as dependent upon their relation to wealth,
transferableness, and tendency to produce pleasure, omitting all reference to
the prevention of pain.

From equation (8), which is the result of the elimination of
$r$ from the original equation, we have
\[
w - s \left(wt + wtp - tp\right) = 0;
\]
whence
\begin{eqnarray*}
s = \frac{w}{wt + wtp - tp}\\
= wtp + wt\left(1-p\right) + \frac{1}{0} w\left(1-t\right)p + \frac{1}{0}w \left(1-t\right)\left(1-p\right)\\
+ 0 \left(1-w\right) tp + \frac{0}{0} \left(1-w\right) t \left(1-p\right) + \frac{0}{0} \left(1-w\right)\left(1-t\right) p\\
+ \frac{0}{0} \left(1-w\right) \left(1-t\right) \left(1-p\right).
\end{eqnarray*}
We will first give the direct interpretation of the above solution,
term by term; afterwards we shall offer some general remarks
which it suggests; and, finally, show how the expression of the
conclusion may be somewhat abbreviated.

First, then, the direct interpretation is, Things limited in
supply consist of \textit{All wealth transferable and productive of pleasure--all
wealth transferable, and not productive of pleasure,--an indefinite
amount of what is not wealth, but is either transferable, and not
productive of pleasure, or intransferable and productive of pleasure,
or neither transferable nor productive of pleasure.}

To which the terms whose coefficients are $\frac{1}{0}$ permit us to add
the following independent relations, viz.:

1st. \textit{Wealth that is intransferable, and productive of pleasure,
does not exist.}

2ndly. \textit{Wealth that is intransferable, and not productive of pleasure,
does not exist.}

14. Respecting this solution I suppose the following remarks
are likely to be made.

First, it may be said, that in the expression above obtained
for ``things limited in supply,'' the term ``All wealth transferable,''
\&c., is in part redundant; since all wealth is (as implied
in the original proposition, and directly asserted in the \textit{independent
relations}) necessarily transferable.

I answer, that although in ordinary speech we should not
deem it necessary to add to ``wealth'' the epithet ``transferable,''
if another part of our reasoning had led us to express the conclusion,
that there is no wealth which is not transferable, yet it
pertains to the perfection of this method that it in all cases fully
defines the objects represented by each term of the conclusion,
by stating the relation they bear to each quality or element of distinction
that we have chosen to employ. This is necessary in order
to keep the different parts of the solution really distinct and independent,
and actually prevents redundancy. Suppose that the
pair of terms we have been considering had not contained the
word ``transferable,'' and had unitedly been ``All wealth,'' we
could then logically resolve the single term ``All wealth'' into
the two terms ``All wealth transferable,'' and ``All wealth
intransferable.'' But the latter term is shown to disappear by
the ``independent relations.'' Hence it forms no part of the description
required, and is therefore redundant. The remaining
term agrees with the conclusion actually obtained.

Solutions in which there cannot, by logical divisions, be produced
any superfluous or redundant terms, may be termed \textit{pure
solutions}. Such are all the solutions obtained by the method of
development and elimination above explained. It is proper to
notice, that if the common algebraic method of elimination were
adopted in the cases in which that method is possible in the present
system, we should not be able to depend upon the purity of
the solutions obtained. Its want of generality would not be its
only defect.

15. In the second place, it will be remarked, that the conclusion
contains two terms, the aggregate significance of which
would be more conveniently expressed by a single term. Instead
of ``All wealth productive of pleasure, and transferable,'' and
``All wealth not productive of pleasure, and transferable,'' we
might simply say, ``All wealth transferable.'' This remark is
quite just. But it must be noticed that whenever any such simplifications
are possible, they are immediately suggested by the
form of the equation we have to interpret; and if that equation
be reduced to its simplest form, then the interpretation to which
it conducts will be in its simplest form also. Thus in the original
solution the terms $wtp$ and $wt(1 - p)$, which have unity for their
coefficient, give, on addition, $wt$; the terms $w \left(1 - t \right) p$ and
$w \left(1 - t \right) \left(1 - p \right)$, which have $\frac{1}{0}$ for their
coefficient give $w \left(1 - t \right)$;
and the terms $\left(1 - w \right) \left(1 - t \right)p$ and
$\left(1 - w \right) \left(1 - t \right) \left(1 -p \right)$, which
have $\frac{0}{0}$ for their coefficient, give $\left(1 - w \right)
\left(1 - t \right)$. Whence the complete solution is

\[
s = wt + \frac{0}{0}\left(1-w\right)\left(1-t\right) +
\frac{0}{0}\left(1-w\right)t\left(1-p\right),
\]
with the independent relation,
\[
w\left(1-t\right) = 0,\textrm{ or }w=\frac{0}{0}t.
\]

The interpretation would now stand thus:--

1st. \textit{Things limited in supply consist of all wealth transferable,
with an indefinite remainder of what is not wealth and not transferable,
and of transferable articles which are not wealth, and are not
productive of pleasure.}

2nd. \textit{All wealth is transferable.}

This is the simplest form under which the general conclusion,
with its attendant condition, can be put.

16. When it is required to eliminate two or more symbols
from a proposed equation we can either employ (6) Prop. I., or
eliminate them in succession, the order of the process being indifferent.
From the equation
\[
w = st \left(p + r - pr\right),
\]
we have eliminated $r$, and found the result,
\[
w - wst - wstp + stp = 0.
\]
Suppose that it had been required to eliminate both $r$ and $t$, then
taking the above as the first step of the process, it remains to
eliminate from the last equation $t$. Now when $t = 1$ the first
member of that equation becomes
\[
w - ws - wsp + sp,
\]
and when $t = 0$ the same member becomes $w$. Whence we have
\[
w \left(w - ws - wsp + sp\right) = 0,
\]
or
\[
w - ws = 0,
\]
for the required result of elimination.

If from the last result we determine $w$, we have

\[
w=\frac{0}{1-s}=\frac{0}{0}s,
\]

whence ``\textit{All wealth is limited in supply}.'' As $p$ does not enter
into the equation, it is evident that the above is true, irrespectively
of any relation which the elements of the conclusion bear
to the quality ``productive of pleasure.''

Resuming the original equation, let it be required to eliminate
$s$ and $t$. We have

\[
w=st(p+r-pr).
\]

Instead, however, of separately eliminating $s$ and $t$ according to
the Rule, it will suffice to treat $st$ as a single symbol, seeing that
it satisfies the fundamental law of the symbols by the equation

\[
st (1 - st) = 0.
\]

Placing, therefore, the given equation under the form

\[
w - st (p + r - pr) = 0;
\]

and making $st$ successively equal to 1 and to 0, and taking the
product of the results, we have

\begin{eqnarray*}
(w - p - r + pr) w = 0,\\
\textrm{or }  w - wp - wr + wpr = 0,
\end{eqnarray*}

for the result sought.

As a particular illustration, let it be required to deduce an
expression for ``things productive of pleasure'' ($p$), in terms of
``wealth'' ($w$), and ``things preventive of pain'' ($r$).

We have, on solving the equation,

\[
p = \frac{w(1-r)}{w(1-r)}
\]
\[
= \frac{0}{0}wr + w(1 - r) + \frac{0}{0}(1 - w)r + \frac{0}{0} (1 - w) (1 - r)
\]

\[
= w(1 - r) + \frac{0}{0}wr + \frac{0}{0}(1 - w).
\]

Whence the following conclusion:--\textit{Things productive of pleasure
are, all wealth not preventive of pain, an indefinite amount
of wealth that is preventive of pain, and an indefinite amount of
what is not wealth.}

From the same equation we get
\[
1-p = 1-\frac{w(1-r)}{w(1-r)} = \frac{0}{w(1-r)},
\]
which developed, gives

\begin{eqnarray*}
w(1-p)
= \frac{0}{0}wr+\frac{0}{0}(1-w){\cdot}r+\frac{0}{0}(1-w)\cdot(1-r)\\
= \frac{0}{0}wr+\frac{0}{0}(1-w).
\end{eqnarray*}

Whence, \textit {Things not productive of pleasure are either wealth, preventive
of pain, or what is not wealth.}

Equally easy would be the discussion of any similar case.

17. In the last example of elimination, we have eliminated
the compound symbol $st$ from the given equation, by treating it
as a single symbol. The same method is applicable to any combination
of symbols which satisfies the fundamental law of individual
symbols. Thus the expression $p+r-pr$ will, on being
multiplied by itself, reproduce itself, so that if we represent
$p+r-pr$ by a single symbol as $y$, we shall have the fundamental
law obeyed, the equation
\[
y = y^2,\textrm{ or } y(1-y) = 0,
\]
being satisfied. For the rule of elimination for symbols is founded
upon the supposition that each individual symbol is subject to
that law; and hence the elimination of any function or combination
of such symbols from an equation, may be effected by a single
operation, whenever that law is satisfied by the function.

Though the forms of interpretation adopted in this and the
previous chapter show, perhaps better than any others, the direct
significance of the symbols 1 and $\frac{0}{0}$, modes of expression
more agreeable to those of common discourse may, with equal
truth and propriety, be employed. Thus the equation (9) may
be interpreted in the following manner: \textit{Wealth is either limited
in supply, transferable, and productive of pleasure, or limited in supply,
transferable, and not productive of pleasure}. And reversely,
\textit{Whatever is limited in supply, transferable, and productive of pleasure,
is wealth}. Reverse interpretations, similar to the above, are
always furnished when the final development introduces terms
having unity as a coefficient.

18. NOTE.--The fundamental equation $f(1)f(0) = 0$, expressing
the result of the elimination of the symbol $x$ from any
equation $f(x) = 0$, admits of a remarkable interpretation.

It is to be remembered, that by the equation $f(x) = 0$ is implied
some proposition in which the individuals represented by
the class $x$, suppose ``men,'' are referred to, together, it may be,
with other individuals; and it is our object to ascertain whether
there is implied in the proposition any relation among the other
individuals, independently of those found in the class \textit{men}. Now
the equation $f(1) = 0$ expresses what the original proposition
would become if \textit{men} made up the universe, and the equation
$f(0) = 0$ expresses what that original proposition would become
if men ceased to exist, wherefore the equation $f(1)f(0) = 0$ expresses
what in virtue of the original proposition would be
equally true on either assumption, i. e. equally true whether
``men'' were ``all things'' or ``nothing.'' Wherefore the theorem
expresses that \textit{what is equally true, whether a given class of
objects embraces the whole universe or disappears from existence,
is independent of that class altogether, and} \textit{vice vers\^{a}}. Herein
we see another example of the interpretation of formal results,
immediately deduced from the mathematical laws of thought, into
general axioms of philosophy.



\chapter[OF REDUCTION]
{\large ON THE REDUCTION OF SYSTEMS OF PROPOSITIONS.}


1. In the preceding chapters we have determined sufficiently
for the most essential purposes the theory of single primary
propositions, or, to speak more accurately, of primary propositions
expressed by a single equation. And we have established
upon that theory an adequate method. We have shown
how any element involved in the given system of equations may
be eliminated, and the relation which connects the remaining
elements deduced in any proposed form, whether of denial, of affirmation,
or of the more usual relation of subject and predicate.
It remains that we proceed to the consideration of systems of
propositions, and institute with respect to them a similar series
of investigations. We are to inquire whether it is possible from
the equations by which a system of propositions is expressed to
eliminate, \textit{ad libitum}, any number of the symbols involved; to
deduce by interpretation of the result the whole of the relations
implied among the remaining symbols; and to determine in particular
the expression of any single element, or of any interpretable
combination of elements, in terms of the other elements,
so as to present the conclusion in any admissible form that may
be required. These questions will be answered by showing that it
is possible to reduce any system of equations, or any of the equations
involved in a system, to an equivalent single equation, to
which the methods of the previous chapters may be immediately
applied. It will be seen also, that in this reduction is involved
an important extension of the theory of single propositions, which
in the previous discussion of the subject we were compelled to
forego. This circumstance is not peculiar in its nature. There
are many special departments of science which cannot be completely
surveyed from within, but require to be studied also from
an external point of view, and to be regarded in connexion with
other and kindred subjects, in order that their full proportions
be understood.

This chapter will exhibit two distinct modes of reducing
systems of equations to equivalent single equations. The first
of these rests upon the employment of arbitrary constant multipliers.
It is a method sufficiently simple in theory, but it has the
inconvenience of rendering the subsequent processes of elimination
and development, when they occur, somewhat tedious. It was,
however, the method of reduction first discovered, and partly on
this account, and partly on account of its simplicity, it has been
thought proper to retain it. The second method does not require
the introduction of arbitrary constants, and is in nearly
all respects preferable to the preceding one. It will, therefore,
generally be adopted in the subsequent investigations of this
work.

2. We proceed to the consideration of the first method.

\begin{center}
\textsc{Proposition I}.
\end{center}

\textit{Any system of logical equations may be reduced to a single equivalent
equation, by multiplying each equation after the first by a distinct
arbitrary constant quantity, and adding all the results, including
the first equation, together}.

By Prop. 2, Chap, VI., the interpretation of any single
equation, $f(x, y..) = 0$ is obtained by equating to 0 those constituents
of the development of the first member, whose coefficients
do not vanish. And hence, if there be given two equations,
$f(x, y..) = 0$, and $F(x, y..) = 0$, their united import will be
contained in the system of results formed by equating to 0 all
those constituents which thus present themselves in both, or in
either, of the given equations developed according to the Rule of
Chap. VI. Thus let it be supposed, that we have the two equations

\begin{equation}
xy - 2x = 0,
\end{equation}

\begin{equation}
x - y = 0;
\end{equation}

The development of the first gives

\[
- xy - 2x (1 - y) = 0;
\]
\begin{equation}
\textrm{whence, }xy = 0, x (1 - y) = 0.
\end{equation}

The development of the second equation gives
\[
x(1-y)-y(1 -x) = 0;
\]
\begin{equation}
\textrm{whence, }x (1 - y) = 0, y (1 - x) = 0.
\end{equation}
The constituents whose coefficients do not vanish in both developments
are $xy$, $x (1 - y)$, and $(1 - x) y$, and these would together
give the system
\begin{equation}
xy = 0, x(1-y) = 0, (l-x)y=0;
\end{equation}
which is equivalent to the two systems given by the developments
separately, seeing that in those systems the equation $x (1 - y) = 0$
is repeated. Confining ourselves to the case of binary systems
of equations, it remains then to determine a single equation,
which on development shall yield the same constituents with
coefficients which do not vanish, as the given equations produce.
Now if we represent by
\[
V_1 = 0, V_2 = 0,
\]
the given equations, $V_1$ and $V_2$ being functions of the logical symbols
$x, y, z,$ \&c.; then the single equation
\begin{equation}
V_1 +cV_2=0,
\end{equation}
$c$ being an arbitrary constant quantity, will accomplish the required object. For let $At$ represent any term in the full development $V$, wherein $t$ is a constituent and $A$ its numerical
coefficient, and let $Bt$ represent the corresponding term in the
full development of $V_2$, then will the corresponding term in the
development of (6) be
\[
(A + cB) t.
\]
The coefficient of $t$ vanishes if $A$ and $B$ both vanish, but not
otherwise. For if we assume that $A$ and $B$ do not both vanish,
and at the same time make
\begin{equation}
A + cB = 0,
\end{equation}

the following cases alone can present themselves.

1st. That $A$ vanishes and $B$ does not vanish. In this case
the above equation becomes
\[
cB = 0,
\]

and requires that $c = 0$. But this contradicts the hypothesis that
$c$ is an \textit{arbitrary} constant.

2nd. That $B$ vanishes and $A$ does not vanish. This assumption reduces (7) to
\[
A = 0,
\]
by which the assumption is itself violated.

3rd. That neither $A$ nor $B$ vanishes. The equation (7) then
gives
\[
    c = \frac{-A}{B}
\]
which is a definite value, and, therefore, conflicts with the hypothesis
that $c$ is arbitrary.

Hence the coefficient $A + cB$ vanishes when $A$ and $B$ both
vanish, but not otherwise. Therefore, the same constituents
will appear in the development of (6), with coefficients which do
not vanish, as in the equations $V_1 = 0, V_2 = 0$, singly or together.
And the equation $V_1 + c V_2 = 0$, will be equivalent to the system
$V_1 = 0, V_2 = 0$.

By similar reasoning it appears, that the general system of
equations
\[
V_1 = 0, V_2 = 0, V_3 = 0, \textrm{ \&c.}\,;
\]
may be replaced by the single equation
\[
V_1 + cV_2 + c'V_3 + \textrm{ \&c.} = 0\,,
\]
$c, c',$ \&c., being arbitrary constants. The equation thus formed
may be treated in all respects as the ordinary logical equations
of the previous chapters. The arbitrary constants $c_1, c_2$, \&c., are
not \textit{logical} symbols. They do not satisfy the law,
\[
c_1 (1 - c_1) = 0, c_2 (1 - c_2) = 0\,.
\]
But their introduction is justified by that general principle which
has been stated in (II. 15) and (V. 6), and exemplified in nearly
all our subsequent investigations, viz., that equations involving
the symbols of Logic may be treated in all respects as if those
symbols were symbols of quantity, subject to the special law
$x (1 - x) = 0$, until in the final stage of solution they assume a
form interpretable in that system of thought with which Logic
is conversant.

3. The following example will serve to illustrate the above
method.

Ex. 1.--Suppose that an analysis of the properties of a particular
class of substances has led to the following general conclusions,
viz.:

1st. That wherever the properties $A$ and $B$ are combined,
either the property $C$, or the property $D$, is present also; but
they are not jointly present.

2nd. That wherever the properties $B$ and $C$ are combined,
the properties $A$ and $D$ are either both present with them, or
both absent.

3rd. That wherever the properties $A$ and $B$ are both absent,
the properties $C$ and $D$ are both absent also; and vice versa, where
the properties $C$ and $D$ are both absent, $A$ and $B$ are both absent
also.

Let it then be required from the above to determine what
may be concluded in any particular instance from the presence of
the property $A$ with respect to the presence or absence of the
properties $B$ and $C$, paying no regard to the property $D$.

\begin{tabular}{c l}
Represent &the property $A$ by $x$; \\
     "    &the property $B$ by $y$; \\
     "    &the property $C$ by $z$; \\
     "    &the property $D$ by $w$.
\end{tabular}

Then the symbolical expression of the premises will be

\[
xy - v (w(1 - z) +z (1 - w));
yz = v (xw + (1 - x ) (1 - w));
(1-x)(1-y)=(1-z)(1-w).
\]

From the first two of these equations, separately eliminating the
indefinite class symbol $v$, we have

\[
xy (1 - w (1 - z) - z (1 - w)) = 0;
\]
\[
yz (1 - xw - (1 - x)(1 - w)) = 0.
\]

Now if we observe that by development

\[
1 - w(1 - z) - z(1 - w) = wz + (1 - w)(1 - z),
\]

and

\[
1 - xw - (1 - x) ( 1 - w) = x(1 - w) + w(1 - x),
\]


and in these expressions replace, for simplicity,

\[
1 - x by \bar{x}, 1 - y by \bar{y}, \&c.,
\]

we shall have from the three last equations,
\setcounter{equation}{0}
\begin{eqnarray}
xy (wz + \bar{w}\bar{z}) = 0; \\
yz (x\bar{w} + \bar{x}w) = 0; \\
\bar{x}\bar{y} = \bar{w}\bar{z};
\end{eqnarray}

and from this system we must eliminate $w$.

Multiplying the second of the above equations by $c$, and the
third by $c'$, and adding the results to the first, we have

\[
xy (wz + \bar{w}\bar{z}) + cyz (x\bar{w} + \bar{x}w) + c'(\bar{x}\bar{y} - \bar{w}\bar{z}) = 0.
\]

When $w$ is made equal to 1, and therefore $\bar{w}$ to 0, the first member
of the above equation becomes

\[
xyz + c\bar{x}yz + c'\bar{x}\bar{y}.
\]

And when in the same member $w$ is made 0 and $\bar{w}$ = 1, it becomes

\[
xy\bar{z} + cxyz + c'\bar{x}\bar{y} - c'\bar{z}.
\]

Hence the result of the elimination of $w$ may be expressed in the
form

\begin{equation}
(xyz + c\bar{x}yz + c'\bar{x}\bar{y}) (xy\bar{z} + cxyz + c'\bar{x}\bar{y} - c'\bar{z}) = 0;
\end{equation}

and from this equation $x$ is to be determined.

Were we now to proceed as in former instances, we should
multiply together the factors in the first member of the above
equation ; but it may be well to show that such a course is not
at all necessary. Let us develop the first member of (4) with
reference to $x$, the symbol whose expression is sought, we find

\[
yz (y\bar{z} + cyz - c'\bar{z}) x + (cyz + c'\bar{y}) (c'\bar{y} - c'{z}) (1 - x) = 0;
\]
\[
\textrm{or, }cyzx + (cyz + c'\bar{y}) (c'\bar{y} - c'\bar{z}) (1 - x) = 0;
\]
whence we find,

\[
x =\frac{(cyz + c'\bar{y}) (c'\bar{y} - c'\bar{z})}{(cyz + c'\bar{y}) (c'\bar{y} - c'\bar{z}) - cyz};
\]

and developing the second member with respect to $y$ and $z$,

\[
x=0yz+\frac{0}{0}y\bar{z}+\frac{c^{\prime 2}}{c^{\prime 2}}\bar{y}z+\frac{0}{0}\bar{y}\bar{z};
\]
or,
\[
x=\left(1-y\right)z+\frac{0}{0}y\left(1-z\right)+\frac{0}{0}\left(1-y\right)\left(1-z\right);
\]
or,
\[
x=\left(1-y\right)z+\frac{0}{0}\left(1-z\right);
\]
the interpretation of which is, \textit{Wherever the property $A$ is present,
there either $C$ is present and $B$ absent, or $C$ is absent.} And inversely,
\textit{Wherever the property $C$ is present, and the property $B$
absent, there the property $A$ is present.}

These results may be much more readily obtained by the
method next to be explained. It is, however, satisfactory to
possess different modes, serving for mutual verification, of arriving
at the same conclusion.

4. We proceed to the second method.

\begin{center}
\textsc{Proposition II.}
\end{center}

\textit{If any equations, $V_1=0$, $V_2=0$, $\&c.$, are such that the developments
of their first members consist only of constituents with positive
coefficients, those equations may be combined together into a single
equivalent equation by addition.}

For, as before, let $At$ represent any term in the development
of the function $V_1$, $Bt$ the corresponding term in the development
of $V_2$ and so on. Then will the corresponding term in the
development of the equation
\setcounter{equation}{0}
\begin{equation}
V_1 + V_2 + \textrm{\&c.} = 0,
\end{equation}
formed by the addition of the several given equations, be
\[
\left(A+B+\textrm{\&c.}\right)t.
\]
But as by hypothesis the coefficients $A$, $B$, \&c. are none of them
negative, the aggregate coefficient $A + B$, \&c. in the derived
equation will only vanish when the separate coefficients $A$, $B$, \&c.
vanish together. Hence the same constituents will appear in the
development of the equation (1) as in the several equations
$V_1=0$, $V_2=0$, \&c. of the original system taken collectively, and
therefore the interpretation of the equation (1) will be equivalent
to the collective interpretations of the several equations from
which it is derived.


\begin{center}
\textsc{Proposition III.}
\end{center}

5. \textit{If $V_1=0, V_2=0, \&c.$ represent any system of equations, the terms of which have by transposition been brought to like first side, then the combined interpretation of the system will be involved in the single equation, }

\[
V_{1}^2 + V_{2}^2 + \&c. = 0,
\]

\textit{formed by adding together the squares of the given equations.}

For let any equation of the system, as $V_1=0$, produce on development an equation

\[
a_1 t_1 + a_2 t_2 + \&c. = 0
\]

in which $t_1, t_2, \&c.$ are constituents, and $a_1, a_2, \&c.$ their corresponding coefficients. Then the equation $V_{1}^2=0$ will produce on development an equation

\[
a_{1}^2 t_1 + a_{2}^2 t_2 + \&c. = 0,
\]

as may be proved either from the law of the development or by
squaring the function $a_1 t_1 + a_2 t_2, \&c.$ in subjection to the conditions

\[
t_{1}^2=t_1 \textrm{, } t_{2}^2=t_2, \textrm{, } t_1 t_2=0
\]

assigned in Prop. 3, Chap. v. Hence the constituents which
appear in the expansion of the equation $V_{1}^2=0$, are the same
with those which appear in the expansion of the equation $V_1=0$,
and they have positive coefficients. And the same remark applies to the equations $V_2=0, \&c.$ Whence, by the last Proposition, the equation

\[
V_{1}^2 + V_{2}^2 +\&c. = 0
\]

will be equivalent in interpretation to the system of equations

\[
V_1=0 \textrm{, } V_2=0\textrm{, \&c.}
\]

\textit{Corollary}.--Any equation, $V=0$, of which the first member
already satisfies the condition

\[
V^2=V \textrm{, or } V(1-V)=0,
\]

does not need (as it would remain unaffected by) the process of
squaring. Such equations are, indeed, immediately developable
into a series of constituents, with coefficients equal to 1, Chap. v.
Prop. 4.


\begin{center}
\textsc{Proposition IV.}
\end{center}

6. \textit{Whenever the equations of a system have by the above process
of squaring, or by any other process, been reduced to a form
such that all the constituents exhibited in their development have
positive coefficients, any derived equations obtained by elimination
will possess the same character, and may be combined with the
other equations by addition.}

Suppose that we have to eliminate a symbol $x$ from any
equation $V = 0$, which is such that none of the constituents, in
the full development of its first member, have negative coefficients.
That expansion may be written in the form
\[
V_{1} x+V_{0}(1-x)=0
\]
$V_{1}$ and $V_{0}$ being each of the form
\[
a_{1}t_{1}+a_{2}t_{2}...+a_{n}t_{n},
\]
in which $t_{1}t_{2}... t_{n}$ are constituents of the other symbols, and
$a_{1}a_{2}...a_{n}$ in each case positive or vanishing quantities. The result of elimination is
\[
V_{1}V_{2}=0;
\]
and as the coefficients in $V_{1}$ and $V_{2}$, are none of them negative,
there can be no negative coefficients in the product $V_{1}V_{2}$.
Hence the equation $V_{1}V_{2}=0$ may be added to any other equation,
the coefficients of whose constituents are positive, and the
resulting equation will combine the full significance of those
from which it was obtained.

\begin{center}
\textsc{Proposition V.}
\end{center}

7. \textit{To deduce from the previous Propositions a practical rule or
method for the reduction of systems of equations expressing propositions in Logic.}

We have by the previous investigations established the following points, viz.:

1st. That any equations which are of the form $V = 0$, $V$ satisfying
the fundamental law of duality $V(1 - V) = 0$, may be
combined together by simple addition.

2ndly. That any other equations of the form $V = 0$ may be
reduced, by the process of squaring, to a form in which the same
principle of combination by mere addition is applicable.

It remains then only to determine what equations in the actual
expression of propositions belong to the former, and what to
the latter, class.

Now the general types of propositions have been set forth in
the conclusion of Chap. IV. The division of propositions which
they represent is as follows:

1st. Propositions, of which the subject is universal, and the
predicate particular.

The symbolical type (IV. 15) is

\[
X = vY,
\]

$X$ and $Y$ satisfying the law of duality. Eliminating $v$, we have
\setcounter{equation}{0}
\begin{equation}
X(1-Y) = 0,
\end{equation}

and this will be found also to satisfy the same law. No further
reduction by the process of squaring is needed.

2nd. Propositions of which both terms are universal, and of
which the symbolical type is

\[
X = Y,
\]

$X$ and $Y$ separately satisfying the law of duality. Writing the
equation in the form $X - Y = 0$, and squaring, we have

\begin{eqnarray}
X - 2XY + Y = 0,\nonumber\\
\textrm{or, } X(1 - Y) + Y(1- X) = 0.
\end{eqnarray}

The first member of this equation satisfies the law of duality, as
is evident from its very form.

We may arrive at the same equation in a different manner.
The equation

\[
X = Y
\]

is equivalent to the two equations

\[
X = vY\textrm{, }Y = vX,
\]

(for to affirm that $X$'s are identical with $Y$'s is to affirm both that
All $X$'s are $Y$'s, and that All $Y$'s are $X$'s). Now these equations
give, on elimination of $v$,
\[
X(1-Y)=0\textrm{, } Y(1-X)=0,
\]
which added, produce (2).

3rd. Propositions of which both terms are particular. The
form of such propositions is
\[
vX = vY,
\]
but $v$ is not quite arbitrary, and therefore must not be eliminated.
For $v$ is the representative of \textit{some}, which, though it may include
in its meaning \textit{all}, does not include \textit{none}. We must therefore
transpose the second member to the first side, and square the
resulting equation according to the rule.
The result will obviously be
\[
vX(1 - Y) + vY(l-X) = 0.
\]
The above conclusions it may be convenient to embody in a
Rule, which will serve for constant future direction.

8. \textsc{Rule}.--- \textit{The equations being so expressed as that the terms $X$
and $Y$ in the following typical forms obey the law of duality, change
the equations}
\[
X = vY \textrm{ into } X(1- Y) = 0,
\]
\[
X = Y \textrm{ into } X(1 - Y) + Y(1 - X) = 0.
\]
\[
vX= vY \textrm{ into } vX(1-Y) + vY(1 - X) = 0.
\]

\textit{Any equation which is given in the form $X = 0$ will not need transformation,
and any equation which presents itself in the form $X = 1$
may be replaced by $1 - X = 0$, as appears from the second of the
above transformations.}

When the equations of the system have thus been reduced,
any of them, as well as any equations derived from them by the
process of elimination, may be combined by addition.

9. \textsc{Note}.--It has been seen in Chapter IV. that in literally
translating the terms of a proposition, without attending to its
real meaning, into the language of symbols, we may produce
equations in which the terms $X$ and $Y$ do not obey the law of
duality. The equation $w = st(p + r)$, given in (3) Prop. 3 of
the chapter referred to, is of this kind. Such equations, however,
as it has been seen, have a meaning. Should it, for curiosity,
or for any other motive, be determined to employ them,
it will be best to reduce them by the Rule (VI. 5).

10. Ex. 2.--Let us take the following Propositions of Elementary
Geometry:

1st. Similar figures consist of all whose corresponding angles
are equal, and whose corresponding sides are proportional.

2nd. Triangles whose corresponding angles are equal have
their corresponding sides proportional, and \textit{vice vers\^{a}}.

To represent these premises, let us make \\
\begin{tabular}{c l l}
$s$ &= &similar. \\
$t$ &= &triangles. \\
$q$ &= &having corresponding angles equal. \\
$r$ &= &having corresponding sides proportional. \\
\end{tabular}\\
Then the premises are expressed by the following equations:
\setcounter{equation}{0}
\begin{eqnarray}
s & = & qr, \\
tq & = & tr.
\end{eqnarray}
Reducing by the Rule, or, which amounts to the same thing,
bringing the terms of these equations to the first side, squaring
each equation, and then adding, we have
\begin{equation}
s + qr - 2qrs + tq + tr - 2tqr = 0.
\end{equation}
Let it be required to deduce a description of dissimilar figures
formed out of the elements expressed by the terms, \textit{triangles},
having corresponding angles equal, having corresponding sides
proportional.

We have from (3),

\begin{eqnarray}
s = \frac{tq + qr + rt - 2tqr}{2qr - 1}, \nonumber \\
\therefore 1-s = \frac{qr - tq - rt + 2tqr - 1}{2qr - 1}.
\end{eqnarray}

And fully developing the second member, we find
\begin{eqnarray}
1 - s = 0tqr + 2tq(1 - r) + 2tr(1 - q) + t(1 - q) (1 - r) \nonumber \\
+ 0(1 - t)qr + (1 - t)q(1 -r) + (1 - t)r(1 - q)\nonumber \\
+ (1 - t)(1 - q) (1 - r).
\end{eqnarray}

In the above development two of the terms have the coefficient
2, these must be equated to 0 by the Rule, then those terms
whose coefficients are 0 being rejected, we have
\begin{eqnarray}
1-s=t(1-q)(1-r)+(1-t)q(1-r)+(1-t)r(1-q)\nonumber\\
+(1-t)(1-q)(1-r); \\
tq(1-r)=0; \\
tr(1-q)=0;
\end{eqnarray}
the direct interpretation of which is

1st. \textit{Dissimilar figures consist of all triangles which have not their
corresponding angles equal and sides proportional, and of all figures
not being triangles which have either their angles equal, and sides not
proportional, or their corresponding sides proportional, and angles
not equal, or neither their corresponding angles equal nor corresponding
sides proportional.}

2nd. \textit{There are no triangles whose corresponding angles are equal.
and sides not proportional.}

3rd. \textit{There are no triangles whose corresponding sides are proportional and angles not equal.}

11. Such are the immediate interpretations of the final equation.
It is seen, in accordance with the general theory, that in
deducing a description of a particular class of objects, viz., dissimilar figures,
in terms of certain other elements of the original
premises, we obtain also the independent relations which exist
among those elements in virtue of the same premises. And that
this is not superfluous information, even as respects the immediate
object of inquiry, may easily be shown. For example, the
independent relations may always be made use of to reduce, if it
be thought desirable, to a briefer form, the expression of that
relation which is directly sought. Thus if we write (7) in the
form
\[
0=tq(l-r),
\]
and add it to (6), we get, since
\begin{eqnarray*}
t(1-q)(1-r)+tq(1-r)=t(1-r),\\
1-s=t(1-r)+(1-t)q(1-r)+(1-t)r(1-q)\\
+(1-t)(1-q)(1-r),
\end{eqnarray*}

which, on interpretation, would give for the first term of the description
of dissimilar figures, ``Triangles whose corresponding
sides are not proportional,'' instead of the fuller description originally
obtained. A regard to convenience must always determine
the propriety of such reduction.

12. A reduction which is always advantageous (VII. 15) consists
in collecting the terms of the immediate description sought,
as of the second member of (5) or (6), into as few groups as
possible. Thus the third and fourth terms of the second member
of (6) produce by addition the single term $(1 - t)(1 - q)$.
If this reduction be combined with the last, we have
\[
1 - s = t (1 - r) + (1 - t) q (1 - r) + (1 - t) (1 - q),
\]
the interpretation of which is

\textit{Dissimilar figures consist of all triangles whose corresponding
sides are not proportional, and all figures not being triangles which
have either their corresponding angles unequal, or their corresponding
angles equal, but sides not proportional.}

The fulness of the general solution is therefore not a superfluity.
While it gives us all the information that we seek, it
provides us also with the means of expressing that information
in the mode that is most advantageous.

13. Another observation, illustrative of a principle which has
already been stated, remains to be made. Two of the terms in
the full development of $1 - s$ in (5) have 2 for their coefficients,
instead of $\frac{1}{0}$. It will hereafter be shown that this circumstance
indicates that the two premises were not independent. To verify
this, let us resume the equations of the premises in their reduced
forms, viz.,
\begin{eqnarray*}
s(1-qr)+qr(1-s) &= 0,\\
tq(1-r)+tr(1-q) &= 0.
\end{eqnarray*}
Now if the first members of these equations have any common
constituents, they will appear on multiplying the equations together.
If we do this we obtain
\[
stq(1 - r) + str(l - q) = 0.
\]

Whence there will result
\[
stq (1 - r) = 0\textrm{, }str (1 - q) = 0,
\]
these being equations which are deducible from either of the
primitive ones. Their interpretations are---

\textit{Similar triangles which have their corresponding angles equal
have their corresponding sides proportional.}

\textit{Similar triangles which have their corresponding sides proportional
have their corresponding angles equal. }

And these conclusions are equally deducible from either premiss \textit{singly}. In this respect, according to the definitions laid
down, the premises are not independent.

14. Let us, in conclusion, resume the problem discussed in
illustration of the first method of this chapter, and endeavour to
ascertain, by the present method, what may be concluded from
the presence of the property $C$, with reference to the properties
$A$ and $B$.

We found on eliminating the symbols $v$ the following equations,
viz.:
\setcounter{equation}{0}
\begin{eqnarray}
xy(wz + \bar{w}\bar{z}) = 0, \\
yz (x\bar{w} + \bar{x}w) = 0,\\
\bar{x}\bar{y} = \bar{w}\bar{z}.
\end{eqnarray}

From these we are to eliminate $w$ and determine $z$. Now (1)
and (2) already satisfy the condition $V(l - V) = 0$. The third
equation gives, on bringing the terms to the first side, and
squaring
\begin{equation}
\bar{x}\bar{y} (1 -\bar{w}\bar{z}) + \bar{w}\bar{z}(1 - \bar{x}\bar{y}) = 0.
\end{equation}
Adding (1) (2) and (4) together, we have
\[
xy(wz + \bar{w}\bar{z}) + yz(x\bar{w} +\bar{x}w)+\bar{x}\bar{y}(1 - \bar{w}\bar{z}) + \bar{w}\bar{z}(l - \bar{x}\bar{y}) = 0.
 \]
Eliminating w, we get
\[
(xyz + yz\bar{x} + \bar{x}\bar{y}) \{xy\bar{z} + yzx + \bar{x}\bar{y}z + \bar{z}(l - \bar{x}\bar{y})\} = 0.
\]
Now, on multiplying the terms in the second factor by those in
the first successively, observing that
\[
x\bar{x} = 0\textrm{, }y\bar{y} = 0\textrm{, }z\bar{z} = 0,
\]

nearly all disappear, and we have only left
\begin{equation}
xyz + \bar{x}\bar{y}z = 0;
\end{equation}
whence
\[
z = \frac{0}{xy+\bar{x}\bar{y}}
\]
\[
= 0xy + \frac{0}{0}x\bar{y} + \frac{0}{0}\bar{x}y + 0\bar{x}\bar{y}
\]
\[
= \frac{0}{0}x\bar{y} + \frac{0}{0}\bar{x}y,
\]
furnishing the interpretation. \textit{Wherever the property $C$ is found,
either the property $A$ or the property $B$ will be found with it, but
not both of them together.}

From the equation (5) we may readily deduce the result arrived
at in the previous investigation by the method of arbitrary
constant multipliers, as well as any other proposed forms of the
relation between $x$, $y$, and $z$; e. g. \textit{If the property $B$ is absent,
either $A$ and $C$ will be jointly present, or $C$ will be absent.} And
conversely, \textit{If $A$ and $C$ are jointly present, $B$ will be absent.}
The converse part of this conclusion is founded on the presence
of a term $xz$ with unity for its coefficient in the developed value
of $\bar{y}$.

\chapter[METHODS OF ABBREVIATION]
{\large ON CERTAIN METHODS OF ABBREVIATION.}

1. Though the three fundamental methods of development,
elimination, and reduction, established and illustrated in
the previous chapters, are sufficient for all the practical ends of
Logic, yet there are certain cases in which they admit, and especially
the method of elimination, of being simplified in an important
degree; and to these I wish to direct attention in the
present chapter. I shall first demonstrate some propositions in
which the principles of the above methods of abbreviation are
contained, and I shall afterwards apply them to particular examples.

Let us designate as class terms any terms which satisfy the
fundamental law $V(1-V) = 0$. Such terms will individually
be constituents; but, when occurring together, will not, as do
the terms of a development, necessarily involve the same symbols
in each. Thus $ax+bxy+cyz$ may be described as an expression
consisting of three class terms, $x$, $xy$, and $yz$, multiplied by the
coefficients $a$, $b$, $c$ respectively. The principle applied in the two
following Propositions, and which, in some instances, greatly
abbreviates the process of elimination, is that of the \textit{rejection of
superfluous class terms}; those being regarded as superfluous
which do not add to the constituents of the final result.

\begin{center}
\textsc{Proposition I.}
\end{center}

2. \textit{From any equation, $V = 0$, in which $V$ consists of a series of
class terms having positive coefficients, we are permitted to reject any
term which contains another term as a factor, and to change every
positive coefficient to unity.}

For the significance of this series of positive terms depends
only upon the number and nature of the constituents of its final
expansion, i.e. of its expansion with reference to all the symbols
which it involves, and not at all upon the actual values of the
coefficients (VI. 5). Now let $x$ be any term of the series, and
$xy$ any other term having $x$ as a factor. The expansion of $x$ with
reference to the symbols $x$ and $y$ will be
\[
xy + x \left(1 - y\right),
\]
and the expansion of the sum of the terms $x$ and $xy$ will be
\[
2xy + x \left(1 - y\right).
\]

But by what has been said, these expressions occurring in the
first member of an equation, of which the second member is 0,
and of which all the coefficients of the first member are positive,
are equivalent; since there must exist simply the two constituents
$xy$ and $x \left(1-y\right)$ in the final expansion, whence will simply arise
the resulting equations
\[
xy = 0\textrm{, } x \left(1-y\right) = 0.
\]
And, therefore, the aggregate of terms $x + xy$ may be replaced by
the single term $x$.

The same reasoning applies to all the cases contemplated in
the Proposition. Thus, if the term $x$ is repeated, the aggregate
$2x$ may be replaced by $x$, because under the circumstances the
equation $x = 0$ must appear in the final reduction.

\begin{center}
\textsc{Proposition II.}
\end{center}

3. \textit{Whenever in the process of elimination we have to multiply
together two factors, each consisting solely of positive terms, satisfying
the fundamental law of logical symbols, it is permitted to reject from
both factors any common term, or from either factor any term which
is divisible by a term in the other factor; provided always, that the
rejected term be added to the product of the resulting factors.}

In the enunciation of this Proposition, the word ``divisible''
is a term of convenience, used in the algebraic sense, in which $xy$
and $x \left(1 - y\right)$ are said to be divisible by $x$.

To render more clear the import of this Proposition, let it be
supposed that the factors to be multiplied together are $x + y + z$
and $x + yw + t$. It is then asserted, that from these two factors
We may reject the term $x$, and that from the second factor we
may reject the term $yw$, provided that these terms be transferred
to the final product. Thus, the resulting factors being $y+z$
and $t$, if to their product $yt+zt$ we add the terms $x$ and $yw$,
we have
\[
x+yw+yt+zt,
\]
as an expression equivalent to the product of the given factors
$x+y+z$ and $x+yw+t$; equivalent namely in the process of
elimination.

Let us consider, first, the case in which the two factors have
a common term $x$, and let us represent the factors by the expressions
$x+P$, $x+Q$, supposing $P$ in the one case and $Q$ in the
other to be the sum of the positive terms additional to $x$.

Now,
\begin{equation}
(x+P)(x+Q) = x+xP+xQ+PQ.
\end{equation}
But the process of elimination consists in multiplying certain
factors together, and equating the result to $0$. Either then the
second member of the above equation is to be equated to $0$, or it
is a factor of some expression which is to be equated to $0$.

If the former alternative be taken, then, by the last Proposition,
we are permitted to reject the terms $xP$ and $xQ$, inasmuch
as they are positive terms having another term $x$ as a factor.
The resulting expression is
\[
x+PQ,
\]
which is what we should obtain by rejecting $x$ from both factors,
and adding it to the product of the factors which remain.

Taking the second alternative, the only mode in which the
second member of (1) can affect the final result of elimination
must depend upon the number and nature of its constituents,
both which elements are unaffected by the rejection of the terms
$xP$ and $xQ$. For that development of $x$ includes all possible constituents
of which $x$ is a factor.

Consider finally the case in which one of the factors contains
a term, as $xy$, divisible by a term, $x$, in the other factor.

Let $x+P$ and $xy+Q$ be the factors. Now
\[
(x+P)(xy+Q) = xy+xQ+xyP+PQ.
\]
But by the reasoning of the last Proposition, the term $xyP$ may be
rejected as containing another positive term $xy$ as a factor, whence
we have

\begin{eqnarray*}
xy + xQ + PQ\\
= xy + (x + P) Q.
\end{eqnarray*}
But this expresses the rejection of the term $xy$ from the second
factor, and its transference to the final product. Wherefore the
Proposition is manifest.

\begin{center}
\textsc{Proposition III.}
\end{center}

4. \textit{If $t$ be any symbol which is retained in the final result of the
elimination of any other symbols from any system of equations, the result
of such elimination may be expressed in the form
\[
Et + E \left(1-t\right)=0,
\]
in which $E$ is formed by making in the proposed system $t = 1$, and eliminating
the same other symbols; and $E^{\prime}$ by making in the proposed
system $t = 0$, and eliminating the same other symbols.}

For let $\phi\left(t\right) = 0$ represent the final result of elimination.
Expanding this equation, we have
\[
\phi\left(1\right)t + \phi\left(0\right)\left(1-t\right)=0.
\]
Now by whatever process we deduce the function $\phi\left(t\right)$ from the
proposed system of equations, by the same process should we deduce
$\phi\left(1\right)$, if in those equations $t$ were changed into $1$; and by
the same process should we deduce $\phi\left(0\right)$, if in the same equations
$t$ were changed into $0$. Whence the truth of the proposition is
manifest.

5. Of the three propositions last proved, it may be remarked,
that though quite unessential to the strict development or application
of the general theory, they yet accomplish important ends
of a practical nature. By Prop. 1 we can simplify the results
of addition; by Prop. 2 we can simplify those of multiplication;
and by Prop. 3 we can break up any tedious process of elimination
into two distinct processes, which will in general be of a
much less complex character. This method will be very frequently
adopted, when the final object of inquiry is the determination
of the value of $t$, in terms of the other symbols which remain
after the elimination is performed.

6. Ex. 1.---Aristotle, in the Nicomachean Ethics, Book II.
Cap. 3, having determined that actions are virtuous, not as possessing
in themselves a certain character, but as implying a certain
condition of mind in him who performs them, viz., that he
perform them knowingly, and with deliberate preference, and for
their own sakes, and upon fixed principles of conduct, proceeds
in the two following chapters to consider the question, whether
virtue is to be referred to the genus of Passions, or Faculties, or
Habits, together with some other connected points. He grounds
his investigation upon the following premises, from which, also,
he deduces the general doctrine and definition of moral virtue, of
which the remainder of the treatise forms an exposition.

\begin{center}
\textsc{premises}.
\end{center}

1. Virtue is either a passion (\textgreek{p'ajos}), or a faculty (\textgreek{d'unamis}),
or a habit (\textgreek{<'exis}).

2. Passions are not things according to which we are praised
or blamed, or in which we exercise deliberate preference.

3. Faculties are not things according to which we are praised
or blamed, and which are accompanied by deliberate preference.

4. Virtue is something according to which we are praised
or blamed, and which is accompanied by deliberate preference.

5. Whatever art or science makes its work to be in a good
state avoids extremes, and keeps the mean in view relative to
human nature (\textgreek{t`o m'eson \dots pr`os >hm}$\tilde{\alpha}\varsigma$)

6. Virtue is more exact and excellent than any art or science.
This is an argument \textit{\`a fortiori}. If science and true art shun
defect and extravagance alike, much more does virtue pursue the
undeviating line of moderation. If \textit{they} cause their work to be
in a good state, much more reason have to we to say that Virtue
causeth her peculiar work to be ``in a good state.'' Let the
final premiss be thus interpreted. Let us also pretermit all reference
to praise or blame, since the mention of these in the premises
accompanies only the mention of deliberate preference, and
this is an element which we purpose to retain. We may then
assume as our representative symbols-- \\
\begin{tabular}{c l}
&$v$ = virtue.\\
&$p$ = passions.\\
&$f$ = faculties.\\
&$h$ = habits.\\
&$d$ = things accompanied by deliberate preference.\\
&$g$ = things causing their work to be in a good state.\\
&$m$ = things keeping the mean in view relative to human nature.
\end{tabular}\\
Using, then, $q$ as an indefinite class symbol, our premises will be
expressed by the following equations:
\begin{eqnarray*}
v &=& q\left\{p\left(1-f\right)\left(1- h\right)%
   + f\left(1-p\right)\left(1-h\right)%
   + h\left(1-p\right)\left(1-f\right)\right\}. \\
p &=& q\left(1-d\right).\\
f &=& q\left(1-d\right).\\
v &=& qd.\\
g &=& qm.\\
v &=& qg.\\
\end{eqnarray*}
And separately eliminating from these the symbols $q$,
%** there is a typo in the equation below, unbalanced {  probably needs } right before = sign
%*** also need to restart equation numbering here
\setcounter{equation}{0}
\begin{eqnarray}
v \{1-p\left(1-f\right)\left(1-h\right)
  - f\left(1-p\right)\left(1-h\right)
  - h\left(1-p\right)\left(1-f\right)\}=0.\\
  pd=0.\\
  fd=0.\\
  v\left(1-d\right)=0.\\
  g\left(1-m\right)=0.\\
  v\left(1-g\right)=0.
\end{eqnarray}
We shall first eliminate from (2), (3), and (4) the symbol $d$, and
then determine $v$ in relation to $p$, $f$, and $h$. Now the addition of
(2), (3), and (4) gives
\[
\left(p+f\right)d+v\left(1-d\right)=0.
\]
From which, eliminating $d$ in the ordinary way, we find
\begin{equation}
\left(p+f\right)v=0.
\end{equation}
Adding this to (1), and determining $v$, we find
\[
v=\frac{0}{p+f+1-p\left(1-f\right)\left(1-h\right)-f\left(1-p\right)\left(1-h\right)-h\left(1-f\right)\left(1-p\right)}.
\]
Whence by development,
\[
v=\frac{0}{0}h\left(1-f\right)\left(1-p\right).
\]
The interpretation of this equation is: \textit{Virtue is a habit, and not
a faculty or a passion.}

Next, we will eliminate $f$, $p$, and $g$ from the original system
of equations, and then determine $v$ in relation to $h$, $d$, and $m$.
We will in this case eliminate $p$ and $f$ together. On addition of
(1), (2), and (3), we get
\begin{eqnarray*}
v\{1-p(1-f)(1-h)-f(1-p)(1-h)-h(1-p)(1-f)\} \\
                       +pd+fd=0.
\end{eqnarray*}

Developing this with reference to $p$ and $f$, we have
\begin{eqnarray*}
(v+2d)pf+(vh+d)p(1-f)+(vh+d)(1-p)f\\
                       +v(1-h)(1-p)(1-f)=0.
\end{eqnarray*}
Whence the result of elimination will be
\[
(v+2d)(vh+d)(vh+d)v(1-h)=0.
\]
Now $v+2d=v+d+d$, which by Prop. I. is reducible to $v + d$.
The product of this and the second factor is
\[
(v+ d)(vh+d),
\]
which by Prop. II. reduces to
$d+v(vh)$ or $vh+d$.

In like manner, this result, multiplied by the third factor, gives
simply $vh+d$. Lastly, this multiplied by the fourth factor,
$v(1-h)$, gives, as the final equation,
\begin{equation}
vd(l-h)=0
\end{equation}
It remains to eliminate $g$ from (5) and (6). The result is
\begin{equation}
v(1-m)=0
\end{equation}

Finally, the equations (4), (8), and (9) give on addition
\[
v(1-d)+vd(1-h)+v(1-m)=0
\]
from which we have
\[
v=\frac{0}{1-d+d(1-h)+1-m}.
\]
And the development of this result gives
\[
v=\frac{0}{0}hdm,
\]
$f$ which the interpretation is,--\textit{Virtue is a habit accompanied by
deliberate preference, and keeping in view the mean relative to
human nature.}

Properly speaking, this is not a definition, but a description
of virtue. It is \textit{all}, however, that can be correctly inferred from
the premises. Aristotle specially connects with it the necessity
of prudence, to determine the safe and middle line of action; and
there is no doubt that the ancient theories of virtue generally
partook more of an intellectual character than those (the theory
of utility excepted) which have most prevailed in modern days.
Virtue was regarded as consisting in the right state and habit of
the whole mind, rather than in the single supremacy of conscience
or the moral facility. And to some extent those theories
were undoubtedly right. For though unqualified obedience to
the dictates of conscience is an essential element of virtuous conduct,
yet the conformity of those dictates with those unchanging
principles of rectitude (\textgreek{a>i'wnia d'ikaia})%($\alpha\iota\acute{\omega}\nu\iota\alpha\;%\delta\acute{\iota}\kappa\alpha\iota\alpha$)
which are founded in, or
which rather are themselves the foundation of the constitution of
things, is another element. And generally this conformity, in
any high degree at least, is inconsistent with a state of ignorance
and mental hebetude. Reverting to the particular theory of
Aristotle, it will probably appear to most that it is of too negative
a character, and that the shunning of extremes does not
afford a sufficient scope for the expenditure of the nobler energies
of our being. Aristotle seems to have been imperfectly conscious
of this defect of his system, when in the opening of his seventh
book he spoke of an ``heroic virtue''\footnote{\textgreek{t`hn 'up`er
`hm<'ac 'aret`hn `hrw"ik'hn tina kai jeian}--\textsc{Nic. Eth.} Book vii.}
rising above the measure
of human nature.

7. I have already remarked (VIII. 1) that the theory of single
equations or propositions comprehends questions which cannot
be fully answered, except in connexion with the theory of
systems of equations. This remark is exemplified when it is
proposed to determine from a given single equation the relation,
not of some single elementary class, but of some compound class,
involving in its expression more than one element, in terms of
the remaining elements. The following particular example, and
the succeeding general problem, are of this nature.

Ex. 2.---Let us resume the symbolical expression of the definition
of wealth employed in Chap, VII., viz.,
\[
w = st \left\{p + r\left(l - p\right)\right\},
\]
wherein, as before, \\
\begin{tabular}{c l}
&$w$ = \textrm{wealth,}\\
&$s$ = \textrm{things limited in supply,}\\
&$t$ = \textrm{things transferable,}\\
&$p$ = \textrm{things productive of pleasure,}\\
&$r$ = \textrm{things preventive of pain;}
\end{tabular}\\
and suppose it required to determine hence the relation of things
transferable and productive of pleasure, to the other elements of
the definition, viz., wealth, things limited in supply, and things
preventive of pain.

The expression for things transferable and productive of pleasure
is $tp$. Let us represent this by a new symbol $y$. We have
then the equations
\begin{eqnarray*}
w &= st\left\{p + r\left(1-p\right)\right\},\\
y &= tp,
\end{eqnarray*}
from which, if we eliminate $t$ and $p$, we may determine $y$ as a
function of $w$, $s$, and $r$. The result interpreted will give the relation
sought.

Bringing the terms of these equations to the first side, we
have %*** This equation restarts from 3 for no apparent reason ***
\setcounter{equation}{2}
\begin{eqnarray}
w - stp - str \left(1 - p\right) = 0.\nonumber\\
y - tp = 0.
\end{eqnarray}
And adding the squares of these equations together,
\begin{equation}
w + stp + str \left(1-p\right) - 2wstp - 2wstr\left(1-p\right) + y + tp - 2ytp = 0.
\end{equation}
Developing the first member with respect to $t$ and $p$, in order to
eliminate those symbols, we have
\begin{eqnarray}
\left(w+s-2ws+1-y\right)tp + \left(w+sr-2wsr+y\right)t\left(1-p\right)\nonumber\\
+\left(w+y\right)\left(1-t\right)p+\left(w+y\right)\left(1-t\right)\left(1-p\right);
\end{eqnarray}
and the result of the elimination of $t$ and $p$ will be obtained by
equating to $0$ the product of the four coefficients of
\[
tp, t\left(1-p\right), \left(1-t\right)p,\textrm{ and } \left(1-t\right)\left(1-p\right).
\]

Or, by Prop. 3, the result of the elimination of $t$ and $p$ from the
above equation will be of the form
\[
Ey+E^\prime\left(1-y\right),
\]
wherein $E$ is the result obtained by changing in the given equation
$y$ into $1$, and then eliminating $t$ and $p$; and $E^\prime$ the result
obtained by changing in the same equation $y$ into $0$, and then
eliminating $t$ and $p$. And the mode in each case of eliminating $t$
and $p$ is to multiply together the coefficients of the four constituents
$tp$, $t\left(1-p\right)$, \&c.

If we make $y=1$, the coefficients become--

1st. $w\left(1-s\right)+s\left(1-w\right)$

2nd. $1+w\left(1-sr\right)+s\left(1-w\right)r$, equivalent to 1 by Prop. I.

3rd and 4th. $1 + w$, equivalent to 1 by Prop. I.

Hence the value of $E$ will be
\[
w\left(1-s\right) + s\left(1-w\right).
\]
Again, in (5) making $y = 0$, we have for the coefficients--

1st. $1+w\left(1-s\right) + s\left(1-w\right)$, equivalent to 1.

2nd. $w\left(1-sr\right) + sr\left(1-w\right)$.

3rd and 4th. $w$.

The product of these coefficients gives
\[
E^\prime=w\left(1-sr\right).
\]

The equation from which $y$ is to be determined, therefore, is
\begin{eqnarray*}
\left\{w(1-s)+s(1-w)\right\}y + w(1-sr)(1-y)=0,\\
\therefore y=\frac{w(1-sr)}{w(1-sr)-w(1-s)-s(1-w)};
\end{eqnarray*}
and expanding the second member,
\begin{eqnarray*}
y&=\frac{0}{0}wsr+ws(1-r)+\frac{1}{0}w(1-s)r+\frac{1}{0}w(1-s)(1-r)\\
&\quad+0(1-w)sr+0(1-w)s(1-r)+\frac{0}{0}(1-w)(1-s)r\\
&\quad+\frac{0}{0}(1-w)(1-s)(1-r);
\end{eqnarray*}
whence reducing.

\begin{eqnarray}
y=ws(1-r) + \frac{0}{0}wsr + \frac{0}{0}(1-w)(1-s),\\
\textrm{with }w (1-s) = 0.
\end{eqnarray}

The interpretation of which is--

1st. \textit{Things transferable and productive of pleasure consist of
all wealth (limited in supply and) not preventive of pain, an
indefinite amount of wealth (limited in supply and) preventive of pain,
and an indefinite amount of what is not wealth and not limited in
supply.}

2nd. \textit{All wealth is limited in supply.}

I have in the above solution written in parentheses that part
of the full description which is implied by the accompanying
independent relation (7).

8. The following problem is of a more general nature, and
will furnish an easy practical rule for problems such as the last.

\begin{center}
\textsc{General Problem}.
\end{center}

\textit{Given any equation connecting the symbols $x$, $y..w$, $z..$}

\textit{Required to determine the logical expression of any class
expressed in any way by the symbols $x$, $y..$ in terms of the remaining
symbols, $w$, $z$, \&c.}

Let us confine ourselves to the case in which there are but
two symbols, $x$, $y$, and two symbols, $w$, $z$, a case sufficient to
determine the general Rule.

Let $V=0$ be the given equation, and let $\phi (x,y)$ represent
the class whose expression is to be determined.

Assume $t = \phi (x,y)$, then, from the above two equations, $x$
and $y$ are to be eliminated.

Now the equation $V=0$ may be expanded in the form
\setcounter{equation}{0}
\begin{equation}
Axy + Bx(1-y) + C(1-x)y + D(1-x) (1-y) = 0,
\end{equation}

$A$, $B$, $C$, and $D$ being functions of the symbols $w$ and $z$.

Again, as $\phi (x,y)$ represents a class or collection of things, it
must consist of a constituent, or series of constituents, whose
coefficients are 1.

Wherefore if the \textit{full} development of $\phi (x, y)$ be represented in the form
\[
axy+ bx(l - y) + c(l - x)y + d(1 - x) (1 - y),
\]
the coefficients $a$, $b$, $c$, $d$ must each be 1 or 0.

Now reducing the equation $t = \phi (x, y)$ by transposition and
squaring, to the form
\[
t{1-\phi(x,y)} + \phi(x,y)(1-t) = 0;
\]
and expanding with reference to $x$ and $y$, we get

\begin{eqnarray*}
{t(1 -a ) + a (1 - t)} xy + {t(1 - b) + b(1 - t)} x(1 - y) \\
+ {t(1-c) + c(1-t)} (1-x)y\\
+ {t(1-d) + d(1-t)} (1-x) (1-y) = 0;
\end{eqnarray*}

whence, adding this to (1), we have

\begin{eqnarray*}
{A + t(1-a) + a(1-t)}xy\\
+ {B + t(l-b) + b(l-t)} x(1-y) + \&c. = 0.
\end{eqnarray*}

Let the result of the elimination of $x$ and $y$ be of the form

\[
Et+E' (1-t)=0,
\]

then E will, by what has been said, be the reduced product of
what the coefficients of the above expansion become when $t = 1$ ,
and $E'$ the product of the same factors similarly reduced by the
condition $t = 0$.

Hence $E$ will be the reduced product
\[
(A+1-a)(B+1-b)(C+1-c)(D+1-d).
\]
Considering any factor of this expression, as $A + 1 - a$, we see
that when $a = 1$ it becomes A, and when $a = 0$ it becomes $1 + A$,
which reduces by Prop. I. to 1. Hence we may infer that E will
be the product of the coefficients of those constituents in the
development of $V$ whose coefficients in the development of $\phi (x, y)$
are 1.

Moreover $E'$ will be the reduced product
\[
(A+a)(B+b)(C+c)(D+d).
\]
Considering any one of these factors, as $A + a$, we see that this
becomes $A$ when $a = 0$, and reduces to 1 when $a = 1$ ; and so on
for the others. Hence $E$ will be the product of the coefficients
of those constituents in the development of $y$, whose coefficients
in the development $\phi (x, y)$ are 0. Viewing these cases together,
we may establish the following Rule:

9. \textit{To deduce from a logical equation the relation of any class
expressed by a given combination of the symbols $x$, $y$, \&c, to the
classes represented by any other symbols involved in the given
equation}.

\textsc{Rule}.--\textit{Expand the given equation with reference to the symbols $x$, $y$. Then form the equation
\[
Et + E'(1-t) = 0,
\]
in which $E$ is the product of the coefficients of all those constituents
in the above development, whose coefficients in the expression of the
given class are 1, and $E'$ the product of the coefficients of those constituents of the development whose coefficients in the expression of the
given class are 0. The value of $t$ deduced from the above equation
by solution and interpretation will be the expression required}.

\textsc{Note}.--\textit{Although in the demonstration of this Rule $V$ is
supposed to consist solely of positive terms, it may easily be shown that
this condition is unnecessary, and the Rule general, and that no
preparation of the given equation is really required}.

10. Ex. 3.--The same definition of wealth being given as in
Example 2, required an expression for \textit{things transferable, but not
productive of pleasure}, $t(1 - p)$, in terms of the other elements
represented by $w$, $s$, and $r$.

The equation
\[
w - stp - str (1 - p) = 0,
\]

gives, when squared,
\[
w + stp + str (1 - p) - 2wstp - 2wstr (1 - p) = 0;
\]
and developing the first member with respect to $t$ and $p$,
\begin{eqnarray*}
(w + s - 2ws) tp + (w + sr - 2wsr) t (1 - p) + w (1 -t)p\\
+ w(1-t)(1-p) = 0.
\end{eqnarray*}

The coefficients of which it is best to exhibit as in the following
equation;
\begin{eqnarray*}
{w(1-s)+s(1-w)}tp+{w(1-sr)+sr(1-w)}t(1-p)+w(1-t)p\\
+w(1-t)(1-p)=0
\end{eqnarray*}

Let the function $t (1 -p)$ to be determined, be represented by $z$;
then the full development of $z$ in respect of $t$ and $p$ is
\[
z = 0 tp + t (1 - p) + 0(1 - t) p + 0(1 - t) (1 - p).
\]
Hence, by the last problem, we have
\begin{eqnarray*}
Ez + E'(1-z)=0;\\
\textrm{where }E = w (1 - sr) + sr (1 - w);\\
E'= {w(1 - s) + s (1 - w) } \times w \times w = w(1 - s) ; \\
\therefore {w(1 - sr) + sr(1 - w)} z + w(1 - s) (1 - z) =0.
\end{eqnarray*}
Hence,
\begin{eqnarray*}
z=\frac{w(1-s)}{2wsr-ws-sr}\\
= \frac{0}{0}wsr + 0ws(1 - r) + \frac{1}{0}w(1 - s)r + \frac{1}{0}w(1 - s)(1 - r),\\
+ 0(1 - w)sr + \frac{0}{0}(1 - w)s(1 - r) + \frac{0}{0}(1 - w)(1 - s)r\\
+ \frac{0}{0}(1 - w)(1 - s)(1 - r). \\
\textrm{Or, } z = \frac{0}{0} wsr + \frac{0}{0}(1-w)s(1-r) + \frac{0}{0} (1- w)(1 - s),\\
\textrm{with }w (1 - s) = 0.
\end{eqnarray*}

Hence, \textit{Things transferable and not productive of pleasure are
either wealth (limited in supply and preventive of pain); or things
which are not wealth, but limited in supply and not preventive of
pain; or things which are not wealth, and are unlimited in supply.}

The following results, deduced in a similar manner, will be
easily verified:

\textit{Things limited in supply and productive of pleasure which are
not wealth,--are intransferable.}

\textit{Wealth that is not productive of pleasure is transferable, limited
in supply, and preventive of pain.}

\textit{Things limited in supply which are either wealth, or are productive of pleasure, but not both,--are either transferable and
preventive of pain, or intransferable.}

11. From the domain of natural history a large number of
curious examples might be selected. I do not, however, conceive
that such applications would possess any independent value.
They would, for instance, throw no light upon the true
principles of classification in the science of zoology. For the
discovery of these, some basis of positive knowledge is requisite,--some
acquaintance with organic structure, with teleological adaptation;
and this is a species of knowledge which can only be derived
from the use of external means of observation and analysis.
Taking, however, any collection of propositions in natural history,
a great number of logical problems present themselves,
without regard to the system of classification adopted. Perhaps
in forming such examples, it is better to avoid, as superfluous,
the mention of that property of a class or species which is immediately
suggested by its name, e.g. the ring-structure in the
annelida, a class of animals including the earth-worm and the
leech.

Ex. 4.--1. The annelida are soft-bodied, and either naked or
enclosed in a tube.

2. The annelida consist of all invertebrate animals having
red blood in a double system of circulating vessels. \\

\begin{tabular}{c l}
\textrm{Assume }&$a$ = \textrm{annelida;} \\
       &$s$ = \textrm{soft-bodied animals;} \\
       &$n$ = \textrm{naked;} \\
       &$t$ = \textrm{enclosed in a tube;} \\
       &$i$ = \textrm{invertebrate;} \\
       &$r$ = \textrm{having red blood, \&c.}
\end{tabular}\\

Then the propositions given will be expressed by the equations
\setcounter{equation}{0}
\begin{eqnarray}
a = vs {n(1- t) + t (1 - n)};  \\
a = ir;
\end{eqnarray}
to which we may add the implied condition,

\begin{equation}
nt = 0.
\end{equation}

On eliminating $v$, and reducing the system to a single equation,
we have

\begin{equation}
a [1-sn(1-t)-st(1-n)] + a(1-ir) + ir (1-a) + nt = 0.
\end{equation}

Suppose that we wish to obtain the relation in which soft-bodied
animals enclosed in tubes arc placed (by virtue of the
premises) with respect to the following elements, viz., the
possession of red blood, of an external covering, and of a vertebral
column.

We must first eliminate $a$. The result is

\[
ir {1 - sn(1-t) - st(1-n)} + nt=0.
\]

Then (IX. 9) developing with respect to $s$ and $t$, and reducing
the first coefficient by Prop. 1, we have

\begin{equation}
nst + ir(1-n)s(1-t) + (ir+n)(1-s)t + ir(1-s)(1-t) = 0.
\end{equation}

Hence, if $st=w$, we find

\[
nw + ir(1-n) \times (ir+n) \times ir(1-w) = 0;
\]

or,

\begin{eqnarray*}
nw + ir(1-n)(1-w) = 0; \\
\therefore w = \frac{ ir(1-n)}{ir(1-n)-n} \\
= 0irn + ir(1-n) + 0i(1-r)n + \frac{0}{0}i(1-r)(1-n) \\
+ 0(1-i)rn + \frac{0}{0}(1-i)r(1-n) + 0(1-i)(1-r)n \\
+ \frac{0}{0}(1-i)(1-r)(1-n); \\
\textrm{or, } w = ir(1-n) + \frac{0}{0}i(1-r)(1-n) + \frac{0}{0}(1-i)(1-n).
\end{eqnarray*}

Hence, \textit{soft-bodied animals enclosed in tubes consist of all
invertebrate animals having red blood and not naked, and an
indefinite remainder of invertebrate animals not having red blood and
not naked, and of vertebrate animals which are not naked.}

And in an exactly similar manner, the following reduced
equations, the interpretation of which is left to the reader, have been
deduced from the development (5).

\begin{eqnarray*}
s(1-t) = irn + \frac{0}{0}i(1-n) + \frac{0}{0}(1-i) \\
(1-s) t = \frac{0}{0}(1-i)r(1-n) + \frac{0}{0}(1-r)(1-n) \\
(1-s)(1-t) = \frac{0}{0}i(1-r) + \frac{0}{0}(1-i).
\end{eqnarray*}

In none of the above examples has it been my object to exhibit
in any special manner the power of the method. That,
I conceive, can only be fully displayed in connexion with the
mathematical theory of probabilities. I would, however, suggest
to any who may be desirous of forming a correct opinion upon
this point, that they examine by the rules of ordinary logic the
following problem, \textit{before} inspecting its solution; remembering
at the same time, that whatever complexity it possesses might
be multiplied indefinitely, with no other effect than to render its
solution by the method of this work more operose, but not less
certainly attainable.

Ex. \textit{5}. Let the observation of a class of natural productions
be supposed to have led to the following general results.

1st, That in whichsoever of these productions the properties
$A$ and $C$ are missing, the property $E$ is found, together with one
of the properties $B$ and $D$, but not with both.

2nd, That wherever the properties $A$ and $D$ are found while
$E$ is missing, the properties $B$ and $C$ will either both be found,
or both be missing.

3rd, That wherever the property $A$ is found in conjunction
with either $B$ or $E$, or both of them, there either the property
$C$ or the property $D$ will be found, but not both of them. And
conversely, wherever the property $C$ or $D$ is found singly, there
the property $A$ will be found in conjunction with either $B$ or $E$,
or both of them.

Let it then be required to ascertain, first, what in any particular
instance may be concluded from the ascertained presence of
the property $A$, with reference to the properties $B$, $C$, and $D$;
also whether any relations exist independently among the properties
$B$, $C$, and $D$. Secondly, what may be concluded in like
manner respecting the property $B$, and the properties $A$, $C$,
and $D$.

It will be observed, that in each of the three data, the information
conveyed respecting the properties $A$, $B$, $C$, and $D$, is complicated
with another element, $E$, about which we desire to say
nothing in our conclusion. It will hence be requisite to eliminate
the symbol representing the property E from the system of equations,
by which the given propositions will be expressed.

Let us represent the property $A$ by $x$, $B$ by $y$, $C$ by $z$, $D$ by
$w$, $E$ by $v$. The data are
\setcounter{equation}{0}
\begin{eqnarray}
\bar{x}\bar{z} = qv(y\bar{w} + w\bar{y});\\
\bar{v}xw = q (yz + \bar{y}\bar{z});\\
xy + xv\bar{y} = w\bar{z} + z\bar{w};
\end{eqnarray}

$\bar{x}$ standing for $1 - x$, \&c., and $q$ being an indefinite class symbol.
Eliminating $q$ separately from the first and second equations,
and adding the results to the third equation reduced by (5),
Chap.VIII., we get

\begin{eqnarray}
\lefteqn{\bar{x}\bar{z} (1 - vy\bar{w} - vw\bar{y}) + \bar{v}xw (y\bar{z} + z\bar{y}) + (xy + xv\bar{y}) (wz + \bar{w}\bar{z})} \nonumber\\
 & & {} + (w\bar{z} + z\bar{w}) (1 -xy- xv\bar{y}) = 0.
\end{eqnarray}

From this equation $v$ must be eliminated, and the value of $x$
determined from the result. For effecting this object, it will
be convenient to employ the method of Prop. 3 of the present
chapter.

Let then the result of elimination be represented by the
equation

\[
Ex + E' (l-x) = 0.
\]

To find $E$ make $x = 1$ in the first member of (4), we find
\[
\bar{v}w (y\bar{z} + z\bar{y}) + (y + v\bar{y}) (wz + \bar{w}\bar{z})+ (w\bar{z} + z\bar{w}) \bar{v}\bar{y}.
\]

Eliminating $v$, we have

\[
(wz + \bar{w}\bar{z}) \left\{w(y\bar{z} + z\bar{y}) +y(wz+ \bar{w}\bar{z}) + \bar{y} (w\bar{z} +z\bar{w})\right\};
\]

which, on actual multiplication, in accordance with the conditions
$w\bar{w} = 0$, $z\bar{z} = 0$, \&c., gives

\[
E = wz + y\bar{w}\bar{z}
\]

Next, to find $E'$ make $x = 0$ in (4), we have

\[
z (1 - vy\bar{w} - v\bar{y}w) + w\bar{z} + z\bar{w}.
\]

whence, eliminating $v$, and reducing the result by Propositions
1 and 2, we find

\[
E' = w\bar{z} + z\bar{w} + \bar{y}\bar{w}\bar{z};
\]

and, therefore, finally we have

\begin{equation}
(wz + y\bar{w}\bar{z})x + (w\bar{z} + z\bar{w} + \bar{y}\bar{w}\bar{z})\bar{x} = 0;
\end{equation}

from which

\[
x =
\frac{w\bar{z} + z\bar{w} + \bar{y}\bar{w}\bar{z}}
{w\bar{z} + z\bar{w} + \bar{y}\bar{w}\bar{z} - wz - y\bar{w}\bar{z}}
\]

wherefore, by development,

\begin{eqnarray*}
x = 0yzw + yz\bar{w} + y\bar{z}w + 0y\bar{z}\bar{w}\\
  + 0\bar{y}zw + \bar{y}z\bar{w} + \bar{y}\bar{z}w + \bar{y}\bar{z}\bar{x};
\end{eqnarray*}

or, collecting the terms in vertical columns,

\begin{equation}
x = z\bar{w} + \bar{z}w + \bar{y}\bar{z}\bar{w};
\end{equation}

the interpretation of which is--

\textit{In whatever substances the property $A$ is found, there will also
be found either the property $C$ or the property $D$, but not both, or
else the properties $B$, $C$, and $D$, will all be wanting. And
conversely, where either the property $C$ or the property $D$ is found
singly, or the properties $B$, $C$, and $D$, are together missing, there
the property $A$ will be found.}

It also appears that there is no independent relation among
the properties $B$, $C$, and $D$.

Secondly, we are to find $y$. Now developing (5) with respect
to that symbol,

\[
(xwz + x\bar{w}\bar{z} + \bar{x}w\bar{z} + \bar{x}z\bar{w})y + (xwz + \bar{x}w\bar{z} + \bar{x}z\bar{w} + \bar{x}\bar{z}\bar{w})\bar{y} = 0;
\]

whence, proceeding as before,

\begin{equation}
y = \bar{x}\bar{w}\bar{z} + \frac{0}{0}(\bar{x}wz + xw\bar{z} + xz\bar{w}),
\end{equation}

\begin{eqnarray}
xzw &= 0;\\
\bar{x}z\bar{z}w &= 0;\\
\bar{x}z\bar{w} &= 0;
\end{eqnarray}

From (10) reduced by solution to the form

\[
\bar{x}z = \frac{0}{0}w;
\]

we have the independent relation,--\textit{If the property $A$ is absent
and $C$ present, $D$ is present.} Again, by addition and solution (8)
and (9) give

\[
xz + \bar{x}\bar{z} = \frac{0}{0} \bar{w}.
\]

Whence we have for the general solution and the remaining
independent relation:

1st. \textit{If the property $B$ be present in one of the productions, either the properties $A$, $C$, and $D$, are all absent, or some one alone of them
is absent.} And conversely, \textit{if they are all absent it may be
concluded that the property A is present} (7).

2nd. \textit{If $A$ and $C$ are both present or both absent, $D$ will be absent,
quite independently of the presence or absence of $B$} (8) and (9).

I have not attempted to verify these conclusions.

\chapter[CONDITIONS OF A PERFECT METHOD]
{\large OF THE CONDITIONS OF A PERFECT METHOD.}

1. The subject of Primary Propositions has been discussed at
length, and we are about to enter upon the consideration
of Secondary Propositions. The interval of transition between
these two great divisions of the science of Logic may afford a fit
occasion for us to pause, and while reviewing some of the past
steps of our progress, to inquire what it is that in a subject like
that with which we have been occupied constitutes perfection of
method. I do not here speak of that perfection only which
consists in power, but of that also which is founded in the conception
of what is fit and beautiful. It is probable that a careful analysis
of this question would conduct us to some such conclusion as the
following, viz., that a perfect method should not only be an efficient
one, as respects the accomplishment of the objects for which
it is designed, but should in all its parts and processes manifest
a certain unity and harmony. This conception would be most
fully realized if even the very forms of the method were suggestive
of the fundamental principles, and if possible of the one fundamental
principle, upon which they are founded. In applying
these considerations to the science of Reasoning, it may be well
to extend our view beyond the mere analytical processes, and to
inquire what is best as respects not only the mode or form of
deduction, but also the system of data or premises from which
the deduction is to be made.

2. As respects mere power, there is no doubt that the first
of the methods developed in Chapter VIII. is, within its proper
sphere, a perfect one. The introduction of arbitrary constants
makes us independent of the forms of the premises, as well as of
any conditions among the equations by which they are represented.
But it seems to introduce a foreign element, and while
it is a more laborious, it is also a less elegant form of solution
than the second method of reduction demonstrated in the same
chapter. There are, however, conditions under which the latter
method assumes a more perfect form than it otherwise bears. To
make the one fundamental condition expressed by the equation

\[
x(1-x) = 0,
\]

the universal type of form, would give a unity of character to
both processes and results, which would not else be attainable.
Were brevity or convenience the only valuable quality of a method,
no advantage would flow from the adoption of such a principle.
For to impose upon every step of a solution the character
above described, would involve in some instances no slight labour
of preliminary reduction. But it is still interesting to know
that this can be done, and it is even of some importance to be
acquainted with the conditions under which such a form of solution
would spontaneously present itself. Some of these points
will be considered in the present chapter.

\begin{center}
\textsc{Proposition I.}
\end{center}

3. \textit{To reduce any equation among logical symbols to the form
$V=0$, in which $V$ satisfies the law of duality,}

\[
V(1 - V) = 0.
\]

It is shown in Chap. V. Prop. 4, that the above condition is
satisfied whenever $V$ is the sum of a series of constituents. And
it is evident from Prop. 2, Chap. VI. that all equations are equivalent
which, when reduced by transposition to the form $V=0$,
produce, by development of the first member, the same series of
constituents with coefficients which do not vanish; the particular
numerical values of those coefficients being immaterial.

\textit{Hence the object of this Proposition may always be accomplished
by bringing all the terms of an equation to the first side,
fully expanding that member, and changing in the result all the coefficients
which do not vanish into unity, except such as have already
that value.}

But as the development of functions containing many symbols
conducts us to expressions inconvenient from their great
length, it is desirable to show how, in the only cases which do
practically offer themselves to our notice, this source of
complexity may be avoided.

The great primary forms of equations have already been discussed
in Chapter VIII. They are--

\begin{eqnarray*}
X &=& vY, \\
X &=& Y, \\
vX &=& vY.
\end{eqnarray*}

Whenever the conditions $X(1-X) = 0, Y(1-Y) = 0$, are
satisfied, we have seen that the two first of the above equations
conduct us to the forms

\begin{eqnarray}
X(1-Y) &=& 0,\\
X(1-Y) &+& Y(1-X) = 0;
\end{eqnarray}

and under the same circumstances it may be shown that the last
of them gives

\begin{equation}
v( X(1-Y) + Y(1-X) ) = 0;
\end{equation}

all which results obviously satisfy, in their first members, the
condition

\[
V(1-V) = 0.
\]

Now as the above are the forms and conditions under which the
equations of a logical system properly expressed do actually
present themselves, it is always possible to reduce them by the
above method into subjection to the law required. Though,
however, the separate equations may thus satisfy the law, their
equivalent sum (VIII. 4) may not do so, and it remains to
show how upon it also the requisite condition may be imposed.

Let us then represent the equation formed by adding the
several reduced equations of the system together, in the form

\begin{equation}
v + v' + v'', \&c. = 0,
\end{equation}

this equation being singly equivalent to the system from which
it was obtained. We suppose $v, v', v''$, \&c. to be class terms
(IX. 1) satisfying the conditions

\[
v(1-v) = 0, v'(1-v') = 0, \&c.
\]

Now the full interpretation of (4) would be found by developing
the first member with respect to all the elementary symbols
$x$, $y$, \&c. which it contains, and equating to 0 all the constituents
whose coefficients do not vanish; in other words, all the constituents
which are found in either $v$, $v'$, $v''$, \&c. But those constituents
consist of--1st, such as are found in $v$; 2nd, such as are
not found in $v$, but are found in $v'$; 3rd, such as are neither found
in $v$ nor $v'$, but are found in $v''$, and so on. Hence they will be
such as are found in the expression
\begin{equation}
v + (1 - v)v' + (1 - v)(1 - v)v''+ \&c.,
\end{equation}
an expression in which no constituents are repeated, and which
obviously satisfies the law $V(1-V)=0$.

Thus if we had the expression
\[
(1 - t) + v + (1 - z) + tzw,
\]
in which the terms $1-t$, $1-z$ are bracketed to indicate that they
are to be taken as single class terms, we should, in accordance
with (5), reduce it to an expression satisfying the condition
$V(1-V)=0$, by multiplying all the terms after the first by $t$,
then all after the second by $1-v$; lastly, the term which remains
after the third by $z$; the result being
\begin{equation}
1 - t + tv + t(1 - v)(1 - z) + t(1 - v)zw.
\end{equation}

4. All logical equations then are reducible to the form $V=0$,
$V$ satisfying the law of duality. But it would obviously be a
higher degree of perfection if equations always presented themselves
in such a form, without preparation of any kind, and not
only exhibited this form in their original statement, but retained
it unimpaired after those additions which are necessary in order
to reduce systems of equations to single equivalent forms. That
they do not spontaneously present this feature is not properly
attributable to defect of method, but is a consequence of the fact
that our premises are not always complete, and accurate, and independent.
They are not complete when they involve material
(as distinguished from formal) relations, which are not expressed.
They are not accurate when they imply relations which are not
intended. But setting aside these points, with which, in the
present instance, we are less concerned, let it be considered in
what sense they may fail of being independent.

5. A system of propositions may be termed independent,
when it is not possible to deduce from any portion of the system
a conclusion deducible from any other portion of it. Supposing
the equations representing those propositions all reduced to the
form
\[
V=0,
\]
then the above condition implies that no constituent which can
be made to appear in the development of a particular function $V$
of the system, can be made to appear in the development of any
other function $V'$ of the same system. When this condition is
not satisfied, the equations of the system are not independent.
This may happen in various cases. Let all the equations satisfy
in their first members the law of duality, then if there appears a
positive term $x$ in the expansion of one equation, and a term $xy$
in that of another, the equations are not independent, for the
term $x$ is further developable into $xy + x ( 1 - y)$, and the equation
\[
xy=0
\]
is thus involved in both the equations of the system. Again, let
a term $xy$ appear in one equation, and a term $xz$ in another.
Both these may be developed so as to give the common constituent
$xyz$. And other cases may easily be imagined in which
premises which appear at first sight to be quite independent are
not really so. Whenever equations of the form $V = 0$ are thus
not truly independent, though individually they may satisfy the
law of duality,
\[
V(1 - V) = 0,
\]
the equivalent equation obtained by adding them together will
not satisfy that condition, unless sufficient reductions by the method
of the present chapter have been performed. When, on
the other hand, the equations of a system both satisfy the above
law, and are independent of each other, their sum will also satisfy
the same law. I have dwelt upon these points at greater
length than would otherwise have been necessary, because it appears
to me to be important to endeavour to form to ourselves,
and to keep before us in all our investigations, the pattern of an
ideal perfection,---the object and the guide of future efforts. In
the present class of inquiries the chief aim of improvement of method
should be to facilitate, as far as is consistent with brevity,
the transformation of equations, so as to make the fundamental
condition above adverted to universal.

In connexion with this subject the following Propositions are
deserving of attention.

\begin{center}
\textsc{Proposition II.}
\end{center}

\textit{If the first member of any equation $V = 0$ satisfy the condition
$V(1-V) = 0$, and if the expression of any symbol $t$ of that equation
be determined as a developed function of the other symbols, the
coefficients of the expansion can only assume the forms $1$, $0$, $\displaystyle\frac{0}{0}$, $\displaystyle\frac{1}{0}$.}

For if the equation be expanded with reference to $t$, we obtain
as the result,
\setcounter{equation}{0}
\begin{equation}
Et + E'(1 - t),
\end{equation}
$E$ and $E'$ being what $V$ becomes when $t$ is successively changed
therein into $1$ and $0$. Hence $E$ and $E'$ will themselves satisfy
the conditions
\begin{equation}
E(1 - E) = 0,\quad E'(1 - E') = 0.
\end{equation}
Now (1) gives
\[
t = \frac{E'}{E' - E},
\]
the second member of which is to be expanded as a function of
the remaining symbols. It is evident that the only numerical
values which $E$ and $E'$ can receive in the calculation of the coefficients
will be $1$ and $0$. The following cases alone can therefore
arise:

1st. $E' = 1$, $E = 1$, then $\displaystyle\frac{E'}{E'-E} = \frac{1}{0}$.

2nd. $E' = 1$, $E = 0$, then $\displaystyle\frac{E'}{E'-E} = 1$.

3rd. $E' = 0$, $E = 1$, then $\displaystyle\frac{E'}{E'-E} = 0$.

4th. $E' = 0$, $E = 0$, then $\displaystyle\frac{E'}{E'-E} = \frac{0}{0}$.

Whence the truth of the Proposition is manifest.

6. It may be remarked that the forms $1$, $0$, and $\frac{0}{0}$ appear in
the solution of equations independently of any reference to the
condition $V(1 - V) = 0$. But it is not so with the coefficient $\frac{1}{0}$.
The terms to which this coefficient is attached when the above
condition is satisfied may receive any other value except the
three values $1$, $0$, and $\frac{0}{0}$, when that condition is not satisfied. It
is permitted, and it would conduce to uniformity, to change any
coefficient of a development not presenting itself in any of the
four forms referred to in this Proposition into $\frac{1}{0}$, regarding this
as the symbol proper to indicate that the coefficient to which it is
attached should be equated to $0$. This course I shall frequently
adopt.

\begin{center}
\textsc{Proposition III.}
\end{center}

7. \textit{The result of the elimination of any symbols $x$, $y$, \&c. from
an equation $V=0$, of which the first member identically satisfies
the law of duality,
\[
V(1-V) = 0,
\]
may be obtained by developing the given equation with reference to
the other symbols, and equating to $0$ the sum of those constituents
whose coefficients in the expansion are equal to unity.}

Suppose that the given equation $V = 0$ involves but three
symbols, $x$, $y$, and $t$, of which $x$ and $y$ are to be eliminated. Let
the development of the equation, with respect to $t$, be
\setcounter{equation}{0}
\begin{equation}
At + B(1-t)=0,
\end{equation}

$A$ and $B$ being free from the symbol $t$.

By Chap. IX. Prop. 3, the result of the elimination of $x$ and $y$
from the given equation will be of the form
\begin{equation}
Et+ E'(1 -t) = 0,
\end{equation}
in which $E$ is the result obtained by eliminating the symbols $x$
and $y$ from the equation $A = 0$, $E'$ the result obtained by eliminating
from the equation $B = 0$.

Now $A$ and $B$ must satisfy the condition

\[
A (1 - A) = 0, B(1 - B) = 0
\]

Hence $A$ (confining ourselves for the present to this coefficient)
will either be 0 or 1, or a constituent, or the sum of a part of the
constituents which involve the symbols $x$ and $y$. If $A = 0$ it is
evident that $E = 0$; if $A$ is a single constituent, or the sum of a
part of the constituents involving $x$ and $y$, $E$ will be 0. For the
\textit{full} development of $A$, with respect to $x$ and $y$, will contain terms
with vanishing coefficients, and $E$ is the product of all the coefficients.
Hence when $A = 1$, $E$ is equal to $A$, but in other cases
$E$ is equal to 0. Similarly, when $B = 1$, $E$ is equal to $B$, but in
other cases $E$ vanishes. Hence the expression (2) will consist of
that part, if any there be, of (1) in which the coefficients $A$, $B$
are unity. And this reasoning is general. Suppose, for instance,
that $V$ involved the symbols $x$, $y$, $z$, $t$, and that it were required
to eliminate $x$ and $y$. Then if the development of $V$, with reference to $z$ and $t$, were

\[
zt + xz(1 - t) + y (1 - z) t + (1 - z) (1 - t),
\]

the result sought would be

\[
zt + (1 - z) (1 - t) = 0,
\]

this being that portion of the development of which the coefficients are unity.

Hence, if from any system of equations we deduce a single
equivalent equation $V = 0$, $V$ satisfying the condition

\[
V(1 - V) = 0,
\]

the ordinary processes of elimination may be entirely dispensed
with, and the single process of development made to supply
their place.

8. It may be that there is no practical advantage in the method
thus pointed out, but it possesses a theoretical unity and
completeness which render it deserving of regard, and I shall accordingly
devote a future chapter (XIV.) to its illustration. The
progress of applied mathematics has presented other and signal
examples of the reduction of systems of problems or equations to
the dominion of some central but pervading law.

9. It is seen from what precedes that there is one class of
propositions to which all the special appliances of the above methods
of preparation are unnecessary. It is that which is characterized
by the following conditions:

First, That the propositions are of the ordinary kind, implied
by the use of the copula \textit{is} or \textit{are}, the predicates being particular.

Secondly, That the terms of the proposition are intelligible
without the supposition of any understood relation among the
elements which enter into the expression of those terms.

Thirdly, That the propositions are independent.

We may, if such speculation is not altogether vain, permit
ourselves to conjecture that these are the conditions which would
be obeyed in the employment of language as an instrument of
expression and of thought, by unerring beings, declaring simply
what they mean, without suppression on the one hand, and without
repetition on the other. Considered both in their relation
to the idea of a perfect language, and in their relation to the processes
of an exact method, these conditions are equally worthy
of the attention of the student.

\chapter[OF SECONDARY PROPOSITIONS]
{\large OF SECONDARY PROPOSITIONS, AND OF THE PRINCIPLES OF THEIR
SYMBOLICAL EXPRESSION.}


1. The doctrine has already been established in Chap. IV.,
that every logical proposition may be referred to one or
the other of two great classes, viz., Primary Propositions and
Secondary Propositions. The former of these classes has been
discussed in the preceding chapters of this work, and we are now
led to the consideration of Secondary Propositions, i.e. of Propositions
concerning, or relating to, other propositions regarded as
true or false. The investigation upon which we are entering will,
in its general order and progress, resemble that which we have already
conducted. The two inquiries differ as to the subjects of
thought which they recognise, not as to the formal and scientific
laws which they reveal, or the methods or processes which are
founded upon those laws. Probability would in some measure favour
the expectation of such a result. It consists with all that we
know of the uniformity of Nature, and all that we believe of the immutable
constancy of the Author of Nature, to suppose, that in the
mind, which has been endowed with such high capabilities, not
only for converse with surrounding scenes, but for the knowledge
of itself, and for reflection upon the laws of its own constitution,
there should exist a harmony and uniformity not less real than
that which the study of the physical sciences makes known to us.
Anticipations such as this are never to be made the primary rule
of our inquiries, nor are they in any degree to divert us from
those labours of patient research by which we ascertain what is
the actual constitution of things within the particular province
submitted to investigation. But when the grounds of resemblance
have been properly and independently determined, it is
not inconsistent, even with purely scientific ends, to make that
resemblance a subject of meditation, to trace its extent, and to
receive the intimations of truth, yet undiscovered, which it may
seem to us to convey. The necessity of a final appeal to fact is
not thus set aside, nor is the use of analogy extended beyond its
proper sphere,--the suggestion of relations which independent
inquiry must either verify or cause to be rejected.

2. \textit{Secondary Propositions are those which concern or relate to
Propositions considered as true or false.} The relations of \textit{things}
we express by primary propositions. But we are able to make
Propositions themselves also the subject of thought, and to express
our judgments concerning them. The expression of any
such judgment constitutes a secondary proposition. There exists
no proposition whatever of which a competent degree of knowledge
would not enable us to make one or the other of these two
assertions, viz., either that the proposition is true, or that it is
false; and each of these assertions is a secondary proposition. ``It
is true that the sun shines;'' ``It is not true that the planets
shine by their own light;'' are examples of this kind. In the
former example the Proposition ``The sun shines,'' is asserted to
be true. In the latter, the Proposition, ``The planets shine by
their own light,'' is asserted to be false. Secondary propositions
also include all judgments by which we express a relation or dependence
among propositions. To this class or division we may
refer conditional propositions, as, ``If the sun shine the day will
be fair.'' Also most disjunctive propositions, as, ``Either the sun
will shine, or the enterprise will be postponed.'' In the former
example we express the dependence of the truth of the Proposition,
``The day will be fair,'' upon the truth of the Proposition,
``The sun will shine.'' In the latter we express a relation between
the two Propositions, ``The sun will shine,'' ``The enterprise will
be postponed,'' implying that the truth of the one excludes the
truth of the other. To the same class of secondary propositions we
must also refer all those propositions which assert the simultaneous
truth or falsehood of propositions, as, ``It is not true both that
`the sun will shine' and that `the journey will be postponed.'~''
The elements of distinction which we have noticed may even be
blended together in the same secondary proposition. It may involve
both the disjunctive element expressed by \textit{either}, \textit{or}, and
the conditional element expressed by \textit{if}; in addition to which,
the connected propositions may themselves be of a compound
character. \textit{If} ``the sun shine,'' \textit{and} ``leisure permit,'' then \textit{either}
``the enterprise shall be commenced,'' \textit{or} ``some preliminary
step shall be taken.'' In this example a number of propositions
are connected together, not arbitrarily and unmeaningly, but in
such a manner as to express a \textit{definite} connexion between them,--a
connexion having reference to their respective truth or falsehood.
This combination, therefore, according to our definition, forms
a Secondary Proposition.

The theory of Secondary Propositions is deserving of attentive
study, as well on account of its varied applications, as
for that close and harmonious analogy, already referred to, which
it sustains with the theory of Primary Propositions. Upon each
of these points I desire to offer a few further observations.

3. I would in the first place remark, that it is in the form of
secondary propositions, at least as often as in that of primary propositions,
that the reasonings of ordinary life are exhibited. The
discourses, too, of the moralist and the metaphysician are perhaps
less often concerning things and their qualities, than concerning
principles and hypotheses, concerning truths and the mutual connexion
and relation of truths. The conclusions which our narrow
experience suggests in relation to the great questions of morals and
society yet unsolved, manifest, in more ways than one, the limitations
of their human origin; and though the existence of universal
principles is not to be questioned, the partial formulae
which comprise our knowledge of their application are subject
to conditions, and exceptions, and failure. Thus, in those departments
of inquiry which, from the nature of their subject-matter,
should be the most interesting of all, much of our actual
knowledge is hypothetical. That there has been a strong tendency
to the adoption of the same forms of thought in writers
on speculative philosophy, will hereafter appear. Hence the introduction
of a general method for the discussion of hypothetical
and the other varieties of secondary propositions, will open to us
a more interesting field of applications than we have before met
with.

4. The discussion of the theory of Secondary Propositions is
in the next place interesting, from the close and remarkable analogy
which it bears with the theory of Primary Propositions. It
will appear, that the formal laws to which the operations of the mind
are subject, are identical in expression in both cases. The mathematical
processes which are founded on those laws are, therefore,
identical also. Thus the methods which have been investigated
in the former portion of this work will continue to be available
in the new applications to which we are about to proceed. But
while the laws and processes of the method remain unchanged,
the rule of interpretation must be adapted to new conditions.
Instead of classes of things, we shall have to substitute propositions,
and for the relations of classes and individuals, we shall
have to consider the connexions of propositions or of events.
Still, between the two systems, however differing in purport and
interpretation, there will be seen to exist a pervading harmonious
relation, an analogy which, while it serves to facilitate the conquest
of every yet remaining difficulty, is of itself an interesting
subject of study, and a conclusive proof of that unity of character
which marks the constitution of the human faculties.

\begin{center}
\textsc{Proposition I.}
\end{center}

5. \textit{To investigate the nature of the connexion of Secondary Propositions
with the idea of Time.}

It is necessary, in entering upon this inquiry, to state clearly
the nature of the analogy which connects Secondary with Primary
Propositions.

Primary Propositions express relations among things, viewed
as component parts of a universe within the limits of which,
whether coextensive with the limits of the actual universe or
not, the matter of our discourse is confined. The relations expressed
are essentially \textit{substantive}. Some, or all, or none, of the
members of a given class, are also members of another class.
The subjects to which primary propositions refer--the relations
among those subjects which they express--are all of the above
character.

But in treating of secondary propositions, we find ourselves concerned
with another class both of subjects and relations. For the
subjects with which we have to do are themselves propositions, so
that the question may be asked,--Can we regard these subjects
also as \textit{things}, and refer them, by analogy with the previous
case, to a universe of their own? Again, the relations among
these subject propositions are relations of coexistent truth or
falsehood, not of substantive equivalence. We do not say, when
expressing the connexion of two distinct propositions, that the
one \textit{is} the other, but use some such forms of speech as the
following, according to the meaning which we desire to convey:
``\textit{Either} the proposition $X$ is true, \textit{or} the proposition $Y$ is true;''
``If the proposition $X$ is true, the proposition $Y$ is true;'' ``The
propositions $X$ and $Y$ are jointly true;'' and so on.

Now, in considering any such relations as the above, we are
not called upon to inquire into the whole extent of their possible
meaning (for this might involve us in metaphysical questions of
causation, which are beyond the proper limits of science); but it
suffices to ascertain some meaning which they undoubtedly
possess, and which is adequate for the purposes of logical deduction.
Let us take, as an instance for examination, the conditional
proposition, ``If the proposition $X$ is true, the proposition $Y$ is
true.'' An undoubted meaning of this proposition is, that the
\textit{time} in which the proposition $X$ is true, is \textit{time} in which the
proposition $Y$ is true. This indeed is only a relation of coexistence,
and may or may not exhaust the meaning of the proposition, but
it is a relation really involved in the statement of the proposition,
and further, it suffices for all the purposes of logical inference.

The language of common life sanctions this view of the
essential connexion of secondary propositions with the notion of
time. Thus we limit the application of a primary proposition by
the word ``some,'' but that of a secondary proposition by the
word ``sometimes.'' To say, ``Sometimes injustice triumphs,''
is equivalent to asserting that there are times in which the
proposition ``Injustice now triumphs,'' is a true proposition. There
are indeed propositions, the truth of which is not thus limited to
particular periods or conjunctures; propositions which are true
throughout all time, and have received the appellation of
``eternal truths.'' The distinction must be familiar to every reader of
Plato and Aristotle, by the latter of whom, especially, it is
employed to denote the contrast between the abstract verities of
science, such as the propositions of geometry which are always
true, and those contingent or ph{\ae}nomenal relations of things
which are sometimes true and sometimes false. But the forms of
language in which both kinds of propositions are expressed
manifest a common dependence upon the idea of time; in the one
case as limited to some finite duration, in the other as stretched
out to eternity.

6. It may indeed be said, that in ordinary reasoning we are
often quite unconscious of this notion of time involved in the very
language we are using. But the remark, however just, only
serves to show that we commonly reason by the aid of words
and the forms of a well-constructed language, without attending
to the ulterior grounds upon which those very forms have been
established. The course of the present investigation will afford an
illustration of the very same principle. I shall avail myself of
the notion of time in order to determine the laws of the expression
of secondary propositions, as well as the laws of combination of
the symbols by which they are expressed. But when those
laws and those forms are once determined, this notion of time
(essential, as I believe it to be, to the above end) may practically
be dispensed with. We may then pass from the forms of
common language to the closely analogous forms of the symbolical
instrument of thought here developed, and use its processes, and
interpret its results, without any conscious recognition of the idea
of time whatever.

\begin{center}
\textsc{Proposition II.}
\end{center}

7. \textit{To establish a system of notation for the expression of
Secondary Propositions, and to show that the symbols which it
involves are subject to the same laws of combination as the
corresponding symbols employed in the expression of Primary
Propositions.}

Let us employ the capital letters $X$, $Y$, $Z$, to denote the
elementary propositions concerning which we desire to make some
assertion touching their truth or falsehood, or among which we
seek to express some relation in the form of a secondary
proposition. And let us employ the corresponding small letters $x$, $y$, $z$,
considered as expressive of mental operations, in the following
sense, viz.: Let $x$ represent an act of the mind by which we fix
our regard upon that portion of time for which the proposition $X$
is true; and let this meaning be understood when it is asserted
that $x$ \textit{denotes} the time for which the proposition $X$ is true. Let
us further employ the connecting signs +, -, =, \&c., in the
following sense, viz.: Let $x+y$ denote the aggregate of those
portions of time for which the propositions $X$ and $Y$ are respectively
true, those times being entirely separated from each other.
Similarly let $x-y$ denote that remainder of time which is left when
we take away from the portion of time for which $X$ is true, that
(by supposition) included portion for which $Y$ is true. Also, let
$x=y$ denote that the time for which the proposition $X$ is true,
is identical with the time for which the proposition $Y$ is true.
We shall term $x$; the \textit{representative symbol} of the proposition $X$, \&c.

From the above definitions it will follow, that we shall
always have

\[
x + y = y + x,
\]

for either member will denote the same aggregate of time.

Let us further represent by $xy$ the performance in succession
of the two operations represented by $y$ and $x$, i.e. the whole
mental operation which consists of the following elements, viz.,
1st, The mental selection of that portion of time for which the
proposition $Y$ is true. 2ndly, The mental selection, out of that
portion of time, of such portion as it contains of the time in
which the proposition $X$ is true,--the result of these successive
processes being the fixing of the mental regard upon the whole
of that portion of time for which the propositions $X$ and $Y$ are
both true.

From this definition it will follow, that we shall always have

\begin{equation}
xy = yx.
\end{equation}

For whether we select mentally, first that portion of time for
which the proposition $Y$ is true, then out of the result that
contained portion for which $X$ is true; or first, that portion of time
for which the proposition $X$ is true, then out of the result that
contained portion of it for which the proposition $Y$ is true; we
shall arrive at the same final result, viz., that portion of time for
which the propositions $X$ and $Y$ are both true.

By continuing this method of reasoning it may be established,
that the laws of combination of the symbols $x$, $y$, $z$, \&c., in the
species of interpretation here assigned to them, are identical in
expression with the laws of combination of the same symbols, in
the interpretation assigned to them in the first part of this
treatise. The reason of this final identity is apparent. For in
both cases it is the same faculty, or the same combination of
faculties, of which we study the operations; operations, the
essential character of which is unaffected, whether we suppose them to
be engaged upon that universe of things in which all existence
is contained, or upon that whole of time in which all events are
realized, and to some part, at least, of which all assertions,
truths, and propositions, refer.

Thus, in addition to the laws above stated, we shall have by
(4), Chap, II., the law whose expression is

\begin{equation}
x(y + z)=xy + xz;
\end{equation}

and more particularly the fundamental law of duality (2) Chap, II.,
whose expression is

\begin{equation}
x^2 = x, or, x (1 - x) = 0;
\end{equation}

a law, which while it serves to distinguish the system of thought
in Logic from the system of thought in the science of quantity,
gives to the processes of the former a completeness and a
generality which they could not otherwise possess.

8. Again, as this law (3) (as well as the other laws) is
satisfied by the symbols $0$ and $1$, we are led, as before, to inquire
whether those symbols do not admit of interpretation in the
present system of thought. The same course of reasoning which we
before pursued shows that they do, and warrants us in the two
following positions, viz.:

1st, That in the expression of secondary propositions, $0$
represents \textit{nothing} in reference to the element of time.

2nd, That in the same system $1$ represents the universe, or
whole of time, to which the discourse is supposed in any manner
to relate.

As in primary propositions the universe of discourse is
sometimes limited to a small portion of the actual universe of things,
and is sometimes co-extensive with that universe; so in secondary
propositions, the universe of discourse may be limited to a
single day or to the passing moment, or it may comprise the
whole duration of time. It may, in the most literal sense, be
``eternal.'' Indeed, unless there is some limitation expressed or
implied in the nature of the discourse, the proper interpretation
of the symbol $1$ in secondary propositions is ``eternity;'' even as
its proper interpretation in the primary system is the actually
existent universe.

9. Instead of appropriating the symbols $x$, $y$, $z$, to the
representation of the truths of propositions, we might with equal
propriety apply them to represent the occurrence of events. In fact,
the occurrence of an event both implies, and is implied by, the
truth of a proposition, viz., of the proposition which asserts the
occurrence of the event. The one signification of the symbol $x$
necessarily involves the other. It will greatly conduce to
convenience to be able to employ our symbols in either of these
really equivalent interpretations which the circumstances of a
problem may suggest to us as most desirable; and of this liberty
I shall avail myself whenever occasion requires. In problems of
pure Logic I shall consider the symbols $x$, $y$, \&c. as representing
elementary propositions, among which relation is expressed in
the premises. In the mathematical theory of probabilities, which,
as before intimated (I. 12), rests upon a basis of Logic, and
which it is designed to treat in a subsequent portion of this work,
I shall employ the same symbols to denote the simple events,
whose implied or required frequency of occurrence it counts
among its elements.

\begin{center}
\textsc{Proposition III.}
\end{center}

10. \textit{To deduce general Rules for the expression of Secondary
Propositions.}

In the various inquiries arising out of this Proposition, fulness
of demonstration will be the less necessary, because of the exact
analogy which they bear with similar inquiries already completed
with reference to primary propositions. We shall first consider
the expression of terms; secondly, that of the propositions by
which they are connected.


As 1 denotes the whole duration of time, and $x$ that portion
of it for which the proposition $X$ is true, $1 - x$ will denote that
portion of time for which the proposition $X$ is false.

Again, as $xy$ denotes that portion of time for which the
propositions $X$ and $Y$ are both true, we shall, by combining this and
the previous observation, be led to the following interpretations,
viz.:

The expression $x(1-y)$ will represent the time during which
the proposition $X$ is true, and the proposition $Y$ false. The
expression $(1-x)(1-y)$ will represent the time during which the
propositions $X$ and $Y$ are simultaneously false.

The expression $x(1-y)+y(1-x)$ will express the time
during which either $X$ is true or $Y$ true, but not both; for that
time is the sum of the times in which they are singly and
exclusively true. The expression $xy+(1-x)(1-y)$ will express the
time during which $X$ and $Y$ are either both true or both false.

If another symbol $z$ presents itself, the same principles remain
applicable. Thus $xyz$ denotes the time in which the propositions
$X$, $Y$, and $Z$ are simultaneously true; $(1-x)(1-y)(1-z)$ the
time in which they are simultaneously false; and the sum of
these expressions would denote the time in which they are either
true or false together.

The general principles of interpretation involved in the above
examples do not need any further illustrations or more explicit
statement.

11. The laws of the expression of propositions may now be
exhibited and studied in the distinct cases in which they present
themselves. There is, however, one principle of fundamental
importance to which I wish in the first place to direct attention.
Although the principles of expression which have been laid down
are perfectly general, and enable us to limit our assertions of the
truth or falsehood of propositions to any particular portions of
that whole of time (whether it be an unlimited eternity, or a
period whose beginning and whose end are definitely fixed, or the
passing moment) which constitutes the universe of our discourse,
yet, in the actual procedure of human reasoning, such limitation
is not commonly employed. When we assert that a proposition
is true, we generally mean that it is true throughout the whole
duration of the time to which our discourse refers; and when
different assertions of the unconditional truth or falsehood of
propositions are jointly made as the premises of a logical demonstration,
it is to the same universe of time that those assertions are
referred, and not to particular and limited parts of it. In that
necessary matter which is the object or field of the exact sciences
every assertion of a truth may be the assertion of an ``eternal
truth.'' In reasoning upon transient ph{\ae}nomena (as of some
social conjuncture) each assertion may be qualified by an
immediate reference to the present time, ``Now.'' But in both cases,
unless there is a distinct expression to the contrary, it is to the
same period of duration that each separate proposition relates.
The cases which then arise for our consideration are the
following:

1st. \textit{To express the Proposition, ``The proposition $X$ is true.''}

We are here required to express that within those limits of
time to which the matter of our discourse is confined the
proposition $X$ is true. Now the time for which the proposition $X$ is
true is denoted by $x$, and the extent of time to which our
discourse refers is represented by 1. Hence we have

\begin{equation}
x=1
\end{equation}

as the expression required.

2nd. \textit{To express the Proposition, ``The proposition $X$ is
false.''}

We are here to express that within the limits of time to which
our discourse relates, the proposition $X$ is false; or that within
those limits there is no portion of time for which it is true. Now
the portion of time for which it is true is $x$. Hence the required
equation will be

\begin{equation}
x = 0.
\end{equation}

This result might also be obtained by equating to the whole
duration of time 1, the expression for the time during which the
proposition $X$ is false, viz., $1-x$. This gives

\begin{eqnarray*}
1-x&=& 1, \\
\textrm{whence } x &=& 0.
\end{eqnarray*}

3rd. \textit{To express the disjunctive Proposition, ``Either the proposition
$X$ is true or the proposition $Y$ is true;'' it being thereby
implied that the said propositions are mutually exclusive, that is to
say, that one only of them is true.}

The time for which either the proposition $X$ is true or the
proposition $Y$ is true, but not both, is represented by the
expression $x(1-y) + y(1-x)$. Hence we have

\begin{equation}
x(1-y) + y(1-x) = 1,
\end{equation}

for the equation required.

If in the above Proposition the particles \textit{either}, \textit{or}, are
supposed not to possess an absolutely disjunctive power, so that the
possibility of the simultaneous truth of the propositions $X$ and $Y$
is not excluded, we must add to the first member of the above
equations the term $xy$. We shall thus have

\begin{eqnarray}
xy + x(1-y) + (1-x)y = 1, \nonumber \\
\textrm{or }x + (1 - x)y = 1.
\end{eqnarray}

4th. \textit{To express the conditional Proposition, ``If the
proposition $Y$ is true, the proposition $X$ is true.''}

Since whenever the proposition $Y$ is true, the proposition $X$
is true, it is necessary and sufficient here to express, that the time
in which the proposition $Y$ is true is time in which the
proposition $X$ is true; that is to say, that it is some indefinite portion
of the whole time in which the proposition $X$ is true. Now the
time in which the proposition $Y$ is true is $y$, and the whole time
in which the proposition $X$ is true is $x$. Let $v$ be a symbol of
time indefinite, then will $vx$ represent an indefinite portion of the
whole time $x$. Accordingly, we shall have

\[
y = vx
\]

as the expression of the proposition given.

12. When $v$ is thus regarded as a symbol of time indefinite,
$vx$ may be understood to represent the whole, or an indefinite
part, or no part, of the whole time $x$; for any one of these
meanings may be realized by a particular determination of the arbitrary
symbol $v$. Thus, if $v$ be determined to represent a time in which
the whole time $x$ is included, $vx$ will represent the whole time $x$.
If $v$ be determined to represent a time, some part of which is included
in the time $x$, but which does not fill up the measure of
that time, $vx$ will represent a part of the time $x$. If, lastly, $v$ is
determined to represent a time, of which no part is common with
any part of the time $x$, $vx$ will assume the value 0, and will be
equivalent to ``no time,'' or ``never.''

Now it is to be observed that the proposition, ``If $Y$ is true,
$X$ is true,'' contains no assertion of the truth of either of the
propositions $X$ and $Y$. It may equally consist with the
supposition that the truth of the proposition $Y$ is a condition
indispensable to the truth of the proposition $X$, in which case we
shall have $v=1$; or with the supposition that although $Y$
expresses a condition which, when realized, assures us of the truth
of $X$, yet $X$ may be true without implying the fulfilment of that
condition, in which case $v$ denotes a time, some part of which is
contained in the whole time $x$; or, lastly, with the supposition
that the proposition $Y$ is not true at all, in which case $v$
represents some time, no part of which is common with any part of
the time $x$. All these cases are involved in the general
supposition that $v$ is a symbol of time indefinite.

5th. \textit{To express a proposition in which the conditional and the
disjunctive characters both exist.}

The general form of a conditional proposition is, ``If $Y$ is
true, $X$ is true,'' and its expression is, by the last section, $y = vx$.
We may properly, in analogy with the usage which has been
established in primary propositions, designate $Y$ and $X$ as the
terms of the conditional proposition into which they enter; and
we may further adopt the language of the ordinary Logic, which
designates the term $Y$, to which the particle \textit{if} is attached, the
``antecedent'' of the proposition, and the term $X$
the ``consequent.''

Now instead of the terms, as in the above case, being simple
propositions, let each or either of them be a disjunctive
proposition involving different terms connected by the particles \textit{either},
\textit{or}, as in the following illustrative examples, in which $X$, $Y$, $Z$,
\&c. denote simple propositions.

1st. If either $X$ is true or $Y$ is true, then $Z$ is true.

2nd. If $X$ is true, then either $Y$ is true or $Z$ true.

3rd. If either $X$ is true or $Y$ is true, then either $Z$ and $W$
are both true, or they are both false.

It is evident that in the above cases the relation of the antecedent
to the consequent is not affected by the circumstance that
one of those terms or both are of a disjunctive character. Accordingly
it is only necessary to obtain, in conformity with the
principles already established, the proper expressions for the
antecedent and the consequent, to affect the latter with the indefinite
symbol $v$, and to equate the results. Thus for the propositions
above stated we shall have the respective equations,

\begin{eqnarray*}
&\textrm{1st }x(1 - y) + (1 - x)y = vz.\\
&\textrm{2nd.}x = v \{y(1-z)+z(1-y)\}.\\
&\textrm{3rd.}x (1 - y) + y (1 - x) = v \{ zw + (1 - z) (1 - w) \}
\end{eqnarray*}
The rule here exemplified is of general application.

Cases in which the disjunctive and the conditional elements
enter in a manner different from the above into the expression of
a compound proposition, are conceivable, but I am not aware that
they are ever presented to us by the natural exigencies of human
reason, and I shall therefore refrain from any discussion of them.
No serious difficulty will arise from this omission, as the general
principles which have formed the basis of the above applications
are perfectly general, and a slight effort of thought will adapt
them to any imaginable case.

13. In the laws of expression above stated those of interpretation
are implicitly involved. The equation
\[
x= 1
\]
must be understood to express that the proposition $X$ is true;
the equation
\[
x = 0,
\]
that the proposition $X$ is false. The equation
\[
xy= 1
\]
will express that the propositions $X$ and $Y$ are both true together;
and the equation
\[
xy= 0
\]
that they are not both together true.

In like manner the equations

\begin{eqnarray*}
x(1-y) + y(1-x) &=& 1,\\
x (1 - y) + y (1 - x) &=& 0,
\end{eqnarray*}

will respectively assert the truth and the falsehood of the disjunctive
Proposition, ``Either $X$ is true or $Y$ is true.'' The equations

\begin{eqnarray*}
y &=& vx \\
y &=& v(1-x)
\end{eqnarray*}

will respectively express the Propositions, ``If the proposition $Y$
is true, the proposition $X$ is true.'' ``If the proposition $Y$ is
true, the proposition $X$ is false.''

Examples will frequently present themselves, in the succeeding
chapters of this work, of a case in which some terms of a
particular member of an equation are affected by the indefinite
symbol $v$, and others not so affected. The following instance
will serve for illustration. Suppose that we have

\[
y = xz + vx (1 - z).
\]

Here it is implied that the time for which the proposition $Y$ is
true consists of all the time for which $X$ and $Z$ are together true,
together with an indefinite portion of the time for which $X$ is
true and $Z$ false. From this it may be seen, 1st, That if $Y$ is
true, either $X$ and $Z$ are together true, or $X$ is true and $Z$ false;
2ndly, If $X$ and $Z$ are together true, $Y$ is true. The latter of
these may be called the reverse interpretation, and it consists in
taking the antecedent out of the second member, and the consequent from the first member of the equation. The existence of
a term in the second member, whose coefficient is unity, renders
this latter mode of interpretation possible. The general principle
which it involves may be thus stated:

14. \textsc{Principle}.--\textit{Any constituent term or terms in a particular
member of an equation which have for their coefficient unity, may
be taken as the antecedent of a proposition, of which all the terms
in the other member form the consequent.}

Thus the equation

\[
y = xz + vx (1 - z) + (1 - x) (1 - z)
\]

would have the following interpretations:

\textsc{Direct Interpretation}.--\textit{If the proposition $Y$ is true, then
either $X$ and $Z$ are true, or $X$ is true and $Z$ false, or $X$ and $Z$
are both false}.

\textsc{Reverse Interpretation}.--\textit{If either $X$ and $Z$ are true, or
$X$ and $Z$ are false, $Y$ is true}.

The aggregate of these partial interpretations will express
the whole significance of the equation given.

15. We may here call attention again to the remark, that
although the idea of time appears to be an essential element in
the theory of the interpretation of secondary propositions, it may
practically be neglected as soon as the laws of expression and of
interpretation are definitely established. The forms to which
those laws give rise seem, indeed, to correspond with the forms of
a perfect language. Let us imagine any known or existing language
freed from idioms and divested of superfluity, and let us
express in that language any given proposition in a manner the
most simple and literal,--the most in accordance with those
principles of pure and universal thought upon which all languages
are founded, of which all bear the manifestation, but from which
all have more or less departed. The transition from such a language
to the notation of analysis would consist of no more than
the substitution of one set of signs for another, without essential
change either of form or character. For the elements, whether
things or propositions, among which relation is expressed, we
should substitute letters; for the disjunctive conjunction we
should write +; for the connecting copula or sign of relatioin, we
should write =. This analogy I need not pursue. Its reality
and completeness will be made more apparent from the study of
those forms of expression which will present themselves in subsequent applications of the present theory, viewed in more immediate
comparison with that imperfect yet noble instrument of
thought--the English language.

16. Upon the general analogy between the theory of Primary
and that of Secondary Propositions, I am desirous of adding a
few remarks before dismissing the subject of the present chapter.

We might undoubtedly, have established the theory of Primary
Propositions upon the simple notion of space, in the same
way as that of secondary propositions has been established upon
the notion of time. Perhaps, had this been done, the analogy
which we are contemplating would have been in somewhat closer
accordance with the view of those who regard space and time
as merely ``forms of the human understanding,'' conditions of
knowledge imposed by the very constitution of the mind upon
all that is submitted to its apprehension. But this view, while
on the one hand it is incapable of demonstration, on the other
hand ties us down to the recognition of ``place,'' \textgreek{t`o po>'n}, as an
essential category of existence. The question, indeed, whether
it is so or not, lies, I apprehend, beyond the reach of our faculties;
but it may be, and I conceive has been, established, that the
formal processes of reasoning in primary propositions do not require,
as an essential condition, the manifestation in space of the
things about which we reason; that they would remain applicable,
with equal strictness of demonstration, to forms of existence,
if such there be, which lie beyond the realm of sensible
extension. It is a fact, perhaps, in some degree analogous to this,
that we are able in many known examples in geometry and dynamics,
to exhibit the formal analysis of problems founded upon
some intellectual conception of space different from that which is
presented to us by the senses, or which can be realized by the
imagination. \footnote {Space is presented to us in perception, as possessing the three dimensions
of length, breadth, and depth. But in a large class of problems relating to the
properties of curved surfaces, the rotations of solid bodies around axes, the vibrations
of elastic media, \&c., this limitation appears in the analytical investigation
to be of an arbitrary character, and if attention were paid to the processes
of solution alone, no reason could be discovered why space should not exist in
four or in any greater number of dimensions. The intellectual procedure in
the imaginary world thus suggested can be apprehended by the clearest light of
analogy.

The existence of space in three dimensions, and the views thereupon of the
religious and philosophical mind of antiquity, are thus set forth by Aristotle:--
\textgreek{Meg'ejos d`e t`o m`en `ef "en, gramm'h t`o d' 'ep`i d'no 'ep'ipedon, t`o d'
`ep`i tr'ia sv<'wma' Ka'i par`a ta>'nta o'nk "esvtin "allo m'egejos, di`a t`o tri'a p'anta
e`inai ka`i t`o tr`is p'anth. K'ajaper g'ar fasvi ka`i o`i Pnjag'oreioi, t`o p>'an ka`i t`a
p'anta to>'is trisv`in "wrisvtai. Telent`h g`ar ka`i m'esvon ka`i 'ark`h t`on  `arijm`on
"ekei t`on to>'n pant'oc' ta>'nta d`e t`on t>'hs tri'ados. Di`o par`a t>'hs f'nsvews
e'ilhf'otes "wsvper n'omons 'eke'inhs, ka`i pr`os t`as `agisvte'ias kr'wmeja t>'wn
je>'wn t>'y 'arijm>'y to'nt>'y.}--\textit{De Caelo}, 1.} I conceive, therefore,
 that the idea of space is not
essential to the development of a theory of primary propositions,
but am disposed, though desiring to speak with diffidence upon
a question of such extreme difficulty, to think that the idea of
time is essential to the establishment of a theory of secondary
propositions. There seem to be grounds for thinking, that
without any change in those faculties which are concerned in
\textit{reasoning}, the manifestation of space to the human mind might
have been different from what it is, but not (at least the same)
grounds for supposing that the manifestation of time could have
been otherwise than we perceive it to be. Dismissing, however,
these speculations as possibly not altogether free from presumption,
let it be affirmed that the real ground upon which the
symbol $1$ represents in primary propositions the universe of
things, and not the space they occupy, is, that the sign of
identity $=$ connecting the members of the corresponding equations,
implies that the things which they represent are identical,
not simply that they are found in the same portion of space.
Let it in like manner be affirmed, that the reason why the symbol
$1$ in secondary propositions represents, not the universe of events,
but the eternity in whose successive moments and periods they
are evolved, is, that the same sign of identity connecting the
logical members of the corresponding equations implies, not that
the events which those members represent are identical, but that
the times of their occurrence are the same. These reasons appear
to me to be decisive of the immediate question of interpretation. In
a former treatise on this subject (Mathematical Analysis of Logic,
p. 49), following the theory of Wallis respecting the Reduction
of Hypothetical Propositions, I was led to interpret the symbol $1$
in secondary propositions as the universe of ``cases'' or ``conjunctures
of circumstances;'' but this view involves the necessity of a
definition of what is meant by a ``case,'' or ``conjuncture of
circumstances;'' and it is certain, that whatever is involved in
the term beyond the notion of time is alien to the objects, and
restrictive of the processes, of formal Logic.


\chapter[METHODS IN SECONDARY PROPOSITIONS]{\large OF THE METHODS AND PROCESSES TO BE ADOPTED IN THE TREATMENT
OF SECONDARY PROPOSITIONS.}


1. It has appeared from previous researches (XI. 7) that the
laws of combination of the literal symbols of Logic are the
same, whether those symbols are employed in the expression of
primary or in that of secondary propositions, the sole existing
difference between the two cases being a difference of interpretation.
It has also been established (V. 6), that whenever distinct
systems of thought and interpretation are connected with
the same system of formal laws, i.e., of laws relating to the combination
and use of symbols, the attendant processes, intermediate
between the expression of the primary conditions of a problem
and the interpretation of its symbolical solution, are the same in
both. Hence, as between the systems of thought manifested in
the two forms of primary and of secondary propositions, this community
of formal law exists, the processes which have been established
and illustrated in our discussion of the former class of
propositions will, without any modification, be applicable to the
latter.

2. Thus the laws of the two fundamental processes of elimination
and development are the same in the system of secondary
as in the system of primary propositions. Again, it has been
seen (Chap. VI. Prop. 2) how, in primary propositions, the interpretation
of any proposed equation devoid of fractional forms
may be effected by developing it into a series of constituents, and
equating to 0 every constituent whose coefficient does not vanish.
To the equations of secondary propositions the same method is
applicable, and the interpreted result to which it finally conducts
us is, as in the former case (VI. 6), a system of co-existent denials.
But while in the former case the force of those denials is expended
upon the existence of certain classes of things, in the
latter it relates to the truth of certain combinations of the elementary
propositions involved in the \textit{terms} of the given premises.
And as in primary propositions it was seen that the system of
denials admitted of conversion into various other forms of propositions
(VI. 7), \&c., such conversion will be found to be possible
here also, the sole difference consisting not in the forms of the
equations, but in the nature of their interpretation.

3. Moreover, as in primary propositions, we can find the expression
of any element entering into a system of equations, in
terms of the remaining elements (VI. 10), or of any selected
number of the remaining elements, and interpret that expression
into a logical inference, the same object can be accomplished by
the same means, difference of interpretation alone excepted, in
the system of secondary propositions. The elimination of those
elements which we desire to banish from the final solution, the
reduction of the system to a single equation, the algebraic solution
and the mode of its development into an interpretable form,
differ in no respect from the corresponding steps in the discussion
of primary propositions.

To remove, however, any possible difficulty, it may be desirable
to collect under a general Rule the different cases which
present themselves in the treatment of secondary propositions.

\textsc{Rule}.--\textit{Express symbolically the given propositions (XI. 11).}

\textit{Eliminate separately from each equation in which it is found the
indefinite symbol} $v$ (VII. 5).

\textit{Eliminate the remaining symbols which it is desired to banish
from the final solution: always before elimination 'reducing to a
single equation those equations in which the symbol or symbols to
be eliminated are found (VIII. 7). Collect the resulting equations
into a single equation $V = 0$}.

\textit{Then proceed according to the particular form in which it is
desired to express the final relation, as--}

1st. \textit{If in the form of a denial, or system of denials, develop the
function $V$, and equate to 0 all those constituents whose coefficients
do not vanish.}

2ndly. \textit{If in the form of a disjunctive proposition, equate to 1
the sum of those constituents whose coefficients vanish.}

3rdly. \textit{If in the form of a conditional proposition having a simple
element, as $x$ or $1-x$, for its antecedent, determine the algebraic
expression of that element, and develop that expression.}

4thly. \textit{If in the form of a conditional proposition having a
compound expression, as $xy$, $xy+(1-x)(1-y)$, \&c., for its antecedent,
equate that expression to a new symbol $t$, and determine $t$
as a developed function of the symbols which are to appear in the
consequent, either by ordinary methods or by the special method}
(IX. 9).

5thly. \textit{Interpret the results by} (XI. 13, 14).

\textit{If it only be desired to ascertain whether a particular elementary
proposition $x$ is true or false, we must eliminate all the symbols
but $x$; then the equation $x = 1$ will indicate that the proposition
is true, $x = 0$ that it is false, $0 = 0$ that the premises are insufficient
to determine whether it is true or false.}

4. Ex. 1.--The following prediction is made the subject of a
curious discussion in Cicero's fragmentary treatise, De Fato:--``Si
quis (Fabius) natus est oriente Canicula, is in mari non morietur.''
I shall apply to it the method of this chapter. Let $y$
represent the proposition, ``Fabius was born at the rising of the
dogstar;'' $x$ the proposition, ``Fabius will die in the sea.''
In saying that $x$ \textit{represents} the proposition, ``Fabius, \&c.,'' it is
only meant that $x$ is a symbol so appropriated (XI. 7) to the
above proposition, that the equation $x = 1$ declares, and the equation
$x = 0$ denies, the truth of that proposition. The equation
we have to discuss will be
\begin{equation}
     y = v(1-x).
\end{equation}
And, first, let it be required to reduce the given proposition to a
negation or system of negations (XII. 3). We have, on transposition,
\[
     y-v(1-x) = 0.
\]
Eliminating $v$,
\begin{eqnarray}
y\left\{y-(1-x)\right\} = 0, \nonumber \\
\textrm{or, }y-y(1-x) = 0,\nonumber \\
\textrm{or, }yx = 0.
\end{eqnarray}
The interpretation of this result is:--``It is not true that Fabius
was born at the rising of the dogstar, and will die in the sea.''
Cicero terms this form of proposition, ``Conjunctio ex repugnantibus;''
and he remarks that Chrysippus thought in this way
to evade the difficulty which he imagined to exist in contingent
assertions respecting the future: ``Hoc loco Chrysippus aestuans
falli sperat Chaldaeos casterosque divinos, neque eos usuros esse
conjunctionibus ut ita sua percepta pronuntient: Si quis natus
est oriente Canicula is in mari non morietur; sed potius ita dicant:
Non et natus est quis oriente Canicul\^{a}, et in mari morietur.
O licentiam jocularem! ... Multa genera sunt enuntiandi, nec
ullum distortius quam hoc quo Chrysippus sperat Chaldaeos contentos
Stoicorum causa fore.''--\textit{Cic. De Fato}, 7, 8.

5. To reduce the given proposition to a disjunctive form.
The constituents not entering into the first member of (2) are
\[
x(1-y), (1-x)y, (1-x)(1-y).
\]
Whence we have
\begin{equation}
y(1-x) + x(1-y) + (1-x)(1-y) = 1.
\end{equation}
The interpretation of which is:--\textit{Either Fabius was born at the
rising of the dogstar, and will not perish in the sea; or he was not
born at the rising of the dogstar, and will perish in the sea; or he
was not born at the rising of the dogstar, and will not perish in
the sea.}

In cases like the above, however, in which there exist constituents differing from each other only by a single factor, it is, as
we have seen (VII. 15), most convenient to collect such constituents into a single term. If we thus connect the first and third
terms of (3), we have
\[
(1 - y)x + 1 - x = 1;
\]
and if we similarly connect the second and third, we have
\[
y(1-x)+1-y=1.
\]
These forms of the equation severally give the interpretations--

\textit{Either Fabius was not born under the day star, and will die in
the sea, or he will not die in the sea.}

\textit{Either Fabius was born under the day star, and will not die in
the sea, or he was not born under the dogstar.}

It is evident that these interpretations are strictly equivalent
to the former one.

Let us ascertain, in the form of a conditional proposition, the
consequences which flow from the hypothesis, that ``Fabius will
perish in the sea.''

In the equation (2), which expresses the result of the elimination
of $v$ from the original equation, we must seek to determine
$x$ as a function of $y$.

We have
\[
x = \frac{0}{y} = 0y+\frac{0}{0}(1-y)\textrm{ on expansion,}
\]
or
\[
x = \frac{0}{0}(1-y);
\]
the interpretation of which is,--\textit{If Fabius shall die in the sea, he
was not born at the rising of the dogstar.}

These examples serve in some measure to illustrate the connexion
which has been established in the previous sections between
primary and secondary propositions, a connexion of which
the two distinguishing features are identity of process and analogy
of interpretation.

6. Ex. 2.--There is a remarkable argument in the second
book of the Republic of Plato, the design of which is to prove
the immutability of the Divine Nature. It is a very fine example
both of the careful induction from familiar instances by which
Plato arrives at general principles, and of the clear and connected
logic by which he deduces from them the particular inferences
which it is his object to establish. The argument is contained
in the following dialogue:

``Must not that which departs from its proper form be
changed either by itself or by another thing? Necessarily so.
Are not things which are in the best state least changed and disturbed,
as the body by meats and drinks, and labours, and every
species of plant by heats and winds, and such like affections? Is
not the healthiest and strongest the least changed? Assuredly.
And does not any trouble from without least disturb and change
that soul which is strongest and wisest? And as to all made
vessels, and furnitures, and garments, according to the same
principle, are not those which are well wrought, and in a good
condition, least changed by time and other accidents? Even so.
And whatever is in a right state, either by nature or by art, or
by both these, admits of the smallest change from any other
thing. So it seems. But God and things divine are in every
sense in the best state. Assuredly. In this way, then, God
should least of all bear many forms? Least, indeed, of all.
Again, should He transform and change Himself? Manifestly He
must do so, if He is changed at all. Changes He then Himself to
that which is more good and fair, or to that which is worse and
baser? Necessarily to the worse, if he be changed. For never
shall we say that God is indigent of beauty or of virtue. You
speak most rightly, said I, and the matter being so, seems it to
you, O Adimantus, that God or man \textit{willingly} makes himself in
any sense worse? Impossible, said he. Impossible, then, it is,
said I, that a god should wish to change himself; but ever being
fairest and best, each of them ever remains absolutely in the same
form.''

The premises of the above argument are the following:

1st. If the Deity suffers change, He is changed either by Himself
or by another.

2nd. If He is in the best state, He is not changed by another.

3rd. The Deity is in the best state.

4th. If the Deity is changed by Himself, He is changed to a
worse state.

5th. If He acts willingly, He is not changed to a worse state.

6th. The Deity acts willingly.

Let us express the elements of these premises as follows:

Let $x$ represent the proposition, ``The Deity suffers change.''

% ** Isn't there a special markup for non-bulleted lists?

$y$, He is changed by Himself.

$z$, He is changed by another.

$s$, He is in the best state.

$t$, He is changed to a worse state.

$w$, He acts willingly.

Then the premises expressed in symbolical language yield, after
elimination of the indefinite class symbols $v$, the following equations:

\setcounter{equation}{0}
\begin{eqnarray}
   xyz+x(1-y)(1-z) = 0,\\
   sz = 0,             \\
   s = 1,              \\
   y(1-t) = 0,         \\
   wt = 0,             \\
   w = 1.
\end{eqnarray}

Retaining $x$, I shall eliminate in succession $z$, $s$, $y$, $t$, and $w$ (this
being the order in which those symbols occur in the above system),
and interpret the successive results.

Eliminating $z$ from (1) and (2), we get
\begin{equation}
     xs(1-y) = 0.
\end{equation}
Eliminating $s$ from (3) and (7),
\begin{equation}
     x(1-y) = 0.
\end{equation}
Eliminating $y$ from (4) and (8),
\begin{equation}
     x(1-t) = 0.
\end{equation}
Eliminating $t$ from (5) and (9),
\begin{equation}
     xw = 0.
\end{equation}
Eliminating $w$ from (6) and (10),
\begin{equation}
     x = 0.
\end{equation}

These equations, beginning with (8), give the following
results:

From (8) we have $x = \frac{0}{0}y$, therefore, \textit{If the Deity suffers
change, He is changed by Himself.}

From (9), $x = \frac{0}{0}t$, \textit{If the Deity suffers change, He is changed
to a worse state.}

From (10), $x = \frac{0}{0}(1-w)$. \textit{If the Deity suffers change, He
does not act willingly.}

From (11), \textit{The Deity does not suffer change.} This is Plato's
result.

Now I have before remarked, that the order of elimination
is indifferent. Let us in the present case seek to verify this fact
by eliminating the same symbols in a reverse order, beginning
with $w$. The resulting equations are,

\[
t=0, y = 0, x(1-x) = 0, z = 0, x = 0;
\]

yielding the following interpretations:

\textit{God is not changed to a worse state.
He is not changed by Himself.
If He suffers change, He is changed by another.
He is not changed by another.
He is not changed.
}

We thus reach by a different route the same conclusion.

Though as an exhibition of the \textit{power} of the method, the
above examples are of slight value, they serve as well as more
complicated instances would do, to illustrate its nature and character.

7. It may be remarked, as a final instance of analogy between
the system of primary and that of secondary propositions, that
in the latter system also the fundamental equation,
\[
x (1 - x) = 0,
\]
admits of interpretation. It expresses the axiom, \textit{A proposition
cannot at the same time be true and false}. Let this be compared
with the corresponding interpretation (III. 15). Solved under
the form
\[
x=\frac{0}{1-x}=\frac{0}{0}x,
\]
by development, it furnishes the respective axioms: ``A thing is
what it is:'' ``If a proposition is true, it is true:'' forms of what has
been termed ``The principle of identity.'' Upon the nature and
the value of these axioms the most opposite opinions have been
entertained. Some have regarded them as the very pith and marrow
of philosophy. Locke devoted to them a chapter, headed,
``On Trifling Propositions.''
\footnote{Essay on the Human Understanding, Book IV. Chap. viii.}
In both these views there seems
to have been a mixture of truth and error. Regarded as supplanting
experience, or as furnishing materials for the vain and
wordy janglings of the schools, such propositions are worse than
trifling. Viewed, on the other hand, as intimately allied with
the very laws and conditions of thought, they rise into at least a
speculative importance.


\chapter[CLARKE AND SPINOZA]
{\large ANALYSIS OF A PORTION OF DR. SAMUEL CLARKE's ``DEMONSTRATION
OF THE BEING AND ATTRIBUTES OF GOD,'' AND OF A
PORTION OF THE ``ETHICA ORDINE GEOMETRICO DEMONSTRATA''
OF SPINOZA. }


1. The general order which, in the investigations of the following
chapter, I design to pursue, is the following. I
shall examine what are the actual premises involved in the demonstrations
of some of the general propositions of the above
treatises, whether those premises be expressed or implied. By
the actual premises I mean whatever propositions are assumed
in the course of the argument, without being proved, and are
employed as parts of the foundation upon which the final conclusion
is built. The premises thus determined, I shall express in
the language of symbols, and I shall then deduce from them by
the methods developed in the previous chapters of this work, the
most important inferences which they involve, in addition to the
particular inferences actually drawn by the authors. I shall in
some instances modify the premises by the omission of some fact
or principle which is contained in them, or by the addition or
substitution of some new proposition, and shall determine how
by such change the ultimate conclusions are affected. In the
pursuit of these objects it will not devolve upon me to inquire,
except incidentally, how far the metaphysical principles laid down
in these celebrated productions are worthy of confidence, but
only to ascertain what conclusions may justly be drawn from
given premises; and in doing this, to exemplify the perfect liberty
which we possess as concerns both the choice and the
order of the elements of the final or concluding propositions, viz.,
as to determining what elementary propositions are true or false,
and what are true or false under given restrictions, or in given
combinations.

2. The chief practical difficulty of this inquiry will consist,
not in the application of the method to the premises once determined,
but in ascertaining what the premises are. In what area
regarded as the most rigorous examples of reasoning applied to
metaphysical questions, it will occasionally be found that different
trains of thought are blended together; that particular but essential
parts of the demonstration are given parenthetically, or out
of the main course of the argument; that the meaning of a premiss
may be in some degree ambiguous; and, not unfrequently,
that arguments, viewed by the strict laws of formal reasoning,
are incorrect or inconclusive. The difficulty of determining and
distinctly exhibiting the true premises of a demonstration may,

in such cases, be very considerable. But it is a difficulty which
must be overcome by all who would ascertain whether a particular
conclusion is proved or not, whatever form they may be
prepared or disposed to give to the ulterior process of reasoning.
It is a difficulty, therefore, which is not peculiar to the method
of this work, though it manifests itself more distinctly in connexion
with this method than with any other. So intimate, indeed,
is this connexion, that it is impossible, employing the method
of this treatise, to form even a conjecture as to the validity
of a conclusion, without a distinct apprehension and exact statement
of all the premises upon which it rests. In the more usual
course of procedure, nothing is, however, more common than to
examine some of the steps of a train of argument, and thence to
form a vague general impression of the scope of the whole, without
any such preliminary and thorough analysis of the premises
which it involves.

The necessity of a rigorous determination of the real premises
of a demonstration ought not to be regarded as an evil;
especially as, when that task is accomplished, every source
doubt or ambiguity is removed. In employing the method of
this treatise, the order in which premises are arranged, the mode
of connexion which they exhibit, with every similar circumstance
may be esteemed a matter of indifference, and the process
inference is conducted with a precision which might almost
termed mechanical.

3. The ``Demonstration of the Being and Attributes of
God,'' consists of a series of propositions or theorems, each
of them proved by means of premises resolvable, for the most
part, into two distinct classes, viz., facts of observation, such
as the existence of a material world, the phenomenon of motion,
\&c., and hypothetical principles, the authority and universality
of which are supposed to be recognised \textit{\`{a} priori}. It is,
of course, upon the truth of the latter, assuming the correctness
of the reasoning, that the validity of the demonstration really depends.
But whatever may be thought of its claims in this respect,
it is unquestionable that, as an intellectual performance, its
merits are very high. Though the trains of argument of which
it consists are not in general very clearly arranged, they are almost
always specimens of correct Logic, and they exhibit a
subtlety of apprehension and a force of reasoning which have
seldom been equalled, never perhaps surpassed. We see in them
the consummation of those intellectual efforts which were awakened
in the realm of metaphysical inquiry, at a period when the
dominion of hypothetical principles was less questioned than it
now is, and when the rigorous demonstrations of the newly risen
school of mathematical physics seemed to have furnished a model
for their direction. They appear to me for this reason (not to
mention the dignity of the subject of which they treat) to be
deserving of high consideration; and I do not deem it a vain
or superfluous task to expend upon some of them a careful
analysis.

4. The Ethics of Benedict Spinoza is a treatise, the object
of which is to prove the identity of God and the universe, and
to establish, upon this doctrine, a system of morals and of philosophy.
The analysis of its main argument is extremely difficult,
owing not to the complexity of the separate propositions which it
involves, but to the use of vague definitions, and of axioms which,
through a like defect of clearness, it is perplexing to determine
whether we ought to accept or to reject. While the reasoning of
Dr. Samuel Clarke is in part verbal, that of Spinoza is so in a much
greater degree; and perhaps this is the reason why, to some
minds, it has appeared to possess a formal cogency, to which in
reality it possesses no just claim. These points will, however,
be considered in the proper place.

\begin{center}
\textsc{clarke's demonstration.}

\textsc{Proposition I.}
\end{center}

5. ``\textit{Something has existed from eternity.}''

The proof is as follows:--

``For since something now is, 'tis manifest that something
always was. Otherwise the things that now are must have risen
out of nothing, absolutely and without cause. Which is a
plain contradiction in terms. For to say a thing is produced,
and yet that there is no cause at all of that production, is to say
that something is effected when it is effected by nothing, that is,
at the same time when it is not effected at all. Whatever exists
has a cause of its existence, either in the necessity of its own
nature, and thus it must have been of itself eternal: or in the
will of some other being, and then that other being must, at least
in the order of nature and causality, have existed before it.''

Let us now proceed to analyze the above demonstration. Its
first sentence is resolvable into the following propositions:

1st. Something is.

2nd. If something is, either something always was, or the
things that now are must have risen out of nothing.

The next portion of the demonstration consists of a proof
that the second of the above alternatives, viz., ``The things that
now are have risen out of nothing,'' is impossible, and it may
formally be resolved as follows:

3rd. If the things that now are have risen out of nothing,
something has been effected, and at the same time that something
has been effected by nothing.

4th. If that something has been effected by nothing, it has
not been effected at all.

The second portion of this argument appears to be a mere
assumption of the point to be proved, or an attempt to make that
point clearer by a different verbal statement.

The third and last portion of the demonstration contains a distinct
proof of the truth of either the original proposition to be
proved, viz., ``Something always was,'' or the point proved in
the second part of the demonstration, viz., the untenable nature
of the hypothesis, that ``the things that now are have risen out
of nothing.'' It is resolvable as follows:--

5th. If something is, either it exists by the necessity of its
own nature, or it exists by the will of another being.

6th. If it exists by the necessity of its own nature, something
always was.

7th. If it exists by the will of another being, then the proposition,
that the things which exist have arisen out of nothing,
is false.

The last proposition is not expressed in the same form in the
text of Dr. Clarke; but his expressed conclusion of the prior existence
of another Being is clearly meant as equivalent to a denial
of the proposition that the things which now are have risen
out of nothing.

It appears, therefore, that the demonstration consists of two
distinct trains of argument: one of those trains comprising what
I have designated as the \textit{first} and \textit{second} parts of the demonstration;
the other comprising the \textit{first} and \textit{third} parts. Let us consider
the latter train.

The premises are:--

1st. Something is.

2nd. If something is, either something always was, or the
things that now are have risen out of nothing.

3rd. If something is, either it exists in the necessity of its
own nature, or it exists by the will of another being.

4th. If it exists in the necessity of its own nature, something
always was.

5th. If it exists by the will of another being, then the hypothesis,
that the things which now are have risen out of nothing,
is false.

We must now express symbolically the above proposition.

Let \\
\begin{tabular}{c l l}
&$x$ = &Something is.\\
&$y$ = &Something always was.\\
&$z$ = &The things which now are have risen from nothing.\\
&$p$ = &It exists in the necessity of its own nature \\
& &(i.e. the \textit{something} spoken of above).\\
&$q$ = &It exists by the will of another Being.
\end{tabular}

It must be understood, that by the expression, Let $x =$
``Something is,'' is meant no more than that $x$ is the representative
symbol of that proposition (XI. 7), the equations
$x = 1$, $x = 0$, respectively declaring its truth and its falsehood.

The equations of the premises are:-- \\
\begin{tabular}{c l}
1st. &$x$ = 1;\\
2nd. &$x=v[y(1-x)+z(1-y)]$;\\
3rd. &$x=v[p(1-q)+q(1-p)]$;\\
4th. &$p=vy$;\\
5th. &$q=v(1-z)$;
\end{tabular}

and on eliminating the several indefinite symbols $v$, we have \\
\begin{eqnarray}
1-x=0;\\
x[yz+(1-y)(1-z)]=0;\\
x[pq+(1-p)(1-q)]=0;\\
p(q-y)=0;\\
qz=0.
\end{eqnarray}

6. First, I shall examine whether any conclusions are deducible
from the above, concerning the truth or falsity of the
single propositions represented by the symbols $y$, $z$, $p$, $q$, viz., of
the propositions, ``Something always was;'' ``The things which
now are have risen from nothing;'' ``The something which is
exists by the necessity of its own nature;'' ``The something
which is exists by the will of another being.''

For this purpose we must separately eliminate all the symbols
but $y$, all these but $z$, \&c. The resulting equation will determine
whether any such separate relations exist.

To eliminate $x$ from (1), (2), and (3), it is only necessary to
substitute in (2) and (3) the value of $x$ derived from (1). We
find as the results,

\begin{eqnarray}
yz+(1-y)(1-z)=0\textrm{.}\\
pq+(1-p)(1-q)=0\textrm{.}
\end{eqnarray}

To eliminate $p$ we have from (4) and (7), by addition,

\begin{equation}
p(1-y)+pq+(1-p)(1-q)=0\textrm{;}
\end{equation}

whence we find,

\begin{equation}
(1-y)(1-q)=0\textrm{.}
\end{equation}

To eliminate $q$ from (5) and (9), we have

\[
qz + (1-y)(1-q) = 0\textrm{;}
\]

whence we find

\begin{equation}
x(1-y) = 0\textrm{.}
\end{equation}

There now remain but the two equations (6) and (10), which,
on addition, give

\[
yz + 1 - y = 0\textrm{.}
\]

Eliminating from this equation $z$, we have

\begin{equation}
1-y = 0, \textrm{or, } y = 1\textrm{.}
\end{equation}

Eliminating from the same equation $y$, we have

\begin{equation}
z = 0\textrm{.}
\end{equation}

The interpretation of (11) is

\textit{Something always was.}

The interpretation of (12) is

\textit{The things which are have not risen from nothing.}

Next resuming the system (6), (7), with the two equations
(4), (5), let us determine the two equations involving $p$ and $q$
respectively.

To eliminate $y$ we have from (4) and (6),

\[
p(1-y) + yz + (1-y)(1-z) = (0)\textrm{;}
\]

whence

\begin{equation}
(p + 1 - z) z = 0, \textrm{or, } pz = 0\textrm{.}
\end{equation}

To eliminate $z$ from (5) and (13), we have

\[
qz + pz = 0\textrm{;}
\]

whence we get,

\[
0=0\textrm{.}
\]

There remains then but the equation (7), from which eliminating
$q$, we have $0=0$ for the final equation, in $p$.

\textit{Hence there is no conclusion derivable from the premises affirming
the simple truth or falsehood of the proposition, ``The
something which is exists in the necessity of its own nature.''} And as,
on eliminating $p$, there is the same result, $0=0$, for the ultimate
equation in $q$, it also follows, that \textit{there is no conclusion deducible
from the premises as to the simple truth or falsehood of the proposition,
``The something which is exists by the will of another Being.''}

Of relations connecting more than one of the propositions
represented by the elementary symbols, it is needless to consider
any but that which is denoted by the equation (7) connecting
$p$ and $q$, inasmuch as the propositions represented by the
remaining symbols are absolutely true or false independently of any
connexion of the kind here spoken of. The interpretation of (7),
placed under the form

\[
p(1-q) + q(1-p) = 1\textrm{, is,}
\]

\textit{The something which is, either exists in the necessity of its
own nature, or by the will of another being.}

I have exhibited the details of the above analysis with a,
perhaps, needless fulness and prolixity, because in the examples
which will follow, I propose rather to indicate the steps by
which results are obtained, than to incur the danger of a
wearisome frequency of repetition. The conclusions which have
resulted from the above application of the method are easily verified
by ordinary reasoning.

The reader will have no difficulty in applying the method
to the other train of premises involved in Dr. Clarke's first
Proposition, and deducing from them the two first of the conclusions
to which the above analysis has led.

\begin{center}
\textsc{Proposition II}.
\end{center}

7. \textit{Some one unchangeable and independent Being has existed
from eternity.}

The premises from which the above proposition is prove
are the following:

1st. Something has always existed.

2nd. If something has always existed, either there has existed
some one unchangeable and independent being, or the whole of
existing things has been comprehended in a succession of
changeable and dependent beings.

3rd. If the universe has consisted of a succession of
changeable and dependent beings, either that series has had a cause from
without, or it has had a cause from within.

4th. It has not had a cause from without (because it includes,
by hypothesis, all things that exist).

5th. It has not had a cause from within (because no part is
necessary, and if no part is necessary, the whole cannot be necessary).

Omitting, merely for brevity, the subsidiary proofs contained
in the parentheses of the fourth and fifth premiss, we may represent
the premises as follows:

\begin{eqnarray*}
\textrm{Let } x &=& \textrm{ Something has always existed.} \\
y &=& \textrm{There has existed some one unchangeable and independent being.} \\
z &=& \textrm{There has existed a succession of changeable and
dependent beings.} \\
p &=& \textrm{That series has had a cause from without.} \\
q &=& \textrm{That series has had a cause from within.}
\end{eqnarray*}

Then we have the following system of equations, viz.:
\begin{eqnarray*}
\textrm{1st. } x &=& 1 ; \\
\textrm{2nd. } x &=& v \{y(l - z) + z(1 - y)\} ; \\
\textrm{3rd. } z &=& v \{p(1 - q) + (1 - p)q\} ; \\
\textrm{4th. } p &=& 0; \\
\textrm{5th. } q &=& 0 :
\end{eqnarray*}
which, on the separate elimination of the indefinite symbols $v$,
gives
\setcounter{equation}{0}
\begin{eqnarray}
l-x = 0; \\ % (1)
x\{yz+(1-y)(1-z)\}=0; \\ %(2)
z\{pq+(1-p)(1-q)\}=0; \\ %(3)
p = 0; \\ %(4)
q = 0.  % (5)
\end{eqnarray}

The elimination from the above system of $x$, $p$, $q$, and $y$,
conducts to the equation
\[z=0.\]
And the elimination of $x$, $p$, $q$, and $z$, conducts in a similar
manner to the equation
\[y=1.\]

Of which equations the respective interpretations are:

1st. \textit{The whole of existing things has not been comprehended
in a succession of changeable and dependent beings.}

2nd. \textit{There has existed some one unchangeable and independent
being.}

The latter of these is the proposition which Dr. Clarke proves.
As, by the above analysis, all the propositions represented by the
literal symbols $x$, $y$, $z$, $p$, $q$, are determined as absolutely true or
false, it is needless to inquire into the existence of any further relations
connecting those propositions together.

Another proof is given of Prop. II., which for brevity I pass
over. It may be observed, that the ``impossibility of infinite
succession,'' the proof of which forms a part of Clarke's argument,
has commonly been assumed as a fundamental principle of
metaphysics, and extended to other questions than that of causation.
Aristotle applies it to establish the necessity of first principles
of demonstration; \footnote{Metaphysics, III. 4; Anal. Post. I, 19, \textit{et seq}.}
the necessity of an end (the good), in
human actions, \&c. \footnote{Nic. Ethics, Book I. Cap. II.} There is, perhaps, no
principle more frequently
referred to in his writings. By the schoolmen it was
similarly applied to prove the impossibility of an infinite subordination
of genera and species, and hence the necessary existence
of universals. Apparently the impossibility of our forming a
definite and complete conception of an infinite series, i.e. of
comprehending it as a \textit{whole}, has been confounded with a logical
inconsistency, or contradiction in the idea itself.

8. The analysis of the following argument depends upon the
theory of Primary Propositions.

\begin{center}
\textsc{Proposition III}.
\end{center}

\textit{That unchangeable and independent Being must be self-existent}.

The premises are:--

1. Every being must either have come into existence out of
nothing, or it must have been produced by some external cause,
or it must be self-existent.

2. No being has come into existence out of nothing.

3. The unchangeable and independent Being has not been
produced by an external cause.

For the symbolical expression of the above, let us assume,

\begin{eqnarray*}
x &=& \textrm{Beings which have arisen out of nothing.} \\
y &=& \textrm{Beings which have been produced by an external cause.} \\
z &=& \textrm{Beings which are self-existent.} \\
w &=& \textrm{The unchangeable and independent Being.}
\end{eqnarray*}

Then we have
\setcounter{equation}{0}
\begin{eqnarray}
x(1-y)(1-z) + y(1-x)(1-z) + z(1-x) (1-y) = l, \\ % (1)
x = 0, \\ % (2)
w = v(1-y),  % (3)
\end{eqnarray}

from the last of which eliminating $v$,

\begin{equation}
wy = 0. % (4)
\end{equation}

Whenever, as above, the value of a symbol is given as 0 or 1, it
is best eliminated by simple substitution. Thus the elimination
of $x$ gives

\begin{eqnarray}
y(1-z) + z(1-y) &=& 1; \\ % (5)
\textrm{or, } yz + (1-y)(1-z) &=& 0. % (6)
\end{eqnarray}

Now adding (4) and (6), and eliminating $y$, we get

\begin{eqnarray*}
w(1-z) = 0, \\
\therefore w = vz;
\end{eqnarray*}

the interpretation of which is,--\textit{The unchangeable and
independent being is necessarily self-existing.}

Of (5), in its actual form, the interpretation is,--\textit{Every being
has either been produced by an external cause, or it is self-existent.}

9. In Dr. Samuel Clarke's observations on the above
proposition occurs a remarkable argument, designed to prove that the
material world is not the self-existent being above spoken of.
The passage to which I refer is the following:

``If matter be supposed to exist necessarily, then in that
necessary existence there is either included the power of gravitation,
or not. If not, then in a world merely material, and in which no
intelligent being presides, there never could have been any
motion; because motion, as has been already shown, and is now
granted in the question, is not necessary of itself. But if the
power of gravitation be included in the pretended necessary
existence of matter: then, it following necessarily that there
must be a vacuum (as the incomparable Sir Isaac Newton has
abundantly demonstrated that there must, if gravitation be an
universal quality or affection of matter), it follows likewise,
that matter is not a necessary being. For if a vacuum actually be,
then it is plainly more than possible for matter not to be.''--(pp. 25, 26).

It will, upon attentive examination, be found that the actual
premises involved in the above demonstration are the following:

1st. If matter is a necessary being, either the property of
gravitation is necessarily present, or it is necessarily absent.

2nd. If gravitation is necessarily absent, and the world is not
subject to any presiding Intelligence, motion does not exist.

3rd. If the property of gravitation is necessarily present,
existence of a vacuum is necessary.

4th. If the existence of a vacuum is necessary, matter is not
necessary being.

5th. If matter is a necessary being, the world is not subject to a
presiding Intelligence.

6th. Motion exists.

Of the above premises the first four are expressed in the
demonstration; the fifth is implied in the connexion of its first
and second sentences; and the sixth expresses a fact, which the
author does not appear to have thought it necessary to state, but
which is obviously a part of the ground of his reasoning. Let us
represent the elementary propositions in the following manner:

\begin{eqnarray*}
\textrm{Let } x &=& \textrm{Matter is a necessary being.}\\
y &=& \textrm{Gravitation is necessarily present.}\\
t &=& \textrm{Gravitation is necessarily absent.}\\
z &=& \textrm{The world is merely material, and not subject to any
presiding Intelligence.}\\
w &=& \textrm{Motion exists.}\\
v &=& \textrm{A vacuum is necessary.}
\end{eqnarray*}

Then the system of premises will be represented by the following
equations, in which $q$ is employed as the symbol of time
indefinite:

\begin{eqnarray*}
x &=& q\{y(1-t) + (1-y)t\}. \\
tz &=& q(1-w). \\
y &=& qv. \\
v &=& q(1-x). \\
x &=& qz. \\
w &=& 1.
\end{eqnarray*}

From which, if we eliminate the symbols $q$, we have the following
system, viz.:
\setcounter{equation}{0}
\begin{eqnarray}
x\{yt+(1-y)(1-t)\}=0.\\
tzw=0.\\
y(1-v)=0.\\
vx=0.\\
x(1-z)=0.\\
1-w=0.
\end{eqnarray}

Now if from these equations we eliminate $w$, $v$, $z$, $y$, and $t$, we
obtain the equation
\[x = 0,\]
which expresses the proposition, \textit{Matter is not a necessary
being}.
This is Dr. Clarke's conclusion. If we endeavour to
eliminate any other set of five symbols (except the set $v$, $z$, $y$, $t$, and $x$,
which would give $w = 1$), we obtain a result of the form $0=0$.
It hence appears that \textit{there are no other conclusions expressive of
the absolute truth or falsehood of any of the elementary propositions
designated by single symbols.}

Of conclusions expressed by equations involving two symbols, there
exists but the following, viz.:-- \textit{If the world is merely material,
and not subject to a presiding Intelligence, gravitation is not
necessarily absent.} This conclusion is expressed by the equation

\[tz =0, \textrm{ whence } z = q (1 - t).\]

If in the above analysis we suppress the concluding premiss, expressing
the fact of the existence of motion, and leave the hypothetical
principles which are embodied in the remaining premises
untouched, some remarkable conclusions follow. To these I
shall direct attention in the following chapter.

10. Of the remainder of Dr. Clarke's argument I shall briefly
state the substance and connexion, dwelling only on certain
% End of Part I
%-----------------------File: 001.png-----------------------------
% Beginning of Part II
portions of it which are of a more complex character than the others,
and afford better illustrations of the method of this work.

In Prop. \textsc{iv}. it is shown that the substance or essence of the
self-existent being is incomprehensible. The tenor of the reasoning
employed is, that we are ignorant of the essential nature of
all other things,--much more, then, of the essence of the
self-existent being.

In Prop. \textsc{v}. it is contended that ``though the substance or
essence of the self-existent being is itself absolutely incomprehensible
to us, yet many of the essential attributes of his nature
are strictly demonstrable, as well as his existence.''

In Prop. \textsc{vi}. it is argued that ``the self-existent being must
of necessity be infinite and omnipresent;'' and it is contended
that his infinity must be ``an infinity of fulness as well as of
immensity.'' The ground upon which the demonstration proceeds
is, that an absolute necessity of existence must be independent
of time, place, and circumstance, free from limitation,
and therefore excluding all imperfection. And hence it is
inferred that the self-existent being must be ``a most simple,
unchangeable, incorruptible being, without parts, figure, motion,
or any other such properties as we find in matter.''

The premises actually employed may be exhibited as follows:

1. If a finite being is self-existent, it is a contradiction to
suppose it not to exist.

2. A finite being may, without contradiction, be absent from
one place.

3. That which may without contradiction be absent from one
place may without contradiction be absent from all places.

4. That which may without contradiction be absent from all
places may without contradiction be supposed not to exist.

Let us assume
\begin{eqnarray*}
x &=& \textrm{Finite beings.}\\
y &=& \textrm{Things self-existent.}\\
z &=& \textrm{Things which it is a contradiction to suppose not to exist.}\\
w &=& \textrm{Things which may be absent without contradiction from
one place.}\\
t &=& \textrm{Things which without contradiction may be absent from
every place.}
\end{eqnarray*}


%-----------------------File: 002.png----------------------------
We have on expressing the above, and eliminating the indefinite
symbols,
\begin{eqnarray}\setcounter{equation}{1}
&xy(1-z) =& 0.\\
&x(1-w)  =& 0.\\
&w(1-t)  =& 0.\\
&tz = 0. %* Not exactly same alignment as in original
\end{eqnarray}
Eliminating in succession $t$, $w$, and $z$, we get
\[
   xy = 0,
\]
\[
   \therefore y = \frac{0}{0}(1-x);
\]
the interpretation of which is,--\emph{Whatever is self-existent is infinite}.

In Prop.~\textsc{vii}.\ it is argued that the self-existent being must of
necessity be One. The order of the proof is, that the self-existent
being is ``necessarily existent,'' that ``necessity absolute in
itself is simple and uniform, and without any possible difference
or variety,'' that all ``variety or difference of existence'' implies
dependence; and hence that ``whatever exists necessarily is the
one simple essence of the self-existent being.''

The conclusion is also made to flow from the following premises:---

1. If there are two or more necessary and independent beings,
either of them may be supposed to exist alone.

2. If either may be supposed to exist alone, it is not a contradiction
to suppose the other not to exist.

3. If it is not a contradiction to suppose this, there are not
two necessary and independent beings.

Let us represent the elementary propositions as follows:--
\begin{eqnarray*}
x &=& \textrm{there exist two necessary independent beings.}\\
y &=& \textrm{either may be supposed to exist alone.}\\
z &=& \textrm{it is not a contradiction to suppose the other not to exist.}
\end{eqnarray*}

We have then, on proceeding as before,
\begin{eqnarray}\setcounter{equation}{1}
x(1-y) &=& 0.\\
y(1-z) &=& 0.\\
zx = 0.
\end{eqnarray}

%-----------------------File: 003.png----------------------------

Eliminating $y$ and $z$, we have
\[
x=0.
\]
Whence, \textit{There do not exist two necessary and independent beings}.

11. To the premises upon which the two previous propositions
rest, it is well known that Bishop Butler, who at the time of the
publication of the ``Demonstration,'' was a student in a non-conformist academy, made objection in some celebrated letters,
which, together with Dr.\ Clarke's replies to them, are usually
appended to editions of the work. The real question at issue is
the validity of the principle, that ``whatsoever is absolutely necessary
at all is absolutely necessary in every part of space, and
in every point of duration,''---a principle assumed in Dr.\ Clarke's
reasoning, and explicitly stated in his reply to Butler's first letter.
In his second communication Butler says: ``I do not conceive
that the idea of ubiquity is contained in the idea of self-existence, or \textit{directly follows from it}, any otherwise than as whatever exists must exist \textit{somewhere}.'' That is to say, necessary
existence implies existence in some part of space, but not in
every part. It does not appear that Dr.\ Clarke was ever able to
dispose effectually of this objection. The whole of the correspondence is extremely curious and interesting. The objections
of Butler are precisely those which would occur to an acute mind
impressed with the conviction, that upon the sifting of first principles,
rather than upon any mechanical dexterity of reasoning,
the successful investigation of truth mainly depends. And the
replies of Dr.\ Clarke, although they cannot be admitted as satisfactory,
evince, in a remarkable degree, that peculiar intellectual
power which is manifest in the work from which the discussion
arose.

12. In Prop.~\textsc{viii}.\ it is argued that the self-existent and original cause of all things must be an Intelligent Being.

The main argument adduced in support of this proposition is,
that as the cause is more excellent than the effect, the self-existent being, as the cause and original of all things, must contain
in itself the perfections of all things; and that Intelligence
is one of the perfections manifested in a part of the creation. It
is further argued that this perfection is not a modification of
%-----------------------File: 004.png----------------------------
figure, divisibility, or any of the known properties of matter;
for these are not perfections, but \textit{limitations}. To this is added
the \textit{\`{a} posteriori} argument from the manifestation of design in the
frame of the universe.

There is appended, however, a distinct argument for the
existence of an intelligent self-existent being, founded upon the
ph{\ae}nomenal existence of motion in the universe. I shall briefly
exhibit this proof, and shall apply to it the method of the present
treatise.

The argument, omitting unimportant explanations, is as follows:--

"'Tis evident there is some such a thing as motion in the
world; which either began at some time or other, or was eternal.
If it began in time, then the question is granted that the first
cause is an intelligent being.... On the contrary, if motion was
eternal, either it was eternally caused by some eternal intelligent
being, or it must of itself be necessary and self-existent, or else,
without any necessity in its own nature, and without any external
necessary cause, it must have existed from eternity by an endless
successive communication. If motion was eternally caused by
some eternal intelligent being, this also is granting the question
as to the present dispute. If it was of itself necessary and self-existent,
then it follows that it must be a contradiction in terms
to suppose any matter to be at rest. And yet, at the same time,
because the determination of this self-existent motion must be
every way at once, the effect of it would be nothing else but a
perpetual rest.... But if it be said that motion, without any necessity
in its own nature, and without any external necessary
cause, has existed from eternity merely by an endless successive
communication, as Spinoza inconsistently enough seems to assert,
this I have before shown (in the proof of the second general
proposition of this discourse) to be a plain contradiction. It remains,
therefore, that motion must of necessity be originally
caused by something that is intelligent."

The premises of the above argument may be thus disposed:

1. If motion began in time, the first cause is an intelligent
being.
%-----------------------File: 005.png----------------------------
2. If motion has existed from eternity, either it has been
eternally caused by some eternal intelligent being, or it is self-existent,
or it must have existed by endless successive communication.

3. If motion has been eternally caused by an eternal intelligent
being, the first cause is an intelligent being.

4. If it is self-existent, matter is at rest and not at rest.

5. That motion has existed by endless successive communication,
and that at the same time it is not self-existent, and has
not been eternally caused by some eternal intelligent being, is
false.

To express these propositions, let us assume---
\begin{eqnarray*}
  x &=& \textrm{Motion began in time (and therefore)}\\
1-x &=& \textrm{Motion has existed from eternity.}\\
  y &=& \textrm{The first cause is an intelligent being.}\\
  p &=& \textrm{Motion has been eternally caused by some eternal intelligent being.}\\
  q &=& \textrm{Motion is self-existent.}\\
  r &=& \textrm{Motion has existed by endless successive communication.}\\
  s &=& \textrm{Matter is at rest.}
\end{eqnarray*}
The equations of the premises then are---
\[
\begin{aligned}
  &x = vy.\\
1-&x = v \left\{p\left(1-q\right)\left(1-r\right) +q\left(1-p\right) \left(1-r\right) +r\left(1-p\right)\left(1-q\right)\right\}.\\
  &p = vy.\\
  &q = vs\left(1-s\right) = 0.\\
  &r\left(1-q\right)\left(1-p\right) = 0.
\end{aligned}
\]
Since, by the fourth equation, $q = 0$, we obtain, on substituting
for $q$ its value in the remaining equations, the system
%\begin{eqnarray*}
%  x &=& vy,\\
%  p &=& vy,\\
%1-x &=& v\left\{p\left(1-r\right)+r\left(1-p\right)\right\},\\
%     r\left(1-p\right) &=& 0,
%\end{eqnarray*}
%**[2nd proofer: Alternate version:]
 \[
 \begin{aligned}
   x &= vy,  \\
   p &= vy,
 \end{aligned}
 \qquad
 \begin{gathered}
   1-x = v\left\{p\left(1-r\right) +r\left(1-p\right)\right\}, \\
   r\left(1-p\right) = 0,
 \end{gathered}
 \]
from which eliminating the indefinite symbols $v$, we have the
final reduced system,
\setcounter{equation}{0}
\begin{eqnarray}
x\left(1-y\right)=0,\\
\left(1-x\right)\left\{pr + \left(1-p\right)\left(1-r\right)\right\}=0,\\
p\left(1-y\right)=0.\\
r\left(1-p\right)=0.
\end{eqnarray}
%-----------------------File: 006.png----------------------------
We shall first seek the value of $y$, the symbol involved in
Dr.\ Clarke's conclusion. First, eliminating $x$ from (1) and (2), we
have
\begin{equation}
(1-y) \lbrace pr + (1-p)(1-r) \rbrace = 0.
\end{equation}
Next, to eliminate $r$ from (4) and (5), we have
\begin{gather*}
r(l-p) + (1-y) \lbrace pr + (1-p)(1-r) \rbrace = 0,   \\
\therefore \lbrace 1 - p + (1-y)p \rbrace \times (1-y)(1-p) = 0;
\end{gather*}
whence
\begin{equation}
(1-y)(1-p) = 0.
\end{equation}

Lastly, eliminating $p$ from (3) and (6), we have
\begin{eqnarray*}
       1 - y &=& 0,   \\
\therefore y &=& 1,
\end{eqnarray*}

which expresses the required conclusion, \emph{The first cause is an
intelligent being}.

Let us now examine what other conclusions are deducible
from the premises.

If we substitute the value just found for $y$ in the equations
(1), (2), (3), (4), they are reduced to the following pair of equations,
viz.,
\begin{equation*}
(1-x) \lbrace pr + (l-p)(l-r) \rbrace = 0,\quad r(l-p) = 0.   \tag{7}
\end{equation*}
Eliminating from these equations $x$, we have
\begin{equation*}
r(1-p) = 0, \text{ whence }r = vp,
\end{equation*}
which expresses the conclusion, \emph{If motion has existed by endless
successive communication, it has been eternally caused by an eternal
intelligent being}.

Again eliminating, from the given pair, $r$, we have
\begin{gather*}
(1-x)(1-p) = 0,   \\
\textrm{or, }\hfill 1-x = vp,  \hfill
\end{gather*}
which expresses the conclusion, \emph{If motion has existed from eternity,
it has been eternally caused by some eternal intelligent being}.

Lastly, from the same original pair eliminating $p$, we get
\begin{equation*}
(1-x)r = 0,
\end{equation*}
which, solved in the form
\begin{equation*}
1-x = r(1-r),
\end{equation*}

%-----------------------File: 007.png----------------------------
gives the conclusion, \emph{If motion has existed from eternity, it has not
existed by an endless successive communication}.

Solved under the form
\begin{equation*}
r = vx,
\end{equation*}
the above equation leads to the equivalent conclusion, \emph{If motion
exists by an endless successive communication, it began in time}.

13. Now it will appear to the reader that the first and last of
the above four conclusions are inconsistent with each other. The
two consequences drawn from the hypothesis that motion exists
by an endless successive communication, viz., 1st, that it has
been eternally caused by an eternal intelligent being; 2ndly, that
it began in time,---are plainly at variance. Nevertheless, they are
both rigorous deductions from the original premises. The opposition
between them is not of a \emph{logical}, but of what is technically
termed a \emph{material}, character. This opposition might, however,
have been formally stated in the premises. We might have
added to them a formal proposition, asserting that ``whatever is
\emph{externally} caused by an eternal intelligent being, does not begin in
time.'' Had this been done, no such opposition as now appears
in our conclusions could have presented itself. Formal logic
can only take account of relations which are formally expressed
(VI.~16); and it may thus, in particular instances, become necessary
to express, in a formal manner, some connexion among
the premises which, without actual statement, is involved in the
very meaning of the language employed.

To illustrate what has been said, let us add to the equations
(2) and (4) the equation
\begin{equation*}
px = 0,
\end{equation*}
which expresses the condition above adverted to. We have
\begin{equation*}
(1-x) \lbrace pr + (1-p)(1-r) \rbrace + r(1-p)+ px = 0.  \tag{8}
\end{equation*}
Eliminating $p$ from this, we find simply
\begin{equation*}
r = 0,
\end{equation*}
which expresses the proposition, \emph{Motion does not exist by an endless
successive communication}. If now we substitute for $r$ its value
in (8), we have
\begin{equation*}
(1-x)(1-p) +  px = 0,\text{ or, }1-x = p;
\end{equation*}

%-----------------------File: 008.png----------------------------

whence we have the interpretation, \emph{If motion has existed from
eternity, it has been eternally caused by an eternal intelligent being;}
together with the converse of that proposition.

In Prop.~\textsc{ix}.\ it is argued, that ``the self-existent and original
cause of all things is not a necessary agent, but a being endued
with liberty and choice.'' The proof is based mainly upon his
possession of intelligence, and upon the existence of final causes,
implying design and choice. To the objection that the supreme
cause operates by necessity for the production of what is best, it
is replied, that this is a necessity of fitness and wisdom, and not
of nature.

14. In Prop.~\textsc{x}.\ it is argued, that ``the self-existent being,
the supreme cause of all things, must of necessity have infinite
power.'' The ground of the demonstration is, that as ``all the
powers of all things are derived from him, nothing can make any
difficulty or resistance to the execution of his will.'' It is defined
that the infinite power of the self-existent being does not
extend to the ``making of a thing which implies a contradiction,''
or the doing of that ``which would imply imperfection (whether
natural or moral) in the being to whom such power is ascribed,''
but that it does extend to the creation of matter, and of an immaterial,
cogitative substance, endued with a power of beginning
motion, and with a liberty of will or choice. Upon this doctrine
of liberty it is contended that we are able to give a satisfactory
answer to ``that ancient and great question, \textgreek{p'ojen t`o kak`on},
what is the cause and original of evil?'' The argument on this
head I shall briefly exhibit,

``All that we call evil is either an evil of imperfection, as the
want of certain faculties or excellencies which other creatures
have; or natural evil, as pain, death, and the like; or moral evil,
as all kinds of vice. The first of these is not properly an evil;
for every power, faculty, or perfection, which any creature enjoys,
being the free gift of God,\dots it is plain the want of any certain
faculty or perfection in any kind of creatures, which never belonged
to their natures is no more an evil to them, than their
never having been created or brought into being at all could properly
have been called an evil. The second kind of evil, which
we call natural evil, is either a necessary consequence of the
%-----------------------File: 009.png----------------------------
former, as death to a creature on whose nature immortality was
never conferred; and then it is no more properly an evil than the
former. Or else it is counterpoised on the whole with as great
or greater good, as the afflictions and sufferings of good men,
and then also it is not properly an evil; or else, lastly, it is a
punishment, and then it is a necessary consequence of the third
and last kind of evil, viz., moral evil. And this arises wholly
from the abuse of liberty which God gave to His creatures for
other purposes, and which it was reasonable and fit to give them
for the perfection and order of the whole creation. Only they,
contrary to God's intention and command, have abused what was
necessary to the perfection of the whole, to the corruption and
depravation of themselves. And thus all sorts of evils have entered into the world without any diminution to the infinite goodness of the Creator and Governor thereof."---p. 112.

The main premises of the above argument may be thus
stated:

1st. All reputed evil is either evil of imperfection, or natural
evil, or moral evil.

2nd. Evil of imperfection is not absolute evil.

3rd. Natural evil is either a consequence of evil of imperfection, or it is compensated with greater good, or it is a consequence of moral evil.

4th. That which is either a consequence of evil of imperfection, or is compensated with greater good, is not absolute evil.

5th. All absolute evils are included in reputed evils.

To express these premises let us assume---
\begin{align*}
  w &= \text{ reputed evil.}                         \\
  x &= \text{ evil of imperfection.}                 \\
  y &= \text{ natural evil.}                         \\
  z &= \text{ moral evil.}                           \\
  p &= \text{ consequence of evil of imperfection.}  \\
  q &= \text{ compensated with greater good.}        \\
  r &= \text{ consequence of moral evil.}            \\
  t &= \text{ absolute evil.}
\end{align*}
Then, regarding the premises as Primary Propositions, of which
%-----------------------File: 010.png----------------------------
all the predicates are particular, and the conjunctions \emph{either, or},
as absolutely disjunctive, we have the following equations:
\begin{eqnarray*}
 w = v\left\{x(1-y)(1-q) + y(1-x)(1-z) + z(1-x)(1-y)\right\} \hfill\\
\hfill x = v(1 - t).\\
%
y = v \left\{p (1-q)(1-r) + q (1-p)(1-r) + r(1-p)(1-q)\right\}\hfill\\
\hfill p(l-q) + q(l-p) = v(1-t).\\
t = vw\,.
\end{eqnarray*}
From which, if we separately eliminate the symbol $v$, we have
\setcounter{equation}{0}
\begin{eqnarray}
w\left\{
   1 - x(1-y)(1-z) - y(1-x)(1-z) - z(1-x)(1-y)
 \right\} =0,  \\
xt = 0,        \\
y\left\{
   1 - p(1-q)(1-r) - q(1-p)(1-r) - r(1-p)(1-q)
 \right\} =0,  \\
\left\{ p(1-q) + q(1-p)\right\}t = 0,\\
t(1-w) = 0.
\end{eqnarray}

Let it be required, first, to find what conclusion the premises
warrant us in forming respecting absolute evils, as concerns their
dependence upon moral evils, and the consequences of moral
evils.

For this purpose we must determine $t$ in terms of $z$ and $r$.

The symbols $w$, $x$, $y$, $p$, $q$ must therefore be eliminated. The
process is easy, as any set of the equations is reducible to a single
equation by addition.

Eliminating $w$ from (1) and (5), we have
\[
t\left\{
   1 - x(1-y)(1-z) - y(1-x)(1-z) - z(1-x)(1-y)
 \right\} = 0.\tag{6}
\]
The elimination of $p$ from (3) and (4) gives
\[
 yqr + yqt + yt(1-r)(1-q) = 0. \tag{7}
\]
The elimination of $q$ from this gives
\[
 yt(1-r) = 0. \tag{8}
\]
The elimination of $x$ between (2) and (6) gives
\[
 t\left\{ yz + (1-y)(1-z) \right\} = 0. \tag{9}
\]
The elimination of $y$ from (8) and (9) gives
\[
 t(1-z)(1-r) = 0.
\]
This is the only relation existing between the elements $t$, $z$, and $r$.
%-----------------------File: 011.png----------------------------
We hence get
\begin{align*}
  t &= \frac{0}{(1-z)(1-r)}   \\
    &= \frac{0}{0}zr + \frac{0}{0}z(1-r)
     + \frac{0}{0}(1-z)r + 0(1-z)(1-r)   \\
    &= \frac{0}{0}z + \frac{0}{0}(1-z)r;
\end{align*}
the interpretation of which is, \emph{Absolute evil is either moral evil, or
it is, if not moral evil, a consequence of moral evil.}

Any of the results obtained in the process of the above solution furnish us with interpretations. Thus from (8) we might
deduce
\begin{multline*}
  t = \frac{0}{y(1-r)} = \frac{0}{0}yr
    + \frac{0}{0}(1-y)r + \frac{0}{0}(1-y)(1-r)   \\
  = \frac{0}{0}yr + \frac{0}{0}(1-y);
\end{multline*}
whence, \emph{Absolute evils are either natural evils, which are the consequences of moral evils, or they are not natural evils at all.}

A variety of other conclusions may be deduced from the given
equations in reply to questions which may be arbitrarily proposed. Of such I shall give a few examples, without exhibiting
the intermediate processes of solution.

Quest. 1.---Can any relation be deduced from the premises
connecting the following elements, viz.: absolute evils, consequences of evils of imperfection, evils compensated with greater
good?

Ans.---\emph{No relation exists.} If we eliminate all the symbols but
$z$, $p$, $q$, the result is $0 = 0$.

Quest. 2.---Is any relation implied between absolute evils,
evils of imperfection, and consequences of evils of imperfection.

Ans.---The final relation between $x$, $t$, and $p$ is
\[
  xt + pt = 0;
\]
whence
\[
  t = \frac{0}{p + x} = \frac{0}{0}(1-p)(1-x).
\]
Therefore, \emph{Absolute evils are neither evils of imperfection, nor consequences of evils of imperfection.}
%-----------------------File: 012.png----------------------------
Quest. 3. --- Required the relation of natural evils to evils of
imperfection and evils compensated with greater good.

We find
\begin{gather*}
  pqy = 0,  \\
  \therefore y = \frac{0}{pq} = \frac{0}{0}p(1-q) + \frac{0}{0}(1-p).
\end{gather*}
Therefore, \emph{Natural evils are either consequences of evils of imperfection which are not compensated with greater good, or they are not
consequences of evils of imperfection at all.}

Quest. 4. --- In what relation do those natural evils which are
not moral evils stand to absolute evils and the consequences of
moral evils?

If $y(1-z) = s$, we find, after elimination,
\begin{gather*}
  ts(1-r) = 0;   \\
 \therefore s = \frac{0}{t(1-r)} = \frac{0}{0}tr + \frac{0}{0}(1-t).
\end{gather*}
Therefore, \emph{Natural evils, which are not moral evils, are either absolute evils, which are the consequences of moral evils, or they are not
absolute evils at all.}

The following conclusions have been deduced in a similar
manner. The subject of each conclusion will show of what particular things a description was required, and the predicate will
show what elements it was designed to involve: ---

\emph{Absolute evils, which are not consequences of moral evils, are
moral and not natural evils.}

\emph{Absolute evils which are not moral evils are natural evils, which
are the consequences of moral evils.}

\emph{Natural evils which are not consequences of moral evils are not
absolute evils.}

Lastly, let us seek a description of evils which are not absolute, expressed in terms of natural and moral evils.

We obtain as the final equation,
\[
  1 - t = yz + \frac{0}{0}y(1-z) + \frac{0}{0}(1-y)z + (1-y)(1-z).
\]
The direct interpretation of this equation is a necessary truth,
but the reverse interpretation is remarkable. \emph{Evils which are both}
%[emphasized sentence continues]
%-----------------------File: 013.png----------------------------
\emph{natural and moral, and evils which are neither natural nor moral,
are not absolute evils.}

This conclusion, though it may not express a truth, is certainly involved in the given premises, as \emph{formally} stated.

15. Let us take from the same argument a somewhat fuller
system of premises, and let us in those premises suppose that the
particles, \emph{either, or,} are not absolutely disjunctive, so that in the
meaning of the expression, ``either evil of imperfection, or natural evil, or moral evil,'' we include whatever possesses one or
more of these qualities.

Let the premises be ---

1. All evil ($w$) is either evil of imperfection ($x$), or natural
evil ($y$), or moral evil ($z$).

2. Evil of imperfection ($x$) is not absolute evil ($t$).

3. Natural evil ($y$) is either a consequence of evil of imperfection ($p$), or it is compensated with greater good ($q$), or it is a
consequence of moral evil ($r$).

4. Whatever is a consequence of evil of imperfection ($p$) is
not absolute evil ($t$).

5. Whatever is compensated with greater good ($q$) is not
absolute evil ($t$).

6. Moral evil ($z$) is a consequence of the abuse of liberty ($u$).

7. That which is a consequence of moral evil ($r$) is a consequence of the abuse of liberty ($u$).

8. Absolute evils are included in reputed evils.

The premises expressed in the usual way give, after the elimination of the indefinite symbols $v$, the following equations:
\begin{gather*}
  w(1-x)(1-y)(1-z) = 0,    \tag{1}   \\
  xt = 0,                  \tag{2}   \\
  y(1-p)(1-q)(1-r) = 0,    \tag{3}   \\
  pt = 0,                  \tag{4}   \\
  qt = 0,                  \tag{5}   \\
  z(1-u) = 0,              \tag{6}   \\
  r(1-u) = 0,              \tag{7}   \\
  t(1-w) = 0.              \tag{8}
\end{gather*}
Each of these equations satisfies the condition $V(1-V) = 0$.

%-----------------------File: 014.png----------------------------
The following results are easily deduced ---

\emph{Natural evil is either absolute evil, which is a consequence of moral evil, or it is not absolute evil at all.}

\emph{All evils are either absolute evils, which are consequences of the
abuse of liberty, or they are not absolute evils.}

\emph{Natural evils are either evils of imperfection, which are not absolute evils, or they are not evils of imperfection at all.}

\emph{Absolute evils are either natural evils, which are consequences of
the abuse of liberty, or they are not natural evils, and at the same
time not evils of imperfection.}

\emph{Consequences of the abuse of liberty include all natural evils
which are absolute evils, and are not evils of imperfection, with an
indefinite remainder of natural evils which are not absolute, and of
evils which are not natural.}

16. These examples will suffice for illustration. The reader
can easily supply others if they are needed. We proceed now to
examine the most essential portions of the demonstration of
Spinoza.

\begin{center}\textsc{definitions.}\end{center}

1. By a \emph{cause of itself (causa sui)}, I understand that of which
the essence involves existence, or that of which the nature cannot be conceived except as existing.

2. That thing is said to be finite or bounded in its own kind
\emph{(in suo genere finita)} which may be bounded by another thing of
the same kind; e.~g. Body is said to be finite, because we can
always conceive of another body greater than a given one. So
thought is bounded by other thought. But body is not bounded
by thought, nor thought by body.

3. By substance, I understand that which is in itself \emph{(in se)},
and is conceived by itself \emph{(per se concipitur)}, i.e., that whose
conception does not require to be formed from the conception of
another thing.

4. By attribute, I understand that which the intellect perceives in
substance, as constituting its very essence.

5. By mode, I understand the affections of substance, or that
which is in another thing, by which thing also it is conceived.

6. By God, I understand the Being absolutely infinite, that
%-----------------------File: 015.png----------------------------
is the substance consisting of infinite attributes, each of which
expresses an eternal and infinite essence.

\emph{Explanation.}---I say absolutely infinite, not infinite in its
own kind. For to whatever is only infinite in its own kind we
may deny the possession of (some) infinite attributes. But when
a thing is absolutely infinite, whatsoever expresses essence and
involves no negation belongs to its essence.

7. That thing is termed \emph{free}, which exists by the sole necessity
of its own nature, and is determined to action by itself alone;
\emph{necessary}, or rather constrained, which is determined by another
thing to existence and action, in a certain and determinate manner.

8. By eternity, I understand existence itself, in so far as it is
conceived necessarily to follow from the sole definition of the
eternal thing.

\emph{Explanation.}---For such existence, as an eternal truth, is
conceived as the essence of the thing, and therefore cannot be explained
by mere duration or time, though the latter should be
conceived as without beginning and without end.

\begin{center}\textsc{axioms.}\end{center}

1. All things which exist are either in themselves \emph{in se} or
in another thing.

2. That which cannot be conceived by another thing ought
to be conceived by itself.

3. From a given determinate cause the effect necessarily follows, and,
contrariwise, if no determinate cause be granted, it is
impossible that an effect should follow.

4. The knowledge of the effect depends upon, and involves,
the knowledge of the cause.

5. Things which have nothing in common cannot be understood by means of
each other; or the conception of the one does
not involve the conception of the other.

6. A true idea ought to agree with its own object. (\emph{Idea
vera debet cum suo ideato convenire.})

7. Whatever can be conceived as non-existing does not involve existence in its essence.
%-----------------------File: 016.png----------------------------
Other definitions are implied, and other axioms are virtually
assumed, in some of the demonstrations. Thus, in Prop. I.,
``Substance is prior in nature to its affections," the proof of
which consists in a mere reference to Defs. 3 and 5, there seems
to be an assumption of the following axiom, viz., ``That by which
a thing is conceived is prior in nature to the thing conceived."
Again, in the demonstration of Prop. V. the converse of this
axiom is assumed to be true. Many other examples of the same
kind occur. It is impossible, therefore, by the mere processes of
Logic, to deduce the whole of the conclusions of the first book of
the Ethics from the axioms and definitions which are prefixed to
it, and which are given above. In the brief analysis which will
follow, I shall endeavour to present in their proper order what
appear to me to be the real premises, whether formally stated or
implied, and shall show in what manner they involve the conclusions to which Spinoza was led.

17. I conceive, then, that in the course of his demonstration,
Spinoza effects several parallel divisions of the universe of possible existence, as,

1st. Into things which are in themselves, $x$, and things which
are in some other thing, $x'$; whence, as these classes of thing together make up the universe, we have
\begin{eqnarray*}
              x + x'= 1; \textrm{ (Ax. \textsc{i}.)}\\
  \textrm{or, }         x = 1 - x'.
\end{eqnarray*}

2nd. Into things which are conceived by themselves, $y$, and
things which are conceived through some other thing,$y'$;
whence
\[
  y = 1 - y'. \textrm{ (Ax. \textsc{ii})}
\]

3rd. Into substance, $z$, and modes, $z'$; whence
\[
  z = 1 - z'. \textrm{ (Def. \textsc{iii. v.})}
\]

4th. Into things free, $f$, and things necessary,$f'$; whence
\[
  f = 1 - f'.   \textrm{ (Def. \textsc{vii.})}
\]

5th. Into things which are causes and self-existent, $e$, and
things caused by some other thing, $e'$; whence
\[
  e = 1 - e'. \textrm{ (Def. \textsc{i}. Ax. \textsc{vii.})}
\]
%-----------------------File: 017.png----------------------------
And his reasoning proceeds upon the expressed or assumed
principle, that these divisions are not only parallel, but equivalent. Thus in Def.~\textsc{iii.}, Substance is made equivalent with that
which is conceived by itself; whence
\[
  z = y .
\]
Again, Ax.~\textsc{iv}., as it is actually applied by Spinoza, establishes the identity of cause with that by which a thing is conceived; whence
\[
y = e.
\]
Again, in Def.~\textsc{vii}., things free are identified with things
self-existent; whence
\[
  f = e.
\]
Lastly, in Def.~\textsc{v} mode is made identical with that which is
in another thing; whence $z' = x'$, and therefore,
\[
  z=x.
\]
All these results may be collected together into the following
series of equations, viz.:
\[
x = y = z = f= e = 1-x' = 1-y' = 1-f'= 1-z' = 1-e'.
\]
And any two members of this series connected together by the
sign of equality express a conclusion, whether drawn by Spinoza
or not, which is a legitimate consequence of his system. Thus
the equation
\[
z=1-e',
\]
expresses the sixth proposition of his system, viz., One substance
cannot be produced by another. Similarly the equation

\[
z = e,
\]
expresses his seventh proposition, viz., ``It pertains to the nature
of substance to exist.'' This train of deduction it is unnecessary
to pursue. Spinoza applies it chiefly to the deduction according
to his views of the properties of the Divine Nature, having first
endeavoured to prove that the only substance is God. In the
steps of this process, there appear to me to exist some fallacies,
dependent chiefly upon the ambiguous use of words, to which it
will be necessary here to direct attention.
%-----------------------File: 018.png----------------------------
18. In Prop.~\textsc{v}.\ it is endeavoured to show, that ``There cannot
exist two or more substances of the same nature or attribute.''
The proof is virtually as follows: If there are more substances
than one, they are distinguished either by attributes or modes;
if by attributes, then there is only one substance of the same attribute;
if by modes, then, laying aside these as non-essential,
there remains no \emph{real} ground of distinction. Hence there exists
but one substance of the same attribute. The assumptions here
involved are inconsistent with those which are found in other
parts of the treatise. Thus substance, Def.~\textsc{iv}., is apprehended
by the intellect through the means of attribute. By Def.~\textsc{vi}.\ it
may have many attributes. One substance may, therefore, \emph{conceivably}
be distinguished from another by a difference in some of
its attributes, while others remain the same.

In Prop.~\textsc{viii}.\ it is attempted to show that, All substance
is necessarily infinite. The proof is as follows. There exists
but one substance, of one attribute, Prop.~\textsc{v}.; and it pertains
to its nature to exist, Prop.~\textsc{vii}. It will, therefore, be of its
nature to exist either as finite or infinite. But not as finite, for,
by Def.~\textsc{ii}.\ it would require to be bounded by another substance
of the same nature, which also ought to exist \emph{necessarily}, Prop.~\textsc{vii}. Therefore, there would be two substances of the same
attribute, which is absurd, Prop.~\textsc{v}. Substance, therefore, is
infinite.

In this demonstration the word ``finite'' is confounded with
the expression, ``Finite in its own kind,'' Def.~\textsc{ii}. It is thus assumed
that nothing can be finite, unless it is bounded by another
thing of the same kind. This is not consistent with the ordinary
meaning of the term. Spinoza's use of the term finite
tends to make space the only form of substance, and all existing
things but affections of space, and this, I think, is really one of
the ultimate foundations of his system.

The first scholium applied to the above Proposition is remarkable.
I give it in the original words: ``Quum finitum esse
revera sit ex parte negatio, et infinitum absoluta affirmatio existentiae
alicujus naturae, sequitur ergo ex sola Prop.~\textsc{vii}.\ omnem
substantiam debere esse infinitam.'' Now this is in reality an
assertion of the principle affirmed by Clarke, and controverted by

%-----------------------File: 019.png----------------------------
Butler (XIII.~11), that necessary existence implies existence
in every part of space. Probably this principle will be found to
lie at the basis of every attempt to demonstrate, \textit{\`{a} priori}, the
existence of an Infinite Being.

From the general properties of substance above stated, and
the definition of God as the substance consisting of infinite attributes,
the peculiar doctrines of Spinoza relating to the Divine
Nature necessarily follow. As substance is self-existent, free,
causal in its very nature, the thing in which other things are,
and by which they are conceived; the same properties are also
asserted of the Deity. He is self-existent, Prop.~\textsc{xi}.; indivisible,
Prop.~\textsc{xiii}.; the only substance, Prop.~\textsc{xiv}.; the Being in
which all things are, and by which all things are conceived,
Prop.~\textsc{xv}.; free, Prop.~\textsc{xvii}.; the immanent cause of all things,
Prop.~\textsc{xviii}. The proof that God is the only substance is drawn
from Def.~\textsc{vi}., which is interpreted into a declaration that ``God
is the Being absolutely infinite, of whom no attribute which expresses
the essence of substance can be denied.'' Every conceivable
attribute being thus assigned by definition to Him, and
it being determined in Prop.~\textsc{v}.\ that there cannot exist two substances
of the same attribute, it follows that God is the only
substance.

Though the ``Ethics'' of Spinoza, like a large portion of his
other writings, is presented in the geometrical form, it does not
afford a good praxis for the symbolical method of this work.
Of course every train of reasoning admits, when its ultimate
premises are truly determined, of being treated by that method;
but in the present instance, such treatment scarcely differs, except
in the use of letters for words, from the processes employed
in the original demonstrations. Reasoning which consists so
largely of a play upon terms defined as equivalent, is not often
met with; and it is rather on account of the interest attaching to
the subject, than of the merits of the demonstrations, highly as
by some they are esteemed, that I have devoted a few pages
here to their exposition.

19. It is not possible, I think, to rise from the perusal of the
arguments of Clarke and Spinoza without a deep conviction of the
futility of all endeavours to establish, entirely \textit{\`{a} priori}, the existence
%-----------------------File: 020.png----------------------------
of an Infinite Being, His attributes, and His relation to the universe.
The fundamental principle of all such speculations, viz., that
whatever we can clearly conceive, must exist, fails to accomplish
its end, even when its truth is admitted. For how shall the finite
comprehend the infinite? Yet must the possibility of such conception
be granted, and in something more than the sense of
a mere withdrawal of the limits of phaenomenal existence, before
any solid ground can be established for the knowledge, \textit{\`{a} priori},
of things infinite and eternal. Spinoza's affirmation of the reality
of such knowledge is plain and explicit: ``Mens humana
adaequatum habet cognitionem aeternae et infinitae essentiae Dei''
(Prop.~\textsc{xlvii}., Part~2nd). Let this be compared with Prop.~\textsc{xxxiv}.,
Part~2nd: ``Omnis idea quae in nobis est absoluta
sive adaequata et perfecta, vera est;'' and with Axiom~\textsc{vi}., Part~1st,
``Idea vera debet cum suo ideato convenire.'' Moreover, this
species of knowledge is made the essential constituent of all other
knowledge: ``De natura rationis est res sub quadam aeternitatis
specie percipere'' (Prop.~\textsc{xliv}., Cor.~\textsc{ii}., Part~2nd). Were it
said, that there is a tendency in the human mind to rise in contemplation
from the particular towards the universal, from the
finite towards the infinite, from the transient towards the eternal;
and that this tendency suggests to us, with high probability, the
existence of more than sense perceives or understanding comprehends;
the statement might be accepted as true for at least a
%**[2nd proofer: duplicate " a " removed.]
large number of minds. There is, however, a class of speculations,
the character of which must be explained in part by
reference to other causes,---impatience of probable or limited
knowledge, so often all that we can really attain to; a desire for
absolute certainty where intimations sufficient to mark out before
us the path of duty, but not to satisfy the demands of the speculative
intellect, have alone been granted to us; perhaps, too,
dissatisfaction with the present scene of things. With the
undue predominance of these motives, the more sober procedure
of analogy and probable induction falls into neglect. Yet the latter
is, beyond all question, the course most adapted to our present
condition. To infer the existence of an intelligent cause
from the teeming evidences of surrounding design, to rise to the
conception of a moral Governor of the world, from the study of
%-----------------------File: 021.png----------------------------
the constitution and the moral provisions of our own nature;--
these, though but the feeble steps of an understanding limited
in its faculties and its materials of knowledge, are of more avail
than the ambitious attempt to arrive at a certainty unattainable
on the ground of natural religion. And as these were the most
ancient, so are they still the most solid foundations, Revelation
being set apart, of the belief that the course of this world is not
abandoned to chance and inexorable fate.
%-----------------------File: 022.png----------------------------



%CHAPTER XIV.

\setcounter{chapter}{13}
\chapter[EXAMPLE OF ANALYSIS]{\large EXAMPLE OF THE ANALYSIS OF A SYSTEM OF EQUATIONS BY THE
METHOD OF REDUCTION TO A SINGLE EQUIVALENT EQUATION
$V=0$, WHEREIN $V$ SATISFIES THE CONDITION $V(1-V)=0$.}

1. Let us take the remarkable system of premises employed
in the previous Chapter, to prove that ``Matter is not a
necessary being;'' and suppressing the 6th premiss, viz., Motion
exists,---examine some of the consequences which flow from the
remaining premises. This is in reality to accept as true Dr.\
Clarke's hypothetical principles; but to suppose ourselves ignorant
%*[2nd proofer: Misspelling corrected,-----------------------^^
%  as probably a printer's error.]
of the fact of the existence of motion. Instances may
occur in which such a selection of a portion of the premises of
an argument may lead to interesting consequences, though it is
with other views that the present example has been resumed. The
premises actually employed will be---

1. If matter is a necessary being, either the property of gravitation
is necessarily present, or it is necessarily absent.

2. If gravitation is necessarily absent, and the world is not
subject to any presiding intelligence, motion does not exist.

3. If gravitation is necessarily present, a vacuum is necessary.

4. If a vacuum is necessary, matter is not a necessary being.

5. If matter is a necessary being, the world is not subject
to a presiding intelligence.

If, as before, we represent the elementary propositions by the
following notation, viz.:
\begin{eqnarray*}%{rcl}
  x &=& \textrm{Matter is a necessary being.}   \\
  y &=& \textrm{Gravitation is necessarily present.}   \\
  w &=& \textrm{Motion exists.}   \\
  t &=& \textrm{Gravitation is necessarily absent.}   \\
  z &=& \textrm{The world is merely material, and not subject to
a presiding intelligence.}   \\
  v &=& \textrm{A vacuum is necessary.}
\end{eqnarray*}

%-----------------------File: 023.png----------------------------

We shall on expression of the premises and elimination of the
indefinite class symbols ($q$), obtain the following system of equations:
\begin{eqnarray*}
xyt + x\bar{y}\bar{t} &= 0,   \\
                  tzw &= 0,   \\
             y\bar{v} &= 0,   \\
                   vx &= 0,   \\
             x\bar{z} &= 0;
\end{eqnarray*}
in which for brevity $\bar{y}$ stands for $1-y$, $\bar{t}$ for $1-t$, and
so on; whence, also, $1-\bar{t} = t$, $1-\bar{y} = y$, \&c.

As the first members of these equations involve only positive
terms, we can form a single equation by adding them together
(VIII. Prop.~2), viz.:
\[
xyt + x\bar{y}\bar{t} + y\bar{v} + vx + x\bar{z} + tzw = 0,
\]
and it remains to reduce the first member so as to cause it to
satisfy the condition $V(1-V) = 0$.

For this purpose we will first obtain its development with
reference to the symbols $x$ and $y$. The result is---
\[
\begin{split}
  (t + \bar{v} + v + \bar{z} + tzw) xy
   + (\bar{t} + v + \bar{z} + tzw) x\bar{y}   \\
   + (\bar{v} + tzw) \bar{x}y + tzw\bar{x}\bar{y} = 0.
\end{split}
\]
And our object will be accomplished by reducing the four coefficients of the development to equivalent forms, themselves satisfying the condition required.

Now the first coefficient is, since $v + \bar{v} = 1$,
\[
1 + t + \bar{z} + tzw,
\]
which reduces to unity (IX. Prop.~1).

The second coefficient is
\[
  \bar{t} + v + \bar{z} + tzw;
\]
and its reduced form (X.~3) is
\[
  \bar{t} + tv + t\bar{v}\bar{z} + t\bar{v}zw.
\]

The third coefficient, $\bar{v} + tzw$, reduces by the same method
to $\bar{v} + tzwv$; and the last coefficient $tzw$ needs no reduction.
Hence the development becomes
%-----------------------File: 024.png----------------------------
\[
  xy
+ \left(\bar{t} + tv  + t\bar{v}\bar{z} + t\bar{v}zw\right)x\bar{y}
+ \left(\bar{v}  + tzwv\right) \bar{x}y
+ tzw\bar{x}\bar{y} = 0; \tag{1}
\]
and this is the form of reduction sought.

2. Now according to the principle asserted in Prop.~\textsc{iii}.,
Chap.~\textsc{x}., the whole relation connecting any particular set of the
symbols in the above equation may be deduced by developing
that equation with reference to the particular symbols in question,
and retaining in the result only those constituents whose coefficients
are unity. Thus, if $x$ and $y$ are the symbols chosen, we
are immediately conducted to the equation
\[
xy=0,
\]
whence we have
\[
y=\frac{0}{0}(1-x),
\]
with the interpretation, \emph{If gravitation is necessarily present, matter
is not a necessary being.}

Let us next seek the relation between $x$ and $w$. Developing
(1) with respect to those symbols, we get
\begin{multline*}
  \left(y + \bar{t}\bar{y} + tv\bar{y} + t\bar{v}\bar{z}\bar{y}
      + t\bar{v}z\bar{y}\right)xw
+ \left(y + \bar{t}\bar{y} + tv\bar{y}
      + t\bar{v}\bar{z}\bar{y}\right)x\bar{w}
\\
+ \left(\bar{v}y + tzvy + tz\bar{y}\right)\bar{x}w
+ \bar{v}y\bar{xw} = 0.
\end{multline*}
The coefficient of $xw$, and it alone, reduces to unity. For
$t\bar{v}\bar{z}\bar{y} + t\bar{v}z\bar{y} = t\bar{v}\bar{y}$, and
$tv\bar{y} + t\bar{v}\bar{y} = t\bar{y}$, and
$\bar{t}\bar{y} + t\bar{y} = \bar{y}$, and lastly,
$y + \bar{y} = 1$. This is always the mode in which such reductions
take place. Hence we get
\begin{gather*}
xw=0,
\\
\therefore w = \frac{0}{0}(1-x),
\end{gather*}
of which the interpretation is, \emph{If motion exists, matter is not a necessary
being.}

If, in like manner, we develop (1) with respect to $x$ and $z$,
we get the equation
\begin{align*}
x\bar{z} &= 0,
\\
\therefore x &= \frac{0}{0}z,
\end{align*}
with the interpretation, \emph{If matter is a necessary being, the world
is merely material, and without a presiding intelligence.}

%-----------------------File: 025.png----------------------------
This, indeed, is only the fifth premiss reproduced, but it
shows that there is no other relation connecting the two elements
which it involves.

If we seek the whole relation connecting the elements $x$, $w$,
and $y$, we find, on developing (1) with reference to those symbols,
and proceeding as before,
\begin{equation*}
xy + xw\bar{y} = 0.
\end{equation*}
Suppose it required to determine hence the consequences of the
hypothesis, ``Motion does not exist,'' relatively to the questions
of the necessity of matter, and the necessary presence of gravitation.
We find
\begin{gather*}
w=\frac{-xy}{x\bar{y}},
\\
\therefore 1-w=\frac{x}{x\bar{y}} =\frac{1}{0}xy+x\bar{y}+\frac{0}{0}\bar{x};
\\
\text{or, } \hfill
1-w=x\bar{y}+\frac{0}{0}\bar{x},\text{ with }xy=0.
\hfill
\end{gather*}
The direct interpretation of the first equation is, \textit{If motion does
not exist, either matter is a necessary being, and gravitation is not
necessarily present, or matter is not a necessary being}.

The reverse interpretation is, \textit{If matter is a necessary being,
and gravitation not necessary, motion does not exist}.

In exactly the same mode, if we sought the full relation between
$x$, $z$, and $w$, we should find
\begin{equation*}
xzw + x\bar{z} = 0.
\end{equation*}
From this we may deduce
\begin{equation*}
z=x\bar{w}+\frac{0}{0}\bar{x},\text{ with }xw=0.
\end{equation*}

Therefore, \textit{If the world is merely material, and not subject to
any presiding intelligence, either matter is a necessary being, and
motion does not exist, or matter is not a necessary being.}

Also, \textit{reversely, If matter is a necessary being, and there is no
such thing as motion, the world is merely material.}

3. We might, of course, extend the same method to the
%-----------------------File: 026.png----------------------------
determination of the consequences of any complex hypothesis $u$,
such as, ``The world is merely material, and without any presiding
intelligence ($z$), but motion exists'' ($w$), with reference to
any other elements of doubt or speculation involved in the original
premises, such as, ``Matter is a necessary being'' ($x$), ``Gravitation
is a necessary quality of matter,'' ($y$). We should, for
this purpose, connect with the general equation (1) a new
equation,
\begin{equation*}
u = wz,
\end{equation*}
reduce the system thus formed to a single equation, $V=0$, in
which $V$ satisfies the condition $V(1-V) = 0$, and proceed as
above to determine the relation between $u$, $x$, and $y$, and finally $u$
as a developed function of $x$ and $y$. But it is very much better
to adopt the methods of Chapters \textsc{viii.} and \textsc{ix.} I shall here
simply indicate a few results, with the leading steps of their deduction,
and leave their verification to the reader's choice.

In the problem last mentioned we find, as the relation connecting
$x$, $y$, $w$, and $z$,
\begin{equation*}
xw + x\bar{w}y + x\bar{w}\bar{y}\bar{z} = 0.
\end{equation*}
And if we write $u = xy$, and then eliminate the symbols $x$ and $y$
by the general problem, Chap.~\textsc{ix.}, we find
\begin{equation*}
xu + xy\bar{u} = 0,
\end{equation*}
whence
\begin{equation*}
u=\frac{1}{0}xy+0x\bar{y}+\frac{0}{0}\bar{x};
\end{equation*}
wherefore
\begin{equation*}
wz=\frac{0}{0}\bar{x}\text{ with }xy=0.
\end{equation*}
Hence, \textit{If the world is merely material, and without a presiding
intelligence, and at the same time motion exists, matter is not a necessary
being.}

Now it has before been shown that \textit{if motion exists, matter is
not a necessary being}, so that the above conclusion tells us even
less than we had before ascertained to be (inferentially) true.
Nevertheless, that conclusion is the proper and complete answer
to the question which was proposed, which was, to determine
simply the consequences of a certain complex hypothesis.
%-----------------------File: 027.png----------------------------
4. It would thus be easy, even from the limited system of
premises before us, to deduce a great variety of additional inferences,
involving, in the conditions which are given, any proposed
combinations of the elementary propositions. If the condition
is one which is inconsistent with the premises, the fact
will be indicated by the form of the solution. The value which
the method will assign to the combination of symbols expressive
of the proposed condition will be 0. If, on the other hand, the
fulfilment of the condition in question imposes no restriction upon
the propositions among which relation is sought, so that every
combination of those propositions is equally possible,---the fact
will also be indicated by the form of the solution. Examples
of each of these cases are subjoined.

If in the ordinary way we seek the consequences which would
flow from the condition that \textit{matter is a necessary being}, and at
the same time that \textit{motion exists}, as affecting the Propositions,
\textit{The world is merely material, and without a presiding intelligence},
and, \textit{Gravitation is necessarily present}, we shall obtain the equation
\begin{equation*}
xw = 0,
\end{equation*}
which indicates that the condition proposed is inconsistent with
the premises, and therefore cannot be fulfilled.

If we seek the consequences which would flow from the condition
that \textit{Matter is not a necessary being}, and at the same time
that \textit{Motion does exist}, with reference to the same elements as
above, viz., \textit{the absence of a presiding intelligence}, and the \textit{necessity
of gravitation},--we obtain the following result,
\begin{equation*}
(1-x)w =\frac{0}{0}yz +\frac{0}{0}y(1-z) +\frac{0}{0}(1-y)z +\frac{0}{0}(1-y)(1-z),
\end{equation*}
which might literally be interpreted as follows:

\textit{If matter is not a necessary being, and motion exists, then
either the world is merely material and without a presiding intelligence,
and gravitation is necessary, or one of these two results follows
without the other, or they both fail of being true.} Wherefore
of the four possible combinations, of which some one is true of
necessity, and of which of necessity one only can be true, it is
%-----------------------File: 028.png----------------------------
affirmed that any one may be true. Such a result is a truism---
a mere \textit{necessary} truth. Still it contains the only answer which
can be given to the question proposed.

I do not deem it necessary to vindicate against the charge of
laborious trifling these applications. It may be requisite to enter
with some fulness into details useless in themselves, in order
to establish confidence in general principles and methods. When
this end shall have been accomplished in the subject of the present
inquiry, let all that has contributed to its attainment, but
has afterwards been found superfluous, be forgotten.
%-----------------------File: 029.png----------------------------
%CHAPTER XV.

\chapter[ARISTOTELIAN LOGIC]{\large THE ARISTOTELIAN LOGIC AND ITS
MODERN EXTENSIONS, EXAMINED BY THE METHOD OF THIS TREATISE.}

1. The logical system of Aristotle, modified in its details,
but unchanged in its essential features, occupies so important
a place in academical education, that some account of its
nature, and some brief discussion of the leading problems which
it presents, seem to be called for in the present work. It is, I
trust, in no narrow or harshly critical spirit that I approach this
task. My object, indeed, is not to institute any direct comparison
between the time-honoured system of the schools and that
of the present treatise; but, setting truth above all other considerations,
to endeavour to exhibit the real nature of the ancient
doctrine, and to remove one or two prevailing misapprehensions
respecting its extent and sufficiency.

That which may be regarded as essential in the spirit and
procedure of the Aristotelian, and of all cognate systems of Logic,
is the attempted classification of the allowable forms of inference,
and the distinct reference of those forms, collectively or individually,
to some general principle of an axiomatic nature, such as
the ``dictum of Aristotle:'' Whatsoever is affirmed or denied of
the genus may in the same sense be affirmed or denied of any
species included under that genus. Concerning such general
principles it may, I think, be observed, that they either state directly,
but in an abstract form, the argument which they are
supposed to elucidate, and, so stating that argument, affirm its
validity; or involve in their expression technical terms which,
after definition, conduct us again to the same point, viz.,
the abstract statement of the supposed allowable forms of inference.
The idea of classification is thus a pervading element
in those systems. Furthermore, they exhibit Logic as resolvable
into two great branches, the one of which is occupied with the
treatment of categorical, the other with that of hypothetical or
%-----------------------File: 030.png----------------------------
conditional propositions. The distinction is nearly identical with
that of primary and secondary propositions in the present work.
The discussion of the theory of categorical propositions is, in all
the ordinary treatises of Logic, much more full and elaborate than
that of hypothetical propositions, and is occupied partly with
ancient scholastic distinctions, partly with the canons of deductive
inference. To the latter application only is it necessary to
direct attention here.

2. Categorical propositions are classed under the four following
heads, viz.:
\begin{center}
\begin{tabular}[h!]{ll@{\hspace{0in}}cl}
 & & &\multicolumn{1}{c}{{\textsc{type}}}\\
1st. & Universal affirmative  & Propositions: & All $Y$'s are $X$'s.\\
2nd. & Universal negative     &  "            & No $Y$'s are $X$'s.\\
3rd. & Particular affirmative &  "          & Some $Y$'s are $X$'s.\\
4th. & Particular negative    &  "      & Some $Y$'s are not $X$'s.\\
\end{tabular}
\end{center}
To these forms, four others have recently been added, so as
to constitute in the whole eight forms (see the next article) susceptible,
however, of reduction to six, and subject to relations
which have been discussed with great fulness and ability by Professor
De~Morgan, in his Formal Logic. A scheme somewhat
different from the above has been given to the world by Sir W.\
Hamilton, and is made the basis of a method of syllogistic inference,
which is spoken of with very high respect by authorities
on the subject of Logic.\footnote{Thomson's Outlines of the Laws of Thought, p. 177.}

The processes of Formal Logic, in relation to the above system
of propositions, are described as of two kinds, viz., ``Conversion''
and ``Syllogism.'' By Conversion is meant the expression of
any proposition of the above kind in an equivalent form, but with
a reversed order of terms. By Syllogism is meant the deduction
from two such propositions having a common term, whether
subject or predicate, of some third proposition inferentially involved
in the two, and forming the ``conclusion.'' It is maintained
by most writers on Logic, that these processes, and according
to some, the single process of Syllogism, furnish the
universal types of reasoning, and that it is the business of the
mind, in any train of demonstration, to conform itself, whether
%-----------------------File: 031.png----------------------------
consciously or unconsciously, to the particular models of the processes which have been classified in the writings of logicians.

3. The course which I design to pursue is to show how
these processes of Syllogism and Conversion may be conducted
in the most general manner upon the principles of the present
treatise, and, viewing them thus in relation to a system of Logic,
the foundations of which, it is conceived, have been laid in the
ultimate laws of thought, to seek to determine their true place
and essential character.

The expressions of the eight fundamental types of proposition in the language of symbols are as follows:
\begin{center}
\begin{tabular}[h!]{ll@{\hspace{0in}}ll}
%\begin{tabbing}
%\qquad\=8. \=Some not-$Y$'s are not-$X$'s,\quad\=$v(1-y)= v(1-x)$.\quad\=\kill
      1. &All $Y$'s are $X$'s,              &$y = vx$.\\
      2. &No $Y$'s are $X$'s,               &$y = v(1-x)$.\\
      3. &Some $Y$'s are $X$'s,             &$vy = vx$.\\
      4. &Some $Y$'s are not-$X$'s,         &$vy = v(1-x)$.\\
      5. &All not-$Y$'s are $X$'s,          &$1-y = vx.$&(1)\\
      6. &No not-$Y$'s are $X$'s,           &$1-y = v(l-x)$.\\
      7. &Some not-$Y$'s are $X$'s,         &$v(l-y) = vx$.\\
      8. &Some not-$Y$'s are not-$X$'s,     &$v(1-y) = v(1-x).$\\
%\end{tabbing}
\end{tabular}
\end{center}

In referring to these forms, it will be convenient to apply, in
a sense shortly to be explained, the epithets of logical quantity,
``universal'' and ``particular,'' and of quality, ``affirmative'' and
``negative,'' to the terms of propositions, and not to the propositions themselves. We shall thus consider the term
``All $Y$'s,''
as universal-affirmative; the term ``$Y$'s,'' or ``Some $Y$'s,'' as
particular-affirmative; the term ``All not-$Y$'s,'' as universal-negative; the term ``Some not-$Y$'s,'' as particular-negative. The
expression ``No $Y$'s,'' is not properly a \emph{term} of a proposition, for
the true meaning of the proposition, ``No $Y$'s are $X$'s,'' is
``All $Y$'s are not-$X$'s.'' The subject of that proposition is, therefore,
universal-affirmative, the predicate particular-negative. That
there is a real distinction between the conceptions of ``men'' and
``not men'' is manifest. This distinction is all that I contemplate when applying as above the designations of affirmative and
negative, without, however, insisting upon the etymological propriety of the application to the terms of propositions. The
designations positive anil privative would have been more
%-----------------------File: 032.png----------------------------
appropriate, but the former term is already employed in a fixed
sense in other parts of this work.

4. From the symbolical forms above given the laws of conversion
immediately follow. Thus from the equation
\[y = vx,\]
representing the proposition, ``All $Y$'s are $X$'s,'' we deduce, on
eliminating $v$,
\[y(1-x) = 0,\]
which gives by solution with reference to $1-x$,
\[1-x=\frac{0}{0}(1-y);\]
the interpretation of which is,
\[
\text{All not-$X$'s are not-$Y$'s.}
\]

This is an example of what is called ``negative conversion.''
In like manner, the equation
\[y = v(1-x),\]
representing the proposition, ``No $Y$'s are $X$'s,'' gives
\[x=\frac{0}{0}(1-y),\]
the interpretation of which is, ``No $X$'s are $Y$'s.'' This is an
example of what is termed simple conversion; though it is in reality
of the same kind as the conversion exhibited in the previous
example. All the examples of conversion which have been noticed
by logicians are either of the above kind, or of that which consists
in the mere transposition of the terms of a proposition, without
altering their quality, as when we change
\begin{align*}
  vy &= vx \textrm{, representing, Some $Y$'s are $X$'s,}   \\
\intertext{into}
  vx &= vy \textrm{, representing, Some $X$'s are $Y$'s;}
\end{align*}
or they involve a combination of those processes with some auxiliary
process of limitation, as when from the equation
\begin{align*}
  y &= vx \textrm{, representing, All $Y$'s are $X$'s,}   \\
\intertext{we deduce on multiplication by $v$,}
  vy &= vx \textrm{, representing, Some $Y$'s are $X$'s,}   \\
\intertext{and hence}
  vx &= vy \textrm{, representing, Some $X$'s are $Y$'s.}
\end{align*}
%-----------------------File: 033.png----------------------------
In this example, the process of limitation precedes that of
transposition.

From these instances it is seen that conversion is a particular
application of a much more general process in Logic, of which
many examples have been given in this work. That process has
for its object the determination of any element in any proposition,
however complex, as a logical function of the remaining elements.
Instead of confining our attention to the subject and predicate,
regarded as simple terms, we can take any element or any
combination of elements entering into either of them; make that
element, or that combination, the ``subject'' of a new proposition;
and determine what its predicate shall be, in accordance with the
data afforded to us. It may be remarked, that even the simple
forms of propositions enumerated above afford some ground for
the application of such a method, beyond what the received laws
of conversion appear to recognise. Thus the equation
\begin{equation*}
y=vx,\text{ representing, All }Y\text{'s are }X\text{'s},
\end{equation*}
gives us, in addition to the proposition before deduced, the three
following:
\begin{center}
\begin{tabular}[h!]{lll}
1st. & $y(1-x)=0.$ & There are no $Y$'s that are not-$X$'s.\\ \\
2nd. & $\displaystyle 1-y=\frac{0}{0}x+(1-x).$ & Things that are not-$Y$'s include all\\
 & & \hspace{0.35in}things that are not-$X$'s, and an \\
 & & \hspace{0.35in}indefinite remainder of things \\
 & & \hspace{0.35in}that are $X$'s. \\ \\
3rd. & $\displaystyle x=y+\frac{0}{0}(1-y).$ & Things that are $X$'s include all things \\
 & & \hspace{0.35in}that are $Y$'s, and an indefinite \\
 & & \hspace{0.35in}remainder of things that are not-\\
 & & \hspace{0.35in}$Y$'s.
\end{tabular}
\end{center}

These conclusions, it is true, merely place the given proposition
in other and equivalent forms,--but such and no more is
the office of the received mode of ``negative conversion.''

Furthermore, these processes of conversion are not elementary,
but they are combinations of processes more simple than
they, more immediately dependent upon the ultimate laws and
axioms which govern the use of the symbolical instrument of
%-----------------------File: 034.png----------------------------
reasoning.  This remark is equally applicable to the case of
Syllogism, which we proceed next to consider.

5. The nature of syllogism is best seen in the particular instance.
Suppose that we have the propositions,
\begin{align*}
  &\text{All }X\text{'s are }Y\text{'s},\\
  &\text{All }Y\text{'s are }Z\text{'s}.
\end{align*}

From these we may deduce the conclusion,
\begin{equation*}
\text{All }X\text{'s are }Z\text{'s}.
\end{equation*}
This is a syllogistic inference. The terms $X$ and $Z$ are called
the extremes, and $Y$ is called the middle term. The function
of the syllogism generally may now be defined. Given two propositions
of the kind whose species are tabulated in (1), and involving
one middle or common term $Y$, which is connected in
one of the propositions with an extreme $X$, in the other with an
extreme $Z$; required the relation connecting the extremes $X$ and
$Z$. The term $Y$ may appear in its affirmative form, as, All $Y$'s,
Some $Y$'s; or in its negative form, as, All not-$Y$'s, Some not-$Y$'s; in either proposition, without regard to the particular form
which it assumes in the other.

Nothing is easier than in particular instances to resolve the
Syllogism by the method of this treatise. Its resolution is, indeed,
a particular application of the process for the reduction of
systems of propositions. Taking the examples above given,
we have,
\begin{gather*}
  x = vy,\\
  y = v'z;
\end{gather*}
whence by substitution,
\begin{equation*}
x = vv'z,
\end{equation*}
which is interpreted into
\begin{equation*}
\text{All }X\text{'s are }Z\text{'s}.
\end{equation*}
Or, proceeding rigorously in accordance with the method developed
in (VIII.7), we deduce
\begin{equation*}
x(1-y)=0,\quad\quad y(1-z)=0.
\end{equation*}
Adding these equations, and eliminating $y$, we have
\begin{equation*}
x(1-z)=0;
\end{equation*}
%-----------------------File: 035.png----------------------------
whence $x = \frac{0}{0}z$, or, All $X$'s are $Z$'s.\\
And in the same way may any other case be treated.

6. Quitting, however, the consideration of special examples,
let us examine the general forms to which all syllogism may be
reduced.

\begin{center}
\textsc{Proposition I.}

\emph{To deduce the general rules of Syllogism.}
\end{center}

By the general rules of Syllogism, I here mean the rules applicable
to premises admitting of every variety both of quantity
and of quality in their subjects and predicates, except the combination
of two universal terms in the same proposition. The
admissible forms of propositions are therefore those of which a
tabular view is given in (1).

Let $X$ and $Y$ be the elements or things entering into the first
premiss, $Z$ and $Y$ those involved in the second. Two cases, fundamentally
different in character, will then present themselves.
The terms involving $Y$ will either be of \textit{like} or of \textit{unlike quality},
those terms being regarded as of like quality when they both
speak of ``$Y$'s,'' or both of ``Not-$Y$'s,'' as of unlike quality when
one of them speaks of ``$Y$'s,'' and the other of ``Not-$Y$'s.'' Any
pair of premises, in which the former condition is satisfied, may
be represented by the equations
\begin{align*}
 \tag{1}
vx &= v'y,\\
 \tag{2}
wz &= w'y;
\end{align*}
for we can employ the symbol $y$ to represent either ``All $Y$'s,''
or ``All not-$Y$'s,'' since the interpretation of the symbol is purely
conventional. If we employ $y$ in the sense of ``All not-$Y$'s,''
then $1-y$ will represent ``All $Y$'s,'' and no other change will
be introduced. An equal freedom is permitted with respect
to the symbols $x$ and $z$, so that the equations (1) and (2) may,
by properly assigning the interpretations of $x$, $y$, and $z$, be made
to represent all varieties in the combination of premises dependent
upon the \textit{quality} of the respective terms. Again, by assuming
proper interpretations to the symbols $v$, $v'$, $w$, $w'$, in those
equations, all varieties with reference to \textit{quantity} may also be
%-----------------------File: 036.png----------------------------
represented. Thus, if we take $v=1$, and represent by $v'$ a class
indefinite, the equation~(1) will represent a universal proposition
according to the ordinary sense of that term, i.~e., a proposition
with universal subject and particular predicate. We may, in
fact, give to subject and predicate in either premiss whatever
\textit{quantities} (using this term in the scholastic sense) we please, except
that by hypothesis, they must not both be universal. The
system (1), (2), represents, therefore, with perfect generality,
the possible combinations of premises which have like middle
terms.

7. That our analysis may be as general as the equations to
which it is applied, let us, by the method of this work, eliminate
$y$ from (1) and (2), and seek the expressions for $x$, $1-x$, and
$vx$, in terms of $z$ and of the symbols $v$, $v'$, $w$, $w'$. The above will
include all the possible forms of the subject of the conclusion.
The form $v(1-x)$ is excluded, inasmuch as we cannot from the
interpretation $vx =$ Some $X$'s, given in the premises, interpret
$v(1-x)$ as Some not-$X$'s. The symbol $v$, when used in the sense
of ``some,'' applies to that term only with which it is connected
in the premises.

The results of the analysis are as follows:
\begin{multline*}
\tag{I.}
  x= \bigl[vv'ww' + \frac{0}{0}\{
   vv'\bigl(1-w\bigr) \bigl(1-w'\bigr)
 + ww'\bigl(1-v\bigr) \bigl(1-v'\bigr)
 +    \bigl(1-v\bigr) \bigl(1-w\bigr)\}\bigr]z\\
  + \frac{0}{0}\{vv'\bigl(1-w'\bigr) + 1-v\} \bigl(1-z\bigr),
\end{multline*}
\begin{multline*}
\tag{II.}
  1-x = \bigl[
   v\bigl(1-v'\bigr) \{ww' + \bigl(1-w\bigr) \bigl(1-w'\bigr)\}
 + v\bigl(1-w\bigr)w'\\
+ \frac{0}{0}\{
   vv'\bigl(1-w\bigr) \bigl(1-w'\bigr)
 + ww'\bigl(1-v\bigr) \bigl(1-v'\bigr)
 + \bigl(1-v\bigr) \bigl(1-w\bigr) \} \bigr]z\\
  +\bigl[v\bigl(1-w\bigr)w'
 + \frac{0}{0}\{vv'\bigl(1-w'\bigr) + 1-v\} \bigr] \bigl(1-z\bigr),
\end{multline*}
\begin{multline*}
\tag{III.}
  vx = \{vv'ww' + \frac{0}{0}vv'\bigl(1-w\bigr) \bigl(1-w'\bigr)\}z
 + \frac{0}{0}\bigl(1-w'\bigr) \bigl(1-z\bigr).
\end{multline*}

Each of these expressions involves in its second member two
terms, of one of which $z$ is a factor, of the other $1-z$. But
syllogistic inference does not, as a matter of form, admit of contrary
classes in its conclusion, as of $Z$'s and not-$Z$'s together.
%-----------------------File: 037.png----------------------------

We must, therefore, in order to determine the rules of that
species of inference, ascertain under what conditions the second
members of any of our equations are reducible to a single term.

The simplest form is (III.), and it is reducible to a single
term if $w' = 1$. The equation then becomes
\[  \tag{3}
  vx = vv'wz,
\]
the first member is identical with the extreme in the first premiss;
the second is of the same quantity and quality as the extreme
in the second premiss. For since $w' = 1$, the second member of
(2), involving the middle term $y$, is universal; therefore, by the
hypothesis, the first member is particular, and therefore, the second
member of (3), involving the same symbol $w$ in its coefficient,
is particular also. Hence we deduce the following law.

\textsc{Condition of Inference.}---One middle term, at least, universal.

\textsc{Rule of Inference.}---Equate the extremes.

From an analysis of the equations (I.) and (II.), it will further
appear, that the above is the only condition of syllogistic inference
when the middle terms are of like quality. Thus the
second member of (I.) reduces to a single term, if $w' = 1$ and
$v = 1$; and the second member of (II.) reduces to a single term,
if $w' = 1$, $v = 1$, $w = 1$. In each of these cases, it is necessary that
$w' = 1$, the solely sufficient condition before assigned.

Consider, secondly, the case in which the middle terms are
of unlike quality. The premises may then be represented under
the forms
\begin{align*}
  vx &= v'y,        \tag{4}   \\
  wz &= w'(l-y);     \tag{5}
\end{align*}
and if, as before, we eliminate $y$, and determine the expressions
of $x$, $1-x$, and $vx$, we get
\begin{multline*}
x= \bigl[vv'(l-w)w' +\frac{0}{0}\{ww'(1-v) + (1-v)(1-v')(1-w)  \\
  + v'(1-w)(1-w')\}\bigr]   \\
+ \bigl[vv'w' +\frac{0}{0}\{(1-v)(1-v') + v'(l-w')\}\bigr](1-z). \tag{IV.}
\end{multline*}

%-----------------------File: 038.png----------------------------
\begin{align*}
  1-x = &\bigl[ww'v + v(1-v')(1-w) + \frac{0}{0}\{ww'(1-v)   \\
    &+ (1-v)(1-v')(1-w) + v'(1-w)(1-w')\}\bigr]z   \\
    &+ \bigl[v(1-v') + \frac{0}{0}\{v'(1-w') + (1-v)(1-v')\}\bigr](1-z).
\tag{V.}
\end{align*}
\begin{multline*}
  vx = \{vv'(1-w)w' + \frac{0}{0}vv'(1-w)(1-w')\}z   \\
    \hfill + \{vv'w' + \frac{0}{0}vv'(1-w')\}(1-z). \hfill \tag{VI.}
\end{multline*}
Now the second member of (VI.) reduces to a single term relatively to $z$, if $w = 1$, giving
\[
  vx = \{vv'w' + \frac{0}{0}vv'(1-w')\}(1-z);
\]
the second member of which is opposite, both in quantity and
quality, to the corresponding extreme, $wz$, in the second premiss.
For since $w = 1$, $wz$ is universal. But the factor $vv'$ indicates
that the term to which it is attached is particular, since by hypothesis $v$ and $v'$ are not both equal to 1. Hence we deduce the
following law of inference in the case of like middle terms:

\textsc{First Condition of Inference.}---\emph{At least one universal
extreme.}

\textsc{Rule of Inference.}---\emph{Change the quantity and quality of
that extreme, and equate the result to the other extreme.}

Moreover, the second member of (V.) reduces to a single term
if $v' = 1$, $w' = 1$; it then gives
\[
  1-x = \{vw + \frac{0}{0}(1-v)w\} z.
\]
Now since $v' = 1$, $w' = 1$, the middle terms of the premises are
both universal, therefore the extremes $vx$, $wz$, are particular.
But in the conclusion the extreme involving $x$ is opposite, both
in quantity and quality, to the extreme $vx$ in the first premiss,
while the extreme involving $z$ agrees both in quantity and quality with the corresponding extreme $wz$ in the second premiss.
Hence the following general law:

%-----------------------File: 039.png----------------------------
\textsc{Second Condition of Inference.}---\emph{Two universal middle
terms.}

\textsc{Rule of Inference.}---\emph{Change the quantity and quality of
either extreme, and equate the result to the other extreme unchanged.}

There are in the case of unlike middle terms no other conditions or rules of syllogistic inference than the above. Thus the
equation (IV.), though reducible to the form of a syllogistic conclusion, when $w=1$ and $v = 1$, does not thereby establish a new
condition of inference; since, by what has preceded, the single
condition $v = 1$, or $w = 1$, would suffice.

8. The following examples will sufficiently illustrate the general rules of syllogism above given:
\begin{align*}
  1. &\text{ All $Y$'s are $X$'s.}\\
     &\text{ All $Z$'s are $Y$'s.}
\end{align*}
This belongs to Case 1. All $Y$'s is the universal middle term.
The extremes equated give as the conclusion
\begin{center}
All $Z$'s are $X$'s;
\end{center}
the universal term, All $Z$'s, becoming the subject; the particular
term (some) $X$'s, the predicate.
\begin{align*}
  2. &\text{ All $X$'s are $Y$'s.}\\
     &\text{ No $Z$'s are $Y$'s.}
\end{align*}
The proper expression of these premises is
\begin{align*}
  &\text{All $X$'s are $Y$'s.}\\
  &\text{All $Z$'s are not-$Y$'s.}
\end{align*}
They belong to Case 2, and satisfy the first condition of inference.
The middle term, $Y$'s, in the first premiss, is particular-affirmative; that in the second premiss, not-$Y$'s, particular-negative.
If we take All $Z$'s as the universal extreme, and change its
quantity and quality according to the rule, we obtain the term
Some not-$Z$'s, and this equated with the other extreme, All $X$'s,
gives,
\begin{center}
  All $X$'s are not-$Z$'s, i. e., No $X$'s are $Z$'s.
\end{center}
If we commence with the other universal extreme, and proceed
similarly, we obtain the equivalent result,
\begin{center}
  No $Z$'s are $X$'s.
\end{center}

%-----------------------File: 040.png----------------------------
\begin{align*}
  3. &\text{ All $Y$'s are $X$'s.}\\
     &\text{ All not-$Y$'s are $Z$'s.}
\end{align*}
Here also the middle terms are unlike in quality. The premises
therefore belong to Case 2, and there being two universal middle
terms, the second condition of inference is satisfied. If by the
rule we change the quantity and quality of the first extreme,
(some) $X$'s, we obtain All not-$X$'s, which, equated with the
other extreme, gives
\begin{center}
  All not-$X$'s are $Z$'s.
\end{center}
The reverse order of procedure would give the equivalent result,
\begin{center}
  All not-$Z$'s are $X$'s.
\end{center}

The conclusions of the two last examples would not be recognised as valid in the scholastic system of Logic, which virtually
requires that the subject of a proposition should be affirmative.
They are, however, perfectly legitimate in themselves, and the
rules by which they are determined form undoubtedly the most
general canons of syllogistic inference. The process of investigation by which they are deduced will probably appear to be of
needless complexity; and it is certain that they might have been
obtained with greater facility, and without the aid of any symbolical instrument whatever. It was, however, my object to
conduct the investigation in the most general manner, and by an
analysis thoroughly exhaustive. With this end in view, the
brevity or prolixity of the method employed is a matter of indifference.
Indeed the analysis is not properly that of the syllogism,
but of a much more general combination of propositions; for we
are permitted to assign to the symbols $v$, $v'$, $w$, $w'$, any class-interpretations
that we please. To illustrate this remark, I will
apply the solution (I.) to the following imaginary case:

Suppose that a number of pieces of cloth striped with different colours were submitted to inspection, and that the two following observations were made upon them:

1st. That every piece striped with white and green was also
striped with black and yellow, and \textit{vice vers\^{a}}.

2nd. That every piece striped with red and orange was also
striped with blue and yellow, and \textit{vice vers\^{a}}.
%-----------------------File: 041.png----------------------------
Suppose it then required to determine how the pieces marked
with green stood affected with reference to the colours white,
black, red, orange, and blue.

Here if we assume $v=$ white, $x =$ green, $v' =$ black, $y =$ yellow,
$w =$ red, $z =$ orange, $w' =$ blue, the expression of our premises will
be
\begin{align*}
  vx &= v'y,   \\
  wz &= w'y,
\end{align*}
agreeing with the system (1) (2). The equation (I.) then leads
to the following conclusion:

Pieces striped with green are either striped with orange,
white, black, red, and blue, together, all pieces possessing which
character are included in those striped with green; or they are
striped with orange, white, and black, but not with red or blue;
or they are striped with orange, red, and blue, but not with white
or black; or they are striped with orange, but not with white or
red; or they are striped with white and black, but not with blue
or orange; or they are striped neither with white nor orange.

Considering the nature of this conclusion, neither the symbolical expression (I.) by which it is conveyed, nor the analysis
by which that expression is deduced, can be considered as needlessly complex.

9. The form in which the doctrine of syllogism has been
presented in this chapter affords ground for an important observation. We have seen that in each of its two great divisions the
entire discussion is reducible, so far, at least, as concerns the determination of rules and methods, to the analysis of a pair of
equations, viz., of the system (1), (2), when the premises have
like middle terms, and of the system (4), (5), when the middle
terms are unlike. Moreover, that analysis has been actually
conducted by a method founded upon certain general laws deduced immediately from the constitution of language, Chap. \textsc{ii.}
confirmed by the study of the operations of the human mind,
Chap. \textsc{iii.}, and proved to be applicable to the analysis of all systems of equations whatever, by which propositions, or combinations of propositions, can be represented, Chap. \textsc{viii}. Here, then,
we have the means of definitely resolving the question, whether
syllogism is indeed the fundamental type of reasoning,---whether
%-----------------------File: 042.png----------------------------
the study of its laws is co-extensive with the study of deductive
logic. For if it be so, some indication of the fact must be given
in the systems of equations upon the analysis of which we have
been engaged. It cannot be conceived that syllogism should be
the one essential process of reasoning, and yet the manifestation
of that process present nothing indicative of this high quality of
pre-eminence. No sign, however, appears that the discussion of
all systems of equations expressing propositions is involved in
that of the particular system examined in this chapter. And yet
writers on Logic have been all but unanimous in their assertion,
not merely of the supremacy, but of the universal sufficiency of
syllogistic inference in deductive reasoning. The language of
Archbishop Whately, always clear and definite, and on the subject
of Logic entitled to peculiar attention, is very express on
this point.  ``For Logic,'' he says, ``which is, as it were, the
Grammar of Reasoning, does not bring forward the regular Syllogism
as a \textit{distinct mode of argumentation}, designed to be \textit{substituted}
for any other mode; but as the form to which \textit{all} correct
reasoning may be ultimately reduced.''%
\footnote{Elements of Logic, p. 13, ninth edition.} And Mr. Mill, in a
chapter of his System of Logic, entitled, ``Of Ratiocination or
Syllogism,'' having enumerated the ordinary forms of syllogism,
observes, ``All valid ratiocination, all reasoning by which from
general propositions previously admitted, other propositions,
equally or less general, are inferred, may be exhibited in some of
the above forms.'' And again: ``We are therefore at liberty,
in conformity with the general opinion of logicians, to consider
the two elementary forms of the first figure as the universal types
of all correct ratiocination.'' In accordance with these views it
has been contended that the science of Logic enjoys an immunity
from those conditions of imperfection and of progress to which
all other sciences are subject;%
\footnote{Introduction to Kant's ``Logik.''} and its origin from the travail of
one mighty mind of old has, by a somewhat daring metaphor,
been compared to the mythological birth of Pallas.

As Syllogism is a species of elimination, the question before
us manifestly resolves itself into the two following ones:---1st.
Whether all elimination is reducible to Syllogism; 2ndly.
%-----------------------File: 043.png----------------------------
Whether deductive reasoning can with propriety be regarded as consisting
only of elimination. I believe, upon careful examination,
the true answer to the former question to be, that it is always
theoretically possible so to resolve and combine propositions that
elimination may subsequently be effected by the syllogistic canons,
but that the process of reduction would in many instances
be constrained and unnatural, and would involve operations
which are not syllogistic. To the second question I reply, that
reasoning cannot, except by an arbitrary restriction of its meaning, be confined to the process of elimination. No definition can
suffice which makes it less than the aggregate of the methods
which are founded upon the laws of thought, as exercised upon
propositions; and among those methods, the process of elimination, eminently important as it is, occupies only a place.

Much of the error, as I cannot but regard it, which prevails
respecting the nature of the Syllogism and the extent of its
office, seems to be founded in a disposition to regard all those
truths in Logic as \emph{primary} which possess the character of simplicity
and intuitive certainty, without inquiring into the relation
which they sustain to other truths in the Science, or to general
methods in the Art, of Reasoning. Aristotle's \textit{dictum de omni et
nullo} is a self-evident principle, but it is not found among those
\emph{ultimate} laws of the reasoning faculty to which all other laws,
however plain and self-evident, admit of being traced, and from
which they may in strictest order of scientific evolution be deduced.
For though of every science the fundamental truths are
usually the most simple of apprehension, yet is not that simplicity
the \emph{criterion} by which their title to be regarded as fundamental
must be judged. This must be sought for in the nature
and extent of the structure which they are capable of supporting.
Taking this view, Leibnitz appears to me to have judged correctly
when he assigned to the ``principle of contradiction'' a
fundamental place in Logic;\footnote{Nouveaux Essais sur l'entendement humain. Liv.~\textsc{iv.} cap.~2. Theodic\'{e}e Pt.~I.\ sec.~44.} for we have seen the consequences
of that law of thought of which it is the axiomatic expression
(III.~15). But enough has been said upon the nature of deductive
inference and upon its constitutive elements. The subject of
%-----------------------File: 044.png----------------------------
induction may probably receive some attention in another part of
this work.

10. It has been remarked in this chapter that the ordinary
treatment of hypothetical, is much more defective than that of
categorical, propositions. What is commonly termed the hypothetical
syllogism appears, indeed, to be no syllogism at all.
\begin{tabbing}
Let the argument--- \= \\
                     \>If $A$ is $B$, $C$ is $D$,   \\
                     \>But $A$ is $B$,   \\
                     \>Therefore $C$ is $D$,   \\
be put in the form---   \\
\hfill If the proposition\ \>$X$ is true, $Y$ is true,   \\
                     \>But $X$ is true,   \\
                     \>Therefore $Y$ is true;
\end{tabbing}
wherein by $X$ is meant the proposition $A$ is $B$, and by $Y$, the
proposition $C$ is $D$. It is then seen that the premises contain
only two terms or elements, while a syllogism essentially involves
three. The following would be a genuine hypothetical syllogism:

\begin{tabular}{rl}
            & If $X$ is true, $Y$ is true;   \\
            & If $Y$ is true, $Z$ is true;   \\
$\therefore$ & If $X$ is true, $Z$ is true.
\end{tabular}

After the discussion of secondary propositions in a former
part of this work, it is evident that the forms of hypothetical
syllogism must present, in every respect, an exact counterpart to
those of categorical syllogism. \emph{Particular} Propositions, such as,
``Sometimes if $X$ is true, $Y$ is true,'' may be introduced, and the
conditions and rules of inference deduced in this chapter for categorical
syllogisms may, without abatement, be interpreted to
meet the corresponding cases in hypotheticals.

11. To what final conclusions are we then led respecting the
nature and extent of the scholastic logic? I think to the following:
that it is not a science, but a collection of scientific truths, too
incomplete to form a system of themselves, and not sufficiently
fundamental to serve as the foundation upon which a perfect
system may rest. It does not, however, follow, that because the
logic of the schools has been invested with attributes to which it
%-----------------------File: 045.png----------------------------
has no just claim, it is therefore undeserving of regard. A system
which has been associated with the very growth of language,
which has left its stamp upon the greatest questions and the
most famous demonstrations of philosophy, cannot be altogether
unworthy of attention. Memory, too, and usage, it must be admitted,
have much to do with the intellectual processes; and
there are certain of the canons of the ancient logic which have
become almost inwoven in the very texture of thought in cultured
minds. But whether the mnemonic forms, in which the particular
rules of conversion and syllogism have been exhibited, possess
any real utility,---whether the very skill which they are supposed
to impart might not, with greater advantage to the mental
powers, be acquired by the unassisted efforts of a mind left to its
own resources,---are questions which it might still be not unprofitable
to examine. As concerns the particular results deduced
in this chapter, it is to be observed, that they are solely
designed to aid the inquiry concerning the nature of the ordinary
or scholastic logic, and its relation to a more perfect theory of
deductive reasoning.
%-----------------------File: 046.png----------------------------
%\centerline{\large CHAPTER XVI.}
%\vspace{0.2in}
\chapter[ON THE THEORY OF PROBABILITIES.]{\large ON THE THEORY OF PROBABILITIES}
%\vspace{0.2in}

1. Before the expiration of another year just two centuries
will have rolled away since Pascal solved the first known
question in the theory of Probabilities, and laid, in its solution,
the foundations of a science possessing no common share of the
attraction which belongs to the more abstract of mathematical
speculations. The problem which the Chevalier de M\'{e}r\'{e}, a reputed
gamester, proposed to the recluse of Port Royal (not yet
withdrawn from the interests of science%
\footnote{See in particular a letter addressed by Pascal to Fermat, who had solicited
his attention to a mathematical problem (Port Royal, par M.\ de Sainte Beuve);
also various passages in the collection of Fragments published by M.\ Prosper
Faug\`{e}re.}%endfootnote
 by the more distracting
contemplation of the ``greatness and the misery of man''), was
the first of a long series of problems, destined to call into existence
new methods in mathematical analysis, and to render valuable
service in the practical concerns of life. Nor does the interest
of the subject centre merely in its mathematical connexion,
or its associations of utility. The attention is repaid which is
devoted to the theory of Probabilities as an independent object
of speculation,---to the fundamental modes in which it has been
conceived,---to the great secondary principles which, as in the
contemporaneous science of Mechanics, have gradually been annexed
to it,---and, lastly, to the estimate of the measure of perfection
which has been actually attained. I speak here of that
perfection which consists in unity of conception and harmony of
processes. Some of these points it is designed very briefly to
consider in the present chapter.

2. A distinguished writer%
\footnote{Poisson, Recherches sur la Probabilit\`{e} des Jugemens.} has thus stated the fundamental
definitions of the science:

%-----------------------File: 047.png----------------------------
``The probability of an event is the reason we have to believe
that it has taken place, or that it will take place."

``The measure of the probability of an event is the ratio of
the number of cases favourable to that event, to the total number
of cases favourable or contrary, and all equally possible"
(equally likely to happen).

From these definitions it follows that the word \textit{probability}, in
its mathematical acceptation, has reference to the state of our
knowledge of the circumstances under which an event may happen
or fail. With the degree of information which we possess
concerning the circumstances of an event, the reason we have to
think that it will occur, or, to use a single term, our \textit{expectation} of
it, will vary. Probability is expectation founded upon partial
knowledge. A perfect acquaintance with \textit{all} the circumstances
affecting the occurrence of an event would change expectation
into certainty, and leave neither room nor demand for a theory
of probabilities.

3. Though our expectation of an event grows stronger with
the increase of the ratio of the number of the known cases favourable
to its occurrence to the whole number of equally possible
cases, favourable or unfavourable, it would be unphilosophical
to affirm that the strength of that expectation, viewed as an
emotion of the mind, is capable of being referred to any numerical
standard. The man of sanguine temperament builds high hopes
where the timid despair, and the irresolute are lost in doubt.
As subjects of scientific inquiry, there is some analogy between
\textit{opinion} and \textit{sensation}. The thermometer and the carefully prepared
photographic plate indicate, not the intensity of the sensations
of heat and light, but certain physical circumstances
which accompany the production of those sensations. So also
the theory of probabilities contemplates the numerical measure
of the circumstances upon which expectation is founded; and this
object embraces the whole range of its legitimate applications.
The rules which we employ in life-assurance, and in the other
statistical applications of the theory of probabilities, are altogether
independent of the \textit{mental} ph{\ae}nomena of expectation. They are
founded upon the assumption that the future will bear a
%-----------------------File: 048.png----------------------------
resemblance to the past; that under the same circumstances the same
event will tend to recur with a definite numerical frequency; not
upon any attempt to submit to calculation the strength of human
hopes and fears.

Now experience actually testifies that events of a given species
do, under given circumstances, tend to recur with definite frequency,
whether their true causes be known to us or unknown.
Of course this tendency is, in general, only manifested when the
area of observation is sufficiently large. The judicial records of
a great nation, its registries of births and deaths, in relation to
age and sex, \&c., present a remarkable uniformity from year to
year. In a given language, or family of languages, the same
sounds, and successions of sounds, and, if it be a written language,
the same characters and successions of characters recur
with determinate frequency. The key to the rude Ogham inscriptions,
found in various parts of Ireland, and in which no
distinction of words could at first be traced, was, by a strict application
of this principle, recovered.\footnote{The discovery is due to the Rev.\
Charles Graves, Professor of Mathematics in the University of Dublin.--
\textit{Vide} Proceedings of the Royal Irish Academy,
Feb.\ 14, 1848. Professor Graves informs me that he has verified the principle
by constructing sequence tables for all the European languages.}%endfootnote
 The same method, it is understood, has been applied
\footnote{By the learned Orientalist, Dr.\ Edward Hincks.} to the
deciphering of the cuneiform records recently disentombed from
the ruins of Nineveh by the enterprise of Mr.\ Layard.

4. Let us endeavour from the above statements and definitions
to form a conception of the legitimate object of the theory
of Probabilities.

Probability, it has been said, consists in the expectation
founded upon a particular kind of knowledge, viz., the knowledge
of the relative frequency of occurrence of events. Hence
the probabilities of events, or of combinations of events, whether
deduced from a knowledge of the particular constitution of
things under which they happen, or derived from the long-continued
observation of a past series of their occurrences and failures,
constitute, in all cases, our data. The probability of some
%-----------------------File: 049.png----------------------------
connected event, or combination of events, constitutes the corresponding
\textit{qu\ae{}situm},  or object sought. Now in the most general,
yet strict meaning of the term ``event," every combination
of events constitutes also an event. The simultaneous occurrence
of two or more events, or the occurrence of an event under
given conditions, or in any conceivable connexion with other
events, is still an event. Using the term in this liberty of application,
the object of the theory of probabilities might be thus
defined. Given the probabilities of any events, of whatever
kind, to find the probability of some other event connected with
them.

5. Events may be distinguished as simple or compound, the
latter term being applied to such events as consist in a combination
of simple events (I.~13). In this manner we might define it
as the practical end of the theory under consideration to determine
the probability of some event, simple or compound, from
the given probabilities of other events, simple or compound,
with which, by the terms of its definition, it stands connected.

Thus if it is known from the constitution of a die that there
is a probability, measured by the fraction $\displaystyle\frac{1}{6}$, that the result of
any particular throw will be an ace, and if it is required to determine
the probability that there shall occur one ace, and only one,
in two successive throws, we may state the problem in the order
of its \textit{data} and its \textit{qu\ae{}situm}, as follows:

\textsc{First Datum}.---Probability of the event that the first throw
will give an ace $\displaystyle =\frac{1}{6}$.

\textsc{Second Datum}.---Probability of the event that the second
throw will give an ace $\displaystyle =\frac{1}{6}$.

\textsc{Qu\ae{}situm}.---Probability of the event that either the first
throw will give an ace, and the second not an ace; or the first
will not give an ace, and the second will give one.

Here the two data are the probabilities of simple events defined
as the first throw giving an ace, and the second throw
giving an ace. The qu\ae{}situm is the probability of a compound
event,---a certain disjunctive combination of the simple events
%-----------------------File: 050.png----------------------------
involved or implied in the data. Probably it will generally happen,
when the numerical conditions of a problem are capable of
being deduced, as above, from the constitution of things under
which they exist, that the data will be the probabilities of \emph{simple}
events, and the \textit{qu\ae{}situm} the probability of a \emph{compound} event
dependent upon the said simple events. Such is the case with a
class of problems which has occupied perhaps an undue share of
the attention of those who have studied the theory of probabilities,
viz., games of chance and skill, in the former of which some
physical circumstance, as the constitution of a die, determines
the probability of each possible step of the game, its issue being
some definite combination of those steps; while in the latter, the
relative dexterity of the players, supposed to be known \textit{\`{a} priori},
equally determines the same element. But where, as in statistical
problems, the elements of our knowledge are drawn, not from
the study of the constitution of things, but from the registered
observations of Nature or of human society, there is no reason
why the data which such observations afford should be the probabilities
of simple events. On the contrary, the occurrence of
events or conditions in marked combinations (indicative of some
secret connexion of a causal character) suggests to us the propriety
of making such concurrences, profitable for future instruction
by a numerical record of their frequency. Now the data
which observations of this kind afford are the probabilities of
compound events. The solution, by some general method, of
problems in which such data are involved, is thus not only essential
to the perfect development of the theory of probabilities, but
also a perhaps necessary condition of its application to a large
and practically important class of inquiries.

6. Before we proceed to estimate to what extent known methods
may be applied to the solution of problems such as the
above, it will be advantageous to notice, that there is another
form under which all questions in the theory of probabilities may
be viewed; and this form consists in substituting for \emph{events} the
propositions which assert that those events have occurred, or
will occur; and viewing the element of numerical probability as
having reference to the \emph{truth} of those \emph{propositions}, not to the
%-----------------------File: 051.png----------------------------
\emph{occurrence} of the \emph{events} concerning which they make assertion.
Thus, instead of considering the numerical fraction $p$ as expressing
the probability of the occurrence of an event $E$, let it
be viewed as representing the probability of the truth of the
proposition $X$, which asserts that the event $E$ will occur. Similarly,
instead of any probability, $q$, being considered as referring
to some compound event, such as the concurrence of the
events $E$ and $F$, let it represent the probability of the truth of
the proposition which asserts that $E$ and $F$ will jointly occur;
and in like manner, let the transformation be made from disjunctive and hypothetical combinations of events to disjunctive and
conditional propositions. Though the new application thus assigned to probability is a necessary concomitant of the old one,
its adoption will be attended with a practical advantage drawn
from the circumstance that we have already discussed the theory
of propositions, have defined their principal varieties, and established
methods for determining, in every case, the amount and
character of their mutual dependence. Upon this, or upon some
equivalent basis, any general theory of probabilities must rest.
I do not say that other considerations may not in certain cases of
applied theory be requisite. The data may prove insufficient for
definite solution, and this defect it may be thought necessary to
\emph{supply} by hypothesis. Or, where the statement of large numbers
is involved, difficulties may arise \emph{after the solution}, from this
source, for which special methods of treatment are required.
But in every instance, some form of the general problem as above
stated (Art.~4) is involved, and in the discussion of that problem
the proper and peculiar work of the theory consists. I desire it
to be observed, that to this object the investigations of the following
chapters are mainly devoted. It is not intended to enter,
except incidentally, upon questions involving supplementary hypotheses,
because it is of primary importance, even with reference
to such questions (I.~17), that a general method, founded upon
a solid and sufficient basis of theory, be first established.

7. The following is a summary, chiefly taken from Laplace, of
the principles which have been applied to the solution of questions
of probability. They are consequences of its fundamental
%-----------------------File: 052.png----------------------------
definitions already stated, and may be regarded as indicating the degree
in which it has been found possible to render those definitions
available.

\textsc{Principle} 1st. If $p$ be the probability of the occurrence of
any event, $1-p$ will be the probability of its non-occurrence.

2nd. The probability of the concurrence of two independent
events is the product of the probabilities of those events.

3rd. The probability of the concurrence of two dependent
events is equal to the product of the probability of one of them
by the probability that if that event occur, the other will happen
also.

4th. The probability that if an event, $E$, take place, an event,
$F$, will also take place, is equal to the probability of the concurrence of the events $E$ and $F$, divided by the probability of the
occurrence of $E$.

5th. The probability of the occurrence of one or the other of
two events which cannot concur is equal to the sum of their separate
probabilities.

6th. If an observed event can only result from some one of $n$
different causes which are \textit{\`{a} priori} equally probable, the probability
of any one of the causes is a fraction whose numerator is the
probability of the event, on the hypothesis of the existence of that
cause, and whose denominator is the sum of the similar probabilities
relative to all the causes.

7th. The probability of a future event is the sum of the products
formed by multiplying the probability of each cause by
the probability that if that cause exist, the said future event
will take place.

8. Respecting the extent and the relative sufficiency of these
principles, the following observations may be made.

1st. It is always possible, by the due combination of these
principles, to express the probability of a compound event, dependent
in any manner upon independent simple events whose
distinct probabilities are given. A very large proportion of the
problems which have been actually solved are of this kind, and
the difficulty attending their solution has not arisen from the insufficiency
of the indications furnished by the theory of probabilities,
but from the need of an analysis which should render
%-----------------------File: 053.png----------------------------
those indications available when functions of large numbers, or
series consisting of many and complicated terms, are thereby introduced.
It may, therefore, be fully conceded, that all problems
having for their data the probabilities of independent
simple events fall within the scope of received methods.

2ndly. Certain of the principles above enumerated, and especially
the sixth and seventh, do not presuppose that all the data
are the probabilities of simple events. In their peculiar application
to questions of causation, they do, however, assume, that the
causes of which they take account are mutually exclusive, so
that no combination of them in the production of an effect is
possible. If, as before explained, we transfer the numerical probabilities
from the events with which they are connected to the
propositions by which those events are expressed, the most general
problem to which the aforesaid principles are applicable
may be stated in the following order of \emph{data} and \emph{qu{\ae}sita}.


\begin{center}\textsc{data}.\end{center}

1st. The probabilities of the $n$ conditional propositions:

\begin{tabular}{ccccc}
If the cause &$A_1$& exist, the event &$E$& will follow;\\
      ''     &$A_2$&        ''        &$E$&      ''     \\
\multicolumn{5}{c}{\dotfill}\\
      ''     &$A_n$&        ''        &$E$&      ''
\end{tabular}

2nd. The condition that the antecedents of those propositions
are mutually conflicting.

\begin{center}\textsc{requirements}.\end{center}

The probability of the truth of the proposition which declares
the occurrence of the event $E$; also, when that proposition is
known to be true, the probabilities of truth of the several propositions which affirm the respective occurrences of the causes
$A_1, A_2 \dotsc A_n$.

Here it is seen, that the data are the probabilities of a series
of compound events, expressed by \emph{conditional} propositions. But
the system is obviously a very limited and particular one. For
the antecedents of the propositions are subject to the condition of
being mutually exclusive, and there is but one consequent, the
event $E$, in the whole system. It does not follow, from our
%-----------------------File: 054.png----------------------------
ability to discuss such a system as the above, that we are able to
resolve problems whose data are the probabilities of \emph{any} system
of conditional propositions; far less that we can resolve problems
whose data are the probabilities of \emph{any system of propositions
whatever}. And, viewing the subject in its material rather
than its formal aspect, it is evident, that the hypothesis of \emph{exclusive}
causation is one which is not often realized in the actual
world, the ph{\ae}nomena of which seem to be, usually, the products
of complex causes, the amount and character of whose co-operation
is unknown. Such is, without doubt, the case in nearly all
departments of natural or social inquiry in which the doctrine of
probabilities holds out any new promise of useful applications.

9. To the above principles we may add another, which has
been stated in the following terms by the Savilian Professor of
Astronomy in the University of Oxford.%
\footnote{On certain Questions relating to the Theory of Probabilities; by W. F.
Donkin, M. A., F. R. S., \&c. Philosophical Magazine, May, 1851.}

``Principle 8. If there be any number of mutually exclusive
hypotheses, $h_1, h_2, h_3,\dotsc$ of which the probabilities relative to a
particular state of information are $p_1, p_2, p_3,\dotsc$ and if new information
be given which changes the probabilities of some of them,
suppose of $h_{m+1}$ and all that follow, without having otherwise
\emph{any reference to the rest}; then the probabilities of these latter
have the same ratios to one another, after the new information,
that they had before, that is,
\[
  p'_1\colon p'_2\colon p'_3\ldots\colon p'_m
= p_1 \colon p_2 \colon p_3 \ldots\colon p_m,
\]
where the accented letters denote the values after the new information
has been acquired.''

This principle is apparently of a more fundamental character
than the most of those before enumerated, and perhaps it might, as
has been suggested by Professor Donkin, be regarded as axiomatic.
It seems indeed to be founded in the very definition of
the measure of probability, as ``the ratio of the number of cases
favourable to an event to the total number of cases favourable or
contrary, and all equally possible.'' For, adopting this definition,
it is evident that in whatever proportion the number of equally
%-----------------------File: 055.png----------------------------
possible cases is diminished, while the number of favourable cases
remains unaltered, in exactly the same proportion will the
probabilities of any events to which these cases have reference be
increased. And as the new hypothesis, viz., the diminution of
the number of possible cases without affecting the number of
them which are favourable to the events in question, increases
the probabilities of those events in a constant ratio, the relative
measures of those probabilities remain unaltered. If the principle
we are considering be then, as it appears to be, inseparably
involved in the very definition of probability, it can scarcely,
\emph{of itself}, conduct us further than the attentive study of
the definition would alone do, in the solution of problems. From these
considerations it appears to be doubtful whether, without some
aid of a different kind from any that has yet offered itself to our
notice, any considerable advance, either in the theory of
probabilities as a branch of speculative knowledge, or in the
practical solution of its problems can be hoped for. And the
establishment, \emph{solely} upon the basis of any such collection of principles as
the above, of a method universally applicable to the solution of
problems, without regard either to the number or to the nature
of the propositions involved in the expression of their data,
seems to be impossible. For the attainment of such an object
other elements are needed, the consideration of which will occupy
the next chapter.
%-----------------------File: 056.png----------------------------
%CHAPTER XVII.
\chapter[GENERAL METHOD IN PROBABILITIES]{\large DEMONSTRATION OF A GENERAL METHOD FOR THE SOLUTION OF
PROBLEMS IN THE THEORY OF PROBABILITIES.}

1. It has been defined (XVI.~2), that ``the measure of the
probability of an event is the ratio of the number of cases
favourable to that event, to the total number of cases favourable
or unfavourable, and all equally possible.'' In the following
investigations the term probability will be used in the above sense
of ``measure of probability.''

From the above definition we may deduce the following
conclusions.

I. When it is certain that an event will occur, the probability
of that event, in the above mathematical sense, is 1. For the
cases which are favourable to the event, and the cases which are
possible, are in this instance the same.

Hence, also, if $p$ be the probability that an event $x$ will happen,
$1 - p$ will be the probability that the said event will not happen.
To deduce this result directly from the definition, let $m$ be the
number of cases favourable to the event $x$, $n$ the number of cases
possible, then $n - m$ is the number of cases unfavourable to the
event $x$. Hence, by definition,
\begin{align*}
  &\frac{m}{n}     = \textrm{ probability that $x$ will happen.}  \\
  &\frac{n - m}{n} = \textrm{ probability that $x$ will not happen.} \\
\end{align*}
But
\[
  \frac{n - m}{n} = 1- \frac{m}{n} = 1 - p .
\]

II. The probability of the concurrence of any two events is
the product of the probability of either of those events by the
probability that if that event occur, the other will occur also.

Let $m$ be the number of cases favourable to the happening
of the first event, and $n$ the number of equally possible cases unfavourable to it;
then the probability of the first event is, by
%-----------------------File: 057.png----------------------------
definition, $\dfrac{m}{m + n}$. Of the $m$ cases favourable to the first event, let $l$
cases be favourable to the conjunction of the first and second
events, then, by definition, $\dfrac{l}{m}$ is the probability that if the first
event happen, the second also will happen. Multiplying these
fractions together, we have
\[
  \frac{m}{m + n} \times \frac{l}{m} = \frac{l}{m + n}.
\]
But the resulting fraction $\frac{l}{m + n}$ has for its numerator
the number of cases favourable to the conjunction of events, and for its
denominator, the number $m + n$ of possible cases. Therefore,
it represents the probability of the joint occurrence of the two
events.

Hence, if $p$ be the probability of any event $x$, and $q$ the probability
that if $x$ occur $y$ will occur, the probability of the conjunction $xy$ will be $pq$.

III. The probability that if an event $x$ occur, the event $y$ will
occur, is a fraction whose numerator is the probability of their
joint occurrence, and denominator the probability of the occurrence of the event $x$.

This is an immediate consequence of Principle 2nd.

IV. The probability of the occurrence of some one of a series
of exclusive events is equal to the sum of their separate probabilities.

For let $n$ be the number of possible cases; $m_1$ the number of
those cases favourable to the first event; $m_2$ the number of cases
favourable to the second, \&c. Then the separate probabilities of
the events are $\dfrac{m_1}{n}$, $\dfrac{m_2}{n}$, \&c. Again, as the events are exclusive,
none of the cases favourable to one of them is favourable to
another; and, therefore, the number of cases favourable to some
one of the series will be $m_1 + m_2 \dotsc$, and the probability of some
one of the series happening will be
$\dfrac{m_1 + m_2 \dotsc}{n}$. But this is the
sum of the previous fractions, $\dfrac{m_1}{n}$, $\dfrac{m_2}{n}$, \&c. Whence
the principle is manifest.
%-----------------------File: 058.png----------------------------
2. \textsc{Definition.}---Two events are said to be independent
when the probability of the happening of either of them is
unaffected by our expectation of the occurrence or failure of
the other.

From this definition, combined with Principle~II., we have
the following conclusion:

V. The probability of the concurrence of two independent
events is equal to the product of the separate probabilities of
those events.

For if $p$ be the probability of an event $x$, $q$ that of an event $y$
regarded as quite independent of $x$, then is $q$ also the probability
that if $x$ occur $y$ will occur. Hence, by Principle~II., $pq$ is the
probability of the concurrence of $x$ and $y$

Under the same circumstances, the probability that $x$ will
occur and $y$ not occur will be $p(1-q)$. For $p$ is the probability
that $x$ will occur, and $1-q$ the probability that $y$ will not occur.
In like manner $(1-p)(1-q)$ will be the probability that both
the events fail of occurring.

3. There exists yet another principle, different in kind from
the above, but necessary to the subsequent investigations of this
chapter, before proceeding to the explicit statement of which I
desire to make one or two preliminary observations.

I would, in the first place, remark that the distinction between
simple and compound events is not one founded in the
nature of events themselves, but upon the mode or connexion in
which they are presented to the mind. How many separate particulars,
for instance, are implied in the terms ``To be in health,''
``To prosper,'' \&c., each of which might still be regarded as
expressing a ``simple event''? The prescriptive usages of language,
which have assigned to particular combinations of events
single and definite appellations, and have left unnumbered other
combinations to be expressed by corresponding combinations of
distinct terms or phrases, is essentially arbitrary. When, then,
we designate as simple events those which are expressed by a
single verb, or by what grammarians term a simple sentence, we
do not thereby imply any real simplicity in the events themselves,
but use the term solely with reference to grammatical
expression.
%-----------------------File: 059.png----------------------------
4. Now if this distinction of events, as simple or compound, is
not founded in their real nature, but rests upon the accidents of
language, it cannot affect the question of their mutual dependence
or independence. If my knowledge of two simple events is
confined to this particular fact, viz., that the probability of the
occurrence of one of them is $p$, and that of the other $q$; then I regard
the events as independent, and thereupon affirm that the
probability of their joint occurrence is $pq$. But the ground of
this affirmation is not that the events are simple ones, but that
the data afford no information whatever concerning any connexion
or dependence between them. When the probabilities of events
are given, but all information respecting their dependence withheld,
the mind regards them as independent. And this mode of
thought is equally correct whether the events, judged according
to actual expression, are simple or compound, i.e., whether each
of them is expressed by a single verb or by a combination of
verbs.

5. Let it, however, be supposed that, together with the probabilities
of certain events, we possess some definite information
respecting their possible combinations. For example, let it be
known that certain combinations are excluded from happening,
and therefore that the remaining combinations alone are possible.
Then still is the same general principle applicable. The mode
in which we avail ourselves of this information in the calculation
of the probability of any conceivable issue of events depends not
upon the nature of the events whose probabilities and whose
limits of possible connexion are given. It matters not whether
they are simple or compound. It is indifferent from what source,
or by what methods, the knowledge of their probabilities and of
their connecting relations has been derived. We must regard
the events as independent of any connexion beside that of which
we have information, \emph{deeming it of no consequence whether such information
has been explicitly conveyed to us in the data, or thence
deduced by logical inference}. And this leads us to the statement
of the general principle in question, viz.:

VI. The events whose probabilities are given are to be regarded
as independent of any connexion but such as is either
expressed, or necessarily implied, in the data; and the mode in
%-----------------------File: 060.png----------------------------
which our knowledge of that connexion is to be employed is independent
of the nature of the source from which such knowledge
has been derived.

The practical importance of the above principle consists
in the circumstance, that whatever may be the nature of the
events whose probabilities are given,---whatever the nature of
the event whose probability is sought, we are always able, by an
application of the Calculus of Logic, to determine the expression
of the latter event as a definite combination of the former events,
and definitely to assign the whole of the implied relations connecting
the former events with each other. In other words, we
can determine what that combination of the given events is whose
probability is required, and what combinations of them are alone
possible. It follows then from the above principle, that we can
reason upon those events as if they were simple events, whose
conditions of possible combination had been directly given by
experience, and of which the probability of some definite combination
is sought. The possibility of a general method in probabilities
depends upon this reduction.

6. As the investigations upon which we are about to enter
are based upon the employment of the Calculus of Logic, it is
necessary to explain certain terms and modes of expression which
are derived from this application.

By the event $x$, I mean that event of which the proposition
which affirms the occurrence is symbolically expressed by the
equation
\[
x= 1.
\]
By the event $\phi(x,y,z,\dotsc)$, I mean that event of which the
occurrence is expressed by the equation
\[
\phi(x,y,z,\dotsc) = 1.
\]
Such an event may be termed a compound event, in relation to
the simple events $x$, $y$, $z$, which its conception involves. Thus,
if $x$ represent the event ``It rains,'' $y$ the event ``It thunders,''
the separate occurrences of those events being expressed by the
logical equations
\[
x = 1,\quad  y = 1,
\]
then will $x(1-y) + y (1-x)$ represent the event or state of
%-----------------------File: 061.png----------------------------
things denoted by the Proposition, ``It either rains or thunders,
but not both;'' the expression of that state of things being
\[
x(1 - y) + y(1 - x) = 1.
\]
If for brevity we represent the function $phi(x, y, z,\dotsc)$, used in
the above acceptation by $V$, it is evident (VI.~13) that the law
of duality
\[
V(1 - V) = 0,
\]
will be identically satisfied.

The simple events $x$, $y$, $z$ will be said to be ``conditioned''
when they are not free to occur in every possible combination;
in other words, when some compound event depending upon
them is precluded from occurring. Thus the events denoted by
the propositions, ``It rains,'' ``It thunders,'' are ``conditioned''
if the event denoted by the proposition, ``It thunders, but does
not rain,'' is excluded from happening, so that the range of possible
is less than the range of conceivable combination. Simple
unconditioned events are by definition independent.

Any compound event is similarly said to be conditioned if it
is assumed that it can only occur under a certain condition, that
is, in combination with some other event constituting, by its presence,
that condition.

7. We shall proceed in the natural order of thought, from
simple and unconditioned, to compound and conditioned events.
\newline
\newline
\begin{center}\textsc{Proposition I.}\newline\end{center}

1st. \emph{If $p$, $q$, $r$ are the respective probabilities of any unconditioned
simple events $x$, $y$, $z$, the probability of any compound
event $V$ will be $[V]$, this function $[V]$ being formed by changing,
in the function $V$, the symbols $x$, $y$, $z$ into $p$, $q$, $r$, \&c.}

2ndly. \emph{Under the same circumstances, the probability that if
the event $V$ occur, any other event $V'$ will also occur, will be
$\frac{[VV']}{V}$, wherein $[VV']$ denotes the result obtained by multiplying
together the logical functions $V$ and $V'$, and changing in the result
$x$, $y$, $z$, \&c. into $p$, $q$, $r$, \&c.}

Let us confine our attention in the first place to the
%-----------------------File: 062.png----------------------------
possible combinations of the two simple events, $x$ and $y$, of which the
respective probabilities are $p$ and $q$. The primary combinations
of those events (V.11), and their corresponding probabilities, are
as follows:
\[
\begin{array}{lll}
 \textsc{ events.}&  &\textsc{ probabilities.}\\
 xy,         &\text{ Concurrence of $x$ and $y$, }     &pq.\\
 x(1-y),     &\text{ Occurrence of $x$ without $y$, }  &p(1-q). \\
 (1-x)y,     &\text{ Occurrence of $y$ without $x$, }  &(1-p)q. \\
 (1-x)(1-y), &\text{ Conjoint failure of $x$ and $y$, }&(1-p)(1-q).\\
\end{array}
\]
We see that in these cases the probability of the compound event
represented by a constituent is the same function of $p$ and $q$ as
the logical expression of that event is of $x$ and $y$; and it is obvious
that this remark applies, whatever may be the number of the
simple events whose probabilities are given, and whose \emph{joint existence
or failure} is involved in the compound event of which we
seek the probability.

Consider, in the second place, any \emph{disjunctive} combination of
the above constituents. The compound event, expressed in ordinary
language as the occurrence of ``either the event $x$ without
the event $y$, or the event $y$ without the event $x$'' is symbolically
expressed in the form $x(1-y)+y(1-x)$, and its probability,
determined by Principles \textsc{iv}.\ and \textsc{v}., is $p(1-q)+q(1-p)$. The
latter of these expressions is the same function of $p$ and $q$ as the
former is of $x$ and $y$. And it is obvious that this is also a particular
illustration of a general rule. The events which are expressed
by any two or more constituents are mutually exclusive.
The only possible combination of them is a \emph{disjunctive} one, expressed
in ordinary language by the conjunction \emph{or}, in the language
of symbolical logic by the sign +. Now the probability of
the occurrence of some one out of a set of mutually exclusive
events is the sum of their separate probabilities, and is expressed
by connecting the expressions for those separate probabilities by
the sign +. Thus the law above exemplified is seen to be \emph{general}.
The probability of any unconditioned event $V$ will be found by
changing in $V$ the symbols $x, y, z, \dotsc$ into $p, q, r, \dotsc$

8. Again, by Principle \textsc{iii}., the probability that if the event
$V$ occur, the event $V'$ will occur with it, is expressed by a
%-----------------------File: 063.png----------------------------
fraction whose numerator is the probability of the joint occurrence
of $V$ and $V'$, and denominator the probability of the occurrence
of $V$.

Now the expression of that event, or state of things, which is
constituted by the joint occurrence of the events $V$ and $V'$, will
be formed by multiplying together the expressions $V$ and $V'$ according
to the rules of the Calculus of Logic; since whatever
constituents are found in both $V$ and $V'$ will appear in the product,
and no others. Again, by what has just been shown, the
probability of the event represented by that product will be determined by
changing therein $x, y, z$ into $p, q, r,\dotsc$ Hence the
numerator sought will be what $[VV']$ by definition represents.
And the denominator will be $[V]$, wherefore
\[
  \text{Probability that if $V$ occur, $V'$ will occur with it }
  = \frac{[VV']}{[V]}.
\]

9. For example, if the probabilities of the simple events
$x, y, z$ are $p, q, r$ respectively, and it is required to find the
probability that if either $x$ or $y$ occur, then either $y$ or $z$ will occur,
we have for the logical expressions of the antecedent and consequent---
\begin{gather*}
  \text{1st. Either $x$ or $y$ occurs, }x(1-y) + y(1-x).  \\
  \text{2nd. Either $y$ or $z$ occurs, }y(1-z) + z(1-y).
\end{gather*}
If now we multiply these two expressions together according to
the rules of the Calculus of Logic, we shall have for the expression of
the concurrence of antecedent and consequent,
\[
  xz(1-y) + y(1-x)(1-z).
\]
Changing in this result $x, y, z$ into $p, q, r,$ and similarly transforming
the expression of the antecedent, we find for the probability sought the value
\[
  \frac{pr(1-q) + q(1-p)(1-r)}{p(1-q) + q(1-p)}.
\]
The special function of the calculus, in a case like the above, is
to supply the office of the reason in determining what are the
conjunctures involved at once in the consequent and the antecedent. But the
advantage of this application is almost entirely
%-----------------------File: 064.png----------------------------
%**[2nd proofer: This page slipped though the first round without
%   being marked up, so it hasn't really been proofed twice.]
prospective, and will be made manifest in a subsequent proposition.


\begin{center}\textsc{Proposition II.}\end{center}

10. \emph{It is known that the probabilities of certain simple events
$x, y, z, \dotsc$ are $p, q, r, \dotsc$ respectively when a certain condition $V$ is
satisfied; $V$ being in expression a function of $x, y, z, \dotsc$. Required
the absolute probabilities of the events $x, y, z, \dotsc$, that is, the
probabilities of their respective occurrence independently of the condition V.}

Let, $p'$, $q'$, $r'$, \&c., be the probabilities required, i.~e.\ the probabilities of the events x, y, z,.., regarded not only as simple,
but as independent events. Then by Prop.~\textsc{i.} the probabilities
that these events will occur when the condition $V$, represented
by the logical equation $V=1$, is satisfied, are
\[
  \frac{\left[ xV \right]}{\left[ V \right]},\quad
  \frac{\left[ yV \right]}{\left[ V \right]},\quad
  \frac{\left[ zV \right]}{\left[ V \right]},\text{ \&c.,}
\]
in which $\left[ xV \right]$ denotes the result obtained by multiplying $V$ by
$x$, according to the rules of the Calculus of Logic, and changing
in the result $x$, $y$, $z$, into $p'$, $q'$, $r'$, \&c. But the above conditioned probabilities are by hypothesis equal to $p, q, r,\dotsc$ respectively. Hence we have,
\[
  \frac{\left[ xV \right]}{\left[ V \right]} = p,\quad
  \frac{\left[ yV \right]}{\left[ V \right]} = q,\quad
  \frac{\left[ zV \right]}{\left[ V \right]} = r,\text{ \&c.,}
\]
from which system of equations equal in number to the quantities
$p', q', r',\dotsc$, the values of those quantities may be determined.

Now $x V$ consists simply of those constituents in $V$ of which
$x$ is a factor. Let this sum be represented by $V_x$, and in like
manner let $yV$ be represented by $Vy$, \&c. Our equations then
assume the form
\[
  \frac{\left[ V_x \right]}{\left[ V \right]} = p,\quad
  \frac{\left[ V_y \right]}{\left[ V \right]} = q,\text{ \&c.,}
\]
where $\left[ V_x \right]$ denotes the results obtained by changing in $V_x$ the
symbols $x$, $y$, $z$, \&c., into $p'$, $q'$, $r'$, \&c.

To render the meaning of the general problem and the
%-----------------------File: 065.png----------------------------
principle of its solution more evident, let us take the following example.
Suppose that in the drawing of balls from an urn
attention had only been paid to those cases in which the balls
drawn were either of a particular colour, ``white,'' or of a particular
composition, ``marble,'' or were marked by both these
characters, no record having been kept of those cases in which a
ball that was neither white nor of marble had been drawn. Let
it then have been found, that whenever the supposed condition
was satisfied, there was a probability $p$ that a white ball would be
drawn, arid a probability $q$ that a marble ball would be drawn: and
from these data alone let it be required to find the probability
that in the next drawing, without reference at all to the condition
above mentioned, a white ball will be drawn; also the probability
that a marble ball will be drawn.

Here if $x$ represent the drawing of a white ball, $y$ that of a
marble ball, the condition $V$ will be represented by the logical
function
\[
  xy + x(1-y) + (1-x)y.
\]
Hence we have
\[
  V_x = xy + x(1-y) = x,\quad  V_y = xy + (1-x)y = y;
\]
whence
\[
  [V_x] = p;\quad [V_y] = q;
\]
and the final equations of the problem are
\[
  \frac{p'}{p'q' + p'(1-q') + q'(1-p')} = p, \quad
  \frac{q'}{p'q' + p'(1-q') + q'(1-p')} = q;
\]
from which we find
\[
  p' = \frac{p + q - 1}{q}, \quad
  q' = \frac{p + q - 1}{p}.
\]

It is seen that $p'$ and $q'$ are respectively proportional to $p$ and
$q$, as by Professor Donkin's principle they ought to be. The
solution of this class of problems might indeed, by a direct application
of that principle, be obtained.

To meet a possible objection, I here remark, that the above
reasoning does not require that the drawings of a white and a
marble ball should be independent, in virtue of the physical constitution
of the balls. The assumption of their independence is
indeed involved in the solution, but it does not rest upon any
%-----------------------File: 066.png----------------------------
prior assumption as to the nature of the balls, and their relations,
or freedom from relations, of form, colour, structure, \&c. It is
founded upon our total ignorance of all these things. Probability
always has reference to the state of our actual knowledge,
and its numerical value varies with varying information.

% [**Note: here follows the statement and proof of a Proposition
% The heading is centered in small caps, and the statement is
% a separate paragraph in italics.
% This could be imitated (after appropriately customizing \section)
% using something like
%\newtheoremstyle{boole}{}{}{\itshape}{}{\upshape}{.}{0pt} %{\thmname{#1}\thmnote{#3}\thmnumber{ #2}}
%\theoremstyle{boole}
%\newtheorem*{Propn}{}
%\newcommand\Proposition[2]{\section*{Proposition #1.}\Propn[#2]}
%
%\begin{Proposition}{III}{11}
%
\begin{center}\textsc{Proposition III.}\end{center} % to use above suggestions, comment out from here...

11. % ...down to here
To determine in any question of probabilities the logical
connexion of the qu\ae{}situm with the data; that is, to assign the event
whose probability is sought, as a logical function of the event whose
probabilities are given.
%\end{Proposition}

Let $S$, $T$, \&c., represent any compound events whose probabilities
are given, $S$ and $T$ being in expression known functions
of the symbols $x$, $y$, $z$, \&c., representing simple events.
Similarly let $W$ represent any event whose probability is sought,
$W$ being also a known function of $x$, $y$, $z$, \&c. As $S, T,\dotsc W$
must satisfy the fundamental law of duality, we are permitted
to replace them by single logical symbols, $s, t,\dotsc w$. Assume
then
\[
s = S, t=T, w = W.
\]

These, by the definition of $S, T,\dotsc W$, will be a series of
logical equations connecting the symbols $s, t,\dotsc w$, with the symbols $x, y, z\dotsc$

By the methods of the Calculus of Logic we can eliminate
from the above system any of the symbols $x, y, z,\dotsc$, representing
events whose probabilities are not given, and determine
$w$ as a developed function of $s$, $t$, \&c., and of such of the symbols
$x$, $y$, $z$, \&c., if any such there be, as correspond to events whose
probabilities are given. The result will be of the form
\[
w = A + 0B +\frac{0}{0} C +\frac{1}{0} D,
\]
where $A$, $B$, $C$, and $D$ comprise among them all the possible
\textit{constituents} which can be formed from the symbols $s$, $t$, \&c., i.~e.\
from all the symbols representing events whose probabilities are
given.

The above will evidently be the complete expression of the
relation sought. For it fully determines the event $W$,
%-----------------------File: 067.png----------------------------
represented by the single symbol $w$, as a function or combination of
the events similarly denoted by the symbols $s$, $t$, \&c., and it assigns
by the laws of the Calculus of Logic the condition
\[
D-0\text{,}
\]
as connecting the events $s$, $t$, \&c., among themselves. We may,
therefore, by Principle \textsc{vi.}, regard $s$, $t$, \&c., as \emph{simple} events, of
which the combination $w$, and the condition with which it is associated
$D$, are definitely determined.

Uniformity in the logical processes of reduction being desirable,
I shall here state the order which will generally be pursued.

12. By (\textsc{viii.} 8), the primitive equations are reducible to
the forms
\begin{align*}
&s(1-S) + S(1-s) = 0;\\
&t(1-T)+ T(1-t) = 0;\tag{1}\\
\multispan2\dotfill\\
&w(1-W)-W(1-w) = 0;
\end{align*}
under which they can be added together without impairing their
significance. We can then eliminate the symbols $x$, $y$, $z$, either
separately or together. If the latter course is chosen, it is necessary,
after adding together the equations of the system, to
develop the result with reference to all the symbols to be eliminated,
and equate to $0$ the product of all the coefficients of the
constituents (\textsc{vii.} 9).

As $w$ is the symbol whose expression is sought, we may also,
by Prop.~\textsc{iii.} Chap.~\textsc{ix.}, express the result of elimination in the
form
\[
Ew + E'(1-w) = 0\text{.}
\]

$E$ and $E'$ being successively determined by making in the
general system~(1), $w = 1$ and $w = 0$, and eliminating the symbols
$x$, $y$, $z$, $\dotsc$ Thus the single equations from which $E$ and $E'$ are
to be respectively determined become
\begin{align*}
&s(1-S)+S(1-s)+t(1-T) + T(1-t)\ldots+1-W=0\text{;}\\
&s(1-S)+S(1-s)+t(1-T) + T(1-t)+W=0\text{.}
\end{align*}
From these it only remains to eliminate $x$, $y$, $z$, \&c., and to determine
$w$ by subsequent development.
%-----------------------File: 068.png----------------------------
In the process of elimination we may, if needful, avail ourselves
of the simplifications of Props.~\textsc{i.} and \textsc{ii.} Chap.~\textsc{ix.}

13. Should the data, beside informing us of the probabilities
of events, further assign among them any explicit connexion, such
connexion must be logically expressed, and the equation or equations
thus formed be introduced into the general system.

% [**Note: here follows the statement and proof of another Proposition
% The heading is centered in small caps, and the statement is
% a separate paragraph in italics.
% See my suggested markup on p066
\begin{center}\textsc{Proposition IV.}\end{center}

14. \emph{Given the probabilities of any system of events; to determine
by a general method the consequent or derived probability of
any other event.}

As in the last Proposition, let $S$, $T$, \&c., be the events whose
probabilities are given, $W$ the event whose probability is sought,
these being known functions of $x$, $y$, $z$, \&c. Let us represent the
data as follows:
\[
\begin{aligned}
\text{Probability of }S &= p;\\
\text{Probability of }T &= q;
\end{aligned}\tag{1}
\]
and so on, $p$, $q$, \&c., being known numerical values. If then
we represent the compound event $S$ by $s$, $T$ by $t$, and $W$ by $w$,
we find by the last proposition,
\[
w = A + 0B +\frac{0}{0} C + \frac{1}{0} D; \tag{2}
\]
$A$, $B$, $C$, and $D$ being functions of $s$, $t$, \&c. Moreover the data
(1) are transformed into
\[
\text{Prob. }s =p,\quad \text{Prob. }t = q, \text{ \&c.} \tag{3}
\]

Now the equation (2) is resolvable into the system
\[
\left.
\begin{gathered}
w = A + qC\\
D=0,
\end{gathered}\quad\right\}\tag{4}
\]
$q$ being an indefinite class symbol (VI.~12). But since by the
properties of constituents (V.~Prop.~\textsc{iii.}), we have
\[
A + B + C + D= 1,
\]
the second equation of the above system may be expressed in the
form
\[
A + B+ C = 1.
\]

%-----------------------File: 069.png----------------------------
If we represent the function $A + B + C$ by $V$, the system (4)
becomes
\begin{gather}
  w = A + qC;   \tag{5}   \\
  V = 1.        \tag{6}
\end{gather}

Let us for a moment consider this result. Since $V$ is the sum
of a series of constituents of $s$, $t$, \&c., it represents the compound
event in which the simple events involved are those denoted by
$s$, $t$, \&c. Hence (6) shows that the events denoted by $s$, $t$, \&c.,
and whose probabilities are $p$, $q$, \&c., have such probabilities not
as \emph{independent events}, but as events subject to a certain condition
$V$. Equation~(5) expresses $w$ as a similarly conditioned combination of the same events.

Now by Principle~\textsc{vi.} the mode in which this knowledge of the
connexion of events has been obtained does not influence the mode
in which, when obtained, it is to be employed. We must reason
upon it as if experience had presented to us the events $s$, $t$, \&c.,
as simple events, free to enter into every combination, but possessing, when actually subject to the condition $V$, the probabilities $p$, $q$, \&c., respectively.

Let then $p', q', \dotsc,$ be the corresponding probabilities of such
events, when the restriction $V$ is removed. Then by Prop.~\textsc{ii.}
of the present chapter, these quantities will be determined by the
system of equations,
\[
  \frac{[V_s]}{[V]} = p, \qquad
  \frac{[V_t]}{[V]} = q, \text{ \&c.;}  \tag{7}
\]
and by Prop.~\textsc{i.} the probability of the event $w$ under the same
condition $V$ will be
\[
  \text{Prob. } w = \frac{[A + cC]}{[V]};   \tag{8}
\]
wherein $V_s$ denotes the sum of those constituents in $V$ of which $s$
is a factor, and $[V_s]$ what that sum becomes when $s, t, \dotsc,$ are
changed into $p', q', \dotsc,$ respectively. The constant $c$ represents
the probability of the indefinite event $q$; it is, therefore, arbitrary,
and admits of any value from 0 to 1.

Now it will be observed, that the values of,$p'$, $q'$, \&c., are determined from (7) only in order that they may be substituted in
(8), so as to render Prob. $w$ a function of known quantities, $p$, $q$,
%-----------------------File: 070.png----------------------------
\&c. It is obvious, therefore, that instead of the letters $p'$, $q'$, \&c.,
we might employ any others as $s$, $t$, \&c., in the same \emph{quantitative}
acceptations. This particular step would simply involve a change
of meaning of the symbols $s$, $t$, \&c.---their ceasing to be \emph{logical},
and becoming \emph{quantitative}. The systems (7) and (8) would then
become
\begin{gather}
  \frac{V_s}{V} = p,\quad
  \frac{V_t}{V} = q,\text{ \&c.;}       \tag{9}   \\
  \text{Prob. }w = \frac{A + cC}{V}.    \tag{10}
\end{gather}
In employing these, it is only necessary to determine from (9)
$s$, $t$, \&c., regarded as quantitative symbols, in terms of $p$, $q$, \&c.,
and substitute the resulting values in (10). It is evident, that
$s$, $t$, \&c., inasmuch as they represent \emph{probabilities}, will be positive
proper fractions.

The system (9) may be more symmetrically expressed in the
form
\[
  \frac{V_s}{p} = \frac{V_t}{q} \dotso = V.     \tag{11}
\]
Or we may express both (9) and (10) together in the symmetrical system
\[
  \frac{V_s}{p} = \frac{V_t}{q} \dotso = \frac{A+cC}{u} = V; \tag{12}
\]
wherein $u$ represents Prob. $w$.

15. It remains to interpret the constant $c$ assumed to represent the probability of the indefinite event $q$. Now the \emph{logical}
equation
\[
  w = A + qC,
\]
interpreted in the reverse order, implies that if either the event
$A$ take place, or the event $C$ in connexion with the event $q$, the
event $w$ will take place, and not otherwise. Hence $q$ represents
that condition under which, if the event $C$ take place, the event
$w$ will take place. But the probability of $q$ is $c$. Hence, therefore, $c =$ probability that if the event $C$ take place the event $w$
will take place.

Wherefore by Principle ~\textsc{ii.},
\[
  c = \frac{ \text{Probability of concurrence of $C$ and $w$} }
           { \text{Probability of $C$} }.
\]
%-----------------------File: 071.png----------------------------
We may hence determine the nature of that new experience
from which the actual value of c may be obtained. For if we
substitute in $C$ for $s$, $t$, \&c., their original expressions as functions
of the simple events $x$, $y$, $z$, \&c., we shall form the expression
of that event whose probability constitutes the denominator
of the above value of $c$; and if we multiply that expression
by the original expression of $w$, we shall form the expression of
that event whose probability constitutes the numerator of $c$, and
\emph{the ratio of the frequency of this event to that of the former one, determined
by new observations} will give the value of $c$. Let it be
remarked here, that the constant $c$ does not \emph{necessarily} make its
appearance in the solution of a problem. It is only when the
data are insufficient to render determinate the probability sought,
that this arbitrary element presents itself, and in this case it is
seen that the final logical equation (2) or (5) informs us how it
is to be determined.

If that new experience by which $c$ may be determined cannot
be obtained, we can still, by assigning to $c$ its limiting values
0 and 1, determine the limits of the probability of $w$. These
are
\[
\begin{array}{ll}
  \textrm{Minor limit of Prob. } w &= \frac{A}{V}.   \\
  \textrm{Superior limit }         &= \frac{A+C}{V}.
\end{array}
\]
Between these limits, it is certain that the probability sought
must lie independently of all new experience which does not absolutely contradict the past.

If the expression of the event $C$ consists of many constituents,
the logical value of $w$ being of the form
\[
  w = A + \frac{0}{0}C_1 + \frac{0}{0}C_2 + \text{ \&c.,}
\]
we can, instead of employing their aggregate as above, present
the final solution in the form
\[
  \textrm{Prob. } w = \frac{ A + c_1C_1 + c_2C_2 + \text{ \&c.} }{V}.
\]

%-----------------------File: 072.png----------------------------
Here $c_1 =$ probability that if the event $C_1$ occur, the event $w$ will
occur, and so on for the others. Convenience must decide which
form is to be preferred.

16. The above is the complete theoretical solution of the
problem proposed. It may be added, that it is applicable equally
to the case in which any of the events mentioned in its original
statement are conditioned. Thus, if one of the data is the probability $p$, that if the event $x$ occur the event $y$ will occur; the
probability of the occurrence of $x$ not being given, we must assume Prob. $x = c$ (an arbitrary constant), then Prob. $xy = cp$, and
these two conditions must be introduced into the data, and employed according to the previous method. Again, if it is sought
to determine the probability that if an event $x$ occur an event $y$
will occur, the solution will assume the form
\[
  \text{Prob. sought }=\frac{ \text{Prob. }xy }{ \text{Prob. }x },
\]
the numerator and denominator of which must be separately determined by the previous general method.

17. We are enabled by the results of these investigations to
establish a general rule for the solution of questions in probabilities.


\begin{center}\textsc{General Rule.}\end{center}

\textsc{Case I.}---\emph{When all the events are unconditioned.}

Form the symbolical expressions of the events whose probabilities are given or sought.

Equate such of those expressions as relate to compound events
to a new series of symbols, $s$, $t$, \&c., which symbols regard as representing the events, no longer as compound but simple, to
whose expressions they have been equated.

Eliminate from the equations thus formed all the logical symbols, except those which express events, $s$, $t$, \&c., whose respective
probabilities $p$, $q$, \&c. are given, or the event $w$ whose probability
is sought, and determine $w$ as a developed function of $s$, $t$, \&c.
in the form
\[
  w = A + 0B + \frac{0}{0}C + \frac{1}{0}D.
\]
%-----------------------File: 073.png----------------------------
Let $A + B + C = V$, and let $V_s$ represent the aggregate of
those constituents in $V$ which contain $s$ as a factor, $V_t$ of those
which contain $t$ as a factor, and thus for all the symbols whose
probabilities are given.

Then, passing from Logic to Algebra, form the equations
\begin{align*}
  &\frac{V_s}{p} = \frac{V_t}{q} = V,  \tag{1}  \\
  &\text{Prob. } w = \frac{A + cC}{V}, \tag{2}
\end{align*}
from (1) determine $s$, $t$, \&c. as functions of $p$, $q$, \&c., and substitute their values in (2). The result will express the solution
required.

Or form the symmetrical system of equations
\[
  \frac{V_s}{p} = \frac{V_t}{q} \dotsc =  \frac{A+cC}{u}
 = \frac{V}{1},  \tag{3}
\]
where $u$ represents the probability sought.

If $c$ appear in the solution, its interpretation will be
\[
  c = \frac{\text{Prob. }Cw}{\text{Prob. }c},
\]
and this interpretation indicates the nature of the experience
which is necessary for its discovery.

\textsc{Case~II.}---\emph{When some of the events are conditioned.}

If there be given the probability $p$ that if the event $X$ occur,
the event $Y$ will occur, and if the probability of the antecedent
$X$ be not given, resolve the proposition into the two following,
viz.:
\[
\begin{array}{ll}
  \text{Probability of } X  &= c,   \\
  \text{Probability of } XY &= cp.
\end{array}
\]
If the qu{\ae}situm be the probability that if the event $W$ occur,
the event $Z$ will occur, determine separately, by the previous
case, the terms of the fraction
\[
  \frac{\text{Prob. } WZ}{\text{Prob. }W},
\]
and the fraction itself will express the probability sought.
%-----------------------File: 074.png----------------------------

It is understood in this case that $X$, $Y$, $W$, $Z$ may be any
compound events whatsoever. The expressions $XY$ and $WZ$
represent the products of the symbolical expressions of $X$ and $Y$
and of $W$ and $Z$, formed according to the rules of the Calculus of
Logic.

The determination of the single constant $c$ may in certain
cases be resolved into, or replaced by, the determination of a series
of arbitrary constants $c_1, c_2\dotsc$ according to convenience, as previously
explained.

$18$. It has been stated (I.~$12$) that there exist two distinct definitions,
or modes of conception, upon which the theory of probabilities
may be made to depend, one of them being connected
more immediately with Number, the other more directly with
Logic. We have now considered the consequences which flow
from the numerical definition, and have shown how it conducts
us to a point in which the necessity of a connexion with Logic
obviously suggests itself. We have seen to some extent what
is the nature of that connexion; and further, in what manner the
peculiar processes of Logic, and the more familiar ones of quantitative
Algebra, are involved in the same general method of solution,
each of these so accomplishing its own object that the two
processes may be regarded as supplementary to each other. It
remains to institute the reverse order of investigation, and, setting
out from a definition of probability in which the logical relation
is more immediately involved, to show how the numerical definition
would thence arise, and how the same general method,
equally dependent upon both elements, would finally, but by a
different order of procedure, be established.

That between the symbolical expressions of the logical calculus
and those of Algebra there exists a close analogy, is a fact
to which attention has frequently been directed in the course of
the present treatise. It might even be said that they possess a
community of forms, and, to a very considerable degree, a community
of laws. With a single exception in the latter respect,
their difference is only one of interpretation. Thus the same
expression admits of a logical or of a quantitative interpretation,
according to the particular meaning which we attach to the
%-----------------------File: 075.png----------------------------
symbols it involves. The expression $xy$ represents, under the former
condition, a concurrence of the events denoted by $x$ and $y$; under
the latter, the product of the numbers or quantities denoted by $x$
and $y$. And thus every expression denoting an event, simple or
compound, admits, under another system of interpretation, of a
meaning purely quantitative. Here then arises the question,
whether there exists any principle of transition, in accordance
with which the logical and the numerical interpretations of the
same symbolical expression shall have an intelligible connexion.
And to this question the following considerations afford an
answer.

$19$. Let it be granted that there exists such a feeling as expectation,
a feeling of which the object is the occurrence of events,
and which admits of differing degrees of intensity. Let it also
be granted that this feeling of expectation accompanies our
knowledge of the circumstances under which events are produced,
and that it varies with the degree and kind of that knowledge.
Then, without assuming, or tacitly implying, that the intensity
of the feeling of expectation, viewed as a mental emotion, admits
of precise numerical measurement, it is perfectly legitimate to
inquire into the possibility of a mode of numerical estimation
which shall, at least, satisfy these following conditions, viz., that
the numerical value which it assigns shall increase when the
known circumstances of an event are felt to justify a stronger
expectation, shall diminish when they demand a weaker expectation,
and shall remain constant when they obviously require an
equal degree of expectation.

Now these conditions at least will be satisfied, if we assume
the fundamental principle of expectation to be this, viz., that the
laws for the expression of expectation, viewed as a numerical
element, shall be the same as the laws for the expression of the
expected event viewed as a logical element. Thus if $\phi(x, y, z)$ represent
any unconditional event compounded in any manner of
the events $x$, $y$, $z$, let the same expression $\phi(x, y, z)$, according
to the above principle, denote the expectation of that event;
$x$, $y$, $z$ representing no longer the simple events involved, but
the expectations of those events.
%-----------------------File: 076.png----------------------------

For, in the first place, it is evident that, under this hypothesis,
the probability of the occurrence of some one of a set of mutually
exclusive events will be equal to the sum of the separate probabilities
of those events. Thus if the alternation in question consist
of $n$ mutually exclusive events whose expressions are
\[
  \phi_1(x, y, z),\phi_2(x, y,z),\dotsc\phi_n(x, y, z),
\]
the expression of that alternation will be
\[
  \phi_1(x, y, z) + \phi_2(x, y,z)\dotsc + \phi_n(x, y, z) = 1;
\]
the literal symbols $x$, $y$, $z$ being logical, and relating to the simple
events of which the three alternatives are compounded:
and, by hypothesis, the expression of the probability that some
one of those alternatives will occur is
\[
  \phi_1(x, y, z) + \phi_2(x, y,z) \dotsc + \phi_n(x, y, z),
\]
$x$, $y$, $z$ here denoting the probabilities of the above simple events.
Now this expression increases, \emph{c\ae teris paribus}, with the increase
of the number of the alternatives which are involved, and diminishes
with the diminution of their number; which is agreeable
to the condition stated.

Furthermore, if we set out from the above hypothetical definition
of the measure of probability, we shall be conducted,
either by necessary inference or by successive steps of suggestion,
which might perhaps be termed \emph{necessary}, to the received numerical
definition. We are at once led to recognise unity ($1$)
as the proper numerical measure of certainty. For it is certain
that any event $x$ or its contrary $1-x$ will occur. The expression
of this proposition is
\[
  x + (1-x) = 1,
\]
whence, by hypothesis, $x + (1-x)$, the measure of the probability
of the above proposition, becomes the measure of certainty.
But the value of that expression is $1$, whatever the particular
value of $x$ may be. Unity, or $1$, is therefore, on the hypothesis
in question, the measure of certainty.

Let there, in the next place, be $n$ mutually exclusive, but
equally possible events, which we will represent by
$t_1, t_2, \dotsc t_n$.
%-----------------------File: 077.png----------------------------
The proposition which affirms that some one of these must occur
will be expressed by the equation
\[
  t_1 + t_2 \dotsc + t_n= 1;
\]
and, as when we pass in accordance with the reasoning of the
last section to numerical probabilities, the same equation remains
true in form, and as the probabilities $t_1, t_2 \dotsc t_n$ are equal, we
have
\[
  nt_1 = 1,
\]
whence $t_l = \frac{1}{n}$, and similarly $t_2 = \frac{1}{n}$,
$t_n = \frac{1}{n}$. Suppose it then required to determine the probability that some one event of the
partial series $t_1, t_2 \dotsc t_m$ will occur, we have for the expression
required
\begin{align*}
  t_1+t_2\dotsc+t_m &=\frac{1}{n}+\frac{1}{m}\dotsc&&\text{to $m$ terms}\\
                    &=\frac{m}{n}.
\end{align*}
Hence, therefore, if there are $m$ cases favourable to the occurrence
of a particular alternation of events out of $n$ possible and
equally probable cases, the probability of the occurrence of that
alternation will be expressed by the fraction $\frac{m}{n}$.

Now the occurrence of any event which may happen in different
equally possible ways is really equivalent to the occurrence
of an alternation, i.e., of some one out of a set of alternatives.
Hence the probability of the occurrence of any event may be
expressed by a fraction whose numerator represents the number
of cases favourable to its occurrence, and denominator the total
number of equally possible cases. But this is the rigorous numerical
definition of the measure of probability. That definition is
therefore involved in the more peculiarly \emph{logical} definition, the
consequences of which we have endeavoured to trace.

$20$. From the above investigations it clearly appears, $1$st,
that whether we set out from the ordinary numerical definition
of the measure of probability, or from the definition which assigns
to the numerical measure of probability such a law of value as
shall establish a formal identity between the logical expressions
%-----------------------File: 078.png----------------------------
of events and the algebraic expressions of their values, we shall
be led to the same system of practical results. $2$ndly, that
either of these definitions pursued to its consequences, and considered
in connexion with the relations which it inseparably involves,
conducts us, by inference or suggestion, to the other
definition. To a scientific view of the theory of probabilities
it is essential that both principles should be viewed together, in
their mutual bearing and dependence.
%-----------------------File: 079.png----------------------------
%CHAPTER XVIII.

\chapter[ELEMENTARY ILLUSTRATIONS]{\large ELEMENTARY ILLUSTRATIONS OF THE GENERAL METHOD IN PROBABILITIES.}

$1$. It is designed here to illustrate, by elementary examples,
the general method demonstrated in the last chapter.
The examples chosen will be chiefly such as, from their simplicity,
permit a ready verification of the solutions obtained.
But some intimations will appear of a higher class of problems,
hereafter to be more fully considered, the analysis of which
would be incomplete without the aid of a distinct method determining
the necessary conditions among their data, in order that
they may represent a possible experience, and assigning the corresponding
limits of the final solutions. The fuller consideration
of that method, and of its applications, is reserved for the next
chapter.

2. Ex.~$1$.---The probability that it thunders upon a given
day is $p$, the probability that it both thunders and hails is $q$, but
of the connexion of the two ph{\ae}nomena of thunder and hail, nothing
further is supposed to be known. Required the probability
that it hails on the proposed day.
\begin{align*}
&\text{Let $x$ represent the event---It thunders.}\\
&\text{Let $y$ represent the event---It hails.}
\end{align*}
Then $xy$ will represent the event---It thunders and hails; and
the data of the problem are
\[
  \mathrm{Prob. },x=p,\quad\mathrm{Prob. },xy=q.
\]
There being here but one compound event $xy$ involved, assume,
according to the rule,
\[
  xy = u.  \tag{1}
\]
Our data then become
\[
  \mathrm{Prob. },x=p,\quad\mathrm{Prob. },u=q;\tag{2}
\]
and it is required to find $\mathrm{Prob. },y$. Now ($1$) gives
%-----------------------File: 080.png----------------------------
\[
  y = \frac{u}{x}
    = ux + \frac{1}{0}u(1-x) + 0(1-u)x + \frac{0}{0}(1-u)(1-x).
\]
Hence (XVII.~17) we find
\begin{align*}
  V   &= ux + (1-u)x + (1-u)(1-x),   \\
  V_x &= ux + (1-u)x = x, \quad V_u = ux;
\end{align*}
and the equations of the General Rule, viz.,
\begin{gather*}
  \frac{V_x}{p} = \frac{V_u}{q} = V.   \\
  \mathrm{Prob. },y = \frac{A + cC}{V}
\end{gather*}
become, on substitution, and observing that $A = ux$,
$C = (1-u)(1-x)$, and that $V$ reduces to $x + (1-u)(1-x)$,
\begin{align*}
 &\frac{x}{p} = \frac{ux}{q} = x + (1-u)(1-x),  \tag{3}   \\
 &\mathrm{Prob. },y = \frac{ux + c(1-u)(1-x)}{x + (1-u)(1-x)},\tag{4}\\
\end{align*}
from which we readily deduce, by elimination of $x$ and $u$,
\[
  \mathrm{Prob. },y = q + c(l-p).   \tag{5}
\]
In this result $c$ represents the unknown probability that if the
event $(1-u)(1-x)$ happen, the event $y$ will happen. Now
$(l-u)(l-x) = (l-xy)(1-x) = 1-x$, on actual multiplication.
Hence $c$ is the unknown probability that if it do not thunder, it
will hail.

The general solution ($5$) may therefore be interpreted as follows:---The
probability that it hails is equal to the probability
that it thunders and hails, $q$, together with the probability that it
does not thunder, $1-p$, multiplied by the probability $c$, that if it
does not thunder it will hail. And common reasoning verifies
this result.

If $c$ cannot be numerically determined, we find, on assigning
to it the limiting values $0$ and $1$, the following limits of $\mathrm{Prob. },y$,
viz.:
\begin{align*}
  &\text{Inferior limit $= q$.}   \\
  &\text{Superior limit $= q + 1-p$.}
\end{align*}
%-----------------------File: 081.png----------------------------
\textbf{3. Ex. 2.}---The probability that one or both of two events
happen is $p$, that one or both of them fail is $q$. What is the
probability that only one of these happens?

Let $x$ and $y$ represent the respective events, then the data are---
\[
\begin{array}{c}
\text{Prob. }xy + x (1-y) + (1-x)y = p, \\
\text{Prob. }x(1-y) + (1-x)y + (1-x)(1-y) = q;
\end{array}
\]
and we are to find
\[\text{Prob. }x(1-y) + y(1-x).\]
Here all the events concerned being compound, assume
\[
\begin{array}{c}
xy + x(1-y) + (1-x)y = s, \\
x(1-y) + (1-x)y + (1-x)(1-y) = t, \\
x(1-y) + (1-x)y = w.
\end{array}
\]
Then eliminating $x$ and $y$, and determining $w$ as a developed
function of $s$ and $t$, we find
\[w = st + 0 s(1-t) + 0 (1-s)t + \frac{1}{0} (1-s)(1-t).\]
Hence $A = st, C=0, V=st + s(1-t) + (1-s)t = s + (1-s)t,
V_s=s, V_t=t$; and the equations of the General Rule (XVII.~17)
become
\begin{equation}
\frac{s}{p} = \frac{t}{q} = s + (1-s)t, \tag{1}
\end{equation}
\[
\text{Prob. }w = \frac{st}{s + (1-s)t};
\]
whence we find, on eliminating $s$ and $t$,
\[
\text{Prob. }w = p + q - 1.
\]
Hence $p + q - 1$ is the measure of the probability sought. This
result may be verified as follows:---Since $p$ is the probability that
one or both of the given events occur, $1-p$ will be the probability
that they both fail; and since $q$ is the probability that one
or both fail, $1-q$ is the probability that they both happen.
Hence $1-p + 1-q$, or $2-p-q$, is the probability that they
either both happen or both fail. But the only remaining alternative
which is possible is that one alone of the events happens.
Hence the probability of this occurrence is $1-(2-p-q)$, or
$p + q - 1$, as above.
%-----------------------File: 082.png----------------------------
4. Ex. 3.---The probability that a witness $A$ speaks the truth
is $p$, the probability that another witness $B$ speaks the truth is $q$,
and the probability that they disagree in a statement is $r$. What
is the probability that if they agree, their statement is true?

Let $x$ represent the hypothesis that $A$ speaks truth; $y$ that
$B$ speaks truth; then the hypothesis that $A$ and $B$ disagree in
their statement will be represented by $x(1-y) + y(1-x)$; the
hypothesis that they agree in statement by $xy + (1-x)(1-y)$,
and the hypothesis that they agree in the truth by $xy$. Hence
we have the following data:
\[
  \text{Prob. } x = p,\quad \text{Prob. } y = q,\quad
  \text{Prob. } x(1-y) + y(1-x) = r,
\]
from which we are to determine
\[
  \frac{ \text{Prob. } xy }{ \text{Prob. } xy + (1-x)(1-y) }.
\]
But as Prob. $x(1-y) + y(1-x) = r$, it is evident that Prob.
$xy + (1-x)(1-y)$ will be $1-r$; we have therefore to seek
\[
  \frac{ \text{Prob. } xy }{1-r}.
\]
Now the compound events concerned being in expression,
$x(1-y) + y(1-x)$ and $xy$, let us assume
\[
\left.
\begin{array}{cl}
  x(1-y) + y(1-x) &= s   \\
  xy              &= w
\end{array}
\right\}   \tag{1}
\]
Our data then are Prob. $x = p$, Prob. $y = q$, Prob. $s = r$, and we
are to find Prob. $w$.

The system~(1) gives, on reduction,
\begin{multline*}
  \{x(1-y) + y(1-x)\}(1-s) + s\{xy + (1-x)(1-y)\}   \\
  + xy(1-w) + w(1-xy) = 0;
\end{multline*}
whence
\begin{gather*}
w  = \frac{x(1-y)(1-s) + y(1-x)(1-s) + sxy + s(1-x)(1-y) + xy}{2xy-1}\\
   = \frac{1}{0}xys + xy(1-s) + 0x(1-y)s + \frac{1}{0}x(1-y)(1-s) \\
   + 0(1-x)ys + \frac{1}{0}(1-x)(1-y)s + \frac{1}{0}(1-x)y(1-s)
%\raisetag{1.5ex} The raisetag command seems to not work on my setup.
\tag{2} \\
   + 0(1-x)(1-y)(1-s).
\end{gather*}
% Ideally a split should be used to have the tag where the original was located
% But split flushes everthing to the right, or if &s are used then the lines are not centered.
%-----------------------File: 083.png----------------------------
In the expression of this development, the coefficient $\dfrac{1}{0}$ has been
made to replace every equivalent form (X. 6). Here we have
\[
  V = xy(1-s) + x(1-y)s + (1-x)ys + (1-x)(1-y)(1-s);
\]
whence, passing from Logic to Algebra,
\begin{gather*}
    \frac{xy(1-s) + x(1-y)s}{p} = \frac{xy(1-s) + (1-x)ys}{q}   \\
  = \frac{x(1-y)s + (1-x)ys}{r}\\
= xy(1-s) + x(1-y)s + (1-x)ys + (1-x)(1-y)(1-s).   \\
  \textrm{Prob. }w
= \frac{xy(1-s)}{xy(1-s) + x(1-y)s + (1-x)ys + (1-x)(1-y)(1-s)},
\end{gather*}
from which we readily deduce
\[
  \textrm{Prob. }w = \frac{p+q-r}{2};
\]
whence we have
\[
  \frac{ \text{Prob. }xy }{1-r} = \frac{p+q-r}{2(1-r)}  \tag{3}
\]
for the value sought.

If in the same way we seek the probability that if $A$ and $B$
agree in their statement, that statement will be false, we must
replace the second equation of the system (1) by the following,
viz.:
\[
  (1-x)(1-y) = w;
\]
the final logical equation will then be
\begin{gather*}
  w = \frac{1}{0}xys + 0xy(1-s) + 0x(1-y)s + \frac{1}{0}x(1-y)(1-s)\\
\hfill + 0(1-x)ys + \frac{1}{0}(1-x)y(1-s) + \frac{1}{0}(1-x)(1-y)s \\
\hfill + (1-x)(1-y)(1-s);   \tag{4}
\end{gather*}
whence, proceeding as before, we finally deduce
\[
  \text{Prob. }w = \frac{2 - p - q - r}{2}.   \tag{5}
\]
Wherefore we have

%-----------------------File: 084.png----------------------------
\[
\frac{\text{Prob. } (1-x)(1-y)}{1-r} = \frac{2-p-g-r}{2(1-r)} \tag{6}
\]
for the value here sought.

These results are mutually consistent. For since it is certain
that the joint statement of $A$ and $B$ must be either true or false,
the second members of (3) and (5) ought by addition to make 1.
Now we have identically,
\[
\frac{p+q-r}{2(1-r)} + \frac{2-p-q-r}{2(1-r)} = 1.
\]

It is probable, from the simplicity of the results (5) and (6),
that they might easily be deduced by the application of known
principles; but it is to be remarked that they do not fall directly
within the scope of known \emph{methods}. The number of the data
exceeds that of the simple events which they involve. M.~Cournot,
in his very able work, ``Exposition de la Theorie des
Chances," has proposed, in such cases as the above, to select
from the original premises different sets of data, each set equal in
number to the simple events which they involve, to assume that
those simple events are independent, determine separately from
the respective sets of the data their probabilities, and comparing
the different values thus found for the same elements, judge how
far the assumption of independence is justified. This method
can only approach to correctness when the said simple events
prove, according to the above criterion, to be nearly or quite independent;
and in the questions of testimony and of judgment,
in which such an hypothesis is adopted, it seems doubtful whether
it is justified by actual experience of the ways of men.

5. Ex. 4.---From observations made during a period of general
sickness, there was a probability $p$ that any house taken at
random in a particular district was visited by fever, a probability
$q$ that it was visited by cholera, and a probability $r$ that it escaped
both diseases, and was not in a defective sanitary condition
as regarded cleanliness and ventilation. What is the probability
that any house taken at random was in a defective sanitary
condition?

With reference to any house, let us appropriate the symbols
$x, y, z,$ as follows, viz.:
%-----------------------File: 085.png----------------------------

\begin{tabular}{rl}
  The symbol $x$& to the visitation of fever.\\
             $y$&\hfill ''\hfill cholera. \\
             $z$& defective sanitary condition.
\end{tabular}

The events whose probabilities are given are then denoted by
$x$, $y$, and $(1-x)(1-y)(1-z)$, the event whose probability is
sought is $z$. Assume then,
\[
(1-x)(1-y)(1-z) = w;
\]
then our data are,
\[
\text{Prob. } x = p,\quad \text{Prob. } y = q,\quad \text{Prob. } w = r,
\]
and we are to find Prob. $z$. Now
\begin{multline}
z = \frac{(1-x)(1-y)-w}{(1-x)(1-y)} \\
= \frac{1}{0}xyw + \frac{0}{0}xy(1-w) + \frac{1}{0}x(1-y) + \frac{0}{0}x(1-y)(1-w) \\
+ \frac{1}{0}(1-x)yw + \frac{0}{0}(1-x)y(1-w) + 0(1-x)(1-y)w \\
\hfill + (1-x)(1-y)(1-w). \tag{1}
\end{multline}
The value of $V$ deduced from the above is
\[
\begin{array}{c}
V = xy(1-w) + x(1-y)(1-w) + (1-x)y(1-w) \\
+ (1-x)(1-y)w + (1-x)(1-y)(1-w) = 1-w+w(1-x)(1-y);
\end{array}
\]
and similarly reducing $V_x, V_y, V_w$, we get
\[
V_x = x(1-w),\quad V_y = y(1-w), \quad V_w = w(1-x)(1-y);
\]
furnishing the algebraic equations
\[
\frac{x(1-w)}{p} = \frac{y(1-w)}{q} = \frac{w(1-x)(1-y)}{r} = 1 - w + w(1-x)(1-y). \tag{2}
\]
As respects those terms of the development characterized by
the coefficients $\frac{0}{0}$, I shall, instead of collecting them into a single
term, present them, for the sake of variety (\textsc{xvii.}~18), in the
form
\[
\frac{0}{0}x(1-w) + \frac{0}{0}(1-x)y(1-w); \tag{3}
\]
the value of Prob. $z$ will then be

%-----------------------File: 086.png----------------------------

\[
  \textrm{Prob. }z = \frac{(1-x)(1-y)(1-w) + cx(1-w) + c'(1-x)y(1w)}
                        {1 - w + w(1 - x)(1 - y)}. \tag{4}
\]

From (2) and (4) we deduce
\[
  \text{Prob. }z = \frac{(1-p-r)(1-q-r)}{1-r}
+ cp + c'\frac{q(1-p-r)}{1-r},
\]
as the expression of the probability required. If in this result
we make $c = 0$, and $c' = 0$, we find for an inferior limit of its value
$\frac{(1-p-r)(1-q-r)}{1-r}$; and if we make $c=1$, $c'=1$, we obtain
for its superior limit $1-r$.

6. It appears from inspection of this solution, that the premises
chosen were exceedingly defective. The constants $c$ and
$c'$ indicate this, and the corresponding terms (3) of the final
logical equation show how the deficiency is to be supplied.
Thus, since
\[
\begin{array}{c}
x(1-w)=x\{1-(1-x)(1-y)(1-z)\}=x, \\
(1-x)y(1-w)=(1-x)y\{1-(1-x)(1-y)(1-z)\}=(1-x)y,
\end{array}
\]
we learn that $c$ is the probability that if any house was visited by
fever its sanitary condition is defective, and that $c'$ is the probability
that if any house was visited by cholera without fever, its
sanitary condition was defective.

If the terms of the logical development affected by the coefficient
$\frac{0}{0}$ had been collected together as in the direct statement of
the general rule, the final solution would have assumed the following form:
\[
  \text{Prob. }z = \frac{(1-p-r)(1-q-r)}{1-r}
+ c\left(p+q-\frac{pq}{1-r}\right)
\]
$c$ here representing the probability that if a house was visited by
either or both of the diseases mentioned, its sanitary condition
was defective. This result is perfectly consistent with the former
one, and indeed the \emph{necessary} equivalence of the different forms
of solution presented in such cases may be formally established.

The above solution may be verified in particular cases. Thus,
taking the second form, if $c = 1$ we find Prob. $z = 1-r$, a correct
result. For if the presence of either fever or cholera \emph{certainly}
%-----------------------File: 087.png----------------------------
indicated a defective sanitary condition, the probability that any
house would be in a defective sanitary state would be simply
equal to the probability that it was \emph{not} found in that category
denoted by $z$, the probability of which would, by the data, be $1-r$,
Perhaps the general verification of the above solution would be
difficult.

The constants $p$, $q$, and $r$ in the above solution are subject to
the conditions
\[
  p + r \stackrel{=}{<} 1, \quad  q + r \stackrel{=}{<} 1.
\]

7. Ex. 5.---Given the probabilities of the premises of a hypothetical syllogism to find the probability of the conclusion.

Let the syllogism in its naked form be as follows:

\begin{tabular}{ll}
  Major premiss: &If the proposition $Y$ is true $X$ is true.\\
  Minor premiss: &If the proposition $Z$ is true $Y$ is true.\\
  Conclusion:    &If the proposition $Z$ is true $X$ is true.
\end{tabular}

Suppose the probability of the major premiss to be $p$, that of the
minor premiss $q$.

The data then are as follows, representing the proposition $X$
by $x$, \&c., and assuming $c$ and $c'$ as arbitrary constants:
\begin{align*}
  \text{Prob. }y &= c, \qquad &\text{ Prob. }xy &= cp;\\
  \text{Prob. }z &= c',\qquad &\text{ Prob. }yz &= c'q;
\end{align*}
from which we are to determine,
\[
  \frac{\text{Prob. }xz}{\text{Prob. }z}
  \text{ or }
  \frac{\text{Prob. }xz}{c'}.
\]

Let us assume,
\[
  xy = u,\quad  yz = v,\quad  xz = w,
\]
then, proceeding according to the usual method to determine $w$
as a developed function of $y$, $z$, $w$, and $v$, the symbols corresponding to propositions whose probabilities are given, we find
\[
\begin{split}
  w &= uzvy + 0u(1-z)(1-v)y + 0(1-u)zvy   \\
    &+ \frac{0}{0}(1-u)z(1-v)(1-y) + 0(1-u)(1-z)(1-v)y   \\
    &+ 0(1-u)(1-z)(1-v)(1-y)
     + \text{ terms whose coefficients are }\frac{1}{0};
\end{split}
\]%**[I don't know how/if that text term will be spilt.]
%-----------------------File: 088.png----------------------------
and passing from Logic to Algebra,
\begin{multline*}
  \frac{uzvy + u(1-z)(1-v)y}{cp}
= \frac{uzvy + (1-u)zvy + (1-u)z(1-v)(1-y)}{c'}   \\
= \frac{uzvy + (1-u)zvy}{c'q}   \\
= \frac{uzvy + u(1-z)(1-v)y + (1-u)zvy + (1-u)(1-z)(1-v)y}{c} = V. \\
  \text{Prob. }w = \frac{uzvy + a(1-u)z(1-v)(1-y)}{V},
\end{multline*}
wherein
\[
\begin{split}
  V = uzvy + u(1-z)(1-v)y + (1-u)zvy + (1-u)z(1-v)(1-y)   \\
    +(1-u)(1-z)(1-v)y + (1-u)(1-z)(1-v)(1-y),
\end{split}
\]
the solution of this system of equations gives
\[
  \text{Prob. }w = c'pq + ac' (1-q),
\]
whence
\[
  \frac{\text{Prob. } xy}{c'} = pq + a(1-q),
\]
the value required. In this expression the arbitrary constant $a$
is the probability that if the proposition $Z$ is true and $Y$ false, $X$
is true. In other words, it is the probability, that if the minor
premiss is false, the conclusion is true.

This investigation might have been greatly simplified by assuming
the proposition $Z$ to be true, and then seeking the probability
of $X$. The data would have been simply
\[
  \text{Prob. }y = q,\quad  \text{Prob. }xy = pq;
\]
whence we should have found Prob. $x = pq + a (1-q)$. It is
evident that under the circumstances this mode of procedure
would have been allowable, but I have preferred to deduce the
solution by the direct and unconditioned application of the
method. The result is one which ordinary reasoning verifies,
and which it does not indeed require a calculus to obtain. General
methods are apt to appear most cumbrous when applied to
cases in which their aid is the least required.

Let it be observed, that the above method is equally applicable
to the categorical syllogism, and not to the syllogism only,
%-----------------------File: 089.png----------------------------
but to every form of deductive ratiocination. Given the probabilities
separately attaching to the premises of \emph{any} train of argument;
it is always possible by the above method to determine
the consequent probability of the truth of a conclusion legitimately
drawn from such premises. It is not needful to remind the
reader, that the truth and the correctness of a conclusion are different
things.

8. One remarkable circumstance which presents itself in such
applications deserves to be specially noticed. It is, that propositions
which, when true, are equivalent, are not necessarily
equivalent when regarded only as probable. This principle will
be illustrated in the following example.

Ex. 6.---Given the probability $p$ of the disjunctive proposition
``Either the proposition $Y$ is true, or both the propositions $X$ and
$Y$ are false,'' required the probability of the conditional proposition,
``If the proposition $X$ is true, $Y$ is true.''

Let $x$ and $y$ be appropriated to the propositions $X$ and $Y$
respectively. Then we have
\[
\text{Prob. }y + (1-x) (1-y) = p,
\]
from which it is required to find the value of
$\frac{\text{Prob. }xy}{\text{Prob. }x}$.
\[
  \text{Assume}\hfill  y + (1-x) (1-y) = t.  \hfill\tag{1}
\]
Eliminating $y$ we get
\[
(1-x) (1-t) = 0.
\]
whence
\[
x=\frac{0}{0} t + 1 - t;
\]
and proceeding in the usual way,
\[
\text{Prob. }x = 1-p + cp.     \tag{2}
\]
Where $c$ is the probability that if either $Y$ is true, or $X$ and $Y$
false, $X$ is true.

Next to find $\text{Prob. }xy$. Assume
\[
xy = w.          \tag{3}
\]
Eliminating $y$ from (1) and (3) we get
\[
z (1-t) = 0;
\]
%-----------------------File: 090.png----------------------------
whence, proceeding as above,
\[
\text{Prob. }z = cp,
\]
$c$ having the same interpretation as before. Hence
\[
\frac{\text{Prob. }xy}{\text{Prob. }x} = \frac{cp}{1-p+cp},
\]
for the probability of the truth of the conditional proposition
given.

Now in the science of pure Logic, which, as such, is conversant
only with truth and with falsehood, the above disjunctive
and conditional propositions are equivalent. They are true and
they are false together. It is seen, however, from the above investigation,
that when the disjunctive proposition has a probability
$p$, the conditional proposition has a different and partly indefinite
probability $\frac{cp}{1-p+cp}$. Nevertheless these expressions
are such, that when \emph{either of them becomes 1 or 0, the other assumes
the same value}. The results are, therefore, perfectly consistent,
and the logical transformation serves to verify the formula
deduced from the theory of probabilities.

The reader will easily prove by a similar analysis, that if the
probability of the conditional proposition were given as $p$, that
of the disjunctive proposition would be $1-c+cp$, where $c$ is the
arbitrary probability of the truth of the proposition $X$.

9. Ex.~7.---Required to determine the probability of an event
$x$, having given either the first, or the first and second, or the
first, second, and third of the following data, viz.:

1st. The probability that the event $x$ occurs, or that it alone
of the three events $x$, $y$, $z$, fails, is $p$.

2nd. The probability that the event $y$ occurs, or that it alone
of the three events $x$, $y$, $z$, fails, is $q$.

3rd. The probability that the event $z$ occurs, or that it alone
of the three events $x$, $y$, $z$, fails, is $r$.


\begin{center}\textsc{solution of the first case.}\end{center}

Here we suppose that only the first of the above data is
given.
%-----------------------File: 091.png----------------------------
We have then,
\[
  \textrm{Prob. }\{x + \bigl( 1 - x \bigr) yz\} = p,
\]
to find Prob. $x$,

\[
  \qquad\textrm{Let}\hfill  x + \bigl( 1 - x \bigr) yz = s, \hfill
\]
then eliminating $yz$ as a single symbol, we get,
\begin{equation*}
x\bigl( l - s \bigr) = 0.
\end{equation*}
Hence
\begin{equation*}
x = \frac{0}{1-s} = \frac{0}{0}s + 0\bigl( 1-s \bigr),
\end{equation*}
whence, proceeding according to the rule, we have
\[
  \textrm{Prob. }x = cp,    \tag{1}
\]
where $c$ is the probability that if $x$ occurs, or alone fails, the
former of the two alternatives is the one that will happen. The
limits of the solution are evidently $0$ and $p$.

This solution appears to give us no information beyond what
unassisted good sense would have conveyed. It is, however, all
that the single datum here assumed really warrants us in inferring.
We shall in the next solution see how an addition to our
data restricts within narrower limits the final solution.

\begin{center}\textsc{solution of the second case.}\end{center}

Here we assume as our data the equations
\begin{align*}
  \textrm{Prob. }\{ x + \bigl( 1 - x \bigr)yz \} &= p,  \\
  \textrm{Prob. }\{ y + \bigl( 1 - y \bigr)xz \} &= q.
\end{align*}
Let us write
\begin{align*}
  x + \bigl( 1 - x \bigr)yz &= s,  \\
  y + \bigl( 1 - y \bigr)xz &= q,
\end{align*}
from the first of which we have, by (VIII.~7),
\begin{gather*}
  \{x+\bigl( 1-x \bigr)yz\} \bigl( 1-s \bigr)
+ s \{1 - x - \bigl( 1 - x \bigr)yz\} = 0,
\\
  \textrm{or}\hfill \bigl( x + \bar{x}yz \bigr)\bar{s}
+ s\bar{x}\bigl( 1 - yz \bigr) = 0;\hfill
\end{gather*}
provided that for simplicity we write $\bar{x}$ for $1 - x$, $\bar{y}$ for $1 - y$, and
so on. Now, writing for $1 - yz$ its value in constituents, we
have
\begin{equation*}
\bigl( x + \bar{x}yz \bigr)\bar{s}
+ s\bar{x}\bigl( y\bar{z} + \bar{y}z + \bar{y}\bar{z} \bigr) = 0,
\end{equation*}
an equation consisting solely of positive terms.
%-----------------------File: 092.png----------------------------
In like manner we have from the second equation,
\[(y+  \bar{y}xz) \bar{t} + t\bar{y} (x\bar{z} + \bar{x}z + \bar{x}\bar{z}) =0;\]
and from the sum of these two equations we are to eliminate $y$
and $z$.

If in that sum we make $y = 1$, $z = 1$, we get the result $\bar{s}+ \bar{t}$.

If in the same sum we make $y = 1$, $z = 0$, we get the result
\[ x\bar{s} + s\bar{x} + \bar{t}.  \]

If in the same sum we make $y = 0$, $z = 1$, we get
\[ x\bar{s} + s \bar{x} + x\bar{t} +t \bar{x}.\]

And if, lastly, in the same sum we make $y = 0$, $z = 0$, we find
\[  x\bar{s} + s \bar{x} + tx + t\bar{x}, \text { or } x\bar{s} + s\bar{x} + t. \]

These four expressions are to be multiplied together. Now
the first and third may be multiplied in the following manner:
\begin{gather*}
(\bar{s} +\bar{t}) (x\bar{s} + s\bar{x} + x\bar{t} + t\bar{x})\\
= x\bar{s} + x\bar{t} + (\bar{s} + \bar{t}) (s\bar{x} + t\bar{x})\text { by (IX.~Prop.~{\sc ii}.)}\\
= x\bar{s} + x\bar{t} + \bar{s}\bar{x}t + s\bar{x}\bar{t}.       \tag{2}
\end{gather*}
Again, the second and fourth give by (IX.~Prop.~{\sc i}.)
\begin{gather*}
(x\bar{s} + s\bar{x} + \bar{t}) (x\bar{s} + s\bar{x} + t)\\
= x\bar{s} + s\bar{x}.           \tag{3}
\end{gather*}
Lastly, (2) and (3) multiplied together give
\begin{gather*}
(x\bar{s} + s\bar{x}) (x\bar{s} + s\bar{x}\bar{t} + x\bar{t} + t\bar{x}\bar{s})\\
= x\bar{s} + s\bar{x} (s\bar{x}\bar{t}+ x\bar{t} + t\bar{x}\bar{s})\\
= x \bar{s} + s \bar{x}\bar{t}.
\end{gather*}
Whence the final equation is
\[(1-s)x +s(1-t)(1-x)=0,    \]
which, solved with reference to $x$, gives
\begin{gather*}
x =\frac{s(1-t)}{s(1-t)-(1-s)} \\
= \frac{0}{0}st +s(1-t)+0(1-s)t +0(1-s)(1-t)  ,
\end{gather*}
%-----------------------File: 093.png----------------------------
and, proceeding with this according to the rule, we have, finally,
\begin{equation*}
  \text{Prob. }x = p (1-q) + cpq.      \tag{4}
\end{equation*}
where $c$ is the probability that if the event $st$ happen, $x$ will
happen. Now if we form the developed expression of $st$ by multiplying
the expressions for $s$ and $t$ together, we find---

$c =$ Prob.\ that if $x$; and $y$ happen together, or $x$ and $z$ happen
together, and $y$ fail, or $y$ and $z$ happen together, and $x$ fail, the
event $x$ will happen.

The limits of Prob.\ $x$ are evidently $p (1 - q)$ and $p$.

This solution is more definite than the former one, inasmuch
as it contains a term unaffected by an arbitrary constant.

\begin{center}\textsc{solution of the third case.}\end{center}

Here the data are---
\begin{tabular}{c}
  Prob.\ $\{x + (1-x)yz\} = p$, \\
  Prob.\ $\{y + (1-y)xz\} = q$, \\
  Prob.\ $\{z + (1-z)xy\} = r$.
\end{tabular}

Let us, as before, write $\bar{x}$ for $1-x$, \&c., and assume
\begin{align*}
  x + \bar{x}yz = s,  \\
  y + \bar{y}xz = t,  \\
  z + \bar{z}xy = u.
\end{align*}

On reduction by (VIII.~8) we obtain the equation
\setcounter{equation}{4}
\begin{equation}
\begin{split}
  &(x + \bar{x}yz)\bar{s}
+ s\bar{x}(y\bar{z} + \bar{y}z + \bar{y}\bar{z})
\\
  &+ (y + \bar{y}xz)\bar{t}
+ t\bar{y}(z\bar{x} + x\bar{z} + \bar{x}\bar{z})
\\
  &+ (z + \bar{z}xy)\bar{u}
+ u\bar{z}(x\bar{y} + \bar{x}y + \bar{x}\bar{y}) = 0.   %\tag{5}
\end{split}
\end{equation}

Now instead of directly eliminating $y$ and $z$ from the above
equation, let us, in accordance with (IX.~Prop,~\textsc{iii}.), assume the
result of that elimination to be
\begin{equation*}
Ex + E'(1-x) = 0,
\end{equation*}
then $E$ will be found by making in the given equation $x = 1$,
and eliminating $y$ and $z$ from the resulting equation, and $E'$ will
be found by making in the given equation $x = 0$, and eliminating
$y$ and $z$ from the result. First, then, making $x = 1$, we have
%-----------------------File: 094.png----------------------------
\[
\bar{s} + (y + \bar{y}z)\bar{t} + t\bar{y}\bar{z} + (z + y\bar{z})\bar{u} + u\bar{y}\bar{z} = 0,
\]
and making in the first member of this equation successively
$y = 1, z = 1, y = 1, z = 0$, \&c., and multiplying together the
results, we have the expression
\[
(\bar{s} + \bar{t} + \bar{u})(\bar{s} + \bar{t} + \bar{u})(\bar{s} + \bar{t} + \bar{u})(\bar{s} + t + u),
\]
which is equivalent to
\[
(\bar{s} + \bar{t} + \bar{u})(\bar{s} + t + u).
\]
This is the expression for $E$. We shall retain it in its present
form. It has already been shown by example (VIII.~3), that
the actual reduction of such expressions by multiplication, though
convenient, is not necessary.

Again in $(5)$, making $x = 0$, we have
\[
yz\bar{s} + s(y\bar{z} + \bar{y}z + \bar{y}\bar{z}) + y\bar{t} + t\bar{y} + z\bar{u} + u\bar{z} = 0;
\]
from which, by the same process of elimination, we find for $E'$ the
expression
\[
(\bar{s} + \bar{t} + \bar{u})(s + \bar{t} + u)(s + t + \bar{u})(s + t + u).
\]
The final result of the elimination of $y$ and $z$ from $(5)$ is therefore
\[
(\bar{s} + \bar{t} + \bar{u})(\bar{s} + t + u)x + (\bar{s} + \bar{t} + \bar{u})(s + \bar{t} + u)(s + t + \bar{u})(s + t + u)(1 - x) = 0.
\]
Whence we have
\[
x = \frac{(\bar{s} + \bar{t} + \bar{u})(s + \bar{t} + u)(s + t + \bar{u})(s + t + u)}
         {(\bar{s} + \bar{t} + \bar{u})(s + \bar{t} + u)(s + t + \bar{u})(s + t + u) - (\bar{s} + \bar{t} + \bar{u})(\bar{s} + t + u)};
\]
or, developing the second member,
\begin{equation}
\begin{split}
x = \frac{0}{0}stu + \frac{1}{0}s\bar{t}u + \frac{1}{0}st\bar{u}
+ s\bar{t}\bar{u}
\\
+ \frac{1}{0}\bar{s}tu + 0\bar{s}\bar{t}u + 0\bar{s}t\bar{u}
+ 0\bar{s}\bar{t}\bar{u}. %\tag{6}
\end{split}
\end{equation}
Hence, passing from Logic to Algebra,
\begin{equation}
\begin{split}
  \frac{stu + s\bar{t}\bar{u}}{p}
= \frac{stu + \bar{s}t\bar{u}}{q}
= \frac{stu + \bar{s}\bar{t}u}{r}
\\
= stu + s\bar{t}\bar{u} + \bar{s}\bar{t}u + \bar{s}t\bar{u}
+ \bar{s}\bar{t}\bar{u}. %\tag{7}
\end{split}
\end{equation}

%-----------------------File: 095.png----------------------------
\[
  \text{Prob. }x = \frac{s\bar{t}\bar{u} + cstu}{stu + s\bar{t}\bar{u} + \bar{s}\bar{t}u + \bar{s}t\bar{u} + \bar{s}\bar{t}\bar{u}}, \tag{8}
\]

To simplify this system of equations, change ${\displaystyle \frac{s}{\bar{s}}}$ into ${\displaystyle s, \frac{t}{\bar{t}}}$ into $t$, \&c., and after the change let $\lambda$ stand for $stu + s + t + 1$ . We then have
\[
  \text{Prob. }x = \frac{s + cstu}{\lambda}, \tag{9}
\]
with the relations
\[
  \frac{stu + s}{p} = \frac{stu + t}{q} = \frac{stu + u}{r} = stu + s + t + u + 1 = \lambda. \tag{10}
\]
From these equations we get
\begin{align*}
                      stu + s &= \lambda p,           \tag{11} \\
                      stu + s &= \lambda - t - u - 1, \\
         \therefore \lambda p &= \lambda - u - t - 1. \\
                        u + t &= \lambda(1 - p) - 1.  \\
\text{Similarly,}\hfill u + s &= \lambda(1 - q) - 1,  \hfill \\
\text{and}\hfill        s + t &= \lambda(1 - r) - 1.  \hfill
\end{align*}

From which equations we find
\begin{gather*}
  s = \frac{\lambda(1 + p - q - r) - 1}{2},  \quad
  t = \frac{\lambda(1 + q - r - p) - 1}{2},  \\
  u = \frac{\lambda(1 + r - p - q) - 1}{2}.  \tag{12}
\end{gather*}
Now, by (10),
\[
  stu = \lambda p - s.
\]
Substitute in this equation the values of $s$, $t$, and $u$ above determined, and we have
\begin{multline*}
  \{(1 + p - q - r)\lambda - 1\}
  \{(1 + q - p - r)\lambda - 1\}
  \{(1 + r - p - q)\lambda - 1\}\\
=4\{(p + q + r - 1)\lambda + 1\}, \tag{13}
\end{multline*}
an equation which determines $\lambda$. The values of $s$, $t$, and $u$, are
then given by (12), and their substitution in (9) completes the
solution of the problem.

%-----------------------File: 096.png----------------------------
10. Now a difficulty, the bringing of which prominently before
the reader has been one object of this investigation, here
arises. How shall it be determined, which root of the above
equation ought to taken for the value of $\lambda$. To this difficulty
some reference was made in the opening of the present chapter,
and it was intimated that its fuller consideration was reserved for
the next one; from which the following results are taken.

In order that the data of the problem may be derived from
a possible experience, the quantities $p$, $q$, and $r$ must be subject
to the following conditions:
\[
\begin{aligned}
  1 + p - q - r \stackrel{=}{>} 0,   \\
  1 + q - p - r \stackrel{=}{>} 0,   \\
  1 + r - p - q \stackrel{=}{>} 0.
\end{aligned}                        \tag{14}
\]
Moreover, the value of $\lambda$ to be employed in the general solution
must satisfy the following conditions:
\[
  \lambda \stackrel{=}{>} \frac{1}{1 + p - q - r}, \quad
  \lambda \stackrel{=}{>} \frac{1}{1 + q - p - r}, \quad
  \lambda \stackrel{=}{>} \frac{1}{1 + r - p - q}.      \tag{15}
\]

Now these two sets of conditions suffice for the limitation of
the general solution. It may be shown, that the central equation
(13) furnishes but one value of $\lambda$, which does satisfy these conditions,
and that value of $\lambda$ is the one required.

Let $1+p-q-r$ be the least of the three coefficients of $\lambda$
given above, then $\dfrac{1}{1+p-q-r}$ will be the greatest of those values,
above which we are to show that there exists but one value
of $\lambda$. Let us write (13) in the form
\begin{gather*}
  \{(1+p-q-r)\lambda - 1\}
  \{(1+q-p-r)\lambda - 1\}
  \{(1+r-p-q)\lambda - 1\}   \\
-4\{(p+q+r-1)\lambda + 1\} = 0;   \tag{16}
\end{gather*}
and represent the first member by $V$.

Assume $\lambda = \dfrac{1}{1+p-q-r}$, then $V$ becomes
\[
  -4\left(\frac{p+q+r-1}{1+p-q-r} + 1\right)
= -4\left(\frac{2p}{1+p-q-r} \right),
\]
which is \emph{negative.}
%-----------------------File: 097.png----------------------------
Let $\lambda = \infty$, then $V$ is \emph{positive} and infinite.

Again,
\begin{multline*}
\frac{d^2V}{d\lambda^2}
= (1 + p - q - r)(1 + q - p - r)\{(1 + r - p - q)\lambda - 1\}   \\
+ \text{ similar positive terms},
\end{multline*}
which expression is positive between the limits
$\lambda = \frac{1}{1 + p - q - r}$ and $\lambda = \infty$.

If then we construct a curve whose abscissa shall be measured
by $\lambda$, and whose ordinates by $V$, that curve will, between the
limits specified, pass from below to above the abscissa $\lambda$, its convexity
always being downwards. Hence it will but once intersect
the abscissa $\lambda$ within those limits; and the equation $(16)$ will, therefore,
have but one root thereto corresponding.

The solution is, therefore, expressed by $(9)$, $\lambda$ being that
root of $(13)$ which satisfies the conditions $(15)$, and $s, t$, and $u$
being given by $(12)$. The interpretation of $c$ may be deduced
in the usual way.

It appears from the above, that the problem is, in all cases,
more or less indeterminate.

%-----------------------File: 098.png----------------------------
%\begin{center}
%CHAPTER XIX.
%\end{center}
\chapter[OF STATISTICAL CONDITIONS.]{\large OF STATISTICAL CONDITIONS.}
1. By the term statistical conditions, I mean those conditions
which must connect the numerical data of a problem in
order that those data may be consistent with each other, and
therefore such as statistical observations might actually have
furnished. The determination of such conditions constitutes an
important problem, the solution of which, to an extent sufficient
at least for the requirements of this work, I purpose to undertake
in the present chapter, regarding it partly as an independent object
of speculation, but partly also as a necessary supplement to
the theory of probabilities already in some degree exemplified.
The nature of the connexion between the two subjects may be
stated as follows:

2. There are innumerable instances, and one of the kind
presented itself in the last chapter, Ex.\ 7, in which the solution
of a question in the theory of probabilities is finally dependent
upon the solution of an algebraic equation of an elevated degree.
In such cases the selection of the proper root must be determined
by certain conditions, partly relating to the numerical values assigned
in the data, partly to the due limitation of the element
required. The discovery of such conditions may sometimes be
effected by unaided reasoning. For instance, if there is a probability
$p$ of the occurrence of an event $A$, and a probability $q$ of
the concurrence of the said event $A$, and another event $B$, it is
evident that we must have
\[
p \stackrel{=}{>} q.
\]
But for the \emph{general} determination of such relations, a distinct
method is required, and this we proceed to establish.

As derived from actual experience, the probability of any
event is the result of a process of approximation. It is the limit
of the ratio of the number of cases in which the event is observed
to occur, to the whole number of equally possible cases which
%-----------------------File: 099.png----------------------------
observation records,--a limit to which we approach the more
nearly as the number of observations is increased. Now let the
symbol $n$, prefixed to the expression of any class, represent the
number of individuals contained in that class. Thus, $x$ representing
men, and $y$ white beings, let us assume
\[
\begin{array}{lll}
nx &=& \text{number of men.}\\
nxy &=& \text{number of white men.}\\
nx(1-y) &=& \text{number of men who are not white; and so on.}
\end{array}
\]
In accordance with this notation $n(1)$ will represent the number
of individuals contained in the universe of discourse, and $\frac{n(x)}{n(1)}$
will represent the probability that any individual being, selected
out of that universe of being denoted by $n(1)$, is a man. If observation
has not made us acquainted with the \emph{total} values of
$n(x)$ and $n(1)$, then the probability in question is the limit to
which $\frac{n(x)}{n(1)}$ approaches as the number of individual observations
is increased.

In like manner if, as will generally be supposed in this chapter,
$x$ represent an event of a particular kind observed, $n(x)$ will
represent the number of occurrences of that event, $n(1)$ the
number of observed events (equally probable) of all kinds, and
$\frac{n(x)}{n(1)}$, or its limit, the probability of the occurrence of the
event $x$.

Hence it is clear that any conclusions which may be deduced
respecting the ratios of the quantities $n(x)$, $n(y)$, $n(1)$, \&c. may
be converted into conclusions respecting the probabilities of the
events represented by $x$, $y$, \&c. Thus, if we should find such a
relation as the following, viz.,
\[
n(x) + n(y) < n(1),
\]
expressing that the number of times in which the event $x$ occurs
and the number of times in which the event $y$ occurs, are together
less than the number of possible occurrences $n(1)$, we might
thence deduce the relation,
\begin{gather*}
  \frac{n(x)}{n(1)} + \frac{n(y)}{n(1)} < 1,
\\
  \text{or}\hfill \text{Prob. } x + \text{Prob. } y < 1. \hfill
\end{gather*}
%-----------------------File: 100.png----------------------------
%** emph text should be merged with the preceeding page.
\emph{
And generally any such statistical relations as the above will be
converted into relations connecting the probabilities of the events
concerned, by changing $n(1)$ into $1$, and any other symbol $n(x)$
into Prob. $x$.}

3. First, then, we shall investigate a method of determining
the numerical relations of classes or events, and more particularly
the major and minor limits of numerical value. Secondly, we
shall apply the method to the limitation of the solutions of questions
in the theory of probabilities.

It is evident that the symbol $n$ is distributive in its operation.
Thus we have
\begin{gather*}
n\{xy + (1-x)(1-y)\} = nxy + n(1-x)(1-y)\\
nx(1-y) = nx - nxy,
\end{gather*}
and so on. The number of things contained in any class resolvable
into distinct groups or portions is equal to the sum of
the numbers of things found in those separate portions. It is
evident, further, that any expression formed of the logical symbols
$x$, $y$, \&c. may be developed or expanded in any way consistent
with the laws of the symbols, and the symbol $n$ applied to
each term of the result, provided that any constant multiplier
which may appear, be placed outside the symbol $n$; without affecting
the value of the result. The expression $n(1)$, should it appear,
will of course represent the number of individuals contained
in the universe. Thus,
\begin{gather*}
n(1-x)(1-y) = n(1-x-y+xy)\\
= n(1) - n(x) - n(y) + n(xy).\\
\text{Again,} \hfill n\{xy + (1-x)(1-y)\} = n(l-x-y+2xy) \hfill \\
= n(1) - nx - ny + 2nxy).
\end{gather*}
In the last member the term $2nxy$ indicates \emph{twice} the number of
individuals contained in the class $xy$.

4. We proceed now to investigate the numerical limits of
classes whose logical expression is given. In this inquiry the
following principles are of fundamental importance:

1st. If all the members of a given class possess a certain property
$x$, the total number of individuals contained in the class $x$
%-----------------------File: 101.png----------------------------
will be a superior limit of the number of individuals contained in
the given class.

2nd. A minor limit of the number of individuals in any class $y$
will be found by subtracting a major numerical limit of the contrary class, $1-y$, from the number of individuals contained in the
universe.

To exemplify these principles, let us apply them to the following problem:

\textsc{Problem}.---Given, $n(1)$, $n(x)$, and $n(y)$, required the superior and inferior limits of $nxy$.

Here our data are the number of individuals contained in the universe of discourse, the number contained in the class $x$, and the number in the class $y$, and it is required to determine the limits of the number contained in the class composed of the individuals that are found at once in the class $x$ and in the class $y$.

By Principle~\textsc{i.} this number cannot exceed the number contained in the class $x$, nor can it exceed the number contained in the class $y$. Its major limit will then be the least of the two values $n(x)$ and $(y)$.

By Principle~\textsc{ii.} a minor limit of the class $xy$ will be given by
the expression
\[
  n(l) - \text{ major limit of }\{x(l-y) + y(l-x) + (1-x)(1-y)\}, \tag{1}
\]
since $x(1-y) + y(1-x) - (1-x)(1-y)$ is the complement of
the class $xy$, i.e.\ what it wants to make up the universe.

Now $x(1-y) + (1-x)(1-y) = 1-y$. We have therefore for (1),
\[
\begin{split}
  n(1) - \text{ major limit of }\{1-y + y 1-x)\}   \\
= n(1) - n(l-y) - \text{ major limit of }y(1-x).  %\tag{2}
\end{split}
\]
The major limit of $y(l-x)$ is the least of the two values $n(y)$ and $n(1-x)$. Let $n(y)$ be the least, then (2) becomes
\[
\begin{split}
  n(1) - n(1-y) - n(y)  \\
= n(1) - n(1) + n(y) -n(y) = 0.
\end{split}
\]

Secondly, let $n(1-x)$ be less than $n(y)$, then
\[
  \text{major limit of }ny(1-x) = n(1-x);
\]
therefore (2) becomes
%-----------------------File: 102.png----------------------------
\begin{gather*}
  n(1) - n(1-y) - n(1-x)  \\
= n(1) - n(1) + n(y) - n(1) + n(x)   \\
= nx + ny - n(1).
\end{gather*}
The minor limit of $nxy$ is therefore either $0$ or
$n(x) + n(y) - n(1)$,
according as $n(y)$ is less or greater than $n(1-x)$, or, which is an
equivalent condition, according as $n(x)$ is greater or less than
$n(1-y)$.

Now as $0$ is necessarily a minor limit of the numerical value
of \emph{any} class, it is sufficient to take account of the second of the
above expressions for the minor limit of $n(xy)$. We have, therefore,
\begin{align*}
  \text{Major limit of }n(xy) &=
      \text{ least of values $n(x)$ and $n(y)$.}   \\
  \text{Minor limit of }n(xy) &= n(x) + n(y) - n(1).
\footnotemark
\end{align*}
\footnotetext{The above expression for the minor limit of $nxy$ is applied by Professor
De~Morgan, by whom it appears to have been first given, to the syllogistic form:
\begin{verse}
  Most men in a certain company have coats.\\
  Most men in the same company have waistcoats.\\
  Therefore some in the company have coats and waistcoats.
\end{verse}
}%endfootnotetext

\begin{center}\textsc{Proposition I.}\end{center}

5. \emph{To express the major and minor limits of a class represented
by any constituent of the symbols $x$, $y$, $z$, \&c., having given the values
of $n(x)$, $n(y)$, $n(z)$, \&c., and $n(1)$.}

Consider first the constituent $xyz$.

It is evident that the major numerical limit will be the least
of the values $n(x)$, $n(y)$, $n(z)$.

The minor numerical limit may be deduced as in the previous
problem, but it may also be deduced from the solution of that
problem. Thus:
\[
  \text{Minor limit of }n(xyz) = n(xy) + n(z) - n(1).   \tag{1}
\]
Now this means that $n(xyz)$ is at least as great as the expression
$n(xy) + n(z) - n(1)$. But $n(xy)$ is at least as great as
$n(x) + n(y) - n(1)$. Therefore $n(xyz)$ is at least as great as
\begin{align*}
                &n(x) + n(y) - n(1) + n(z) -n(1),  \\
\text{or }\qquad &n(x) + n(y) + n(z) - 2n(1).
\end{align*}
%-----------------------File: 103.png----------------------------
Hence we have
\[
  \text{Minor limit of }n(xyz) = n(x) + n(y) + n(z) - 2n(1).
\]

By extending this mode of reasoning we shall arrive at the
following conclusions:

1st. The major numerical limit of the class represented by
any constituent will be found by prefixing $n$ separately to each
factor of the constituent, and taking the least of the resulting
values.

2nd. The minor limit will be found by adding all the values
above mentioned together, and subtracting from the result as
many, less one, times the value of $n(1)$.

Thus we should have
\begin{multline*}
 \begin{aligned}
  \text{Major limit of }nxy(1-z) &=
      \text{ least of the values $nx$, $ny$, and $n(1-z)$.}   \\
  \text{Minor limit of }nxy(1-z) &= n(x) + n(y) + n(1-z) - 2n(1)
 \end{aligned}
\\= nx + n(y) -n(z) - n(1).
\end{multline*}

In the use of general symbols it is perhaps better to regard all
the values $n(x)$, $n(y)$, $n(1-z)$, as major limits of $n\{xy(1-z)\}$,
since, in fact, it cannot exceed any of them. I shall in the following investigations adopt this mode of expression.

\begin{center}\textsc{Proposition II.}\end{center}

6. \emph{To determine the major numerical limit of a class expressed
by a series of constituents of the symbols $x$, $y$, $z$, \&$c$., the values of
$n(x)$, $n(y)$, $n(z)$, \&c., and $n(1)$, being given.}

Evidently one mode of determining such a limit would be to
form the least possible sum of the major limits of the several constituents.
Thus a major limit of the expression
\[
  n\{xy + (1-x)(1-y)\}
\]
would be found by adding the least of the two values $nx$, $ny$, furnished
by the first constituent, to the least of the two values
$n(1-x)$, $n(1-y)$, furnished by the second constituent. If we
do not know which is in each case the least value, we must form
the four possible sums, and reject any of these which are equal to
or exceed $n(1)$. Thus in the above example we should have
%-----------------------File: 104.png----------------------------
\begin{align*}
&nx\phantom{()} + n(l-x) = n(l).\\
&n(x)           + n(1-y) = n(1) + n(x)-n(y).\\
&n(y)           + n(l-y) = n(l) + n(y)-n(x).\\
&n(y)           + n(1-y) = n(1).
\end{align*}
Rejecting the first and last of the above values, we have
\[n(1) +n(x)-n(y), \text{ and } n(1) + n(y)- n(x),\]
for the expressions required, one of which will (unless $nx = ny$)
be less than $n(l)$, and the other greater. The least must of
course be taken.

When two or more of the constituents possess a common factor,
as $x$, that factor can only, as is obvious from Principle~{\sc i}.,
furnish a single term $n(x)$ in the final expression of the major
limit. Thus if $n(x)$ appear as a major limit in two or more
constituents, we must, in adding those limits together, replace
$nx + nx$ by $nx$, and so on. Take, for example, the expression
$n\{xy + x (1-y)z\}$. The major limits of this expression, immediately
furnished by addition, would be---
\begin{alignat*}{2}
&1.\; nx.                  &4.\;& ny + nx. \\
&2.\; nx + n (1-y). \qquad &5.\;& ny + n (1-y).  \\
&3.\; nx + n (z).          &6.\;& ny + nz.
\end{alignat*}
Of these the first and sixth only need be retained; the second,
third, and fourth being greater than the first; and the fifth being
equal to $n(1)$. The limits are therefore
\[ n(x) \text{ and } n(y) + n(z),\]
and of these two values the last, supposing it to be less than $n(1)$,
must be taken.

These considerations lead us to the following Rule:\\

\textsc{Rule.}---{\it Take one factor from each constituent, and prefix to
it the symbol $n$, add the several terms or results thus formed
together, rejecting all repetitions of the same term; the sum thus
obtained will be a major limit of the expression, and the least of all
such sums will be the major limit to be employed.}

Thus the major limits of the expression
\[ xyz + x(1-y) (1-z) + (1-x) (1-y) (1-z)\]
would be
%-----------------------File: 105.png----------------------------
%** Not sure how to left align the 'or' in the equation.
%*[2nd proofer: here's my version:]
\begin{align*}
     n(x) + n(1 - y), &\text{ and } n(x) + n(1 - z),\\
\text{or}\hfill
  n(x) + n(1) - n(y), &\text{ and } n(x) + n(1) - n(z).\hfill
\end{align*}

If we began with $n(y)$, selected from the first term, and took
$n(x)$ from the second, we should have to take $n(1 - y)$ from the
third term, and this would give
\[
n(y) + n(x) + n(1 - y), \text{ or } n(1) + n(x).
\]
But as this result exceeds $n(1)$, which is an obvious major limit
to \emph{every} class, it need not be taken into account.

\begin{center}
\textsc{Proposition III.}
\end{center}

\emph{7. To find the minor numerical limit of any class expressed by
constituents of the symbols $x, y, z$, having given
$n(x), n(y), n(z) \dotsc n(1)$.}

This object may be effected by the application of the preceding
Proposition, combined with Principle~\textsc{ii.}, but it is better
effected by the following method:

Let any two constituents, which differ from one another only
by a single factor, be added, so as to form a single class term
as $x(1 - y) + xy$ form $x$, and this species of aggregation having
been carried on as far as possible, i.e., there having been selected
out of the given series of constituents as many sums of this kind
as can be formed, each such sum comprising as many constituents
as can be collected into a single term, without regarding whether
any of the said constituents enter into the composition of other
terms, let these ultimate aggregates, together with those constituents
which do not admit of being thus added together, be
written down as distinct terms. Then the several minor limits
of those terms, deduced by Prop.~I., will be the minor limits of
the expression given, and one only of those minor limits will at
the same time be positive.

Thus from the expression $xy + (1 - x)y + (1 - x)(1 - y)$ we
can form the aggregates $y$ and $1 - x$, by respectively adding the
first and second terms together, and the second and third.
Hence $n(y)$ and $n(1 - x)$ will be the minor limits of the expression
given. Again, if the expression given were
%-----------------------File: 106.png----------------------------
\begin{multline*}
xyz + x (1-y) z + (1-x) yz + (l-x) (1-y) z\\
 + xy(1-z) +(1-x)(1-y)(1-z),
\end{multline*}
we should obtain by addition of the first four terms the single
term $z$, by addition of the first and fifth term the single term $xy$,
and by addition of the fourth and sixth terms the single term
$(1-x) (1-y)$; and there is no other way in which constituents
can be collected into single terms, nor are there are any constituents
left which have not been thus taken account of. The
three resulting terms give, as the minor limits of the given
expression, the values
\begin{gather*}
n(z),\; n(x) + n(y) - n(1),\\
\text{and}\hfill
 n(1-x) + n(1-y) - n(1), \text{ or } n(1) - n(x) - n(y). \hfill
\end{gather*}

8. The proof of the above rule consists in the proper application
of the following principles:\,---\,1st. The minor limit of any
collection of constituents which admit of being added into a single
term, will obviously be the minor limit of that single term.
This explains the first part of the rule. 2nd. The minor limit
of the sum of any two terms which either are distinct constituents,
or consist of distinct constituents, but do not admit of being
added together, will be the sum of their respective minor limits,
if those minor limits are both positive; but if one be positive, and
the other negative, it will be equal to the positive minor limit
alone. For if the negative one were added, the value of the limit
would be diminished, i.~e.\ it would be less for the sum of two
terms than for a single term. Now whenever two constituents
differ in more than one factor, so as not to admit of being added
together, the minor limits of the two cannot be both positive.
Thus let the terms be $xyz$ and $(1-x) (1-y) z$, which differ in
two factors, the minor limit of the first is $ n(x + y + z-2)$, that
of the second $n(1-x + 1-y + z-2)$, or,
\[
\text{1st. }n\{x + y-1-(1-z)\}.\quad  \text{2nd. }n\{1-x-y-(1-z)\}.
\]
If $n(x + y-1)$ is positive, $n(1-x-y)$ is negative, and the second must be negative. If $n(x + y-1)$ is negative, the first is
negative; and similarly for cases in which a larger number of
factors are involved. It may in this manner be shown that,
according to the mode in which the aggregate terms are formed in
%-----------------------File: 107.png----------------------------
the application of the rule, no two minor limits of distinct terms
can be added together, for either those terms will involve some
common constituent, in which case it is clear that we cannot add
their minor limits together,---or the minor limits of the two will
not be both positive, in which case the addition would be useless.


\textsc{Proposition IV.}

9. \textit{Given the respective numbers of individuals comprised in
any classes, $s$, $t$, \&c. logically defined, to deduce a system, of numerical
limits of any other class $w$, also logically defined.}

As this is the most general problem which it is meant to discuss
in the present chapter, the previous inquiries being merely
introductory to it, and the succeeding ones occupied with its application,
it is desirable to state clearly its nature and design.

When the classes $s$, $t \dotsc w$ are said to be logically defined, it
is meant that they are classes so defined as to enable us to write
down their symbolical expressions, whether the classes in question
be simple or compound. By the general method of this
treatise, the symbol $w$ can then be determined directly as a developed
function of the symbols $s$, $t$, \&c. in the form
\begin{equation*}
w = A + 0B + \frac{0}{0}C + \frac{1}{0}D,      \tag{1}
\end{equation*}
wherein $A$, $B$, $C$, and $D$ are formed of the constituents of $s$, $t$, \&c.
How from such an expression the numerical limits of $w$ may in
the most general manner be determined, will be considered hereafter.
At present we merely purpose to show how far this object
can be accomplished on the principles developed in the previous
propositions; such an inquiry being sufficient for the purposes of
this work. For simplicity, I shall found my argument upon the
particular development,
\begin{equation*}
w = st + 0s(1 - t) + \frac{1}{0}(1 - s) t + \frac{0}{0}(1 - s)(1 - t),
\tag{2}
\end{equation*}
in which all the varieties of coefficients present themselves.

Of the constituent $(1-s) (1-t)$, which has for its coefficient $\frac{0}{0}$, it is implied that some, none, or all of the class denoted
%-----------------------File: 108.png----------------------------
by that constituent are found in $w$. It is evident that $n(w)$ will
have its highest numerical value when all the members of the
class denoted by $(1-s)(1-f)$ are found in $w$. Moreover, as
none of the individuals contained in the classes denoted by
$s(1-t)$ and $(1-s)t$ are found in $w$, the superior numerical limits
of $w$ will be identical with those of the class $st + (1-s)(1-t)$.
They are, therefore,
\[
  ns + n(1-t) \text{ and } nt + n(1-s).
\]
\emph{In like manner a system of superior numerical limits of the
development $A + 0 B + \frac{0}{0}C + \frac{1}{0}D$, may be found from those of
$A + C$ by Prop.~2.}

Again, any minor numerical limit of $w$ will, by Principle~\textsc{ii.},
be given by the expression
\[
  n(1) - \text{ major limit of } n(1-w),
\]
but the development of $w$ being given by (1), that of $1-w$ will
obviously be
\[
  1 - w = 0A + B + \frac{0}{0}C + \frac{1}{0}D.
\]
This may be directly proved by the method of Prop.~2, Chap.~\textsc{x.}
Hence
\begin{align*}
\text{Minor limit of } n(w) &= n(1) - \text{ major limit } (B + C) \\
                            &= \text{ minor limit of } (A + D),
\end{align*}
by Principle~\textsc{ii.}, since the classes $A + D$ and $B + C$ are supplementary.
Thus the minor limit of the second member of (2)
would be $n(t)$, and, generalizing this mode of reasoning, we have
the following result:

\emph{A system of minor limits of the development
\[
  A + 0B + \frac{0}{0}C + \frac{1}{0}D
\]
will be given by the minor limits of $A + D$.}

This result may also be directly inferred. For of minor numerical
limits we are bound to seek the greatest. Now we obtain
in general a higher minor limit by connecting the class $D$
%-----------------------File: 109.png----------------------------
with $A$ in the expression of $w$, a combination which, as shown in
various examples of the Logic we are permitted to make, than
we otherwise should obtain.

Finally, as the concluding term of the development of $w$ indicates the equation $D = 0$, it is evident that $n(D) = 0$. Hence
we have
\[
  \text{Minor limit of } n(D) \stackrel{=}{<} 0,
\]
and this equation, treated by Prop.~3, gives the requisite conditions among the numerical elements $n(s)$, $n(t)$, \&c., in order that
the problem may be real, and may embody in its data the results of a possible experience,

Thus from the term $\frac{1}{0}(1-s)t$ in the second member of (2)
we should deduce
\begin{gather*}
  n(1-s) + n(t) - n(1) \stackrel{=}{<} 0,   \\
  \therefore n(t) \stackrel{=}{<}  n(s).
\end{gather*}
These conclusions may be embodied in the following rule:

10. \textsc{Rule.}---\emph{Determine the expression of the class $w$ as a developed logical function of the symbols $s$, $t$, \&c. in the form}
\[
  w = A + 0B + \frac{0}{0}C + \frac{1}{0}D.
\]
\emph{Then will}
\begin{align*}
  \text{Maj. lim. } w &= \text{ Maj. lim. } A + C.   \\
  \text{Min. lim. } w &= \text{ Min. lim. } A + D.
\end{align*}
\emph{The necessary numerical conditions among the data being given by
the inequality}
\[
  \text{Min. lim. } D \stackrel{=}{<} n(1).
\]

To apply the above method to the limitation of the solutions
of questions in probabilities, it is only necessary to replace in
each of the formula $n(x)$ by Prob. $x$, $n(y)$ by Prob. $y$, \&c., and,
finally, $n(1)$ by $1$. The application being, however, of great importance,
it may be desirable to exhibit in the form of a rule
the chief results of transformation.

11. Given the probabilities of any events $s$, $t$, \&c., whereof
another event $w$ is a developed logical function, in the form
\[
  w = A + 0B + \frac{0}{0}C + \frac{1}{0}D,
\]
%-----------------------File: 110.png----------------------------
required the systems of superior and inferior limits of Prob. $w$,
and the conditions among the data.

\textsc{Solution.}---The superior limits of Prob. $(A + C)$, and the
inferior limits of Prob. $(A + D)$ will form two such systems as are
sought. The conditions among the constants in the data will be
given by the inequality,
\[
  \text{Inf. lim. Prob. } D \stackrel{=}{<} 0.
\]
In the application of these principles we have always
\[
  \text{Inf. lim. Prob. } x_1 x_2 \dotsc x_n = \text{ Prob. } x_1
+ \text{ Prob. }x_2 \dotsc + \text{ Prob. }x_n - (n-1).
\]
Moreover, the inferior limits can only be determined from \emph{single}
terms, either given or formed by aggregation. Superior limits
are included in the form $\sum \text{ Prob. }x$, Prob. $x$ applying only to
symbols which are different, and are taken from different terms in
the expression whose superior limit is sought. Thus the superior
limits of Prob. $xyz + x(1-y)(1-z)$ are
\[
  \text{Prob. }x,\quad \text{Prob. }y + \text{ Prob. }(1-z),
  \text{ and Prob. }z + \text{ Prob. }(1-y).
\]
Let it be observed, that if in the last case we had taken Prob. $z$
from the first term, and Prob. $(1-z)$ from the second,---a connexion
not forbidden,---we should have had as their sum $1$, which
as a result would be useless because \textit{\`{a} priori} necessary. It is
obvious that we may reject any limits which do not fall between
0 and 1.

Let us apply this method to Ex. 7, Case \textsc{iii.} of the last
chapter.

The final logical solution is
\[
\begin{split}
  x &= \frac{0}{0}stu + \frac{1}{0}s\bar{t}u
     + \frac{1}{0}st\bar{u} + s\bar{t}\bar{u}   \\
    &+ \frac{1}{0}\bar{s}tu + 0\bar{s}\bar{t}u
     + 0\bar{s}t\bar{u} + 0\bar{s}\bar{t}\bar{u},
\end{split}
\]
the data being
\[
 \text{Prob. }s = p,\quad \text{Prob. }t = q,\quad \text{Prob. }u = r.
\]

We shall seek both the numerical limits of $x$, and the conditions
connecting $p$, $q$, and $r$.
%-----------------------File: 111.png----------------------------
The superior limits of $x$ are, according to the rule, given by
those of $stu + s\bar{t}\bar{u}$. They are, therefore,
\[
  p, q + 1 - r,\quad r + 1 - q.
\]
The inferior limits of $x$ are given by those of
\[
  s\bar{t}u + st\bar{u} + s\bar{t}\bar{u} + \bar{s}tu.
\]

We may collect the first and third of these constituents in the
single term $s\bar{t}$, and the second and third in the single term $s\bar{u}$.
The inferior limits of $x$ must then be deduced separately from
the terms $s(1-t)$, $s(1-u)$, $(1-s)tu$, which give
\begin{multline*}
  p + 1 - q - 1,\quad p + 1 - r - 1,\quad 1 - p + q + r - 2,   \\
\text{or }\hfill p - q,\quad p - r,\quad \text{and }q + r - p- 1.\hfill
\end{multline*}

Finally, the conditions among the constants $p$, $q$, and $r$, are
given by the terms
\[
  s\bar{t}u, \quad st\bar{u}, \quad \bar{s}tu,
\]
from which, by the rule, we deduce
\begin{align*}
& p + 1 - q + r - 2 \stackrel{=}{<} 0, \quad
  p + q + 1 - r - 2 \stackrel{=}{<} 0, \quad
  1 - p + q + r - 2 \stackrel{=}{<} 0. \\
&\text{or } \hfill
  1 + q - p - r \stackrel{=}{>} 0, \quad
  1 + r - p - q \stackrel{=}{>} 0, \quad
  1 + p - q - r \stackrel{=}{>} 0. \hfill
\end{align*}
These are the limiting conditions employed in the analysis of
the final solution. The conditions by which in that solution $\lambda$ is
limited, were determined, however, simply from the conditions
that the quantities $s$, $t$, and $u$ should be positive. Narrower
limits of that quantity might, in all probability, have been deduced from the above investigation.

12. The following application is taken from an important problem,
the solution of which will be given in the next chapter.
There are given,
\[
  \text{Prob. }x = c_1,   \quad
  \text{Prob. }y = c_2,   \quad
  \text{Prob. }s = c_1p_1,\quad
  \text{Prob. }t = c_2p_2,
\]
together with the logical equation
\[
\begin{split}
  z = stxy + s\bar{t}x\bar{y} + \bar{s}t\bar{x}y + 0\bar{s}\bar{t} \\
  + \frac{1}{0}\left\{
  \begin{split}
    stx\bar{y} + st\bar{x}y + st\bar{x}\bar{y} + s\bar{t}xy
      + s\bar{t}\bar{x}y   \\
    + s\bar{t}\bar{x}\bar{y} + \bar{s}txy + \bar{s}tx\bar{y}
      + \bar{s}t\bar{x}\bar{y};
\end{split}
\right.
\end{split}
\]
%-----------------------File: 112.png----------------------------
and it is required to determine the conditions among the constants
$c_1$, $c_2$, $p_1$, $p_2$, and the major and minor limits of $z$.

First let us seek the conditions among the constants. Confining
our attention to the terms whose coefficients are $\frac{1}{0}$, we
readily form, by the aggregation of constituents, the following
terms, viz.:
\[
s(1-x), \qquad t(1-y), \qquad sy(1-t),\qquad  tx(1-s);
\]
nor can we form any other terms which are not included under
these. Hence the conditions among the constants are,
\begin{align*}
&n(s) + n(1-x) - n(1) \stackrel{=}{<} 0,\\
&n(t) + n(1-y) - n(1) \stackrel{=}{<} 0,\\
&n(s) + n(y) + n(1-t) - 2n(1) \stackrel{=}{<} 0,\\
&n(t) + n(x) + n(1-s) - 2n(1) \stackrel{=}{<} 0.
\end{align*}

Now replace $n(x)$ by $c_1$, $n(y)$ by $c_2$, $n(s)$ by $c_1 p_1$, $n(t)$ by
$c_2 p_2$, and $n(1)$ by $1$, and we have, after slight reductions,
\[
\begin{array}{rl}
  c_1 p_1 \stackrel{=}{<} c_1, \qquad
& c_2 p_2 \stackrel{=}{<} c_2,\\
  c_1 p_1 \stackrel{=}{<} 1 - c_2(1-p_2), \qquad
& c_2 p_2 \stackrel{=}{<} 1 - c_1(1-p_1).
\end{array}
\]
Such are, then, the requisite conditions among the constants.

Again, the major limits of $z$ are identical with those of the
expression
\[
stxy + s(1-t) x(1-y) + (1-s)t(1-x)y;
\]
which, if we bear in mind the conditions
\[
n (s) \stackrel{=}{<} n (x), \qquad  n(t) \stackrel{=}{<} n (y),
\]
above determined, will be found to be
\begin{align*}
n(s) + n(t),   &\text{ or, } c_1p_1 + c_2p_2,\\
n(s) + n(1-x), &\text{ or, } 1-c_1 (1-p_1)
n(t) + n(1-y), &\text{ or, } 1-c_2 (1-p_2).
\end{align*}

Lastly, to ascertain the minor limits of $z$, we readily form
from the constituents, whose coefficients are 1 or $\frac{1}{0}$, the single
terms $s$ and $t$, nor can any other terms not included under these be
%-----------------------File: 113.png----------------------------
formed by selection or aggregation. Hence, for the minor limits
of $z$ we have the values $c_1p_1$ and $c_2p_2$.

13. It is to be observed, that the method developed above
does not always assign the narrowest limits which it is possible
to determine. But it in all cases, I believe, sufficiently limits the
solutions of questions in the theory of probabilities.

The problem of the determination of the \emph{narrowest} limits of
numerical extension of a class is, however, always reducible to a
purely algebraical form.\footnote{The author regrets the loss of a manuscript, written about four years ago,
in which this method, he believes, was developed at considerable length. His
recollection of the contents is almost entirely confined to the \emph{impression} that the
principle of the method was the same as above described, and that its sufficiency was proved.
The prior methods of this chapter are, it is almost needless
to say, easier, though certainly less general.} Thus, resuming the equations
\[
  w = A + 0B + \frac{0}{0}C + \frac{1}{0}D,
\]
let the highest inferior numerical limit of $w$ be represented by
the formula $an(s) + bn(t) \dotsc + dn(1)$, wherein $a, b, c,\dotsc d$ are
numerical constants to be determined, and $s$, $t$, \&c., the logical
symbols of which $A$, $B$, $C$, $D$ are constituents. Then
\[
\begin{split}
\hfill an(s)+ bn(t)\dotsc + dn(1)=\text{ minor limit of $A$ subject}\\
\hfill \text{ to the condition $D = 0$.}
\end{split}
\]
Hence if we develop the function
\[
  as + bt \dotsc + d,
\]
reject from the result all constituents which are found in $D$, the
coefficients of those constituents which remain, and are found
also in $A$, ought not individually to exceed unity in value, and
the coefficients of those constituents which remain, and which
are not found in $A$, should individually not exceed 0 in value.
Hence we shall have a series of inequalities of the form
$f \stackrel{=}{<} 1$,
and another series of the form $g \stackrel{=}{<} 0$, $f$ and $g$ being linear functions
of $a$, $b$, $c$, \&c. Then those values of $a, b \dotsc d$, which, while
satisfying the above conditions, give to the function
\[
  an(s) + bn(t) \dotsc + dn(1),
\]
its highest value must be determined, and the highest value in
%-----------------------File: 114.png----------------------------
question will be the highest minor limit of $w$. To the above we
may add the relations similarly formed for the determination of
the relations among the given constants $n(s), n(t) \dotsc n(1)$.

14. The following somewhat complicated example will show
how the limitation of a solution is effected, when the problem
involves an arbitrary element, constituting it the representative
of a system of problems agreeing in their data, but unlimited in
their qu{\ae}sita.

\textsc{Problem.}---Of $n$ events $x_1\; x_2 \dotsc x_n$, the following particulars
are known:

1st. The probability that either the event $x_1$ will occur, or
all the events fail, is $p_1$.

2nd. The probability that either the event $x_2$ will occur, or
all the events fail, is $p_2$. And so on for the others.

It is required to find the probability of any single event, or
combination of events, represented by the general functional form
$\phi(x_1 \dotsc x_n)$, or $\phi$.

Adopting a previous notation, the data of the problem are
\[
  \text{Prob. }(x_1 + \bar{x}_1 \dotsc \bar{x}_n) = p_1 \dotsc
  \text{Prob. }(x_n + \bar{x}_1 \dotsc \bar{x}_n) = p_n.
\]
And Prob. $\phi(x_1 \dotsc x_n)$ is required.

Assume generally
\begin{gather*}
  x_r + \bar{x}_1 \dotsc \bar{x}_n = s_r,    \tag{1}   \\
  \phi = w.                            \tag{2}
\end{gather*}
We hence obtain the collective logical equation of the problem
\[
  \sum \bigl\{\bigl( x_r + \bar{x}_1 \dotsc \bar{x}_n \bigr)\bar{s}_r
    + s_r \bigl( \bar{x}_r - \bar{x}_1 \dotsc \bar{x}_n \bigr)\bigr\}
  + \phi\bar{w} + w\bar{\phi} = 0.  \tag{3}
\]
From this equation we must eliminate the symbols $x_1, \dotsc x_n$, and
determine $w$ as a developed logical function of $s_1 \dotsc s_n$.

Let us represent the result of the aforesaid elimination in the
form
\[
  Ew + E'(1-w) = 0;
\]
then will $E$ be the result of the elimination of the same symbols
from the equation
\[
  \sum \bigl\{\bigl( x_r + \bar{x}_1 \dotsc \bar{x}_n \bigr)\bar{s}_r
    + s_r \bigl( \bar{x}_r - \bar{x}_1 \dotsc \bar{x}_n \bigr)\bigr\}
  + 1 - \phi = 0.  \tag{4}
\]

Now $E$ will be the product of the coefficients of all the constituents
(considered with reference to the symbols $x_1, x_2 \dotsc x_n$)
%-----------------------File: 115.png----------------------------
which are found in the development of the first member of the
above equation. Moreover, $\phi$, and therefore $1 - \phi$, will consist
of a series of such constituents, having unity for their respective
coefficients. In determining the forms of the coefficients in the
development of the first member of (4), it will be convenient to
arrange them in the following manner:

1st. The coefficients of constituents found in $1 - \phi$.

2nd. The coefficient of $\bar{x}_1, \bar{x}_2 \dotsc \bar{x}_n$, if found in $\phi$.

3rd. The coefficients of constituents found in $\phi$, excluding the
constituent $\bar{x}_1, \bar{x}_2 \dotsc \bar{x}_n$.

The above is manifestly an exhaustive classification.

First then; the coefficient of any constituent found in $1 - \phi$,
will, in the development of the first member of (4), be of the form
\[
  1 + \text{ positive terms derived from }\sum.
\]
Hence, every such coefficient may be replaced by unity, Prop.~\textsc{i.}
Chap.~\textsc{ix.}

Secondly; the coefficient of $\bar{x}_1 \dotsc \bar{x}_n$, if found in $\phi$, in the
development of the first member of (4) will be
\[
  \sum \bar{s}_r, \text{ or } \bar{s}_1 + \bar{s}_2 \dotsc + \bar{s}_n
\]

Thirdly; the coefficient of any other constituent, $x_1 \dotsc x_i,
\bar{x}_{i+1} \dotsc \bar{x}_n$, found in $\phi$, in the development of the first member
of (4) will be
$\bar{s}_1 \dotsc \bar{s}_i + s_{i+1} \dotsc + s_n$.

Now it is seen, that $E$ is the product of all the coefficients
above determined; but as the coefficients of those constituents
which are not found in $\phi$ reduce to unity, $E$ may be regarded as
the product of the coefficients of those constituents which are found
in $\phi$. From the mode in which those coefficients are formed, we
derive the following rule for the determination of $E$, viz., in
each constituent found in $\phi$, except the constituent
$\bar{x}_1 \; \bar{x}_2 \dotsc \bar{x}_n$,
for $x_1$ write $\bar{s}_1$, for $\bar{x}_l$ write $s_1$, and so on, and add the results;
but for the constituent $\bar{x}_1, \bar{x}_2 \dotsc \bar{x}_n$, if it occur in $\phi$, write $\bar{s}_1 + \bar{s}_2 \dotsc + \bar{s}_n$,
the product of all these sums is $E$.

To find $E'$ we must in (3) make $w = 0$, and eliminate
$x_1, x_2 \dotsc x_n$,
from the reduced equation. That equation will be
\[
  \sum \bigl\{
    \bigl( x_r + \bar{x}_1 \dotsc + \bar{x}_n \bigr)\bar{s}_r
         + s_r\bigl(\bar{x}_r - \bar{x}_1 \dotsc \bar{x}_n \bigr)
  \bigr\} + \phi = 0.  \tag{5}
\]
%-----------------------File: 116.png----------------------------
Hence $E'$ will be formed from the constituents in $1 - \phi$, i.~e.
from the constituents \emph{not} found in $\phi$ in the same way as $E$ is
formed from the constituents found in $\phi$.

Consider next the equation
\[
  Ew + E'(1-w) = 0.
\]
This gives
\[
  w = \frac{E'}{E'-E}.   \tag{6}
\]

Now $E$ and $E'$ are functions of the symbols $s_1, s_2 \dotsc s_n$. The
expansion of the value of $w$ will, therefore, consist of all the constituents which can be formed out of those symbols, with their
proper coefficients annexed to them, as determined by the rule
of development.

Moreover, $E$ and $E'$ are each formed by the multiplication of
factors, and neither of them can vanish unless some one of the
factors of which it is composed vanishes. Again, any factor, as
$\bar{s}_1 \dotsc + \bar{s}_n$ can only vanish when all the terms by the addition of
which it is formed vanish together, since in development we attribute to these terms the values 0 and 1, only. It is further evident, that no two factors differing from each other can vanish
together. Thus the factors $\bar{s}_1 + \bar{s}_2 \dotsc + \bar{s}_n$, and $s_1 + \bar{s}_2 \dotsc + \bar{s}_n$, cannot
simultaneously vanish, for the former cannot vanish unless
$\bar{s}_1 = 0$, or $s_1 = 1$; but the latter cannot vanish unless $s_1 = 0$.

First, let us determine the coefficient of the constituent
$\bar{s}_1 \bar{s}_2 \dotsc \bar{s}_n$ in the development of the value of $w$.

The simultaneous assumption $\bar{s}_1 = 1$, $\bar{s}_2 = 1 \dotsc \bar{s}_n = 1$, would
cause the factor $s_1 + s_2 \dotsc + s_n$ to vanish if this should occur in
$E$ or $E'$; and no other factor under the same assumption would
vanish; but $s_1 + s_2 \dotsc + s_n$ does not occur as a factor of either
$E$ or $E'$; neither of these quantities, therefore, can vanish; and,
therefore, the expression $\frac{E'}{E'-E}$, is neither $1$, $0$, nor $\frac{0}{0}$.

\emph{Wherefore the coefficient of
$\bar{s}_1 \; \bar{s}_2 \dotsc \bar{s}_n$ in the expanded value
of $w$,} may be represented by $\frac{1}{0}$.

Secondly, let us determine the coefficient of the constituent
$s_1 \; s_2 \dotsc s_n$.
%-----------------------File: 117.png----------------------------
The assumptions $s_1 = 1, s_2 = 1, \dotsc s_n = 1$, would cause the factor
$\bar{s}_1 + \bar{s}_2 \dotsc + \bar{s}_n$ to vanish. Now this factor is found in $E$ and not
in $E'$ whenever $\phi$ contains both the constituents
$x_1\; x_2 \dotsc x_n$ and
$\bar{x}_1\; \bar{x}_2 \dotsc \bar{x}_n$. Here then $\frac{E'}{E'-E}$ becomes $\frac{E'}{E'}$ or 1.  The factor
$\bar{s}_1 + \bar{s}_2 \dotsc + \bar{s}_n$ is found in $E'$ and not in $E$, if $\phi$ contains neither
of the constituents $x_1\; x_2 \dotsc x_n$ and
$\bar{x}_1\; \bar{x}_2 \dotsc \bar{x}_n$. Here then
$\frac{E'}{E'-E}$ becomes $\frac{0}{-E}$ or 0. Lastly, the factor
$\bar{s}_1 + \bar{s}_2 \dotsc + \bar{s}_n$ is
contained in both $E$ and $E'$, if one of the constituents $x_1\; x_2 \dotsc x_n$ and
$\bar{x}_1\; \bar{x}_2 \dotsc \bar{x}_n$ is found in $\phi$, and one is not. Here then $\frac{E'}{E'-E}$
becomes $\frac{0}{0}$.

\emph{The coefficient of the constituent $s_1\; s_2 \dotsc s_n$, will therefore be
$1$, $0$, or $\frac{0}{0}$, according as $\phi$ contains both the constituents $x_1\; x_2 \dotsc x_n$
and $\bar{x}_1\; \bar{x}_2 \dotsc \bar{x}_n$, or neither of them, or one of them and not the
other.}

Lastly, to determine the coefficient of any other constituent
as $s_1 \dotsc s_i$ $\bar{s}_{i+1} \dotsc \bar{s}_n$.
%[** you may want to replace the \; thickspace with $ $ to avoid overfull hbox]

The assumptions $s_1 = 1, \dotsc s_i = 1,\; s_{i+1} = 0,\; s_n = 0$, would
cause the factor $\bar{s}_1 \dotsc + \bar{s}_i + s_{i+1} \dotsc + s_n$ to vanish. Now this factor
is found in $E$, if the constituent
$x_1 \dotsc x_i \; \bar{x}_{i+1} \dotsc \bar{x}_n$ is found in
$\phi$ and in $E'$, if the said constituent is not found in $\phi$. In the
former case we have $\frac{E'}{E'-E} = \frac{E'}{E'} = 1$; in the latter case we have
$\frac{E'}{E'-E} = \frac{0}{0-E} = 0$.

\emph{Hence the coefficient of any other constituent
$s_1 \dotsc s_i, \; \bar{s}_{i+1} \dotsc \bar{s}_n$
is 1 or 0 according as the similar constituent
$x_1 \dotsc x_i \; \bar{x}_{i+1} \dotsc \bar{x}_n$
is or is not found in $\phi$.}

We may, therefore, practically determine the value of $w$ in
the following manner. Rejecting from the given expression of
$\phi$ the constituents $x_1 \; x_2 \dotsc x_c$ and %and
%[** and and is in the original]
$\bar{x}_1 \; \bar{x}_2 \dotsc \bar{x}_n$, should both or
either of them be contained in it, let the symbols
$x_1, \; x_2, \dotsc x_n$,
in the result be changed into $s_1, \; s_2, \dotsc s_n$ respectively. Let the coefficients
of the constituents $s_1 \; s_2 \dotsc s_n$ and $\bar{s}_1 \; \bar{s}_2 \dotsc \bar{s}_n$ be determined
%-----------------------File: 118.png----------------------------
according to the special rules for those cases given above, and let
every other constituent have for its coefficient 0. The result
will be the value of $w$ as a function of $s_1, s_2, \dotsc s_n$.

As a particular case, let $\phi = x_1$. It is required from the
given data to determine the probability of the event $x_1$.

The symbol $x_1$ expanded in terms of the entire series of symbols
$x_1, x_2, \dotsc x_n$, will generate all the constituents of those
symbols which have $x_1$ as a factor. Among those constituents
will be found the constituent $x_1 \; x_2 \dotsc x_n$, but not the constituent
$\bar{x}_1 \; \bar{x}_2 \dotsc \bar{x}_n$.

Hence in the expanded value of $x_1$ as a function of the symbols
$s_1, s_2, \dotsc s_n$, the constituent $s_1 \; s_2 \dotsc s_n$ will have the coefficient
$\frac{0}{0}$, and the constituent
$\bar{s}_1 \; \bar{s}_2 \dotsc \bar{s}_n$ the coefficient $\frac{1}{0}$.

If from $x_1$ we reject the constituent $x_1 \; x_2 \dotsc x_n$, the result
will be $x_1 - x_1x_2 \dotsc x_n$, and changing therein $x_1$ into $s_1$ \&c., we
have $s_1 - s_1s_2 \dotsc s_n$ for the corresponding portion of the expression of $x_1$ as a function of $s_1, s_2, \dotsc s_n$.

Hence the final expression for $x_1$ is
\begin{equation*}
\begin{split}
  x_1 = s_1 - s_1s_2 \dotsc s_n
      + \frac{0}{0}s_1s_2 \dotsc s_n
      + \frac{1}{0}\bar{s}_1\bar{s}_2 \dotsc \bar{s}_n   \\
  + \text{ constituents whose coefficients are }0.
\end{split}
  \tag{7}
\end{equation*}

The sum of all the constituents in the above expansion whose
coefficients are either 1, 0, or $\frac{0}{0}$, will be
$1 - \bar{s}_1\bar{s}_2 \dotsc \bar{s}_n$.

We shall, therefore, have the following \emph{algebraic} system for
the determination of Prob. $x_1$, viz.:
\[
  \text{Prob. }x_1 = \frac{s_1 - s_1s_2 \dotsc s_n + cs_1s_2 \dotsc s_n}{1 - \bar{s}_1\bar{s}_2 \dotsc \bar{s}_n},  \tag{8}
\]
with the relations
\begin{equation*}
\begin{split}
  \frac{s_1}{p_1} = \frac{s_2}{p_2} \dotsc = \frac{s_n}{p_n}   \\
  = 1 - \bar{s}_1 \bar{s}_2 \dotsc \bar{s}_n = \lambda.
\end{split}
   \tag{9}
\end{equation*}

It will be seen, that the relations for the determination of
$s_1 \; s_2 \dotsc s_n$ are quite independent of the form of the function $\phi$,
and the values of these quantities, determined once, will serve
%-----------------------File: 119.png----------------------------
for all possible problems in which the data are the same, however
the \emph{qu{\ae}sita} of those problems may vary. The nature of
that event, or combination of events, whose probability is sought,
will affect only the form of the function in which the determined
values of $s_1 \; s_2 \dotsc s_n$ are to be substituted.
We have from (9)
\[
  s_1 = p_1 \lambda, \quad
  s_2 = p_2 \lambda, \dotsc
  s_n = p_n \lambda.
\]
Whence
\[
  1 - (1 - p_1 \lambda)(1 - p_2 \lambda) \dotsc (1 - p_n \lambda)
 = \lambda.
\]
Or,
\[
  1 - \lambda
  = (1 - p_1 \lambda)(1 - p_2 \lambda) \dotsc (1 - p_n \lambda);  \tag{10}
\]
from which equation the value of $\lambda$ is to be determined.

Supposing this value determined, the value of Prob. $x_1$ will be
\[
  \frac{p_1\lambda - (1-c)p_1p_2 \dotsc p_n\lambda^n}
       {1 - (1-p_1\lambda)(1-p_2\lambda)\dotsc (1-p_n\lambda)},
\]
or, on reduction by (10),
\[
  \text{Prob. }x_1 = p_1 - (1-c)p_1p_2 \dotsc p_n \lambda^{n-1}. \tag{11}
\]

Let us next seek the conditions which must be fulfilled
among the constants $p_1, p_2, \dotsc p_n$, and the limits of the value of
Prob. $x_1$.

As there is but one term with the coefficient $\frac{1}{0}$, there is but
one condition among the constants, viz.,
\begin{gather*}
  \text{Minor limit, }(1-s_1)(1-s_2) \dotsc (1-s_n)
    \stackrel{=}{<} 0.  \\
 \begin{array}{lc}
  \text{Or,} &
    n(1-s_1) + n(1-s_2) \dotsc + n(1-s_n) - (n-1)n(1)
    \stackrel{=}{<} 0. \\
  \text{Or,} & n(1) - n(s_1) - n(s_2) \dotsc - n(s_n)
    \stackrel{=}{<} 0. \\
  \text{Whence} & p_1 + p_2 \dotsc + p_n
    \stackrel{=}{>} 1,
 \end{array}
\end{gather*}
the condition required.

The major limit of Prob. $x_1$ is the major limit of the sum of
those constituents whose coefficients are 1 or $\frac{0}{0}$. But that sum is $s_1$.

Hence,
\[
  \text{Major limit, Prob. }x_1 = \text{ major limit }s_1 = p_1.
\]
%-----------------------File: 120.png----------------------------
The minor limit of Prob. $x_1$ will be identical with the minor
limit of the expression
\[
  s_1 - s_1 s_2 \dotsc s_n + (1 -s_1)(1 - s_2)\dotsc (1 - s_n).
\]

A little attention will show that the different aggregates,
terms which can be formed out of the above, each including the
greatest possible number of constituents, will be the following,
viz.:
\[
  s_1(1-s_2),\quad s_1(1 - s_3), \dotsc s_1(1 - s_n),\quad
  (1 - s_2)(1 - s_3) \dotsc (1 - s_n).
\]

From these we deduce the following expressions for the minor
limit, viz.:
\[
  p_1 - p_2,\quad  p_1 - p_3 \dotsc p_1 - p_n,\quad
  1 - p_2 - p_3 \dotsc - p_n .
\]

The value of Prob. $x_1$ will, therefore, not fall short of any of
these values, nor exceed the value of $p_1$.

Instead, however, of employing these conditions, we may
directly avail ourselves of the principle stated in the demonstration of the general method in probabilities. The condition
that $s_1, s_2, \dotsc s_n$ must each be less than unity, requires that $\lambda$
should be less than each of the quantities
$\frac{1}{p_1}, \frac{1}{p_2}, \dotsc \frac{1}{p_n}$. And
the condition that $s_1, s_2, \dotsc s_n$, must each be greater than 0, requires that $\lambda$ should also be greater than 0. Now
$p_1 \; p_2 \dotsc p_n$
being proper fractions satisfying the condition
\[
  p_1 + p_2 \dotsc + p_n > 1,
\]
it may be shown that but one positive value of $\lambda$ can be deduced
from the central equation (10) which shall be less than each of
the quantities
$\frac{1}{p_1}, \frac{1}{p_2}, \dotsc \frac{1}{p_n}$. That value of $\lambda$ is, therefore, the
one required.

To prove this, let us consider the equation
\[
  (l - p_1\lambda)(1 - p_2\lambda) \dotsc (1 - p_n\lambda)
  - 1  + \lambda = 0.
\]

When $\lambda = 0$ the first member vanishes, and the equation is
satisfied. Let us examine the variations of the first member
between the limits $\lambda = 0$ and $\lambda = \frac{1}{p_1}$, supposing $p_1$ the greatest of
the values $p_1 \; p_2 \dotsc p_n$.
%-----------------------File: 121.png----------------------------
Representing the first member of the equation by $V$, we have
\[
  \frac{dV}{d\lambda}
  = -p_1(1-p_2\lambda) \dotsc (1-p_n\lambda) \dotsc
    -p_n(1-p_1\lambda) \dotsc (1-p_{n-1}\lambda) + 1,
\]
which, when $\lambda = 0$, assumes the form
$-p_1 - p_2 \dotsc - p_n + 1$, and
is negative in value.

Again, we have
\[
  \frac{d^2V}{d\lambda^2}
  = p_1p_2(1-p_3\lambda)(1-p_n\lambda) + \text{ \&c.,}
\]
consisting of a series of terms which, under the given restrictions
with reference to the value of $\lambda$, are \emph{positive}.

Lastly, when $\lambda = \frac{1}{p_1}$, we have
\[
  V = -1 + \frac{1}{p_1},
\]
which is positive.

From all this it appears, that if we construct a curve, the ordinates of which shall represent the value of $V$ corresponding to
the abscissa $\lambda$, that curve will pass through the origin, and will
for small values of $\lambda$ lie beneath the abscissa. Its convexity will,
between the limits $\lambda = 0$ and $\lambda = \frac{1}{p_1}$ be downwards, and at the
extreme limit $\frac{1}{p_1}$ the curve will be above the abscissa, its ordinate
being positive. It follows from this description, that it will intersect the abscissa once, and only once, within the limits specified, viz., between the values $\lambda = 0$, and
$\lambda = \frac{1}{p_1}$.

The solution of the problem is, therefore, expressed by (11),
the value of $\lambda$ being that root of the equation~(10), which lies
within the limits 0 and
$\frac{1}{p_1}, \frac{1}{p_2}, \dotsc \frac{1}{p_n}$.

The constant $c$ is obviously the probability, that if the events
$x_1, x_2, \dotsc x_n$, all happen, or all fail, they will all happen.

This determination of the value of $\lambda$ suffices for all problems
in which the data are the same as in the one just considered. It
is, as from previous discussions we are prepared to expect, a determination independent of the form of the function $\phi$.
%-----------------------File: 122.png----------------------------
Let us, as another example, suppose
\[
  \phi = \text{ or }
  w = x_1(1-x_2) \dotsc (1-x_n) \dotsc
    + x_n(1-x_1) \dotsc (1-x_{n-1}).
\]
This is equivalent to requiring the probability, that of the events
$x_1, x_2, \dotsc x_n$ one, and only one, will happen. The value of $w$ will
obviously be
\[
  w = s_1(1-s_2) \dotsc (1-s_n) \dotsc
    + s_n(1-s_1) \dotsc (1-s_{n-1})
    + \frac{1}{0}(1-s_1) \dotsc (1-s_n),
\]
from which we should have
\begin{gather*}
  \text{Prob. }
    \{ x_1(1-x_2) \dotsc (1-x_n) \dotsc
     + x_n(1-x_1) \dotsc (1-x_{n-1}) \}   \\
  = \frac{s_1(1-s_2) \dotsc (1-s_n) \dotsc
        + s_n(1-s_1) \dotsc (1-s_{n-1})}
         {1 - (1-s_1) \dotsc (1-s_n)}  \\
  = \frac{p_1\lambda(1-p_2\lambda) \dotsc (1-p_n\lambda) \dotsc
        + p_n\lambda(1-p_1\lambda) \dotsc (1-p_{n-1}\lambda)}
         {\lambda}  \\
  = \frac{p_1(1-\lambda)}{1-p_1\lambda}
  + \frac{p_2(1-\lambda)}{1-p_2\lambda} \dotsc
  + \frac{p_n(1-\lambda)}{1-p_n\lambda}
\end{gather*}
This solution serves well to illustrate the remarks made in the
introductory chapter (I.~16) The essential difficulties of the
problem are founded in the nature of its data and not in that of
its qu{\ae}sita. The central equation by which $\lambda$ is determined, and
the peculiar discussions connected therewith, are equally pertinent to every form which that problem can be made to assume,
by varying the interpretation of the arbitrary elements in its
original statement.
%-----------------------File: 123.png----------------------------

%CHAPTER XX.
\chapter[PROBLEMS ON CAUSES]{\large PROBLEMS RELATING TO THE CONNEXION OF CAUSES AND
EFFECTS.}

% ** Automatic above, manual bellow, use whichever or neither.

%\begin{center}
%{\Large \textbf{CHAPTER XX.}}\\[1cm]
%PROBLEMS RELATING TO THE CONNEXION OF CAUSES AND\\
%EFFECTS.
%\end{center}



1. So to apprehend in all particular instances the relation of
cause and effect, as to connect the two extremes in thought
according to the order in which they are connected in nature
(for the \textit{modus operandi} is, and must ever be, unknown to us),
is the final object of science. This treatise has shown, that there
is special reference to such an object in the constitution of the
intellectual faculties. There is a sphere of thought which comprehends things only as coexistent parts of a universe; but
there is also a sphere of thought (Chap.~\textsc{xi.}) in which they are
apprehended as links of an unbroken, and, to human appearance, an endless chain---as having their place in an order connecting them both with that which has gone before, and with
that which shall follow after. In the contemplation of such
a series, it is impossible not to feel the pre-eminence which is due,
above all other relations, to the relation of cause and effect.

Here I propose to consider, in their abstract form, some problems in which the above relation is involved. There exists
among such problems, as might be anticipated from the nature
of the relation with which they are concerned, a wide diversity.
From the probabilities of causes assigned \textit{\`{a} priori}, or given by
experience, and their respective probabilities of association with
an effect contemplated, it may be required to determine the probability of that effect; and this either, 1st, absolutely, or 2ndly,
under given conditions. To such an object some of the earlier
of the following problems relate. On the other hand, it may be
required to determine the probability of a particular cause, or of
some particular connexion among a system of causes, from observed effects, and the known tendencies of the said causes, singly
or in connexion, to the production of such effects. This class of
questions will be considered in a subsequent portion of the
%-----------------------File: 124.png----------------------------
chapter, and other forms of the general inquiry will also be
noticed. I would remark, that although these examples are designed chiefly as illustrations of a \emph{method}, no regard has been
paid to the question of ease or convenience in the application of
that method. On the contrary, they have been devised, with
whatever success, as types of the class of problems which might
be expected to arise from the study of the relation of cause and
effect in the more complex of its actual and visible manifestations.

2. \textsc{Problem I.}---The probabilities of two causes $A_1$ and $A_2$
are $c_1$ and $c_2$ respectively. The probability that if the cause $A_1$
present itself, an event $E$ will accompany it (whether as a consequence of the cause $A_1$ or not) is $p_1$, and the probability that if
the cause $A_2$ present itself, that event $E$ will accompany it,
whether as a consequence of it or not, is $p_2$. Moreover, the
event $E$ cannot appear in the absence of both the causes $A_1$ and
$A_2$.
%
\footnote{The mode in which such data as the above might be furnished by experience is easily conceivable. Opposite the window of the room in which I write
is a field, liable to be overflowed from two causes, distinct, but capable of being
combined, viz., floods from the upper sources of the River Lee, and tides from
the ocean. Suppose that observations made on $N$ separate occasions have
yielded the following results: On $A$ occasions the river was swollen by freshets,
and on $P$ of those occasions it was inundated, whether from this cause or not.
On $B$ occasions the river was swollen by the tide, and on $Q$ of those occasions it
was inundated, whether from this cause or not. Supposing, then, that the field
cannot be inundated in the absence of \emph{both} the causes above mentioned, let it be
required to determine the total probability of its inundation.

Here the elements $a$, $b$, $p$, $q$ of the general problem represent the ratios
\[
  \frac{A}{N},\; \frac{P}{A},\; \frac{B}{N},\; \frac{Q}{B},
\]
or rather the values to which those ratios approach, as the value of $N$ is indefinitely increased.} %[endfootnote]
%
Required the probability of the event $E$.

The solution of what this problem becomes in the case in
which the causes $A_1$, $A_2$ are mutually exclusive, is well known
to be
\[
  \text{Prob. }E = c_1p_1 + c_2p_2;
\]
and it expresses a particular case of a fundamental and very important principle in the received theory of probabilities. Here
it is proposed to solve the problem free from the restriction above
stated.

%-----------------------File: 125.png----------------------------
Let us represent
\[\begin{array}{l}
  \text{The cause }  A_1 \text{ by } x.\\
  \text{The cause }  A_2 \text{ by } y.\\
  \text{The effect } E   \text{ by } z.
\end{array}\]
Then we have the following numerical data:
\begin{align*}
  &\text{Prob. }x = c_1,     &\text{Prob. }y = c_2,   \\
  &\text{Prob. }xz = c_1p_1, &\text{Prob. }yz = c_2p_2.  \tag{1}
\end{align*}
Again, it is provided that if the causes $A_1$, $A_2$ are both absent, the effect $E$ does not occur; whence we have the logical
equation
\[
  (1-x)(1-y) = v(1-z).
\]
Or, eliminating $v$,
\[
  z(1-x)(1-y) = 0.   \tag{2}
\]

Now assume,
\[
  xz = s,\quad  yz = t.    \tag{3}
\]

Then, reducing these equations (VIII. 7), and connecting the
result with (2),
\[
  xz(1-s) + s(1-xz) + yz(1-t) + t(1-yz) + z(1-x)(1-y) = 0.  \tag{4}
\]

From this equation, $z$ must be determined as a developed
logical function of $x$, $y$, $s$, and $t$, and its probability thence deduced by means of the data,
\[
  \text{Prob. }x = c_1,\quad \text{Prob. }y = c_2,\quad
  \text{Prob. }s = c_1p_1,\quad \text{Prob. }t = c_2p_2.   \tag{5}
\]

Now developing (4) with respect to $z$, and putting $\bar{x}$ for
$1 - x$, $\bar{y}$ for $1 - y$, and\footnotemark\ so on, we have
%**[I changed "1-y"-^ from "1=y".]
%**[I agree 1-y makes sense.]
\footnotetext{The original text was \lq\lq$\bar{y}$ for $1 = y$\rq\rq, corrected here by Distributed Proofreaders.}
\begin{gather*}
  (x\bar{s} + s\bar{x} + y\bar{t} + t\bar{y} + \bar{x}\bar{y})z
   + (s + t)\bar{z} = 0,   \\
  \therefore z + \frac{s+t}{s + t
 - x\bar{s} - s\bar{x} - y\bar{t} - t\bar{y} - \bar{x}\bar{y}}  \\
%
  =            st     x      y
  + \frac{1}{0}st     x \bar{y}
  + \frac{1}{0}st\bar{x}     y
  + \frac{1}{0}st\bar{x}\bar{y}   \\
%
  + \frac{1}{0}s\bar{t}     x      y
  +            s\bar{t}     x \bar{y}
  + \frac{1}{0}s\bar{t}\bar{x}     y
  + \frac{1}{0}s\bar{t}\bar{x}\bar{y}\\
%
  + \frac{1}{0}\bar{s}t     x      y
  + \frac{1}{0}\bar{s}t     x \bar{y}
  +            \bar{s}t\bar{x}     y
  + \frac{1}{0}\bar{s}t\bar{x}\bar{y} \\
%
  + 0\bar{s}\bar{t}     x      y
  + 0\bar{s}\bar{t}     x \bar{y}
  + 0\bar{s}\bar{t}\bar{x}     y
  + 0\bar{s}\bar{t}\bar{x}\bar{y}.  \tag{6}
\end{gather*}

%-----------------------File: 126.png----------------------------
From this result we find (XVII.~17),
\begin{gather*}
V=stxy + s \bar{t}x\bar{y} + \bar{s}t\bar{x}y + \bar{s}\bar{t}xy+\bar{s}\bar{t}x\bar{y} \\
+ \bar{s}\bar{t}\bar{x}y + \bar{s}\bar{t}\bar{x}\bar{y}\\
= stxy + s\bar{t}x\bar{y} + \bar{s}t\bar{x}y+\bar{s}\bar{t}.
\end{gather*}
Whence, passing from Logic to Algebra, we have the following
system of equations, $u$ standing for the probability sought:
\begin{equation*}\tag{7}
\begin{split}
 \frac{stxy + s\bar{t}x\bar{y} + \bar{s}\bar{t}x}{c_1}
=\frac{stxy + \bar{s}t\bar{x}y + \bar{s}\bar{t}y}{c_2} \\
%**[2nd proofer: I changed this--^^^^^^^^^^^^^^^,
%   from double-barred{s}ty, which looks like a printer's error.]
=\frac{stxy + s\bar{t}x\bar{y}}{c_1 p_1}
=\frac{stxy + \bar{s}t\bar{x}y}{c_2 p_2} \\
=\frac{stxy + s\bar{t}x\bar{y} + \bar{s}t\bar{x}y}{u}
=\frac{stxy + s\bar{t}x\bar{y} + \bar{s}t\bar{x}y + \bar{s}\bar{t}}{1}
=V,
\end{split}
\end{equation*}
from which we must eliminate $s$, $t$, $x$, $y$, and $V$.

Now if we have any series of equal fractions, as
\begin{equation*}
\frac{a}{a'}=\frac{b}{b'}=\frac{c}{c'} \dotso =\lambda,
\end{equation*}
we know that
\begin{equation*}
\frac{la + mb + nc}{la' + mb' + nc'}=\lambda.
\end{equation*}
And thus from the above system of equations we may deduce
\begin{equation*}
 \frac{\bar{s}t\bar{x}y}{u-c_1 p_1}
=\frac{s\bar{t}x\bar{y}}{u-c_2 p_2}
=\frac{\bar{s}\bar{t}}{1-u}=V;
\end{equation*}
whence we have, on equating the product of the three first members
to the cube of the last,
\begin{equation*}\tag{8}
\frac{s \bar{s}^2 t \bar{t}^2 x \bar{x} y \bar{y}}
     {(u-c_1 p_1)(u-c_2 p_2)(1-u)}=V^3.
\end{equation*}

Again, from the system (7) we have
\begin{equation*}
 \frac{\bar{s}\bar{t}\bar{x}}{1-u-c_1 + c_1 p_1}
=\frac{\bar{s}\bar{t}\bar{y}}{1-u-c_2 + c_2 p_2}
=\frac{stxy}{c_1 p_1 + c_2 p_2 - u}=V,
\end{equation*}
whence proceeding as before
\begin{equation*}\tag{9}
\frac{s \bar{s}^2 t \bar{t}^2 x \bar{x} y \bar{y}}
     {(1-c_1 + c_1 p_1 - u)(1-c_2 + c_2 p_2 - u)
      (c_1 p_1 + c_2 p_2 - u)}=V^3.
\end{equation*}
%-----------------------File: 127.png----------------------------
Equating the values of $V^3$ in (8) and (9), we have
\[
\begin{split}
(u-c_1 p_1)(u-c_2 p_2)(1-u) \\
= \{1-c_1(1-p_1)-u)\}\{1-c_2(1-p_2)-u\}(c_1 p_1 + c_2 p_2 -u),
\end{split}
\]
which may be more conveniently written in the form
\begin{equation*}\tag{10}
\frac{(u-c_1 p_1)(u-c_2 p_2)}{c_1 p_1 + c_2 p_2 - u} =\frac{\{1-c_1(1-p_1)-u\}\{1-c_2(1-p_2)-u\}}{1-u}.
\end{equation*}

From this equation the value of $u$ may be found. It remains
only to determine which of the roots must be taken for this purpose.

3. It has been shown (XIX.~12) that the quantity $u$, in
order that it may represent the probability required in the above
case, must exceed each of the quantities $c_1 p_1$, $c_2 p_2$, and fall
short of each of the quantities $1-c_1(1-p_1)$, $1-c_2(1-p_2)$ and
$c_1 p_1 + c_2 p_2$; the condition among the constants, moreover, being
that the three last quantities must individually exceed each of
the two former ones. Now I shall show that these conditions
being satisfied, the final equation (10) has but one root which
falls within the limits assigned. That root will therefore be the
required value of $u$.

Let us represent the lower limits $c_1 p_1$, $c_2 p_2$, by $a$, $b$ respectively,
and the upper limits $1-c_1(1-p_1)$, $1-c_2(1-p_2)$ and
$c_1 p_1 + c_2 p_2$ by $a'$, $b'$, $c'$ respectively. Then the general equation
may be expressed in the form
\begin{gather*}
(u-a) (u-b) (1-u)-(a'-u) (b'-u) (c'-u) = 0,  \tag{11}\\
\text{or}\hfill (1-a'-b')u^2-\{ab-a'b'+(1-a'-b')c'\} u + ab - a'b'c' = 0.\hfill
\end{gather*}
Representing the first member of the above equation by $V$, we
have
\begin{equation*}\tag{12}
\frac{d^2 V}{du^2}=2(1 - a' - b').
\end{equation*}
Now let us suppose $a$ the highest of the lower limits of $u$, $a'$ the
lowest of its higher limits, and trace the progress of the values
of $V$ between the limits $u = a$ and $u = a'$.

When $u = a$, we see from the form of the first member of (11)
that $V$ is negative, and when $u = a'$ we see that $V$ is positive.
%-----------------------File: 128.png----------------------------
Between those limits $V$ varies continuously without becoming
infinite, and $\frac{d^2 V}{du^2}$ is always of the same sign.

Hence if $u$ represent the abscissa $V$ the ordinate of a plane
curve, it is evident that the curve will pass from a point below
the axis of $u$ corresponding to $u = a$, to a point above the axis of
$u$ corresponding to $u = a'$, the curve remaining continuous, and
having its concavity or convexity always turned in the same
direction. A little attention will show that, under these circumstances,
it must cut the axis of $u$ once, and only once.

Hence between the limits $u = a$, $u = a'$, there exists one value
of $u$, and only one, which satisfies the equation~(11). It will
further appear, if in thought the curve be traced, that the other
value of $u$ will be less than $a$ when the quantity $1-a'-b'$ is
positive and greater than any one of the quantities $a'$, $b'$, $c'$ when
$1-a'-b'$ is negative. It hence follows that in the solution of
(11) the positive sign of the radical must be taken. We thus
find
\begin{equation*}\tag{13}
u=\frac{ab-a'b'+(1-a'-b')c'+\surd{Q}}{2(1-a'-b')},
\end{equation*}
where  $Q = \{ab - a'b' + (1-a'-b')c'\}^2 - 4(1-a'-b') (ab-a'b'c')$.

4. The results of this investigation may to some extent be
verified. Thus, it is evident that the probability of the event $E$
must in general exceed the probability of the concurrence of the
event $E$ and the cause $A_1$ or $A_2$. Hence we must have, as the
solution indicates,
\begin{equation*}
u > c_1 p_1, \quad u > c_2 p_2.
\end{equation*}

Again, it is clear that the probability of the effect $E$ must in
general be less than it would be if the causes $A_1, A_2$ were
mutually exclusive. Hence
\begin{equation*}
u \stackrel{=}{<} c_1 p_1 + c_2 p_2.
\end{equation*}

Lastly, since the probability of the failure of the effect $E$
concurring with the presence of the cause $A_1$ must, in general, be
less than the absolute probability of the failure of $E$, we have
\begin{gather*}
c_1 (1-p_1) \stackrel{=}{<} 1-u, \\
\therefore u \stackrel{=}{<} 1 - c_1(1-p_1).
\end{gather*}
%-----------------------File: 129.png----------------------------
Similarly,
\begin{equation*}
u \stackrel{=}{<} 1-c_2(1-p_2).
\end{equation*}

And thus the conditions by which the general solution was
limited are confirmed.

Again, let $p_1 = 1$, $p_2=1$. This is to suppose that when either
of the causes $A_1$, $A_2$ is present, the event $E$ will occur. We have
then $a = c_1$, $b = c_2$, $a' = 1$, $b' = 1$, $c'= c_1 + c_2$, and substituting in
(13) we get
\begin{gather*}
u=\frac{ c_1 c_2 - c_1 - c_2 - 1
       + \surd\{ (c_1 c_2 - c_1 - c_2 - 1)^2
               +4(c_1 c_2 - c_1 - c_2)      \} }{-2} \\
= c_1 + c_2 - c_1 c_2 \text{ on reduction}\\
= 1 - (1 - c_1) (1 - c_2).
\end{gather*}
Now this is the known expression for the probability that one
cause at least will be present, which, under the circumstances, is
evidently the probability of the event $E$.

Finally, let it be supposed that $c_1$ and $c_2$ are very small, so
that their product may be neglected; then the expression for $u$
reduces to $c_1 p_1 + c_2 p_2$. Now the smaller the probability of each
cause, the smaller, in a much higher degree, is the probability of
a conjunction of causes. Ultimately, therefore, such reduction
continuing, the probability of the event $E$ becomes the same as
if the causes were mutually exclusive.

I have dwelt at greater length upon this solution, because it
serves in some respect as a model for those which follow, some of
which, being of a more complex character, might, without such
preparation, appear difficult.

5. \textsc{Problem II}.---In place of the supposition adopted in the
previous problem, that the event $E$ cannot happen when both the
causes $A_1$, $A_2$ are absent, let it be assumed that the causes $A_1$, $A_2$
cannot both be absent, and let the other circumstances remain as
before. Required, then, the probability of the event $E$.

Here, in place of the equation~(2) of the previous solution, we
have the equation
\begin{equation*}
(1-x)(1-y) = 0.
\end{equation*}
The developed logical expression of $z$ is found to be
%-----------------------File: 130.png----------------------------
\begin{equation*}\begin{split}
& z = stxy + \frac{1}{0} stx\bar{y} + \frac{1}{0}st\bar{x}y + \frac{1}{0} st\bar{x}\bar{y} \\
& + \frac{1}{0}s\bar{t}xy + s\bar{t}x\bar{y} + \frac{1}{0}s\bar{t}\bar{x}y + \frac{1}{0} s\bar{t}\bar{x}\bar{y} \\
& + \frac{1}{0}\bar{s}txy + \frac{1}{0} \bar{s}tx\bar{y} + \bar{s}t\bar{x}y + \frac{1}{0}\bar{s}t\bar{x}\bar{y} \\
& + 0 \bar{s}\bar{t}xy + 0 \bar{s}\bar{t}x\bar{y} + 0\bar{s}\bar{t}\bar{x}y + \frac{1}{0}\bar{s}\bar{t}\bar{x}\bar{y};
\end{split}
\end{equation*}
and the final solution is
\begin{equation*}
\operatorname{Prob. } E = u;
\end{equation*}
the quantity $u$ being determined by the solution of the equation
\begin{equation*}\tag{1}
\frac{(u-a)(u-b)}{a+b-u}=\frac{(a'-u)(b'-u)}{u-a'-b'+1},
\end{equation*}
wherein $a=c_1 p_1$, $b=c_2 p_2$, $a'=1 - c_1(1-p_1)$, $b'=1-c_2(1-p_2)$.

The conditions of limitation are the following:---That value
of $u$ must be chosen which exceeds each of the three quantities
\begin{equation*}
a, b, \text{ and } a' + b'-1,
\end{equation*}
and which at the same time falls short of each of the three quantities
\begin{equation*}
a', b', \text{ and } a + b.
\end{equation*}

Exactly as in the solution of the previous problem, it may be
shown that the quadratic equation~(1) will have one root, and
only one root, satisfying these conditions. The conditions themselves
were deduced by the same rule as before, excepting that
the minor limit $a' + b'-1$ was found by seeking the major limit
of $1-z$.

It may be added that the constants in the data, beside
satisfying the conditions implied above, viz., that the quantities $a'$, $b'$,
and $a + b$, must individually exceed $a$, $b$, and $a' + b'-1$, must
also satisfy the condition $c_1 + c_2 \stackrel{=}{>} 1$. This also appears from the
application of the rule.

6. \textsc{Problem III.}---The probabilities of two events $A$ and $B$
are $a$ and $b$ respectively, the probability that if the event $A$ take
place an event $E$ will accompany it is $p$, and the probability that
%-----------------------File: 131.png----------------------------
if the event $B$ take place, the same event $E$ will accompany it
is $q$. Required the probability that if the event $A$ take place the
event $B$ will take place, or \textit{vice vers\^{a}}, the probability that if $B$
take place, $A$ will take place.

Let us represent the event $A$ by $x$, the event $B$ by $y$, and the
event $E$ by $z$. Then the data are---
\begin{align*}
\operatorname{Prob. } x & = a, & \operatorname{Prob. } y & = b. \\
\operatorname{Prob. } x z & = a p,& \operatorname{Prob. } y z & = b q.
\end{align*}
Whence it is required to find
\begin{equation*}
\frac{\operatorname{Prob. } x y}{\operatorname{Prob. } x} \text{ or }
\frac{\operatorname{Prob. } x y}{\operatorname{Prob. } y}.
\end{equation*}
Let
\begin{equation*}
xy=s, \quad yz = t, \quad xy=w.
\end{equation*}
Eliminating $z$, we have, on reduction,
\begin{gather*}
s \bar{x} + t \bar{y} + s y \bar{t} + x t \bar{s} + x y \bar{w}
+ (1 - x y)w = 0, \\
\therefore w = \frac{s \bar{x} + t \bar{y} + s y \bar{t}
+ x t \bar{s} + x y}{2 x y -1}
\begin{split}
& = xyst + \frac{1}{0}x\bar{y}st
  + \frac{1}{0}\bar{x}yst + \frac{1}{0}\bar{x}\bar{y}st
\\
& + \frac{1}{0}xys\bar{t} + 0 x\bar{y}s\bar{t}
  + \frac{1}{0}\bar{x}ys\bar{t} + \frac{1}{0}\bar{x}\bar{y}s\bar{t}
\\
& + \frac{1}{0}xy\bar{s}t + \frac{1}{0}x \bar{y}\bar{s}t
  + 0\bar{x}y\bar{s}t + \frac{1}{0}\bar{x}\bar{y}\bar{s}t
\\
& + xy\bar{s}\bar{t} + 0 x\bar{y}\bar{s}\bar{t}
+ 0\bar{x}y\bar{s}\bar{t} + 0 \bar{x}\bar{y}\bar{s}\bar{t}.
\end{split}
\tag{1}
\end{gather*}
Hence, passing from Logic to Algebra,
\begin{equation*}
\operatorname{Prob. } xy = \frac{xyst + xy\bar{s}\bar{t}}{V},
\end{equation*}
$x$, $y$, $s$, and $t$ being determined by the system of equations
\begin{gather*}
\frac{xyst + x\bar{y}s\bar{t} + xy\bar{s}\bar{t} + x\bar{y}\bar{s}\bar{t}}{a}
= \frac{xyst + \bar{x}y\bar{s}t + xy\bar{s}\bar{t} + \bar{x}y\bar{s}\bar{t}}{b} \\
= \frac{xyst + x\bar{y}s\bar{t}}{ap}=\frac{xyst + \bar{x}y\bar{s}t}{bq} \\
= xyst + x\bar{y}s\bar{t} + \bar{x}y\bar{s}t + xy\bar{s}\bar{t} + x\bar{y}\bar{s}\bar{t}
+\bar{x}y\bar{s}\bar{t}+\bar{x}\bar{y}\bar{s}\bar{t}=V.
\end{gather*}
%-----------------------File: 132.png----------------------------
To reduce the above system to a more convenient form, let every
member be divided by $\bar{x}\bar{y}\bar{s}\bar{t}$, and in the result let
\begin{equation*}
\frac{xs}{\bar{x}\bar{s}}=m, \quad
\frac{yt}{\bar{y}\bar{t}}=m',\quad
\frac{x}{\bar{x}}=n, \quad
\frac{y}{\bar{y}}=n'.
\end{equation*}
We then find
\begin{gather*}
\frac{m m' + m + nn' + n}{a}=\frac{mm' + m' + nn' + n'}{b} \\
=\frac{mm'+m}{ap} = \frac{mm'+m'}{bq} \\
= mm' + m + m' + nn' + n + n' + 1.
\end{gather*}
Also,
\begin{equation*}
\operatorname{Prob. } xy = \frac{mm' + nn'}{mm'+m+m'+nn'+n+n'+1}.
\end{equation*}
These equations may be reduced to the form
\begin{gather*}
 \frac{mm'+m}{ap}    =\frac{mm'+m'}{bq}
=\frac{nn'+n}{a(1-p)}=\frac{nn'+n'}{b(1-q)} \\
=(m+1)(m'+1) + (n+1)(n'+1) - 1. \\
\operatorname{Prob. } xy = \frac{mm' + nn'}{(m+1)(m'+1)+(n+1)(n'+1)-1}.
\end{gather*}
Now assume
\begin{equation*}\tag{2}
(m+1)(m'+1)=\frac{\mu}{\nu + \mu - 1}, \quad (n+1)(n'+1)=\frac{\nu}{\nu + \mu - 1}.
\end{equation*}
Then since $\displaystyle mm'+m=\frac{m(m'+1)(m+1)}{m+1}=\frac{m\mu}{(m+1)(\nu + \mu - 1)}$,
and so on for the other numerators of the system, we find, on
substituting and multiplying each member of the system by
$\nu + \mu - 1$, the following results:
\begin{gather*}
 \frac{m \mu}{(m +1)ap}
=\frac{m'\mu}{(m'+1)bq}
=\frac{n \nu}{(n +1)a(1-p)}
=\frac{n'\nu}{(n'+1)b(1-q)} = 1.
\\
\operatorname{Prob. } xy = (mm' + nn')(\nu + \mu - 1). \tag{3}
\end{gather*}
From the above system we have
\begin{equation*}
\frac{m}{m+1}=\frac{ap}{\mu}, \text{ whence } m=\frac{ap}{\mu-ap}.
\end{equation*}
%-----------------------File: 133.png----------------------------
Similarly
\[
  m'= \frac{bq}{\mu-bq},\quad
  n = \frac{a(1-p)}{\nu - a(1-p)},\quad
  n'= \frac{b(1-q)}{\nu - b(1-q)}.
\]
Hence
\[
  m + 1 = \frac{\mu}{\mu-ap},\quad
  n + 1 = \frac{\nu}{\nu - a(1-p), \text{ \&c.}}
\]
Substitute these values in (2) reduced to the form
\[
  \nu + \mu - 1 = \frac{\mu}{(m+1)(m'+1)} = \frac{\nu}{(n+1)(n'+1)},
\]
and we have
\[
  \nu + \mu - 1 = \frac{(\mu-ap)(\mu-bq)}{\mu}
  = \frac{\{\nu - a(1-p)\}\; \{\nu - b(1-q)\}}{\nu},  \tag{4}
\]
Substitute also for $m$, $m'$, \&c. their values in (3), and we have
\begin{gather*}
  \text{Prob. }xy   \\
  = \left[ \frac{abpq}{(\mu-ap)(\mu-bq)}
         + \frac{ab(1-p)(1-q)}
                {\{\nu - a(1-p)\} \{\nu - b(1-q)\}}\right]
    (\nu + \mu - 1)   \\
  = \frac{abpq}{\mu} + \frac{ab(1-p)(1-q)}{\nu} \text{ by (4).}
\end{gather*}
Now the first equation of the system (4) gives
\begin{gather*}
  \nu + \mu - 1 = \mu - ap - bq + \frac{apbq}{\mu},  \tag{5}  \\
  \therefore \frac{apbq}{\mu} = \nu - 1 + ap + bq.
\end{gather*}
Similarly,
\[
  \frac{ab(1-p)(1-q)}{\nu} = \mu - 1 + a(1-p) + b(1-q).
\]
Adding these equations together, and observing that the first
member of the result becomes identical with the expression just
found for Prob. $xy$, we have
\[
  \text{Prob. }xy = \nu + \mu + a + b - 2.
\]
Let us represent Prob. $xy$ by $u$, and let $a + b - 2 = m$, then
\[
  \mu + \nu = u - m.  \tag{6}
\]
Again, from (5) we have
\[
  \mu\nu = abpq - (ap + bq - 1)\mu.  \tag{7}
\]
%-----------------------File: 134.png----------------------------
Similarly from the first and third members of (4) equated we
have
\begin{equation*}
\mu \nu = ab(1-p)(1-q) - \{a(1-p)+b(1-q)-1\}\nu.
\end{equation*}
Let us represent $ap + bq - 1$ by $h$, and $a(1 - p) + b(1 - q) - 1$ by
$h'$. We find on equating the above values of $\mu \nu$,
\begin{align*}
h \mu - h'\nu &= ab \{pq + (1-p)(1-q)\}  \\
              &= ab(p + q - 1).
\end{align*}
Let $ab (p + q - 1) = l$, then
\begin{equation*}\tag{8}
h \mu - h'\nu = l.
\end{equation*}
Now from (6) and (8) we get
\begin{equation*}
\mu = \frac{h' (u-m)+l}{m}, \quad \nu = \frac{h(u-m)-l}{m}.
\end{equation*}
Substitute these values in (7) reduced to the form
\begin{equation*}
\mu(\nu + h) = abpq,
\end{equation*}
and we have
\begin{equation*}\tag{9}
(hu - l) \{h'(u - m) + l\} = abpqm^2,
\end{equation*}
a quadratic equation, the solution of which determines $u$, the
value of $\operatorname{Prob. } xy$ sought.

The solution may readily be put in the form
\begin{equation*}
h=\frac{lh'+h(h'm-l)\pm\surd[\{lh'-h(h'm-l)\}^2+4hh'abpqm^2]}{2hh'}.
\end{equation*}
But if we further observe that
\begin{equation*}
lh' - h (h'm -l) = l(h + h') - hh'm = (l - hh') m,
\end{equation*}
since
\begin{equation*}
h = ap + bq - 1, \quad h' = a (1 - p) + b (1 - q) - 1,
\end{equation*}
whence
\begin{equation*}
h+h'=a + b-2 = m,
\end{equation*}
we find
\begin{equation*}\tag{10}
\operatorname{Prob. } xy
= \frac{lh'+h(h'm-l)\pm m\surd\{(l-hh')^2+4hh'abpq\}}{2hh'}.
\end{equation*}
It remains to determine which sign must be given to the
radical. We might ascertain this by the general method exemplified
in the last problem, but it is far easier, and it fully suffices in the
present instance, to determine the sign by a comparison of the
%-----------------------File: 135.png----------------------------
above formula with the result proper to some known case. For
instance, if it were certain that the event $A$ is \emph{always}, and the
event $B$ \emph{never}, associated with the event $E$, then it is certain that
the events $A$ and $B$ are never conjoined. Hence if $p = 1$, $q = 0$,
we ought to have $u = 0$. Now the assumptions $p = 1$, $q = 0$,
give
\[
  h = a - 1,\quad h'= b - 1,\quad l = 0,\quad m = a + b - 2.
\]
Substituting in (10) we have
\[
  \text{Prob. }xy
 = \frac{(a-1)(b-1)(a+b-2) \pm (a + b - 2)(a-1)(b-1)}{2(a-1)(b-1)},
\]%**[Missing ^--open parenthesis added.]
%[** agreed. added note.]
and\footnote{The numerator was originally \lq\lq$(a-1)b-1)(a+b-2)\dots$\rq\rq, and was fixed in 2004 by Distributed Proofreaders.
} this expression vanishes when the lower sign is taken.
Hence the final solution of the general problem will be expressed
in the form
\[
  \frac{\text{Prob. }xy}{\text{Prob. }x}
  = \frac{ lh' + h(h'm-l) - m\surd\{(l-hh')^2 + 4hh'abpq\} }{2ahh'},
\]
wherein $h = ap + bq - 1,\quad h' = a(1-p) + b(1-q) - 1$,
\[
  l = ab(p + q - 1),\quad m = a + b - 2.
\]
As the terms in the final logical solution affected by the coefficient $\frac{1}{0}$ are the same as in the first problem of this chapter,
the conditions among the constants will be the same, viz.,
\[
  ap \stackrel{=}{<} 1 - b(1-q),\quad bq \stackrel{=}{<} 1 - a(1-p).
\]
7. It is a confirmation of the correctness of the above solution
that the expression obtained is symmetrical with respect to the
two sets of quantities $p$, $q$, and $1 - p$, $1 - q$, i.e. that on changing
$p$ into $1 - p$, and $q$ into $1 - q$, the expression is unaltered This
is apparent from the equation
\[
  \text{Prob. }xy
  = ab\left\{ \frac{pq}{\mu} + \frac{(1-p)(1-q)}{\nu} \right\}
\]
employed in deducing the final result. Now if there exist probabilities $p$, $q$ of the event $E$, as consequent upon a knowledge
of the occurrences of $A$ and $B$, there exist probabilities $1 - p$, $1 - q$
of the contrary event, that is, of the non-occurrence of $E$ under
the same circumstances. As then the data are unchanged in
%-----------------------File: 136.png----------------------------
form, whether we take account in them of the occurrence or of
the non-occurrence of $E$, it is evident that the solution ought to
be, as it is, a symmetrical function of $p$, $q$ and $1 - p$, $1 - q$.

Let us examine the particular case in which $p = 1$, $q = 1$.
We find
\begin{equation*}
h = a + b - 1, \quad h' = -1, \quad l = ab, \quad m = a + b - 2,
\end{equation*}
and substituting
\begin{gather*}
\frac{\operatorname{Prob. } xy}{\operatorname{Prob. } x}
=\frac{-ab + (a+b-1)(2-a-b-ab)-(a+b-2)(ab-a-b+1)}{-2a(a+b-1)} \\
=\frac{-2ab(a+b-1)}{-2a(a+b-1)}=b.
\end{gather*}
It would appear, then, that in this case the events $A$ and $B$ are
virtually independent of each other. The supposition of their
invariable association with some other event $E$, of the frequency
of whose occurrence, except as it may be inferred from this
particular connexion, absolutely nothing is known, does not establish
any dependence between the events $A$ and $B$ themselves. I apprehend
that this conclusion is agreeable to reason, though particular
examples may appear at first sight to indicate a different
result. For instance, if the probabilities of the casting up, 1st,
of a particular species of weed, 2ndly, of a certain description of
zoophytes upon the sea-shore, had been separately determined,
and if it had also been ascertained that neither of these events
could happen except during the agitation of the waves caused by
a tempest, it would, I think, justly be concluded that the events
in question were not independent. The picking up of a piece of
seaweed of the kind supposed would, it is presumed, render more
probable the discovery of the zoophytes than it would otherwise
have been. But I apprehend that this fact is due to our
knowledge of another circumstance not implied in the actual conditions
of the problem, viz., that the occurrence of a tempest is but an
\textit{occasional} phenomenon. Let the range of observation be
confined to a sea \textit{always} vexed with storm. It would then, I
suppose, be seen that the casting up of the weeds and of the
zoophytes ought to be regarded as independent events. Now,
to speak more generally, there are conditions common to all
%-----------------------File: 137.png----------------------------
ph\ae{}nomena,---conditions which, it is felt, do not affect their mutual
independence. I apprehend therefore that the solution indicates,
that when a particular condition has prevailed through the whole
of our \textit{recorded experience}, it assumes the above character with
reference to the class of ph\ae{}nomena over which that experience
has extended.

8. \textsc{Problem} IV.---To illustrate in some degree the above
observations, let there be given, in addition to the data of the
last problem, the absolute probability of the event $E$, the
completed system of data being
\begin{gather*}
\operatorname{Prob. } x = a, \quad \operatorname{Prob. } y = b, \quad \operatorname{Prob. } z = c, \\
\operatorname{Prob. } xz = ap, \quad \operatorname{Prob. } yz = bq,
\end{gather*}
and let it be required to find $\operatorname{Prob. } xy$.

Assuming, as before, $xz = s$, $yz=t$, $xy = w$, the final logical
equation is
\begin{equation*}
\begin{split}
w = xystz + xy\bar{s}\bar{t}\bar{z}
+ 0(x\bar{y}s\bar{t}z + \bar{x}y\bar{s}tz
  + x\bar{y}\bar{z}\bar{s}\bar{t}+ xy\bar{z}\bar{s}\bar{t}
\hfill \\ \hfill
+ \bar{x}\bar{y}z\bar{s}\bar{t}+ \bar{x}\bar{y}\bar{z}\bar{s}\bar{t})
\\
+ \text{ terms whose coefficients are } \frac{1}{0}.  %\tag{1}
\end{split}
\end{equation*}
The algebraic system having been formed, the subsequent
eliminations may be simplified by the transformations adopted in the
previous problem. The final result is
\begin{equation*}
\operatorname{Prob. } xy
= ab \left\{ \frac{pq}{c} + \frac{(1-p)(1-q)}{1-c} \right\}.  \tag{2}
\end{equation*}
The conditions among the constants are
\begin{equation*}
c \stackrel{=}{>} ap, \quad
c \stackrel{=}{>} bq, \quad
c \stackrel{=}{<} 1 - a(1 - p), \quad
c \stackrel{=}{<} 1-b(1-q).
\end{equation*}

Now if $p = 1$, $q = 1$, we find
\begin{equation*}
\operatorname{Prob. } xy = \frac{ab}{c},
\end{equation*}
$c$ not admitting of any value less than $a$ or $b$. It follows hence
that if the event $E$ is known to be an \textit{occasional} one, its
invariable attendance on the events $x$ and $y$ \textit{increases} the probability
of their conjunction in the inverse ratio of its own frequency.
%-----------------------File: 138.png----------------------------
The formula~(2) may be verified in a large number of cases.
As a particular instance, let $q = c$, we find
\begin{equation*}\tag{3}
\operatorname{Prob. } xy = ab.
\end{equation*}
Now the assumption $q = c$ involves, by Definition (Chap.~XVI.)
the independence of the events $B$ and $E$. If then $B$ and $E$ are
independent, no relation which may exist between $A$ and $E$ can
establish a relation between $A$ and $B$; wherefore $A$ and $B$ are
also independent, as the above equation~(3) implies.

It may readily be shown from (2) that the value of $\operatorname{Prob. } z$,
which renders $\operatorname{Prob. } xy$ a minimum, is
\begin{equation*}
\operatorname{Prob. } z = \frac{\surd(pq)}{\surd(pq)+\surd(1-p)(1-q)}.
\end{equation*}
If $p = q$, this gives
\begin{equation*}
\operatorname{Prob. } z = p;
\end{equation*}
a result, the correctness of which may be shown by the same
considerations which have been applied to (3).

\textsc{Problem} V.---Given the probabilities of any three events,
and the probability of their conjunction; required the
probability of the conjunction of any two of them.

Suppose the data to be
\begin{equation*}
\operatorname{Prob. } x = p, \quad
\operatorname{Prob. } y = q, \quad
\operatorname{Prob. } z = r, \quad
\operatorname{Prob. } xyz = m,
\end{equation*}
and the qu\ae{}situm to be $\operatorname{Prob. } xy$.

Assuming $xyz = s$, $xy = t$, we find as the final logical
equation,
\begin{equation*}
t=xyzs + xy\bar{z}\bar{s} + 0(x\bar{y}\bar{s} + \bar{x}\bar{s})
+ \frac{1}{0}(\text{sum of all other constituents});
\end{equation*}
whence, finally,
\begin{equation*}
\operatorname{Prob. } xy
= \frac{H-\surd(H^2 - 4pq\bar{r}^2 - 4\bar{p}\bar{q}\bar{r}m)}
       {2\bar{r}},
\end{equation*}
wherein
\begin{equation*}
\bar{p} = 1-p, \text{ \&c}. \quad H= \bar{p}\bar{q} + (p + q)\bar{r}.
\end{equation*}
This admits of verification when $p = 1$, when $q = 1$, when $r = 0$,
and therefore $m = 0$, \&c.

Had the condition, $\operatorname{Prob. } z = r$, been omitted, the solution
would still have been definite. We should have had
%-----------------------File: 139.png----------------------------
\[
  \text{Prob. }xy = \frac{pq(1-m) + (1-p)(1-q)m}{1 - m};
\]
and it may be added, as a final confirmation of their correctness,
that the above results become identical when $m = pqr$.

9. The following problem is a generalization of Problem~I.,
and its solution, though necessarily more complex, is obtained by
a similar analysis.

\textsc{Problem~VI.}---If an event can only happen as a consequence of one or more of certain causes $A_1, A_2, \dotsc A_n$, and if
generally $c_i$ represent the probability of the cause $A_i$ and $p_i$ the
probability that if the cause $A_i$ exist, the event $E$ will occur,
then the series of values of $c_i$ and $p_i$ being given, required the
probability of the event $E$.\footnote{
 It may be proper to remark, that the above problem was proposed to the
notice of mathematicians by the author in the Cambridge and Dublin Mathematical Journal, Nov. 1851, accompanied by the subjoined observations:

``The motives which have led me, after much consideration, to adopt, with
reference to this question, a course unusual in the present day, and not upon
slight grounds to be revived, are the following:---First, I propose the question
as a test of the sufficiency of received methods. Secondly, I anticipate that its
discussion will in some measure add to our knowledge of an important branch
of pure analysis. However, it is upon the former of these grounds alone that I
desire to rest my apology.

``While hoping that some may be found who, without departing from the line
of their previous studies, may deem this question worthy of their attention, I
wholly disclaim the notion of its being offered as a trial of personal skill or
knowledge, but desire that it may be viewed solely with reference to those public and scientific ends for the sake of which alone it is proposed.''

The author thinks it right to add, that the publication of the above problem
led to some interesting private correspondence, but did not elicit a solution.}%endfootnote

Let the causes $A_1, A_2, \dotsc A_n$ be represented by
$x_1, x_2, \dotsc x_n$,
and the event $E$ by $z$.

Then we have generally,
\[
  \text{Prob. }x_i = c_i,\qquad  \text{Prob. }x_i z = c_i p_i.
\]
Further, the condition that $E$ can only happen in connexion with
some one or more of the causes $A_1, A_2, \dotsc A_n$ establishes the logical condition,
\[
  z(1-x_1)(1-x_2) \dotsc (1-x_n) = 0.   \tag{1}
\]

%-----------------------File: 140.png----------------------------
Now let us assume generally
\[
  x_i z = t_i,
\]
which is reducible to the form
\[
  x_i z(1-t_i) + t_i(1-x_i z) = 0,
\]
forming the type of a system of $n$ equations which, together with
(1), express the logical conditions of the problem. Adding all
these equations together, as after the previous reduction we are
permitted to do, we have
\[
  \sum\{ x_i z(1-t_i) + t_i(1-x_i z) \}
  + z(1-x_1)(1-x_2) \dotsc (1-x_n) = 0,   \tag{2}
\]
(the summation implied by $\sum$ extending from $i = 1$ to $i = n$), and
this single and sufficient logical equation, together with the $2n$
data, represented by the general equations
\[
  \text{Prob. }x_i = c_i,\quad \text{Prob. }t_i = c_i p_i,  \tag{3}
\]
constitute the elements from which we are to determine Prob. $z$.

Let (2) be developed with respect to $z$. We have
\begin{multline*}
  \left[ \sum\{ x_i (1-t_i) + t_i(1-x_i ) \}
         + (1-x_1)(1-x_2) \dotsc (1-x_n) \right] z \\
  + \sum t_i(1-z) = 0,
\end{multline*}
whence
\[
  z = \frac{ \sum t_i }
           { \sum t_i - \sum\{ x_i (1-t_i) + t_i(1-x_i) \}
             - (1-x_1)(1-x_2) \dotsc (1-x_n) }.   \tag{4}
\]
Now any constituent in the expansion of the second member of
the above equation will consist of $2n$ factors, of which $n$ are taken
out of the set $x_1, x_2, \dotsc x_n, 1-x_1, 1-x_2, \dotsc 1-x_n$, and $n$ out of
the set $t_1, t_2, \dotsc t_n, 1-t_1, 1-t_2 \dotsc 1-t_n$, no such combination as
$x_1(1-x_1), t_1(1-t_1)$, being admissible. Let us consider first
those constituents of which $(1-t_1), (1-t_2) \dotsc (1-t_n)$ forms the
$t$-factor, that is the factor derived from the set
$t_1,\dotsc 1-t_1$.

The coefficient of any such constituent will be found by
changing $t_1, t_2, \dotsc t_n$ respectively into $0$ in the second member of
(4), and then assigning to $x_1, x_2, \dotsc x_n$ their values as dependent
upon the nature of the $x$-factor of the constituent. Now simply
substituting for $t_1, t_2, \dotsc t_n$ the value $0$, the second member becomes
\[
  \frac{0}{ -\sum x_i - (1-x_1)(1-x_2) \dotsc (1-x_n) },
\]
%-----------------------File: 141.png----------------------------
and this vanishes whatever values, 0, 1, we subsequently assign
to $x_1, x_2, \dotsc x_n$. For if those values are not all equal to 0, the
term $\sum x_i$ does not vanish, and if they are all equal to 0, the term
$-(1-x_1) \dotsc (1-x_n)$ becomes $-1$, so that in either case the denominator
does not vanish, and therefore the fraction does. Hence
the coefficients of all constituents of which $(1-t_1) \dotsc (1-t_n)$ is a
factor will be 0, and as the sum of all possible $x$-constituents is
unity, there will be an aggregate term $0(1-t_1)\dotsc (1-t_n)$ in the
development of $z$.

Consider, in the next place, any constituent of which the
$t$-factor is $t_1\; t_2 \dotsc t_r(1-t_{r+1}) \dotsc (1-t_n)$, $r$ being equal
to or greater
than unity. Making in the second member of (4),
$t_1 = 1,\dotsc t_r = 1, t_{r+1} = 0, \dotsc t_n = 0$, we get the expression
\[
  \frac{r}{x_1\dotsc + x_r - x_{r+1} \dotsc - x_n
           - (1-x_1)(1-x_2)\dotsc (1-x_n)}
\]
Now the only admissible values of the symbols being 0 and 1,
it is evident that the above expression will be equal to 1 when
$x_1 = 1\dotsc x_r = 1, x_{r+1} = 0,\dotsc x_n = 0$, and that for all other
combinations of value that expression will assume a value greater than
unity. Hence the coefficient 1 will be applied to all constituents
of the final development which are of the form
\[
  x_1\dotsc x_r(1-x_{r+1})\dotsc (1-x_n)\,
  t_1\dotsc t_r(1-t_{r+1})\dotsc (1-t_n),
\]
the $x$-factor being similar to the $t$-factor, while other constituents
included under the present case will have the virtual coefficient $\frac{1}{0}$.
Also, it is manifest that this reasoning is independent
of the particular arrangement and succession of the individual
symbols.

Hence the complete expansion of $z$ will be of the form
\begin{multline*}
  z = \sum(XT) + 0(1-t_1)(1-t_2)\dotsc (1-t_n)   \\
      + \text{ constituents whose coefficients are }\frac{1}{0},
  \tag{5}
\end{multline*}
where $T$ represents any $t$-constituent except
$(1-t_1)\dotsc (1-t_n)$,
and $X$ the corresponding or similar constituent of $x_1\dotsc x_n$.
%-----------------------File: 142.png----------------------------
For instance, if $n = 2$, we shall have
\begin{equation*}
\sum (XT) = x_1 x_2 t_1 t_2 + x_1 \bar{x}_2 t_1 \bar{t}_2 + \bar{x}_1 x_2 \bar{t}_1 t_2,
\end{equation*}
$\bar{x}_1$, $\bar{x}_2$, \&c. standing for $1 - x_1$, $1 - x_2$, \&c.; whence
\begin{gather*} \tag{6}
\begin{split}
z = x_1 x_2 t_1 t_2
+ x_1 \bar{x}_2 t_1 \bar{t}_2
+ \bar{x}_1 x_2 \bar{t}_1 t_2 \\
+ 0(x_1 x_2 \bar{t}_1 \bar{t}_2
+ x_1 \bar{x}_2 \bar{t}_1 \bar{t}_2
+ \bar{x}_1 x_2 \bar{t}_1 \bar{t}_2
+ \bar{x}_1 \bar{x}_2 \bar{t}_1 \bar{t}_2) \\
+ \text{ constituents whose coefficients are }\frac{1}{0}.
\end{split}
\end{gather*}
This result agrees, difference of notation being allowed for, with
the developed form of $z$ in Problem~I. of this chapter, as it
evidently ought to do.

10. To avoid complexity, I purpose to deduce from the above
equation~(6) the necessary conditions for the determination of
$\operatorname{Prob. } z$ for the particular case in which $n = 2$, in such a form as
may enable us, by pursuing in thought the same line of
investigation, to assign the corresponding conditions for the more
general case in which $n$ possesses any integral value whatever.

Supposing then $n = 2$, we have
\begin{gather*}
\begin{split}
V = x_1 x_2 t_1 t_2
+ x_1 \bar{x}_2 t_1 \bar{t}_2
+ \bar{x}_1 x_2 \bar{t}_1 t_2
+ x_1 x_2 \bar{t}_1 \bar{t}_2
+ x_1 \bar{x}_2 \bar{t}_1 \bar{t}_2
\\
+ \bar{x}_1 x_2 \bar{t}_1 \bar{t}_2
+ \bar{x}_1 \bar{x}_2 \bar{t}_1 \bar{t}_2.
\end{split}
\\
\operatorname{Prob. } z
= \frac{x_1 x_2 t_1 t_2
+ x_1 \bar{x}_2 t_1 \bar{t}_2
+ \bar{x}_1 x_2 \bar{t}_1 t_2  }{V},
\end{gather*}
the conditions for the determination of $x_1$, $t_1$, \&c., being
\[
\begin{split}
 \frac{x_1 x_2 t_1 t_2
 + x_1 \bar{x}_2 t_1 \bar{t}_2
 + x_1 x_2 \bar{t}_1 \bar{t}_2
 + x_1 \bar{x}_2 \bar{t}_1 \bar{t}_2}{c_1}
\\
=\frac{x_1 x_2 t_1 t_2
 + \bar{x}_1 x_2 \bar{t}_1 t_2
 + x_1 x_2 \bar{t}_1 \bar{t}_2
 + \bar{x}_1 x_2 \bar{t}_1 \bar{t}_2}{c_2}
\\
=\frac{x_1 x_2 t_1 t_2 + x_1 \bar{x}_2 t_1 \bar{t}_2}{c_1 p_1}
=\frac{x_1 x_2 t_1 t_2 + \bar{x}_1 x_2 \bar{t}_1 t_2}{c_2 p_2} = V.
\end{split}
\]

Divide the members of this system of equations by
$\bar{x}_1 \bar{x}_2 \bar{t}_1 \bar{t}_2$,
and the numerator and denominator of $\operatorname{Prob. } z$ by the same
quantity, and in the results assume
\begin{equation*}\tag{7}
 \frac{x_1 t_1}{\bar{x}_1 \bar{t}_1} = m_1, \quad
 \frac{x_2 t_2}{\bar{x}_2 \bar{t}_2} = m_2, \quad
 \frac{x_1}{\bar{x}_1}=n_1, \quad \frac{x_2}{\bar{x}_2}=n_2;
\end{equation*}

%-----------------------File: 143.png----------------------------
we find
\begin{gather*}
\operatorname{Prob. } z = \frac{m_1 m_2 + m_1 + m_2}{m_1 m_2 + m_1 + m_2 + n_1 n_2 + n_1 + n_2 + 1},\\
\text{and } \frac{m_1 m_2 + m_1 + n_1 n_2 + n_1}{c_1}=\frac{m_1 m_2 + m_2 + n_1 n_2 + n_2}{c_2}
\\
\tag{8}
= \frac{m_1 m_2 + m_1}{c_1 p_1} + \frac{m_1 m_2 + m_2}{c_2 p_2}
=m_1 m_2 + m_1 + m_2 + n_1 n_2 + n_1 + n_2 + 1,
\end{gather*}
whence, if we assume,
\begin{equation*}\tag{9}
(m_1 + 1) (m_2+1)=M, \quad (n_1 + 1) (n_2 + 1) = N,
\end{equation*}
we have, after a slight reduction,
\begin{gather*}
\operatorname{Prob. } z = \frac{M-1}{M+N-1}, \\
\frac{n_1 (n_2 + 1)}{c_1 (1-p_1)} = \frac{n_2 (n_1 + 1)}{c_2 (1-p_2)} = \frac{m_1 (m_2 + 1)}{c_1 p_1} = \frac{m_2 (m_1 + 1)}{c_2 p_2}
=M+N-1;
\end{gather*}
or,
\[
\begin{split}
\frac{m_1 M}{(m_1 + 1)c_1 p_1}=\frac{m_2 M}{(m_2 + 1)c_2 p_2} =\frac{n_1 N}{(n_1 + 1)c_1(1-p_1)}\\
=\frac{n_2 N}{(n_2 + 1)c_2(1-p_2)}=M+N-1.
\end{split}
\]

Now let a similar series of transformations and reductions be
performed in thought upon the final logical equation (5). We
shall obtain for the determination of $\operatorname{Prob. } z$ the following
expression:
\begin{equation*}\tag{10}
\operatorname{Prob. } z = \frac{M-1}{M+N-1}
\end{equation*}
wherein
\begin{gather*}
M=(m_1 + 1) (m_2 + 1)\dotsc (m_n + 1), \\
N = (n_1 + 1) (n_2 + 1)\dotsc (n_n + 1),
\end{gather*}
$m_1, \dotsc, m_n, n_1,\dotsc,n_n$, being given by the system of equations,
\begin{equation*}\tag{11}\begin{split}
\frac{m_1 M}{(m_1 + 1) c_1 p_1} = \frac{m_2 M}{(m_2 + 1) c_2 p_2} \dotso = \frac{m_n M}{(m_n + 1) c_n p_n} &\\
 =\frac{n_1 N}{(n_1 + 1)c_1 (1-p_1)}\dotso
= \frac{n_n N}{(n_n+1)c_n(1-p_n)} =M+N-1.&
\end{split}\end{equation*}

Still further to simplify the results, assume
%-----------------------File: 144.png----------------------------
\begin{equation*}
\frac{M+N-1}{M}=\frac{1}{\mu}, \quad \frac{M+N-1}{N}=\frac{1}{\nu};
\end{equation*}
whence
\begin{equation*}
M=\frac{\mu}{\mu+\nu-1}, \quad N=\frac{\nu}{\mu+\nu-1}.
\end{equation*}
We find
\begin{gather*}
\frac{m_1}{(m_1+1)c_1 p_1}=\frac{m_2}{(m_2+1)c_2 p_2}\dotso =\frac{m_n}{(m_n+1)c_n p_n}=\frac{1}{\mu}, \\
\frac{n_1}{(n_1+1)c_1(1-p_1)}=\frac{n_2}{(n_2+1)c_2(1-p_2)}\dotso =\frac{n_n}{(n_n+1)c_n(1-p_n)}=\frac{1}{\nu};
\end{gather*}
whence
\begin{equation*}
m_1=\frac{c_1 p_1}{\mu - c_1 p_1}, \dotsc
m_n=\frac{c_n p_n}{\mu - c_n p_n};
\end{equation*}
and finally,
\begin{gather*}
m_1+1=\frac{\mu}{\mu-c_1 p_1}, \dotsc
m_n+1=\frac{\mu}{\mu-c_n p_n}, \\
n_1+1=\frac{\nu}{\nu-c_1(1-p_1)}, \dotsc
n_n+1=\frac{\nu}{\nu-c_n(1-p_n)}.
\end{gather*}

Substitute these values with those of $M$ and $N$ in (9), and
we have
\begin{gather*}
\frac{\mu^n}{(\mu-c_1 p_1)(\mu-c_2 p_2) \dotsc (\mu - c_n p_n)} =\frac{\mu}{\mu + \nu -1}, \\
\frac{\nu^n}{ \{\nu-c_1(1-p_1)\} \{\nu-c_2(1-p_2)\} \dotsc \{\nu-c_n(1-p_n)\} } =\frac{\nu}{\mu+\nu-1},
\end{gather*}
which may be reduced to the symmetrical form
\begin{equation*}\tag{12}\begin{split}
  \mu+\nu-1=\frac{(\mu-c_1 p_1) \dotsc (\mu - c_n p_n)}{\mu^{n-1}} \\
= \frac{ \{\nu-c_1(1-p_1)\} \dotsc \{\nu-c_n(1-p_n)\} }{\nu^{n-1}}.
\end{split}\end{equation*}
Finally,
\begin{equation*}\tag{13}
\operatorname{Prob. } z = \frac{M-1}{M+N-1}=1-\nu.
\end{equation*}
Let us then assume $1-\nu = u$, we have then
\begin{gather*}
\mu - u = \frac{(\mu-c_1 p_1) \dotsc (\mu-c_n p_n)}{\mu^{n-1}} \\
= \frac{ \{1-c_1(1-p_1)\} \dotsc \{1-c_n(1-p_n)-u\} }{(1-u)^{n-1}}.
\end{gather*}

%-----------------------File: 145.png----------------------------
If we make for simplicity
\begin{equation*}
c_1 p_1 = a_1, \quad c_n p_n=a_n, \quad 1-c_1(1-p_1)=b_1,\text{ \&c.},
\end{equation*}
the above equations may be written as follows:
\begin{equation}\tag{14}
\mu - u = \frac{(\mu - a_1) \dotsc (\mu - a_n)}{\mu^{n-1}},
\end{equation}
wherein
\begin{equation}\tag{15}
\mu = u + \frac{(b_1 - u) \dotsc (b_n - u)}{(1-u)^{n-1}}.
\end{equation}

This value of $\mu$ substituted in (14) will give an equation
involving only $u$, the solution of which will determine $\operatorname{Prob. } z$,
since by (13) $\operatorname{Prob. } z = u$. It remains to assign the limits of $u$.

11. Now the very same analysis by which the limits were
determined in the particular case in which $n = 2$, (XIX.~12)
conducts us in the present case to the following result. The
quantity $u$, in order that it may represent the value of $\operatorname{Prob. } z$, must
must have for its inferior limits the quantities $a_1, a_2, \dotsc a_n$, and
for its superior limits the quantities $b_1, b_2,\dotsc b_n$, $a_1 + a_2 \dotsc + a_n$.
We may hence infer, \textit{\`{a} priori}, that there will always exist one
root, and only one root, of the equation (14) satisfying these
conditions. I deem it sufficient, for practical verification, to show
that there will exist one, and only one, root of the equation (14),
between the limits $a_1, a_2, \dotsc a_n$, and $b_1, b_2,\dotsc b_n$.

First, let us consider the nature of the changes to which $\mu$ is
subject in (15), as $u$ varies from $a_1$, which we will suppose the
greatest of its minor limits, to $b_1$, which we will suppose the least
of its major limits. When $u = a_1$ it is evident that $\mu$ is positive
and greater than $a_1$. When $u = b_1$, we have $\mu = b_1$, which is also
positive. Between the limits $u = a_1$, $u = b_1$ it may be shown
that $\mu$ increases with $u$. Thus we have
\begin{equation}\tag{16}\begin{split}
\frac{d\mu}{du}
=1-\frac{(b_2 - u)     \dotsc (b_n - u)}{(1-u)^{n-1}}
  -\frac{(b_1-u)(b_3-u)\dotsc (b_n - u)}{(1-u)^{n-1}}\dotsc \\
+ (n-1)\frac{(b_1 - u)(b_2-u)\dotsc (b_n-u)}{(1-u)^n}.
\end{split}\end{equation}
Now let
\begin{equation*}
\frac{b_1 - u}{1-u}=x_1 \dotsc \frac{b_n - u}{1-u}=x_n.
\end{equation*}
%-----------------------File: 146.png----------------------------
Evidently $x_1, x_2,\dotsc x_n$, will be proper fractions, and we have
\[
\begin{split}
\frac{d\mu}{du}=1 - x_2 x_3\dotsc x_n - x_1 x_3 \dotsc x_n \dotsc
- x_1 x_2 \dotsc x_{n-1} + (n-1)x_1 x_2 \dotsc x_n \\
=1 - (1-x_1)x_2 x_3 \dotsc x_n - x_1 (1-x_2)x_3 \dotsc x_n \dotsc \\
- x_1 x_2 \dotsc x_{n-1}(1-x_n) - x_1 x_2 \dotsc x_n.
\end{split}
\]

Now the negative terms in the second member are (if we may
borrow the language of the logical developments) \emph{constituents}
formed from the fractional quantities $x_1, x_2, \dotsc x_n$. Their sum
cannot therefore exceed unity; whence $\displaystyle \frac{d\mu}{du}$ is positive, and $\mu$
increases with $u$ between the limits specified.

Now let (14) be written in the form
\begin{equation}\tag{17}
\frac{(\mu-a_1)\dotsc(\mu-a_n)}{\mu^{n-1}} - (\mu-u) = 0,
\end{equation}
and assume $u = a_1$. The first member becomes
\begin{equation}\tag{18}
(\mu-a_1)\left\{\frac{(\mu-a_2)\dotsc(\mu-a_n)}{\mu^{n-1}} -1\right\},
\end{equation}
and this expression is negative in value. For, making the same
assumption in (15), we find
\begin{equation*}
\mu - a_1 = \frac{(b_1 - u)\dotsc(b_n - u)}{(1-u)^{n-1}}
= \text{ a positive quantity}.
\end{equation*}
At the same time we have
\begin{equation*}
\frac{(\mu-a_2)\dotsc (\mu - a_n)}{\mu^{n-1}}
=\frac{\mu-a_2}{\mu} \dotsc \frac{\mu-a_n}{\mu},
\end{equation*}
and since the factors of the second member are positive fractions,
that member is less than unity, whence (18) is negative.
Wherefore \emph{the assumption $u = a_1$ makes the first member of (17) negative.}

Secondly, let $u = b_1$, then by (15) $\mu=u=b_1$, and the \emph{first
member of (17) becomes positive}.

Lastly, between the limits $u=a_1$ and $u=b_1$ the first member
of (17) continuously increases. For the first term of that
expression written under the form
\begin{equation*}
(\mu-a_1)\frac{\mu - a_1}{\mu}\dotsc \frac{\mu-a_n}{\mu}
\end{equation*}
%-----------------------File: 147.png----------------------------
increases, since $\mu$ increases, and, with it, every factor contained.
Again, the negative term $\mu-u$ diminishes with the increase of
$u$, as appears from its value deduced from (15), viz.,
\begin{equation*}
\frac{(b_1-u)\dotsc(b_n-u)}{(1-u)^{n-1}}.
\end{equation*}
Hence then, between the limits $u = a_1$, $u = b_1$, the first member
of (17) continuously increases, changing in so doing from a
negative to a positive value. Wherefore, between the limits assigned,
there exists one value of $u$, and only one, by which the said
equation is satisfied.

12. Collecting these results together, we arrive at the
following solution of the general problem.

The probability of the event $E$ will be that value of $u$
deduced from the equation
\begin{equation}\tag{19}
\mu-u=\frac{(\mu-c_1 p_1)\dotsc(\mu-c_n p_n)}{\mu^{n-1}},
\end{equation}
wherein
\begin{equation*}
\mu=u+\frac{\{1-c_1(1-p_1)-u\}\dotsc\{1-c_n(1-p_n)-u\}}{(1-u)^{n-1}},
\end{equation*}
which (value) lies between the two sets of quantities,
\begin{equation*}
c_1 p_1, c_2 p_2, \dotsc, c_n p_n \text{ and }
1-c_1(1-p_1), 1-c_2(1-p_2) \dotsc 1-c_n(1-p_n),
\end{equation*}
the former set being its inferior, the latter its superior, limits.

And it may further be inferred in the general case, as it has
been proved in the particular case of $n = 2$, that the value of $u$,
determined as above, will not exceed the quantity
\begin{equation*}
c_1 p_1 + c_2 p_2 \dotsc + c_n p_n.
\end{equation*}

13. Particular verifications are subjoined.

1st. Let $p_1 = 1, p_2=1, \dotsc p_n = 1$. This is to suppose it
certain, that if any one of the events $A_1, A_2 \dotsc A_n$ happen, the
event $E$ will happen. In this case, then, the probability of the
occurrence of $E$ will simply be the probability that the events or
causes $A_1, A_2 \dotsc A_n$ do not all fail of occurring, and its expression
will therefore be $1-(1-c_1)(1-c_2)\dotsc(1-c_n)$.

Now the general solution (19) gives
%-----------------------File: 148.png----------------------------
\[
  \mu - u = \frac{(\mu-c_1) \dotsc (\mu-c_n)}{\mu^{n-1}},
\]
wherein
\[
  \mu = u + \frac{(1-u)^n}{(1-u)^{n-1}} = 1.
\]
Hence,
\begin{gather*}
  1 - u = (1-c_1) \dotsc (1-c_n),  \\
  \therefore u = 1 - (1-c_1) \dotsc (1-c_n),
\end{gather*}
equivalent to the \textit{\`{a} priori} determination above.

2nd. Let $p_1 = 0$, $p_2 = 0$, $p_n = 0$, then (19) gives
\begin{gather*}
  \mu - u = \mu,   \\
  \therefore u = 0,
\end{gather*}
as it evidently ought to be.

3rd. Let $c_1, c_2\dotsc c_n$ be small quantities, so that their squares
and products may be neglected. Then developing the second
members of the equation~(19),
\begin{gather*}
\begin{split}
  \mu - u
  &= \frac{ \mu^n - (c_1p_1 + c_2p_2 \dotsc + c_np_n)\mu^{n-1} }
          { \mu^{n-1} }   \\
  &= \mu - (c_1p_1 + c_2p_2 \dotsc + c_np_n),
\end{split}   \\
  \therefore u = c_1p_1 + c_2p_2 \dotsc + c_np_n.
\end{gather*}

Now this is what the solution would be were the causes
$A_1, A_2\dotsc A_n$ mutually exclusive. But the smaller the probabilities of those causes, the more do they approach the condition
of being mutually exclusive, since the smaller is the probability of
any concurrence among them. Hence the result above obtained
will undoubtedly be the limiting form of the expression for the
probability of $E$.

4th. In the particular case of $n = 2$, we may readily eliminate
$\mu$ from the general solution. The result is
\[
  \frac{(u-c_1p_1)(u-c_2p_2)}{c_1p_1 + c_2p_2 - u}
  =  \frac{ \{1 - c_1(1-p_1) - u\} \{1 - c_2(1-p_2) - u\} }{1 - u},
\]
which agrees with the particular solution before obtained for this
case, Problem \textsc{i.}

Though by the system (19), the solution is in general made
to depend upon the solution of an equation of a high order, its
%-----------------------File: 149.png----------------------------
practical difficulty will not be great. For the conditions relating
to the limits enable us to select at once a near value of $u$, and
the forms of the system (19) are suitable for the processes of
successive approximation.

14. \textsc{Problem} 7.---The data being the same as in the last
problem, required the probability, that if any definite and given
combination of the causes $A_1, A_2,\dotsc A_n$, present itself, the event
$E$ will be realized.

The cases $A_1, A_2,\dotsc A_n$, being represented as before by
$x_1, x_2, \dotsc x_n$ respectively, let the definite combination of them,
referred to in the statement of the problem, be represented by
the $\phi(x_1, x_2 \dotsc x_n)$ so that the actual occurrence of that
combination will be expressed by the logical equation,
\begin{equation*}
\phi(x_1, x_2, \dotsc x_n)=1.
\end{equation*}

The data are
\begin{equation}\tag{1}\begin{aligned}
&\operatorname{Prob. } x_1 = c_1, \dotsc \; &
&\operatorname{Prob. } x_n = c_n, \\
&\operatorname{Prob. } x_1 z = c_1 p_1, \; &
&\operatorname{Prob. } x_n z = c_n p_n;
\end{aligned}\end{equation}
and the object of investigation is
\begin{equation}\tag{2}
\frac{\operatorname{Prob. } \phi(x_1, x_2\dotsc x_n)z}
     {\operatorname{Prob. } \phi(x_1, x_2\dotsc x_n) }.
\end{equation}
We shall first seek the value of the numerator.

Let us assume,
\begin{gather*}\tag{3}
x_1 z = t_1 \dotsc x_n z = t_n,
\\\tag{4}
\phi(x_1, x_2\dotsc x_n) z = w.
\end{gather*}
Or, if for simplicity, we represent $\phi(x_1, x_2\dotsc x_n)$ by $\phi$, the last
equation will be
\begin{equation}\tag{5}
\phi z = w,
\end{equation}
to which must be added the equation
\begin{equation}\tag{6}
\bar{x}_1 \bar{x}_2 \dotsc \bar{x}_n z = 0.
\end{equation}

Now any equation $x_r z = t_r$ of the system (3) may be reduced
to the form
\begin{equation*}
x_r z \bar{t}_r + t_r(1-x_r z)=0.
\end{equation*}
Similarly reducing (5), and adding the different results together,
we obtain the logical equation
%-----------------------File: 150.png----------------------------
\[
  \Sigma \{x_rz\bar{t}_r + t_r(1-x_rz)\}
+ \bar{x}_1\dotsc \bar{x}_nz + \phi z\bar{w} + w(1-\phi z) = 0, \tag{7}
\]
from which $z$ being eliminated, $w$ must be determined as a developed logical function of $x_1,\dotsc x_n, t_1,\dotsc t_n$.

Now making successively $z = 1, z = 0$ in the above equation,
and multiplying the results together, we have
\[
  \{\Sigma (x_r\bar{t}_r + \bar{x}_rt_r)
+ \bar{x}_1\dotsc \bar{x}_n + \phi\bar{w} + w\bar{\phi}\}
  \times (\Sigma t_r + w) = 0.
\]
Developing this equation with reference to $w$, and replacing
in the result $\sum t_r + 1$ by $1$, in accordance with Prop. \textsc{i.} Chap. \textsc{ix.},
we have
\[
  Ew + E'(1-w) = 0;
\]
wherein
\begin{gather*}
  E = \Sigma (x_r\bar{t}_r + t_r\bar{x}_r)
    + \bar{x}_1 \dotsc \bar{x}_n + \bar{\phi},   \\
  E'= \Sigma t_r \{ \Sigma (x_r\bar{t}_r + t_r\bar{x}_r)
    + \bar{x}_1 \dotsc \bar{x}_n + \phi \}.
\end{gather*}
And hence
\[
  w = \frac{E'}{E'-E}. \tag{8}
\]

The second member of this equation we must now develop
with respect to the double series of symbols
$x_1, x_2,\dotsc x_n, t_1, t_2,\dotsc t_n$.
In effecting this object, it will be most convenient to arrange
the constituents of the resulting development in three distinct
classes, and to determine the coefficients proper to those classes
separately.

First, let us consider those constituents of which
$\bar{t}_1 \dotsc \bar{t}_n$ is a
factor. Making $t_1 = 0 \dotsc t_n = 0$, we find
\[
  E' = 0,\quad E = \Sigma x_r + \bar{x}_1 \dotsc \bar{x}_n + \bar{\phi}.
\]
It is evident, that whatever values $(0, 1)$ are given to the $x$-symbols, $E$ does not vanish. Hence the coefficients of all constituents
involving $\bar{t}_1 \dotsc \bar{t}_n$ are $0$.

Consider secondly, those constituents which do not involve the
factor $\bar{t}_1 \dotsc \bar{t}_n$, and which are symmetrical with reference to the two
sets of symbols $x_1 \dotsc x_n$ and $t_1 \dotsc t_n$. By symmetrical constituents
is here meant those which would remain unchanged if $x_1$ were
converted into $t_1$, $x_2$ into $t_2$, \&c., and \textit{vice vers\^{a}}. The constituents
$x_1 \dotsc x_n\ t_1 \dotsc t_n,
\bar{x}_1 \dotsc \bar{x}_n\ \bar{t}_1 \dotsc \bar{t}_n$, \&c., are in this sense symmetrical.
%-----------------------File: 151.png----------------------------
For all symmetrical constituents it is evident that
\[
  \sum (x_r \bar{t}_r + t_r \bar{x}_r)
\]
vanishes. For those which do not involve $\bar{t}_1 \dotsc \bar{t}_n$, it is further
evident that $\bar{x}_1 \dotsc \bar{x}_n$ also vanishes, whence
\begin{gather*}
  E = \bar{\phi} \qquad E' = \Sigma t_r(\phi),   \\
  w = \frac{\sum t_r(\phi)}{ \sum t_r(\phi) - \bar{\phi} }.
\end{gather*}
For those constituents of which the $x$-factor is found in $\phi$ the
second member of the above equation becomes 1; for those of
which the $x$-factor is found in $\bar{\phi}$ it becomes 0. Hence \emph{the coefficients of symmetrical constituents not involving
$\bar{t}_1 \dotsc \bar{t}_n$, of which
the $x$-factor is found in $\phi$ will be 1; of those of which the $x$-factor
is not found in $\phi$ it will be 0.}

Consider lastly, those constituents which are unsymmetrical
with reference to the two sets of symbols, and which at the same
time do not involve $\bar{t}_1 \dotsc \bar{t}_n$.

Here it is evident, that neither $E$ nor $E'$ can vanish, whence
the numerator of the fractional value of $w$ in (8) must exceed
the denominator. That value cannot therefore be represented
by 1, 0, or $\frac{0}{0}$. It must then, in the logical development, be represented by $\frac{1}{0}$. Such then will be the coefficient of this class
of constituents.

15. Hence the final logical equation by which $w$ is expressed
as a developed logical function of
$x_1, \dotsc x_n$, $t_1, \dotsc t_n$, will be of
the form
\[
  w = \sum\nolimits_1 (XT)
  + 0\{ \sum\nolimits_2(XT) + \bar{t}_1\dotsc \bar{t}_n \}
  + \frac{1}{0}
  \text{\raisebox{-1.5ex}{$\genfrac{}{}{0pt}{}{\text{(sum of other con-}}{\text{stituents),}}$}} \tag{9}
\]%[** The code above is ugly but it reproduces the original.  Below is simpler code but is not exactly as the original]
%  \parbox{8em}{(sum of other constituents),} \tag{9}
%
wherein $\sum_1 (XT)$ represents the sum of all symmetrical constituents of which the factor $X$ is found in $\phi$, and $\sum_2(XT)$, the
sum of all symmetrical constituents of which the factor $X$ is not
found in $\phi$,---the constituent $\bar{x}_1 \dotsc \bar{x}_n\; \bar{t}_1 \dotsc \bar{t}_n$, should it appear,
being in either case rejected.

Passing from Logic to Algebra, it may be observed, that
%-----------------------File: 152.png----------------------------
here and in all similar instances, the function $V$, by the aid of
which the algebraic system of equations for the determination of
the values of $x_1, \dotsc x_n, t_1 \dotsc t_n$ is formed, is independent of the
nature of any function $\phi$ involved, not in the expression of the
\emph{data}, but in that of the \textit{qu{\ae}situm} of the problem proposed. Thus
we have in the present example,
\begin{multline*}
  \hfill \text{Prob. }w = \frac{\sum\nolimits_1 (XT)}{V}, \hfill \\
  \text{wherein }\hfill
  V = \sum\nolimits_1 (X T) + \sum\nolimits_2 (X T)
      + t_1 \dotsc \bar{t}_n \hfill \\
  \hfill = \sum (XT) + \bar{t}_1 \dotsc t_n. \hfill \tag{10}
\end{multline*}
Here $\sum (X T)$ represents the sum of all symmetrical constituents
of the $x$ and $t$ symbols, except the constituent
$\bar{x}_1 \dotsc \bar{x}_n,\; \bar{t}_1 \dotsc \bar{t}_n$.
This value of $V$ is the same as that virtually employed in the solution of the preceding problem, and hence we may avail ourselves of the results there obtained.

If then, as in the solution referred to, we assume
\[
  \frac{x_1 t_1}{\bar{x}_1 \bar{t}_1} = m_1,\quad
  \frac{x_n t_n}{\bar{x}_n \bar{t}_n} = m_n,\quad
  \frac{x_1}{\bar{x}_1} = n_1, \text{ \&c.,}
\]
we shall obtain a result which may be thus written:
\[
  \text{Prob. }w = \frac{M_1}{M + N - 1},   \tag{11}
\]
$M_1$ being formed by rejecting from the function $\phi$ the constituent
$x_1 \dotsc \bar{x}_n$, if it is there found, dividing the result by the same constituent $\bar{x}_1 \dotsc \bar{x}_n$ and then changing
$\frac{x_1}{\bar{x}_1}$ into $m_1$,
$\frac{x_2}{\bar{x}_2}$ into $m_2$, and
so on. The values of $M$ and $N$ are the same as in the preceding
problem. Reverting to these and to the corresponding values of
$m_1$, $m_2$, \&c., we find
\[
  \text{Prob. }w = M_1(\mu + \nu - 1),
\]
the general values of $m_r$, $n_r$ being
\[
  m_r = \frac{c_r p_r}{\mu - c_rp_r},\qquad
  n_r = \frac{c_r(1-p_r)}{\mu - c_r(1-p_r)},
\]
and $\mu$ and $\nu$ being given by the solution of the system of equations,
%-----------------------File: 153.png----------------------------
\[
\mu + \nu -1 = \frac{(\mu-c_1p_1)\dotsc (\mu-c_np_n)}{\mu^{n-1}}
= \frac{ \{\nu-c_1(1-p_1)\} \dotsc \{\nu-c_n(1-p_n)\} }{\nu^{n-1}} .
\]
The above value of Prob. $w$ will be the numerator of the fraction
(2). It now remains to determine its denominator.

For this purpose assume
\[
\phi(x_1,x_2 \dotsc x_n)=v,
\]
or
\[
\phi = v ;
\]
whence
\[
\phi \bar{v} + v \bar{\phi} = 0.
\]

Substituting the first member of this equation in (7) in place of
the corresponding form $\phi z\bar{w}+w(1-\phi z)$ we obtain as the primary
logical equation,
\[
\sum \{x_r z \bar{t}_r + t_r(1-x_r z)\} + \bar{x}_1 \dotsc \bar{x}_n z + \phi \bar{v} + v \bar{\phi} = 0,
\]
whence eliminating $z$, and reducing by Prop.~II.\ Chap.~IX.,
\[
\phi \bar{v} + v \bar{\phi} + \sum t_r \{\sum (x_r\bar{t}_r + t_r\bar{x}_r) + \bar{x}_1 \dotsc \bar{x}_n\}=0.
\]
Hence
\[
v = \frac{\phi + \sum t_r \{\sum(x_r\bar{t}_r + t_r\bar{x}_r) + \bar{x}_1 \dotsc \bar{x}_n\} }{2\phi-1}
\]
and developing as before,
\[
\begin{split}
v = \sum\nolimits_1(XT)+\bar{t}_1\dotsc\bar{t}_n \sum\nolimits_1(X) + 0\{\sum\nolimits_2(XT)+\bar{t}_1\dotsc\bar{t}_n\sum\nolimits_2(X)\}
\\
+ \frac{1}{0}(\text{sum of other constituents}).
\end{split}  \tag{12}
\]

Here $\sum_1(X)$ indicates the sum of all constituents found in $\phi$,
$\sum_2(X)$ the sum of all constituents not found in $\phi$. The expressions
are indeed used in place of $\phi$ and $1-\phi$ to preserve symmetry.

It follows hence that $\sum_1(X)+\sum_2(X)=1$, and that, as before,
$\sum_1(XT)+\sum_2(XT) = \sum(X T)$. Hence $V$ will have the
same value as before, and we shall have
\[
\operatorname{Prob. } v
= \frac{\sum_1(XT)+\bar{t}_1\dotsc\bar{t}_n \sum_1(X)}{V},
\]

Or transforming, as in the previous case,
\[
\operatorname{Prob. } v = \frac{M_1+N_1}{M+N-1},  \tag{13}
\]

%-----------------------File: 154.png----------------------------
wherein $N_1$ is formed by dividing $\phi$ by $\bar{x}_1\dotsc\bar{x}_n$, and changing in
the result $\frac{x_1}{\bar{x}_1}$ into $n_1$,
$\frac{x_2}{\bar{x}_2}$ into $n_2$, \&c.

Now the final solution of the problem proposed will be given
by assigning their determined values to the terms of the fraction
\[
\frac{\operatorname{Prob. } \phi(x_1,\dotsc x_n)z}
     {\operatorname{Prob. } \phi(x_1,\dotsc x_n)}, \mathrm{ or }
\frac{\operatorname{Prob. } w}{\operatorname{Prob. } v}.
\]
Hence, therefore, by (11) and (13) we have
\[
\textrm{Prob. sought} = \frac{M_1}{M_1 + N_1}
\]

A very slight attention to the mode of formation of the functions
$M_1$ and $N_1$ will show that the process may be greatly simplified.
We may, indeed, exhibit the solution of the general
problem in the form of a rule, as follows:

\textit{Reject from the function $\phi(x_1,x_2\dotsc x_n)$ the constituent $\bar{x}_1 \dotsc \bar{x}_n$ if
it is therein contained, suppress in all the remaining constituents
the factors $\bar{x}_1$, $\bar{x}_2$, \&c., and change generally in the result $x_r$ into
$\frac{c_r p_r}{\mu - c_r p_r}$. Call this result $M_1$.}

\textit{Again, replace in the function $\phi(x_1,x_2\dotsc x_n)$ the constituent
$\bar{x}_1 \dotsc \bar{x}_n$ if is therein found, by unity; suppress in all the remaining
constituents the factors $\bar{x}_1$, $\bar{x}_2$, \&c., and change generally in the result
$x_r$ into $\frac{c_r (1-p_r)}{\nu - c_r (1-p_r)}$.}

\textit{Then the solution required will be expressed by the formula
\[\frac{M_1}{M_1 + N_1},\tag{14}\]
$\mu$ and $\nu$ being determined by the solution of the system of equations
\begin{multline*}
\mu + \nu - 1
= \frac{(\mu - c_1 p_1)\dotsc (\mu - c_n p_n)}{\mu^{n-1}}   \\
= \frac{ \{\nu - c_1 (1 - p_1)\} \dotsc \{\mu - c_n (1 - p_n)\} }
       {\nu^{n-1}}.
\tag{15}
\end{multline*}
}
It may be added, that the limits of $\mu$ and $\nu$ are the same as in
the previous problem. This might be inferred from the general
principle of continuity; but conditions of limitation, which are
%-----------------------File: 155.png----------------------------
probably sufficient, may also be established by other considerations.

Thus from the demonstration of the general method in probabilities, Chap. XVII. Prop,\textsc{iv.}, it appears that the quantities
$x_1, \dotsc x_n, t_1, \dotsc t_n$ in the primary system of algebraic equations,
must be \emph{positive proper fractions}. Now
\[
  \frac{x_r}{1-x_r} = n_r =  \frac{c_r(1-p_r)}{\nu - c_r(1-p_r)}.
\]
Hence generally $n_r$ must be a positive quantity, and therefore
we must have
\[
  \nu \stackrel{=}{>} c_r(1-p_r).
\]
In like manner since we have
\[
  \frac{x_r t_r}{(1-x_r)(1-t_r)}= m_r = \frac{c_r p_r}{\mu - c_r p_r},
\]
we must have generally
\[
  \mu \stackrel{=}{>} c_r p_r.
\]

16. It is probable that the two classes of conditions thus represented are together sufficient to determine generally which of
the roots of the equations determining $\mu$ and $\nu$ are to be taken.
Let us take in particular the case in which $n = 2$. Here we have
\begin{gather*}
  \mu + \nu - 1 = \frac{(\mu - c_1 p_1)(\mu - c_2\ p_2)}{\mu}
= \mu - (c_1 p_1 + c_2 p_2) + \frac{c_1 p_1 c_2 p_2}{\mu},
\\
  \therefore\nu = 1 - c_1 p_1 - c_2 p_2 +\frac{c_1 p_1\ c_2 p_2}{\mu}
= 1 - c_1 p_1 - \frac{(\mu - c_1 p_1) c_2 p_2}{\mu}.
\end{gather*}
Whence, since $\mu \stackrel{=}{>} c_1 p_1$ we have generally
\[
  \nu \stackrel{=}{<} 1 - c_1 p_1.
\]
In like manner we have
\[
  \nu \stackrel{=}{<} 1 - c_2 p_2,\quad
  \mu \stackrel{=}{<} 1 - c_1(1-p_1),\quad
  \mu \stackrel{=}{<} 1 - c_2(1-p_2).
\]

Now it has already been shown that there will exist but one
value of $\mu$ satisfying the whole of the above conditions relative
to that quantity, viz.
\[
  \mu \stackrel{=}{>} c_r p_r,\quad
  \mu \stackrel{=}{<} 1 - c_r(1-p_r),
\]
whence the solution for this case, at least, is determinate. And I
%-----------------------File: 156.png----------------------------
apprehend that the same method is generally applicable and
sufficient. But this is a question upon which a further degree of
light is desirable.

To verify the above results, suppose $\phi(x_1,\dotsc x_n) = 1$, which is
virtually the case considered in the previous problem. Now the
development of 1 gives all possible constituents of the symbols %uncertain character here, picture quality is too low.
$x_1,\dotsc x_n$. Proceeding then according to the Rule, we find
\begin{align*}
& M_1 = \frac{\mu^n}{(\mu-c_1 p_1)\dotsc (\mu-c_n p_n)} - 1
= \frac{\mu}{\mu + \nu - 1}-1 \text{ by (15)}. \\
& N_1
= \frac{\nu^n}{ \{\nu - c_1(1-p_1)\}\dotsc \{\nu - c_n(1-p_n)\} } - 1 = \frac{\nu}{\mu+\nu-1}-1.
\end{align*}
Substituting in (14) we find
\begin{equation*}
\operatorname{Prob. } z = 1-\nu,
\end{equation*}
which agrees with the previous solution.

Again, let $\phi(x_1, \dotsc x_n) = x_1$, which, after development and
suppression of the factors $\bar{x}_2, \dotsc \bar{x}_n$, gives
$x_1 (x_2 + 1)\dotsc (x_n+1)$, whence
we find
\begin{align*}
& M_1
= \frac{c_1 p_1 \mu^{n-1}}{(\mu-c_1 p_1)\dotsc (\mu-c_n p_n)}
= \frac{c_1 p_1}{\mu + \nu - 1} \text{ by (15)}. \\
& N_1
= \frac{c_1 (1-p_1)\nu^{n-1}}{ \{\nu - c_1(1-p_1)\} \dotsc \{\nu - c_n(1-p_n)\} }
= \frac{c_1(1-p_1)}{\mu+\nu-1}.
\end{align*}
Substituting, we have
\begin{equation*}
\text{Probability that if the event $A_1$, occur, $E$ will occur} = p_1.
\end{equation*}
And this result is verified by the data. Similar verifications
might easily be added.

Let us examine the case in which
\begin{equation*}
\phi(x_1,\dotsc x_n)
= x_1 \bar{x}_2 \dotsc \bar{x}_n
+ x_2 \bar{x}_1 \bar{x}_3 \dotsc \bar{x}_n \dotsc
+ x_n \bar{x}_1 \dotsc \bar{x}_{n-1}.
\end{equation*}
Here we find
\begin{gather*}
M_1 = \frac{c_1 p_1}{\mu - c_1 p_1} \dotsc
    + \frac{c_n p_n}{\mu - c_n p_n}, \\
N_1 = \frac{c_1 (1-p_1)}{\nu - c_1(1-p_1)} \dotsc
    + \frac{c_n (1-p_n)}{\nu - c_n(1-p_n)};
\end{gather*}
whence we have the following result---

%-----------------------File: 157.png----------------------------
\begin{equation*}
\left.\begin{minipage}{.4\linewidth}Probability that if some one
alone of the causes $A_1, A_2\dotsc A_n$
present itself, the event $E$
will follow.\end{minipage}\right\}
=\frac{\sum \displaystyle \frac{c_r p_r}{\mu - c_r p_r}}{\sum {\displaystyle \frac{c_r p_r}{\mu - c_r p_r}}+\sum \displaystyle \frac{c_r(1-p_r)}{\nu - c_r(1-p_r)}}
\end{equation*}

Let it be observed that this case is quite different from the
well-known one in which the mutually exclusive character of
the causes $A_1,\dotsc A_n$ is one of the elements of the data, expressing
a condition under which the very observations by which the
probabilities of $A_1, A_2,$ \&c. are supposed to have been determined,
were made.

Consider, lastly, the case in which $\phi(x_1,\dotsc x_n)=x_1 x_2 \dotsc x_n$.
Here
\begin{gather*}
M_1 = \frac{c_1 p_1 \dotsc c_n p_n}
           {(\mu - c_1 p_1)\dotsc (\mu-c_n p_n)}
=\frac{c_1 p_1 \dotsc c_n p_n}{\mu^{n-1}(\mu + \nu - 1)}, \\
N_1 = \frac{c_1(1-p_1)\dotsc c_n(1-p_n)}
           {\{\nu - c_1(1-p_1)\} \dotsc \{\nu-c_n(1-p_n)\}}
=\frac{c_1(1-p_1)\dotsc c_n(1-p_n)}{\nu^{n-1}(\mu + \nu -1)}.
\end{gather*}
Hence the following result---
\begin{equation*}
\left.\begin{minipage}{.4\linewidth}Probability that if all the
causes $A_1, \dotsc A_n$
conspire, the event $E$ will follow.
\end{minipage}\right\}
=\frac{p_1 \dotsc p_n \nu^{n-1}}
      {p_1 \dotsc p_n \nu^{n-1} + (1-p_1)\dotsc (1-p_n)\mu^{n-1}}.
\end{equation*}
This expression assumes, as it ought to do, the value 1 when any
one of the quantities $p_1, \dotsc p_n$ is equal to 1.

17. \textsc{Problem} VIII.---Certain causes $A_1, A_2,\dotsc A_n$ being so
restricted that they cannot all fail, but still can only occur in
certain definite combinations denoted by the equation
\begin{equation*}
\phi(A_1, A_2 \dotsc A_n)=1,
\end{equation*}
and there being given the separate probabilities $c_1,\dotsc c_n$ of the
said causes, and the corresponding probabilities $p_1, \dotsc p_n$ that an
event $E$ will follow if those respective causes are realized, required
the probability of the event $E$.

This problem differs from the one last considered in several
particulars, but chiefly in this, that the restriction denoted by the
equation $\phi(A_1, \dotsc A_n)=1$, forms one of the data, and is supposed
%-----------------------File: 158.png----------------------------
to be furnished by or to be accordant with the very experience
from which the knowledge of the numerical elements of the
problem is derived.

Representing the events $A_1,\dotsc A_n$ by $x_1,\dotsc x_n$ respectively,
and the event $E$ by $z$, we have---
\begin{equation}\tag{1}
\operatorname{Prob. } x_r=c_r, \quad
 \operatorname{Prob. } x_r z = c_r p_r.
\end{equation}
Let us assume, generally,
\begin{equation*}
x_r z = t_r,
\end{equation*}
then combining the system of equations thus indicated with the
equations
\begin{equation*}
\bar{x}_1 \dotsc \bar{x}_n = 0, \quad \phi(x_1, \dotsc x_n)=1, \quad \text{or } \phi=1,
\end{equation*}
furnished in the data, we ultimately find, as the developed
expression of $z$,
\begin{equation}\tag{2}
z= {\textstyle \sum}(XT) + 0 \bar{t}_1 \bar{t}_2 \dotsc
 \bar{t}_n {\textstyle \sum}(X),
\end{equation}
where $X$ represents in succession each constituent found in $\phi$,
and $T$ a similar series of constituents of the symbols
$t_1,\dotsc t_n$;
$\sum(XT)$ including only \emph{symmetrical} constituents with reference
to the two sets of symbols.

The method of reduction to be employed in the present case
is so similar to the one already exemplified in former problems,
that I shall merely exhibit the results to which it leads. We
find
\begin{equation}\tag{3}
\operatorname{Prob. } z = \frac{M}{M+N}
\end{equation}
with the relations
\begin{equation}\tag{4}
\frac{M_1}{c_1 p_1}\dotso = \frac{M_n}{c_n p_n}
=\frac{N_1}{c_1 (1-p_1)}=\frac{N_n}{c_n(1-p_n)}=M+N.
\end{equation}
Wherein $M$ is formed by suppressing in $\phi(x_2,\dotsc x_n)$ all the
factors $\bar{x}_1, \dotsc \bar{x}_n$, and changing in the result $x_1$ into $m_1$, $x_n$ into $m_n$,
while $N$ is formed by substituting in $M$, $n_1$ for $m_1$, \&c.; moreover
$M_1$ consists of that portion of $M$ of which $m_1$ is a factor,
$N_1$ of that portion of $N$ of which $n_1$ is a factor; and so on.

Let us take, in illustration, the particular case in which the
causes $A_1 \dotsc A_n$ are mutually exclusive. Here we have
\begin{equation*}
\phi(x_1, \dotsc x_n) = x_1 \bar{x}_2 \dotsc \bar{x}_n \dotsc
+ x_n \bar{x}_1 \dotsc \bar{x}_{n-1}.
\end{equation*}
%-----------------------File: 159.png----------------------------
Whence
\begin{align*}
M &= m_1 + m_2 \dotsc + m_n,\\
N &= n_1 + n_2 \dotsc + n_n,\\
M_1 &= m_1, N_1 = n_1, \text{ \&c.}
\end{align*}
Substituting, we have
\[
\frac{m_1}{c_1 p_1} \dotso = \frac{m_n}{c_n p_n}
= \frac{n_1}{c_1 (1-p_1)} \dotso
= \frac{n_n}{c_n (1-p_n)} = M + N.
\]
Hence we find
\[
\frac{m_1 + m_2 \dotsc + m_n}
     {c_1 p_1 + c_2 p_2 \dotsc + c_n p_n}
= M + N,
\]
or
\[
\frac{M}{c_1 p_1 \dotsc + c_n p_n} = M + N.
\]
Hence, by (3),
\[
\operatorname{Prob. } z = c_1 p_1 \dotsc + c_n p_n,
\]
a known result.

There are other particular cases in which the system (4) admits
of ready solution. It is, however, obvious that in most
instances it would lead to results of great complexity. Nor does
it seem probable that the existence of a functional relation among
causes, such as is assumed in the data of the general problem, will
often be presented in actual experience; if we except only the
particular cases above discussed.

Had the general problem been modified by the restriction
that the event $E$ cannot occur, all the causes $A_1 \dotsc A_n$ being absent,
instead of the restriction that the said causes cannot all fail,
the remaining condition denoted by the equation $\phi(A_1,\dotsc A_n) = 1$
being retained, we should have found for the final logical equation
\[
z = \sum_1(XT) + 0 \sum(X),
\]
$\sum(X)$ being, as before, equal to $\phi(x_1,\dotsc x_n)$, but $\sum_1(XT)$ formed
by rejecting from $\phi$ the particular constituent
$\bar{x_1}\dotsc\bar{x_n}$ if therein
contained, and then multiplying each $x$-constituent of the result
by the corresponding $t$-constituent. It is obvious that in the particular
case in which the causes are mutually exclusive the value
of Prob.~$z$ hence deduced will be the same as before.

18. \textsc{Problem} IX.---Assuming the data of any of the
%-----------------------File: 160.png----------------------------
previous problems, let it be required to determine the probability
that if the event $E$ present itself, it will be associated with the
particular cause $A_r$; in other words, to determine the
\textit{\`{a} posteriori}
probability of the cause $A_r$ when the event $E$ has been observed
to occur.

In this case we must seek the value of the fraction
\begin{equation}\tag{1}
\frac{\operatorname{Prob. } x_r z}{\operatorname{Prob. } z}, \text{ or }
\frac{c_r p_r}{\operatorname{Prob. } z}, \text{ by the data.}
\end{equation}
As in the previous problems, the value of $\operatorname{Prob. } z$ has been
assigned upon different hypotheses relative to the connexion or
want of connexion of the causes, it is evident that in all those
cases the present problem is susceptible of a determinate solution
by simply substituting in (1) the value of that element thus
determined.

If the \textit{\`{a} priori} probabilities of the causes are equal, we have
$c_1 = c_2 \dotso = c_r$. Hence for the different causes the value (1) will
vary directly as the quantity $p_r$. Wherefore \emph{whatever the nature
of the connexion among the causes}, the \textit{\`{a} posteriori} probability of
each cause will be proportional to the probability of the observed
event $E$ when that cause is known to exist. The particular case
of this theorem, which presents itself when the causes are
mutually exclusive, is well known. We have then
\begin{equation*}
\frac{\operatorname{Prob. } x_r z}{\operatorname{Prob. } z}=
\frac{c_r p_r}{{\textstyle \sum} c_r p_r}=\frac{p_r}{p_1 + p_2\dotsc + p_n},
\end{equation*}
the values of $c_1, \dotsc c_n$ being equal.

Although, for the demonstration of these and similar
theorems in the particular case in which the causes are mutually
exclusive, it is not necessary to introduce the functional symbol $\phi$,
which is, indeed, to claim for ourselves the choice of all possible
and conceivable hypotheses of the connexion of the causes, yet,
under every form, the solution by the method of this work of
problems, in which the number of the data is indefinitely great,
must always partake of a somewhat complex character. Whether
the systematic evolution which it presents, first, of the logical,
secondly, of the numerical relations of a problem, furnishes
any compensation for the length and occasional tediousness of its
%-----------------------File: 161.png----------------------------
processes, I do not presume to inquire. Its chief value undoubtedly
consists in its power,---in the mastery which it gives us over
questions which would apparently baffle the unassisted strength
of human reason. For this cause it has not been deemed superfluous
to exhibit in this chapter its application to problems, some
of which may possibly be regarded as repulsive, from their
difficulty, without being recommended by any prospect of immediate
utility. Of the ulterior value of such speculations it is, I
conceive, impossible for us, at present, to form any decided judgment.

19. The following problem is of a much easier description
than the previous ones.

\textsc{Problem} X.---\emph{The probability of the occurrence of a certain
natural ph\ae nomenon under given circumstances is $p$. Observation
has also recorded a probability $a$ of the existence of a permanent
cause of that ph\ae nomenon, i.e.\ of a cause which would always
produce the event under the circumstances supposed. What is the
probability that if the ph\ae nomenon is observed to occur $n$ times in
succession under the given circumstances, it will occur the ${n+1}^{\text{th}}$
time? What also is the probability, after such observation, of the
existence of the permanent cause referred to?}

\textsc{First Case}.---Let $t$ represent the existence of a permanent
cause, and $x_1, x_2\dotsc x_{n+1}$ the successive occurrences of the natural
ph\ae nomenon.

If the permanent cause exist, the events $x_1, x_2\dots x_{n+1}$ are
necessary consequences. Hence
\begin{equation*}
t=v x_1, \quad t=v x_2, \text{ \&c.},
\end{equation*}
and eliminating the indefinite symbols,
\begin{equation*}
t(1-x_1)=0, \quad t(1-x_2)=0, \quad t(1-x_{n+1})=0.
\end{equation*}
Now we are to seek the probability that if the combination
$x_1 x_2 \dotsc x_n$ happen, the event $x_{n+1}$ will happen, i.e.\ we are to seek
the value of the fraction
\begin{equation*}
\frac{\operatorname{Prob. } x_1 x_2 \dotsc x_{n+1}}
     {\operatorname{Prob. } x_1 x_2 \dotsc x_n    }.
\end{equation*}

We will first seek the value of
$\operatorname{Prob. } x_1 x_2 \dotsc x_n$.
%-----------------------File: 162.png----------------------------
Represent the combination $x_1\, x_2 \dotsc x_n$ by $w$, then we have the
following logical equations:
\begin{gather*}
  t(1-x_1) = 0,\quad t(1-x_2) = 0, \dotsc t(1-x_n) = 0,   \\
  x_1\, x_2 \dotsc x_n = w.
\end{gather*}
Reducing the last to the form
\[
  (x_1\, x_2 \dotsc x_n)(1-w) + w(1 - x_1\, x_2 \dotsc x_n) = 0,
\]
and adding it to the former ones, we have
\[
  \sum t(1-x_i) + x_1\, x_2 \dotsc x_n(1-w)
  + w(1- x_1\, x_2 \dotsc x_n) = 0, \tag{1}
\]
wherein $\sum$ extends to all values of $i$ from 1 to $n$, for the one logical equation of the data. With this we must connect the numerical conditions,
\[
  \text{Prob. }x_1 = \text{ Prob. }x_2 \dotso
  = \text{ Prob. }x_n = p,\quad \text{Prob. }t = a;
\]
and our object is to find Prob, $w$.

From (1) we have
\begin{gather*}
  w = \frac{\sum t(1-x_i) + x_1\, x_2\dotsc x_n}
           {2x_1\, x_2\dotsc x_n - 1}
\\
  = \frac{\sum (1-x_i) + x_1\,x_2\dotsc x_n}
         {2x_1\, x_2\dotsc x_n - 1} t
  + \frac{x_1\, x_2\dotsc x_n}
         {2x_1\, x_2\dotsc x_n - 1} (1-t),   \tag{2}
\end{gather*}
on developing with respect to $t$. This result must further be
developed with respect to $x_1, x_2,\dotsc x_n$.

Now if we make $x_1 = 1, x_2 = 1, \dotsc x_n = 1$, the coefficients both
of $t$ and of $1-t$ become 1. If we give to the same symbols any
other set of values formed by the interchange of 0 and 1, it is
evident that the coefficient of $t$ will become negative, while that
of $1-t$ will become 0. Hence the full development (2) will be
\begin{multline*}
  w = x_1 x_2 \dotsc x_n t + x_1 x_2 \dotsc x_n(1-t)
    + 0(1 - x_1 x_2 \dotsc x_n)(1-t)   \\
  + \text{ constituents whose coefficients are $\frac{1}{0}$, or equivalent to$\frac{1}{0}$.}
\end{multline*}
Here we have
\begin{multline*}
  V = x_1 x_2 \dotsc x_n t + x_1 x_2 \dotsc x_n (1-t)
    + (1 - x_1 x_2 \dotsc x_n) (1-t)   \\
  = x_1 x_2 \dotsc x_n t + 1 - t;
\end{multline*}
whence, passing from Logic to Algebra,
%-----------------------File: 163.png----------------------------
\begin{multline*}
    \frac{x_1 x_2 \dotsc x_n t + x_1(1-t)}{p}
  = \frac{x_1 x_2 \dotsc x_n t + x_2(1-t)}{p} \dotso
\\ \hfill
  = \frac{x_1 x_2 \dotsc x_n t + x_n(1-t)}{p}
  = \frac{x_1 x_2 \dotsc x_n t}{a} = x_1 x_2 \dotsc x_n t + 1 - t.
\\ \hfill
  \text{Prob. }w
= \frac{x_1 x_2 \dotsc x_n}{x_1 x_2 \dotsc x_n t + 1 - t}. \hfill
\end{multline*}

From the forms of the above equations it is evident that we
have $x_1 = x_2 \dotso = x_n$. Replace then each of these quantities by $x$,
and the system becomes
\begin{align*}
  \frac{x^n t + (1-t)x}{p} &= \frac{x^n t}{a} = x^n t + 1 - t,   \\
  \text{Prob. }w &= \frac{x^n}{x^n t + 1 - t};
\end{align*}
from which we readily deduce
\[
  \text{Prob. }w = \text{ Prob. }x_1 x_2 \dotsc x_n
  = a + (p-a)\left( \frac{p-a}{1-a} \right)^{n-1}
\]
If in this result we change $n$ into $n + 1$, we get
\[
  \text{Prob. }x_1 x_2 \dotsc x_{n+1}
  = a + (p-a)\left( \frac{p-a}{1-a} \right)^n
\]
Hence we find---
\[
  \frac{\text{Prob. }x_1 x_2 \dotsc x_{n+1}}
       {\text{Prob. }x_1 x_2 \dotsc x_n}
  =
  \frac{a + (p-a)\left( \frac{p-a}{1-a} \right)^n}
       {a + (p-a)\left( \frac{p-a}{1-a} \right)^{n-1}}
\tag{3}
\]
as the expression of the probability that if the ph{\ae}nomenon be $n$
times repeated, it will also present itself the $n + 1^{th}$ time. By the
method of Chapter~XIX. it is found that $a$ cannot exceed $p$ in
value.

The following verifications are obvious:---

1st. If $a =0$, the expression reduces to $p$, as it ought to do.
For when it is certain that no permanent cause exists, the successive occurrences of the ph{\ae}nomenon are independent.

2nd. If $p = 1$, the expression becomes 1, as it ought to do.

3rd. If $p = a$, the expression becomes 1, unless $a = 0$. If the
probability of a ph{\ae}nomenon is equal to the probability that there
%-----------------------File: 164.png----------------------------
exists a cause which under given circumstances would always
produce it, then the fact that that ph\ae nomenon has ever been noticed under those circumstances, renders certain its re-appearance
under the same.\footnote{As we can neither re-enter nor recall the state of infancy, we are unable to
say how far such results as the above serve to explain the confidence with which
young children connect events whose association they have once perceived.
But we may conjecture, generally, that the strength of their expectations is
due to the necessity of inferring (as a part of their rational nature), and the
narrow but impressive experience upon which the faculty is exercised.  Hence
the reference of every kind of sequence to that of cause and effect. A little
friend of the author's, on being put to bed, was heard to ask his brother the
pertinent question,---"Why does going to sleep at night make it light in the
morning?"  The brother, who was a year older, was able to reply, that it
would be light in the morning even if little boys did not go to sleep at night.}%endfootnote

4th.  As $n$ increases, the expression approaches in value to
unity.  This indicates that the probability of the recurrence of
the event increases with the frequency of its successive appearances,---a result agreeable to the natural laws of expectation.

\textsc{Second Case}.---We are now to seek the probability \textit{\`{a} posteriori} of the existence of a permanent cause of the
ph\ae nomenon.
This requires that we ascertain the value of the fraction
\[
  \frac{\operatorname{Prob. } t x_1 x_2 \dotsc x_n}
       {\operatorname{Prob. }   x_1 x_2 \dotsc x_n}
\]
the denominator of which has already been determined.

To determine the numerator assume
\[
  t x_1 x_2 \dotsc x_n = w,
\]
then proceeding as before, we obtain for the logical development,
\[
  w = t x_1 x_2 \dotsc x_n + 0 (1-t).
\]
Whence, passing from Logic to Algebra, we have at once
\[
  \operatorname{Prob. } w = a,
\]
a result which might have been anticipated.  Substituting then
for the numerator and denominator of the above fraction their
values, we have for the \textit{\`{a} posteriori} probability of a permanent
cause, the expression

%-----------------------File: 165.png----------------------------
\begin{equation*}
\frac{a}{a+(p-a)\left({\displaystyle \frac{p-a}{1-a}}\right)^{n-1}}.
\end{equation*}
It is obvious that the value of this expression increases with the
value of $n$.

I am indebted to a learned correspondent,\footnote{Professor Donkin.} whose original
contributions to the theory of probabilities have already been referred
to, for the following verification of the first of the above
results (3).

``The whole \textit{\`{a} priori} probability of the event (under the
circumstances) being $p$, and the probability of some cause $C$ which
would necessarily produce it, $a$, let $x$ be the probability that it
will happen if no such cause as $C$ exist. Then we have the
equation
\begin{equation*}
p = a + (1-a) x,
\end{equation*}
whence
\begin{equation*}
x=\frac{p-a}{1-a}.
\end{equation*}
Now the ph\ae nomenon observed is the occurrence of the event $n$
times. The \textit{\`{a} priori} probability of this would be---
\begin{equation*}\begin{split}
& 1 \text{ supposing $C$ to exist,} \\
& x^n \text{ supposing $C$ not to exist;}
\end{split}\end{equation*}
whence the \textit{\`{a} posteriori} probability that $C$ exists is
\begin{equation*}
\frac{a}{a+(1-a)x^n},
\end{equation*}
that $C$ does not exist is
\begin{equation*}
\frac{(1-a)x^n}{a+(1-a)x^n}.
\end{equation*}
Consequently the probability of another occurrence is
\begin{equation*}
\frac{a}{a+(1-a)x^n}\times 1 + \frac{(1-a)x^n}{a+(1-a)x^n}\times a,
\end{equation*}
or
\begin{equation*}
\frac{a+(1-a)x^{n+1}}{a+(1-a)x^n},
\end{equation*}
%-----------------------File: 166.png----------------------------
which, on replacing $n$ by its value $\displaystyle \frac{p-a}{1-a}$, will be found to agree
with (3)."

Similar verifications might, it is probable, also be found for
the following results, obtained by the direct application of the
general method.

The probability, under the same circumstances, that if, out of
$n$ occasions, the event happen $r$ times, and fail $n-r$ times, it will
happen on the ${n + 1}^{\mathit{th}}$ time is
\begin{equation*}
\frac{a+m(p-a )\left({\displaystyle \frac{p-la}{1-a}}\right)^r}
     {a+m(p-la)\left({\displaystyle \frac{p-la}{1-a}}\right)^{r-1}}
\end{equation*}
%**note: LaTeX will not easily left justify arguments of a fraction
%--at least I don't know how
%**[2nd proofer: How about throwing a \phantom{-1} into the exponent
%   of the numerator?]
wherein $m=\displaystyle \frac{n(n-1)\dotsc n-r+1}{1 \centerdot 2 \dotsc r}$ and $\displaystyle l=\frac{r}{n}$.

The probability of a permanent cause ($r$ being less than $n$)
is 0. This is easily verified.

If $p$ be the probability of an event, and $c$ the probability that
if it occur it will be due to a permanent cause; the probability
after $n$ successive observed occurrences that it will recur on the
${n + 1}^{\mathit{th}}$ similar occasion is
\begin{equation*}
\frac{c+(1-c)x^n}{c+(1-c)x^{n-1}},
\end{equation*}
%**[As above.]
wherein $\displaystyle x = \frac{p(1-c)}{1-cp}$.

20. It is remarkable that the solutions of the previous problems
are void of any arbitrary element. We should scarcely,
from the appearance of the data, have anticipated such a
circumstance. It is, however, to be observed, that in all those problems
the probabilities of the \emph{causes} involved are supposed to be known
\textit{\`{a} priori}. In the absence of this assumed element of knowledge,
it seems probable that arbitrary constants would \emph{necessarily} appear
in the final solution. Some confirmation of this remark is
afforded by a class of problems to which considerable attention
has been directed, and which, in conclusion, I shall briefly
consider.
%-----------------------File: 167.png----------------------------
It has been observed that there exists in the heavens a large
number of double stars of extreme closeness. Either these apparent
instances of connexion have some physical ground or they
have not. If they have not, we may regard the phenomenon of a
double star as the accidental result of a ``random distribution" of
stars over the celestial vault, i.e.\ of a distribution which would
render it just as probable that either member of the binary system
should appear in one spot as in another. If this hypothesis be
assumed, and if the number of stars of a requisite brightness be
known, we can determine what is the probability that two of
them should be found within such limits of mutual distance as
to constitute the observed phenomenon. Thus Mitchell,\footnote{Phil. Transactions, An. 1767.}
estimating that there are 230 stars in the heavens equal in brightness
to $\beta$ Capricorni, determines that it is 80 to 1 against such a
combination being presented were those stars distributed at
random. The probability, when such a combination has been
observed, that there exists between its members a physical ground
of connexion, is then required.

Again, the sum of the inclinations of the orbits of the ten
known planets to the plane of the ecliptic in the year 1801 was
$91^{\circ}\cdot$4187, according to the French measures. Were all
inclinations equally probable, Laplace\footnote{Th\'{e}orie Analytique
des Probabilit\'{e}s, p. 276.} determines, that there would be
only the excessively small probability .00000011235 that the
mean of the inclinations should fall within the limit thus
assigned. And he hence concludes, that there is a very high
probability in favour of a disposing cause, by which the inclinations
of the planetary orbits have been confined within such narrow
bounds. Professor De Morgan,\footnote{Encyclopedia Metropolitana. Art.\ Probabilities.} taking the sum of the inclinations
at $92^{\circ}$, gives to the above probability the value .00000012,
and infers that ``it is $1\colon .00000012$, that there was a necessary
cause in the formation of the solar system for the inclinations
being what they are." An equally determinate conclusion has
been drawn from observed coincidences between the direction of
%-----------------------File: 168.png----------------------------
circular polarization in rock-crystal, and that of certain oblique
faces in its crystalline structure.\footnote{Edinburgh Review, No.~185, p.~32.  This
article, though not entirely free from error, is well worthy of attention.}

These problems are all of a similar character. A certain hypothesis
is framed, of the various possible consequences of which
we are able to assign the probabilities with perfect rigour. Now
some actual result of observation being found among those
consequences, and its \emph{hypothetical} probability being therefore known,
it is required thence to determine the probability of the hypothesis
assumed, or its contrary. In Mitchell's problem, the hypothesis
is that of a ``random distribution of the stars,"---the
possible and observed consequence, the appearance of a close
double star. The very small probability of such a result is held
to imply that the probability of the hypothesis is equally small,
or, at least, of the same order of smallness. And hence the high
and, and as some think, \emph{determinate} probability of a disposing
cause in the stellar arrangements is inferred. Similar remarks
apply to the other examples adduced.

21. The general problem, in whatsoever form it may be presented,
admits only of an \emph{indefinite} solution. Let $x$ represent the
proposed hypothesis, $y$ a ph\ae nomenon which might occur as one
of its possible consequences, and whose calculated probability, on
the assumption of the truth of the hypothesis, is $p$, and let it be
required to determine the probability that if the ph\ae nomenon $y$ is
observed, the hypothesis $x$ is true. The very data of this problem
cannot be expressed without the introduction of an arbitrary
element. We can only write
\begin{equation}\tag{1}
\operatorname{Prob. } x = a, \quad \operatorname{Prob. } xy = ap;
\end{equation}
$a$ being perfectly arbitrary, except that it must fall within the
limits 0 and 1 inclusive. If then $P$ represent the conditional
probability sought, we have
\begin{equation}\tag{2}
P=\frac{\operatorname{Prob. } xy}{\operatorname{Prob. } y} =\frac{ap}{\operatorname{Prob. } y}.
\end{equation}
It remains then to determine $\operatorname{Prob. } y$.
%-----------------------File: 169.png----------------------------
Let $xy = t$, then
\begin{equation}\tag{3}
y = \frac{t}{x}=tx + \frac{1}{0}t(1-x)+0(1-t)x+\frac{0}{0}(1-t)(1-x).
\end{equation}
Hence observing that $\operatorname{Prob. } x = a$, $\operatorname{Prob. } t = ap$, and passing from
Logic to Algebra, we have
\begin{equation*}
\operatorname{Prob. } y = \frac{tx + c(1-t)x}{tx+1-t},
\end{equation*}
with the relations
\begin{equation*}
\frac{tx + (1-t)x}{a}=\frac{tx}{ap}=tx+1-t.
\end{equation*}

Hence we readily find
\begin{equation}\tag{4}
\operatorname{Prob. } y = ap + c (1-a).
\end{equation}

Now recurring to (3), we find that $c$ is the probability, that if
the event $(1-t) (1-x)$ occur, the event $y$ will occur. But
\begin{equation*}
(1-t)(1-x)=(1-xy)(1-x)=1-x.
\end{equation*}

Hence $c$ is the probability \emph{that if the event $x$ do not occur,
the event $y$ will occur.}

Substituting the value of $\operatorname{Prob. } y$ in (2), we have the
following theorem:

\emph{The calculated probability of any ph\ae nomenon $y$, upon an
assumed physical hypothesis $x$, being $p$, the \textit{\`{a} posteriori} probability $P$
of the physical hypothesis, when the ph\ae nomenon has been observed,
is expressed by the equation}
\begin{equation}\tag{5}
P=\frac{ap}{ap+c(1-a)},
\end{equation}
\emph{where $a$ and $c$ are arbitrary constants, the former representing the
\textit{\`{a} priori} probability of the hypothesis, the latter the probability that
if the hypothesis were false, the event $y$ would present itself.}

The principal conclusion deducible from the above theorem
is that, other things being the same, the value of $P$ increases and
diminishes simultaneously with that of $p$. Hence the greater or
less the probability of the ph\ae nomenon when the \emph{hypothesis} is
\emph{assumed}, the greater or less is the probability of the hypothesis
when the \emph{ph\ae nomenon} has been \emph{observed}. When $p$ is very small,
then generally $P$ also is small, unless either $a$ is large or $c$ small.
%-----------------------File: 170.png----------------------------
Hence, secondly, if the probability of the ph{\ae}nomenon is very
small when the hypothesis is assumed, the probability of the hypothesis is very small when the ph{\ae}nomenon is observed, unless
either the \textit{\`{a} priori} probability $a$ of the hypothesis is large, or the
probability of the ph{\ae}nomenon upon any other hypothesis small.

The formula (5) admits of exact verification in various cases,
as when $c = 0$, or $a = 1$, or $a = 0$. But it is evident that it does
not, unless there be means for determining the values of $a$ and $c$,
yield a \emph{definite} value of $P$. Any solutions which profess to accomplish this object, either are erroneous in principle, or involve
a tacit assumption respecting the above arbitrary elements. Mr.\
De~Morgan's solution of Laplace's problem concerning the existence of a determining cause of the narrow limits within which
the inclinations of the planetary orbits to the plane of the ecliptic
are confined, appears to me to be of the latter description. Having
found a probability $p =.00000012$, that the sum of the inclinations would
be less than $92^\circ$ were all degrees of inclination
equally probable in each orbit, this able writer remarks: \lq\lq If
there be a reason for the inclinations being as described, the
probability of the event is $1$. Consequently, it is
$1 \colon .00000012$
(i.e. $1 \colon p$), that there was a necessary cause in the formation of
the solar system for the inclinations being what they are.\rq\rq\ Now
this result is what the equation~(5) would really give, if, assigning
to $p$ the above value, we should assume $c = 1$, $a =\frac{1}{2}$. For we
should thus find,
\begin{gather*}
  P = \frac{ \frac{1}{2}p }{ \frac{1}{2}p + \frac{1}{2} } = \frac{p}{1+p}   \\
  \therefore 1 - P \colon P \colon\colon 1 \colon p.    \tag{6}
\end{gather*}

But $P$ representing the probability, \textit{\`{a} posteriori}, that all
inclinations are equally probable, $1 - P$ is the probability, \textit{\`{a} posteriori},
that such is not the case, or, adopting Mr.\ De~Morgan's
alternative, that a determining cause exists. The equation~(6),
therefore, agrees with Mr.\ De~Morgan's result.

22. Are we, however, justified in assigning to $a$ and $c$ particular values?
I am strongly disposed to think that we are not.
%-----------------------File: 171.png----------------------------
The question is of less importance in the special instance than
in its ulterior bearings. In the received applications of the theory
of probabilities, arbitrary constants do not explicitly appear;
but in the above, and in many other instances sanctioned by the
highest authorities, some virtual determination of them has been
attempted. And this circumstance has given to the results of
the theory, especially in reference to questions of causation, a
character of definite precision, which, while on the one hand it
has seemed to exalt the dominion and extend the province of
numbers, even beyond the measure of their ancient claim to rule
the world;\footnote{Mundum regunt numeri.} on the other hand has called forth vigorous protests
against their intrusion into realms in which conjecture is the only
basis of inference. The very fact of the appearance of arbitrary
constants in the solutions of problems like the above, treated
by the method of this work, seems to imply, that definite solution
is impossible, and to mark the point where inquiry ought to stop.
We possess indeed the means of interpreting those constants, but
the experience which is thus indicated is as much beyond our
reach as the experience which would preclude the necessity of
any attempt at solution whatever.

Another difficulty attendant upon these questions, and inherent, perhaps,
in the very constitution of our faculties, is that of
precisely defining what is meant by Order. The manifestations
of that principle, except in very complex instances, we have no
difficulty in detecting, nor do we hesitate to impute to it an almost
necessary foundation in causes operating under Law. But
to assign to it a standard of \emph{numerical} value would be a vain,
not to say a presumptuous, endeavour. Yet must the attempt be
made, before we can aspire to weigh with accuracy the
probabilities%**[Misspelling corrected.]
\footnote{Original text was \lq\lq probabibilities\rq\rq\ and
was fixed in 2004 by Distributed Proofreaders.}
\footnote{The following footnote was in the original text but was
not referenced in the text, so it is referenced here\footnotemark\ in
2004 by Distributed Proofreaders.}
%**[Where's this footnote supposed to go?
%   I can't find the dagger on this page.]
\footnotetext{See an interesting paper by Prof. Forbes in the Philosophical Magazine,
Dec. 1850; also Mill's Logic, chap, xviii.} %
of different constitutions of the universe, so as to determine the
elements upon which alone a definite solution of the
problems in question can be established.

23. The most usual mode of endeavouring to evade the \emph{necessary}
arbitrariness of the solution of problems in the theory of
%-----------------------File: 172.png----------------------------
probabilities which rest upon insufficient data, is to assign to some
element whose real probability is unknown all possible degrees
of probability; to suppose that these degrees of probability are
themselves equally probable; and, regarding them as so many distinct causes of the phenomenon observed, to apply the theorems
which represent the case of an effect due to some one of a number
of equally probable but mutually exclusive causes (Problem~9).
For instance, the rising of the sun after a certain interval of
darkness having been observed $m$ times in succession, the probability of its again rising under the same circumstances is determined, on received principles, in the following manner. Let $p$
be any unknown probability between 0 and 1, and $c$ (infinitesimal
and constant) the probability, that the probability of the sun's
rising after an interval of darkness lies between the limits $p$ and
$p + dp$. Then the probability that the sun will rise $m$ times in
succession is
\[
  c\int_0^1 p^m dp;
\]
and the probability that he will do this, and will rise again, or,
which is the same thing, that he will rise $m + 1$ times in succession, is
\[
  c\int_0^1 p^{m+1} dp,
\]
Hence the probability that if he rise $m$ times in succession, he will
rise the ${m + 1}^{th}$ time, is
\[
  \frac{c\int_0^1 p^{m+1} dp}{c\int_0^1 p^m dp} = \frac{m+1}{m+2},
\]
the known and generally received solution.

The above solution is usually founded upon a supposed analogy
of the problem with that of the drawing of balls from an urn containing a mixture of black and white balls, between which all
possible numerical ratios are assumed to be equally probable.
And it is remarkable, that there are two or three distinct hypotheses which lead to the same final result. For instance, if the
balls are finite in number, and those which arc drawn are not
%-----------------------File: 173.png----------------------------
replaced, or if they are infinite in number, whether those drawn
are replaced or not, then, supposing that $m$ successive drawings
have yielded only white balls, the probability of the issue of a
white ball at the ${m + 1}^{th}$ drawing is
\[
  \frac{m + 1}{m + 2}.%
\footnote{See a memoir by Bishop Terrot, Edinburgh Phil. Trans. vol. xx. Part iv.}
\]

It has been said, that the principle involved in the above
and in similar applications is that of the equal distribution of
our knowledge, or rather of our ignorance---the assigning to
different states of things of which we know nothing, and upon
the very ground that we know nothing, equal degrees of probability. I apprehend, however, that this is an arbitrary method of
procedure. Instances may occur, and one such has been adduced,
in which different hypotheses lead to the same final conclusion.
But those instances are exceptional. With reference to the particular problem in question, it is shown in the memoir cited, that
there is one hypothesis, viz., when the balls are finite in number
and not replaced, which leads to a different conclusion, and it is
easy to see that there are other hypotheses, as strictly involving
the principle of the \lq\lq equal distribution of knowledge or ignorance,\rq\rq\ which would also conduct to conflicting results.

24. For instance, let the case of sunrise be represented by
the drawing of a white ball from a bag containing an infinite
number of balls, which are all either black or white, and let the
assumed principle be, that \emph{all possible constitutions of the system
of balls are equally probable}. By a constitution of the system, I
mean an arrangement which assigns to every ball in the system
a determinate colour, either black or white. Let us thence seek
the probability, that if $m$ white balls are drawn in $m$ drawings,
a white ball will be drawn in the ${m + 1}^{th}$ drawing.

First, suppose the number of the balls to be $\mu$, and let the
symbols $x_1, x_2, \dotsc x_\mu$ be appropriated to them in the following
manner. Let $x_i$ denote that event which consists in the $i^{th}$ ball
of the system being white, the proposition declaratory of such a
state of things being $x_i = 1$. In like manner the compound
%-----------------------File: 174.png----------------------------
symbol $1-x_i$ will represent the circumstance of the $i^{th}$ ball being
black. It is evident that the several constituents formed of the
entire set of symbols $x_1, x_2,\dotsc x_\mu$ will represent in like manner
the several possible constitutions of the system of balls with
respect to blackness and whiteness, and the number of such constitutions being $2^\mu$, the probability of each will, in accordance
with the hypothesis, be $\dfrac{1}{2^\mu}$. This is the value which we should
find if we substituted in the expression of any constituent for

each of the symbols $x_1, x_2,\dotsc x_\mu$, the value $\dfrac{1}{2}$. Hence, then, the
probability of any event which can be expressed as a series of
constituents of the above description, will be found by substituting in such expression the value $\dfrac{1}{2}$ for each of the above
symbols.

Now the larger $\mu$ is, the less probable it is that any ball
which has been drawn and replaced will be drawn again. As $\mu$.
approaches to infinity, this probability approaches to $0$. And
this being the case, the state of the balls actually drawn can be
expressed as a logical function of $m$ of the symbols
$x_1, x_2,\dotsc x_\mu$,
and therefore, by development, as a series of constituents of the
said $m$ symbols. Hence, therefore, its probability will be found
by substituting for each of the symbols, whether in the undeveloped or the developed form, the value $\dfrac{1}{2}$. But this is the very
substitution which it would be necessary, and which it would
suffice, to make, if the probability of a white ball at each drawing
were known, \textit{\`{a} priori}, to be $\dfrac{1}{2}$.

It follows, therefore, that if the number of balls be infinite,
and all constitutions of the system equally probable, the probability of drawing $m$ white balls in succession will be $\dfrac{1}{2^m}$, and the
probability of drawing $m + 1$ white balls in succession $\dfrac{1}{2^{m+1}}$;

whence the probability that after m white balls have been drawn,
the next drawing will furnish a white one, will be $\dfrac{1}{2}$. In other
%-----------------------File: 175.png----------------------------
words, past experience does not in this case affect future expectation.

25. It may be satisfactory to verify this result by ordinary
methods. To accomplish this, we shall seek---

First: The probability of drawing $r$ white balls, and $p-r$
black balls, in $p$ trials, out of a bag containing $\mu$ balls, every ball
being replaced after drawing, and all constitutions of the systems
being equally probable, \textit{\`{a} priori}.

Secondly: The value which this probability assumes when
$\mu$ becomes infinite.

Thirdly: The probability hence derived, that if $m$ white
balls are drawn in succession, the ${m + 1}^{th}$ ball drawn will be
white also.

The probability that $r$ white balls and $p-r$ black ones will be
drawn in $p$ trials out of an urn containing $\mu$ balls, each ball
being replaced after trial, and all constitutions of the system as
above defined being equally probable, is equal to the sum of the
probabilities of the same result upon the separate hypotheses of
there being no white balls, 1 white ball,---lastly $\mu$ white balls in
the urn. Therefore, it is the sum of the probabilities of this result on the hypothesis of there being $n$ white balls, $n$ varying
from 0 to $\mu$.

Now supposing that there are $n$ white balls, the probability
of drawing a white ball in a single drawing is $\frac{n}{\mu}$, and the probability of drawing $r$ white balls and $p-r$ black ones in a particular order in $p$ drawings, is
\[
  \left( \frac{n}{\mu} \right)^r
  \left( 1 - \frac{n}{\mu} \right)^{p-r}
\]
But there being as many such orders as there are combinations
of $r$ things in $p$ things, the total probability of drawing $r$ white
balls in $p$ drawings out of the system of $\mu$ balls of which $n$ are
white, is
\[
  \frac{ p(p-1) \dotsc (p-r+1) }{ 1 \cdot 2 \dotsc r }
  \left( \frac{n}{\mu} \right)^r
  \left( 1 - \frac{n}{\mu} \right)^{p-r} \tag{1}
\]
Again, the number of constitutions of the system of $\mu$ balls, which
admit of exactly $n$ balls being white, is
%-----------------------File: 176.png----------------------------
\[
  \frac{\mu(\mu-1) \dotsc (\mu-n+1)}{1 \centerdot 2 \dotsc n},
\]
and the number of possible constitutions of the system is $2^\mu$.
Hence the probability that exactly $n$ balls are white is
\[
  \frac{\mu(\mu-1) \dotsc (\mu-n+1)}{1 \centerdot 2 \dotsc n 2^\mu},
\]
Multiplying (1) by this expression, and taking the sum of the
products from $n = 0$ to $n = \mu$, we have
\[
  \frac{p(p-1) \dotsc p-r+1}{1 \centerdot 2 \dotsc r}
  \sum_{n=0}^{n=\mu}
  \frac{\mu(\mu-1) \dotsc (\mu-n+1)}{1 \centerdot 2 \dotsc n 2^\mu}
  \left( \frac{n}{\mu} \right)^r
  \left( 1-\frac{n}{\mu} \right)^{p-r} ,   \tag{2}
\]
for the expression of the total probability, that out of a system
of $\mu$ balls of which all constitutions are equally probable, $r$ white
balls will issue in $p$ drawings. Now
\begin{gather*}
  \sum_{n=0}^{n=\mu}
  \frac{\mu(\mu-1) \dotsc (\mu-n+1)}
       {1 \centerdot 2 \dotsc n \centerdot 2^\mu}
  \left( \frac{n}{\mu} \right)^r
  \left( 1-\frac{n}{\mu} \right)^{p-r}
\\
= \sum_{n=0}^{n=\mu}
  \frac{\mu(\mu-1) \dotsc (\mu-n+1)}
       {1 \centerdot 2 \dotsc n 2^\mu}
  \left( \frac{n}{\mu} \right)^r
  \left( 1-\frac{n}{\mu} \right)^{p-r}
  \varepsilon^{n\theta} \dotso (\theta = 0)
\\
= \frac{1}{2^\mu}
  \left( \frac{D}{\mu} \right)^r
  \left( 1-\frac{D}{\mu} \right)^{p-r}
  \sum_{n=0}^{n=\mu}
  \frac{\mu(\mu-1) \dotsc (\mu-n+1)}
       {1 \centerdot 2 \dotsc n}
  \varepsilon^{n\theta}
\\
= \frac{1}{2^\mu}
  \left( \frac{D}{\mu} \right)^r
  \left( 1-\frac{D}{\mu} \right)^{p-r}
  (1 + \varepsilon^\theta)^\mu,        \tag{3}
\end{gather*}
$D$ standing for the symbol $\frac{d}{d\theta}$, so that
$\phi(D) \varepsilon_{n\theta} = \phi(n) \varepsilon^{n\theta}$.
But by a known theorem,
\begin{gather*}
  t^m = 1 + \Delta 0^m t + \frac{\Delta^2 0^m}{1\centerdot 2} t(t-1)
      + \frac{\Delta^3 0^m}{1\centerdot 2\centerdot 3} t(t-1)(t-2).%*
\\
  \therefore D^m (1+\varepsilon^\theta)^\mu
= \{1 + \Delta 0^mD + \frac{\Delta^2 0^m}{1\centerdot 2} D(D-1)
    + \text{ \&c.}\} (1+\varepsilon^\theta)^\mu.                   %*
\end{gather*}
In the second member let $\varepsilon^\theta = x$, then
\[
  D^m (1+\varepsilon^\theta)^\mu
= (1 + \Delta 0^m x\frac{d}{dx}
    + \frac{\Delta^2 0^m}{1\centerdot 2} x^2 \frac{d^2}{dx^2}
    + \text{ \&c.}) (1+x)^\mu,                                     %*
\]
since
\[
  D(D-1) \dotsc (D-i+1) = x^i \left( \frac{d}{dx} \right)^i.
\]

%**[These corrections seem to have already been applied.]
%**[I'm amazed, and unsure to what 3,5 and 6 ever refer to.]
%
%* ERRATA. --- 3,5, and 6 from bottom, \textit{for $1$ read $0^m$}.
%-----------------------File: 177.png----------------------------
In the second member of the above equation, performing the differentiations
and making $x=1$ (since $\theta = 0$), we get
\[
  D^m(1 + \varepsilon^\theta)^\mu
= \mu(\Delta 0^m)2^{\mu-1}
  + \frac{\mu(\mu-1)}{1\centerdot 2}(\Delta^2 0^m)2^{\mu-2}
  + \text{ \&c.}
\]
The last term of the second member of this equation will be
\[
  \frac{\mu(\mu-1)\dotsc (\mu-m+1)\Delta^m 0^m}
       {1\centerdot 2\dotsc m}
  2^{\mu-m}
= \mu(\mu-1)\dotsc (\mu-m+1)2^{\mu-m};
\]
since $\Delta^m 0^m = 1\centerdot 2\dotsc m$. When $\mu$ is a large quantity this term
exceeds all the others in value, and as $\mu$ approaches to infinity
tends to become infinitely great in comparison with them. And
as moreover it assumes the form $\mu^m 2^{\mu-m}$, we have, on passing to
the limit,
\[
  D^m(1 + \varepsilon^\theta)^\mu = \mu^m 2^{\mu-m}
= \left( \frac{\mu}{2} \right)^m 2^\mu.
\]
Hence if $\phi(D)$ represent any function of the symbol $D$, which
is capable of being expanded in a series of ascending powers of $D$,
we have
\[
  \phi(D)(1 + \varepsilon^\theta)^\mu = \phi\left(\frac{\mu}{2}\right)2^\mu ,  \tag{4}
\]
if $\theta = 0$ and $\mu = \infty$. Strictly speaking, this implies that the ratio of
the two members of the above equation approaches a state of
equality, as $\mu$ increases towards infinity, $\theta$ being equal to $0$.

By means of this theorem, the last member of (3) reduces to
the form
\[
  \frac{1}{2^\mu} \left(\frac{1}{2}\right)^r
  \left(1 - \frac{1}{2}\right)^{p-r} 2^\mu
= \left(\frac{1}{2}\right)^p .
\]
Hence (2) gives
\[
  \frac{p(p-1)\dotsc (p-r+1)}
       {1\centerdot 2\dotsc r}
  \left( \frac{1}{2} \right)^p ,
\]
as the expression for the probability that from an urn containing
an infinite number of black and white balls, all constitutions of
the system being equally probable, $r$ white balls will issue in $p$
drawings.

Hence, making $p = m$, $r = m$, the probability that in $m$ drawings
all the balls will be white is $\left( \frac{1}{2} \right)^m$, and the probability that this
%-----------------------File: 178.png----------------------------
will be the case, and that moreover the ${m + 1}^{th}$ drawing will
yield a white ball is $\left(\dfrac{1}{2}\right)^{m+1}$, whence the probability, that if the
first $m$ drawings yield white balls only, the ${m + 1}^{th}$ drawing will
also yield a white ball, is
\[
  \left( \frac{1}{2} \right)^{m+1} \div
  \left( \frac{1}{2} \right)^m  = \frac{1}{2};
\]
and generally, any proposed result will have the same probability
as if it were an even chance whether each particular drawing
yielded a white or a black ball. This agrees with the conclusion
before obtained.

26. These results only illustrate the fact, that when the defect
of data is supplied by hypothesis, the solutions will, in general,
vary with the nature of the hypotheses assumed; so that the
question still remains, only more definite in form, whether the
principles of the theory of probabilities serve to guide us in the
election of such hypotheses. I have already expressed my conviction that they do not---a conviction strengthened by other reasons
than those above stated. Thus, a definite solution of a problem
having been found by the method of this work, an equally definite solution is sometimes attainable by the same method when
one of the data, suppose Prob. $x =p_1$ is omitted. But I have not
been able to discover any mode of deducing the second solution
from the first by \emph{integration}, with respect to $p$ supposed variable
within limits determined by Chap.~\textsc{xix}. This deduction would,
however, I conceive, be possible, were the principle adverted to
in Art.~23 valid. Still it is with diffidence that I express my
dissent on these points from mathematicians generally, and more
especially from one who, of English writers, has most fully entered into the spirit and the methods of Laplace; and I venture
to hope, that a question, second to none other in the Theory of
Probabilities in importance, will receive the careful attention
which it deserves.

%-----------------------File: 179.png----------------------------



%CHAPTER XXI.

\chapter[PROBABILITY OF JUDGEMENTS]{\large PARTICULAR APPLICATION OF THE PREVIOUS GENERAL METHOD
TO THE QUESTION OF THE PROBABILITY OF JUDGMENTS.}

1. On the presumption that the general method of this treatise
for the solution of questions in the theory of probabilities,
has been sufficiently elucidated in the previous chapters, it is proposed here to enter upon one of its practical applications selected
out of the wide field of social statistics, viz., the estimation of the
probability of judgments. Perhaps this application, if weighed
by its immediate results, is not the best that could have been
chosen. One of the first conclusions to which it leads is that of
the \emph{necessary} insufficiency of any data that experience alone can
furnish, for the accomplishment of the most important object of
the inquiry. But in setting clearly before us the \emph{necessity} of
hypotheses as supplementary to the data of experience, and in
enabling us to deduce with rigour the consequences of \emph{any} hypothesis which may be assumed, the method accomplishes all
that properly lies within its scope. And it may be remarked,
that in questions which relate to the conduct of our own species,
hypotheses are more justifiable than in questions such as those referred to in the concluding sections of the previous chapter. Our
general experience of human nature comes in aid of the scantiness
and imperfection of statistical records.

2. The elements involved in problems relating to criminal
assize are the following:---

1st. The probability that a particular member of the jury
will form a correct opinion upon the case.

2nd. The probability that the accused party is guilty.

3rd. The probability that he will be condemned, or that he
will be acquitted.

4th. The probability that his condemnation or acquittal will
be just.

5th. The constitution of the jury.

%-----------------------File: 180.png----------------------------
6th. The data furnished by experience, such as the relative
numbers of cases in which unanimous decisions have been arrived
at, or particular majorities obtained; the number of cases in
which decisions have been reversed by superior courts, \&c.

Again, the class of questions under consideration may be
regarded as either direct or inverse. The direct questions of probability are those in which the probability of correct decision
for each member of the tribunal, or of guilt for the accused
party, are supposed to be known \textit{\`{a} priori}, and in which the probability of a decision of a particular kind, or with a definite majority,
is sought. Inverse problems are those in which, from the data furnished by experience, it is required to determine some element
which, though it stand to those data in the relation of cause to
effect, cannot directly be made the subject of observation; as
when from the records of the decisions of courts it is required to
determine the probability that a member of a court will judge
correctly. To this species of problems, the most difficult and
the most important of the whole series, attention will chiefly be
directed here.

3. There is no difficulty in solving the direct problems referred to in the above enumeration. Suppose there is but one
juryman. Let $k$ be the probability that the accused person is
guilty; $x$ the probability that the juryman will form a correct
opinion; $X$ the probability that the accused person will be condemned: then---
\begin{align*}
  kx &= \text{ probability that the accused party is guilty, and that the}\\
  & \qquad \text{juryman judges him to be guilty.}
\\
  (l-k)(l-x) &= \text{ probability that the accused person is inno-}\\
  & \qquad \text{cent, and that the juryman pronounces him guilty.}
\end{align*}

Now these being the only cases in which a verdict of condemnation can be given, and being moreover mutually exclusive,
we have
\[
  X = kx + (1-k)(1-x).    \tag{1}
\]

In like manner, if there be $n$ jurymen whose separate probabilities of correct judgment are $x_1, x_2, \dotsc x_n$, the probability of an
unanimous verdict of condemnation will be
\[
  X = k x_1 x_2 \dotsc x_n + (1-k)(1-x_1)(1-x_2)\dotsc (1-x_n).
\]
%-----------------------File: 181.png----------------------------
Whence, if the several probabilities $x_1, x_2 \dotsc x_n$ are equal, and are
each represented by $x$, we have
\[
  X = kx^n + (1-k)(1-x)^n.   \tag{2}
\]
The probability in the latter case, that the accused person is guilty,
will be
\[
  \frac{kx^n}{kx^n + (1-k)(1-x)^n}
\]
All these results assume, that the events whose probabilities
are denoted by $k$, $x_1$, $x_2$, \&c., are independent, an assumption
which, however, so far as we are concerned, is involved in the
fact that those events are the only ones of which the probabilities
are given.

The probability of condemnation by a given number of voices
may be found on the same principles. If a jury is composed of
three persons, whose several probabilities of correct decision are
$x$, $x'$, $x''$, the probability $X_2$ that the accused person will be declared guilty by two of them will be
\[
\begin{split}
  X_2 = k \{xx'(1-x'') + xx''(1-x') + x'x''(1-x)\}   \\
      + (1-k)\{(1-x)(1-x')x'' + (1-x)(1-x'')x' + (1-x')(1-x'')x\},
\end{split}
\]
which if $x = x' = x''$ reduces to
\[
  3kx^2(1-x) + 3(1-k)x(1-x)^2.
\]

And by the same mode of reasoning, it will appear that if
$X_i$ represent the probability that the accused person will be declared guilty by $i$ voices out of a jury consisting of $n$ persons,
whose separate probabilities of correct judgment are equal, and
represented by $x$, then
\[
  X_i = \frac{n(n-1) \dotsc (n-i+1)}{1 \centerdot 2 \dotsc i}
        \{ kx^i(1-x)^{n-i} + (1-k)x^{n-i}(1-x)^i \}.         \tag{3}
\]
If the probability of condemnation by a determinate majority $a$
is required, we have simply
\[
  i - a = n - i,
\]
whence
\[
  i = \frac{n+a}{2},
\]
%-----------------------File: 182.png----------------------------
which must be substituted in the above formula. Of course $a$
admits only of such values as make $i$ an integer. If $n$ is even,
those values are 0, 2, 4, \&c.; if odd, 1, 3, 5, \&c., as is otherwise
obvious.

The probability of a condemnation by a majority of at least a
given number of voices $m$, will be found by adding together the
following several probabilities determined as above, viz.:
\begin{equation*}\begin{minipage}{.8\linewidth}
1st. The probability of a condemnation by an exact majority $m$;
\end{minipage}\end{equation*}
\begin{equation*}\begin{minipage}{.8\linewidth}
2nd. The probability of condemnation by the next greater
majority $m + 2$;
\end{minipage}\end{equation*}
%**[2nd proofer: I probably would have done this with
%   \begin{verse}..\\..\end{verse}, but I don't know if
%   either way is better at getting the indentation right.]
and so on; the last element of the series being the probability of
unanimous condemnation. Thus the probability of condemnation
by a majority of 4 at least out of 12 jurors, would be
\begin{equation*}
X_8 + X_9 \dotsc + X_{12},
\end{equation*}
the values of the above terms being given by (3) after making
therein $n = 12$.

4. When, instead of a jury, we are considering the case of a
simple deliberative assembly consisting of $n$ persons, whose separate
probabilities of correct judgment are denoted by $x$, the above
{formul\ae} are replaced by others, made somewhat more simple by
the omission of the quantity $k$.

The probability of unanimous decision is
\begin{equation*}
X = x^n+(1-x)^n.
\end{equation*}

The probability of an agreement of $i$ voices out of the whole
number is
\begin{equation}\tag{4}
X_i = \frac{n(n-1)\dotsc (n-i+1)}{1 \cdot 2 \dotsc i} \{x^i(1-x)^{n-i}+x^{n-i}(1-x)^i\}.
\end{equation}

Of this class of investigations it is unnecessary to give any
further account. They have been pursued to a considerable extent
by Condorcet, Laplace, Poisson, and other writers, who
have investigated in particular the modes of calculation and reduction
which are necessary to be employed when $n$ and $i$ are
large numbers. It is apparent that the whole inquiry is of a very
speculative character. The values of $x$ and $k$ cannot be
%-----------------------File: 183.png----------------------------
determined by direct observation. We can only presume that they
must both in general exceed the value $\dfrac{1}{2}$; that the former, $x$, must
increase with the progress of public intelligence; while the latter,
$k$, must depend much upon those preliminary steps in the administration of the law by which persons suspected of crime are
brought before the tribunal of their country. It has been remarked by Poisson, that in periods of revolution, as during the
Reign of Terror in France, the value of $k$ may fall, if account be
taken of political offences, far below the limit $\dfrac{1}{2}$. The history of
Europe in days nearer to our own would probably confirm this
observation, and would show that it is not from the wild license
of democracy alone, that the accusation of innocence is to be
apprehended.

Laplace makes the assumption, that all values of $x$ from
\[
  x = \frac{1}{2}; \text{ to }x = 1,
\]
are equally probable. He thus excludes the supposition that a
juryman is more likely to be deceived than not, but assumes that
within the limits to which the probabilities of individual correctness of judgment are confined, we have no reason to give
preference to one value of $x$ over another. This hypothesis is
entirely arbitrary, and it would be unavailing here to examine
into its consequences.

Poisson seems first to have endeavoured to deduce the values
of $x$ and $k$, inferentially, from experience. In the six years from
1825 to 1830 inclusively, the number of individuals accused of
crimes against the person before the tribunals of France was
11016, and the number of persons condemned was 5286. The
juries consisted each of 12 persons, and the decision was pronounced by a simple majority. Assuming the above numbers
to be sufficiently large for the estimation of probabilities, there
would therefore be a probability measured by the fraction $\dfrac{5286}{11016}$
or $.4782$ that an accused person would be condemned by a simple
majority. We should have the equation
\[
  X_7 + X_8 \dotsc + X_{12} = .4782,    \tag{5}
\]
%-----------------------File: 184.png----------------------------
the general expression for $X_i$ being given by (3) after making
therein $n = 12$. In the year 1831 the law, having received alteration, required a majority of at least four persons for condemnation, and the number of persons tried for crimes against the
person during that year being $2046$, and the number condemned
$743$, the probability of the condemnation of an individual by the
above majority was $\frac{743}{2046}$, or $.3631$. Hence we should have
\[
  X_8 + X_9 \dotsc + X_{12} = .3631\,.    \tag{6}
\]

Assuming that the values of $k$ and $x$ were the same for the
year 1831 as for the previous six years, the two equations (5) and
(6) enable us to determine approximately their values. Poisson
thus found,
\[
  k =.5354,\quad x =.6786\,.
\]

For crimes against property during the same periods, he
found by a similar analysis,
\[
  k =.6744,\quad x = .7771\,.
\]

The solution of the system (5) (6) conducts in each case to
two values of $k$, and to two values of $x$, the one value in each
pair being greater, and the other less, than $\dfrac{1}{2}$. It was assumed,
that in each case the larger value should be preferred, it being
conceived more probable that a party accused should be guilty
than innocent, and more probable that a juryman should form
a correct than an erroneous opinion upon the evidence.

5. The data employed by Poisson, especially those which were
furnished by the year 1831, are evidently too imperfect to permit
us to attach much confidence to the above determinations of $x$ and
$k$; and it is chiefly for the sake of the method that they are here
introduced. It would have been possible to record during the
six years, 1825-30, or during any similar period, the number of
condemnations pronounced with each possible majority of voices.
The values of the several elements $X_8, X_9, \dotsc X_{12}$ were there
no reasons of policy to forbid, might have been accurately ascertained. Here then the conception of the general problem, of
which Poisson's is a particular case, arises. How shall we, from
%-----------------------File: 185.png----------------------------
this apparently supernumerary system of data, determine the
values of $x$ and $k$? If the hypothesis, adopted by Poisson and
all other writers on the subject, of the absolute independence of
the events whose probabilities are denoted by $x$ and $k$ be retained,
we should be led to form a system of five equations of the type~(3),
and either select from these that particular pair of equations which
might appear to be most advantageous, or combine together the
equations of the system by the method of least squares. There
might exist a doubt as to whether the latter method would be
strictly applicable in such cases, especially if the values of $x$ and $k$
afforded by different selected pairs of the given equations were very
different from each other. M.~Cournot has considered a somewhat
similar problem, in which, from the records of individual votes in
a court consisting of four judges, it is proposed to investigate the
separate probabilities of a correct verdict from each judge. For
the determination of the elements $x$, $x'$, $x''$, $x'''$, he obtains eight
equations, which he divides into two sets of four equations, and
he remarks, that should any considerable discrepancy exist between the values of $x$, $x'$, $x''$, $x'''$, determined from those sets, it
might be regarded as an indication that the hypothesis of the independence of the opinions of the judges was, in the particular
case, untenable. The principle of this mode of investigation has
been adverted to in (XVIII.~4).

6. I proceed to apply to the class of problems above indicated,
the method of this treatise, and shall inquire, first, whether the
records of courts and deliberative assemblies, \emph{alone}, can furnish
any information respecting the probabilities of correct judgment
for their individual members, and, it appearing that they cannot,
secondly, what kind and amount of necessary hypothesis will best
comport with the actual data.

\begin{center}\textsc{Proposition I.}\end{center}

\emph{From the mere records of the decisions of a court or deliberative
assembly, it is not possible to deduce any definite conclusion respecting the correctness of the individual judgments of its members.}

Though this Proposition may appear to express but the conviction of unassisted good sense, it will not be without interest to
show that it admits of rigorous demonstration.
%-----------------------File: 186.png----------------------------
Let us suppose the case of a deliberative assembly consisting
of $n$ members, no hypothesis whatever being made respecting
the dependence or independence of their judgments. Let the
logical symbols $x_1, x_2, \dotsc x_n$ be employed according to the following definition, viz.: Let the generic symbol $x_i$ denote that
event which consists in the uttering of a correct opinion by the
$i^{th}$ member, $A_i$ of the court. We shall consider the values of
Prob. $x_1$, Prob. $x_2,\dotsc $Prob. $x_n$, as the \textit{qu{\ae}sita} of a problem, the
expression of whose possible data we must in the next place
investigate.

Now those data are the probabilities of events capable of
being expressed by definite logical functions of the symbols
$x_1, x_2,\dotsc x_n$. Let $X_1, X_2, \dotsc X_m$ represent the functions in question,
and let the actual system of data be
\[
  \text{Prob. }X_1 = a_1,\quad
  \text{Prob. }X_2 = a_2,\quad
  \text{Prob. }X_m = a_m.
\]
Then from the very nature of the case it may be shown that
$X_1, X_2, \dotsc X_m$, are functions which remain unchanged if
$x_1, x_2, \dotsc x_n$ are therein changed into
$1-x_1, 1-x_2, \dotsc 1-x_n$
respectively. Thus, if it were recorded that in a certain proportion of instances the votes given were unanimous, the event
whose probability, supposing the instances sufficiently numerous,
is thence determined, is expressed by the logical function
\[
  x_1 x_2 \dotsc x_n + (1-x_1)(1-x_2)\dotsc (1-x_n),
\]
a function which satisfies the above condition. Again, let it be
recorded, that in a certain proportion of instances, the vote of an
individual, suppose $A_1$, differs from that of all the other members of the court. The event, whose probability is thus given,
will be expressed by the function
\[
  x_1(1-x_2)\dotsc (1-x_n) + (1-x_1)x_2\dotsc x_n;
\]
also satisfying the above conditions. Thus, as agreement in
opinion may be an agreement in either truth or error; and as,
when opinions are divided, either party may be right or wrong;
it is manifest that the expression of any particular state, whether
of agreement or difference of sentiment in the assembly, will
depend upon a logical function of the symbols $x_1, x_2, \dotsc x_n$,
%-----------------------File: 187.png----------------------------
which similarly involves the privative symbols
$1-x_1, 1-x_2, \dotsc 1-x_n$. But in the records of assemblies, it is not presumed
to declare which set of opinions is right or wrong. Hence the
functions $X_1, X_2, \dotsc X_m$ must be solely of the kind above described.

7. Now in proceeding, according to the general method, to
determine the value of Prob. $x_1$, we should first equate the functions $X_1, \dotsc X_m$ to a new set of symbols $t_1, \dotsc t_m$. From the
equations
\[
  X_1 = t_1, X_2 = t_2, \dotsc X_m = t_m,
\]
thus formed, we should eliminate the symbols $x_2, x_3,\dotsc x_n$, and
then determine $x_1$ as a developed logical function of the symbols
$t_1, t_2, \dotsc t_m$, expressive of events whose probabilities are given.
Let the result of the above elimination be
\[
  Ex_1 + E'(1-x_1) = 0;    \tag{1}
\]
$E$ and $E'$ being function of $t_1, t_2, \dotsc t_m$. Then
\[
  x_1 = \frac{E'}{E'-E}.   \tag{2}
\]

Now the functions $X_1, X_2,\dotsc X_m$ are symmetrical with reference to the symbols $x_1,\dotsc x_n$ and $1-x_1,\dotsc 1-x_n$. It is evident, therefore, that in the equation $E'$ must be identical with $E$.
Hence (2) gives
\[
  x = \frac{E}{0},
\]
and it is evident, that the only coefficients which can appear in the
development of the second member of the above equation are
$\frac{0}{0}$ and $\frac{1}{0}$. The former will present itself whenever the values
assigned to $t_1,\dotsc t_m$ in determining the coefficient of a constituent,
are such as to make $E = 0$, the latter, or an equivalent result, in
every other case. Hence we may represent the development
under the form
\[
  x_1 = \frac{0}{0}C + \frac{1}{0}D    \tag{3}
\]
$C$ and $D$ being constituents, or aggregates of constituents, of the
symbols $t_1, t_2, \dotsc t_m$.
%-----------------------File: 188.png----------------------------
Passing then from Logic to Algebra, we have
\[
  \text{Prob. }x_1 = \frac{cC}{C} = c,
\]
the function $V$ of the general Rule (XVII.~17) reducing in the
present case to $C$. The value of Prob. $x_1$ is therefore wholly arbitrary, if we except the condition that it must not transcend
the limits $0$ and $1$. The individual values of Prob. $x_2$, $\dotsc$Prob. $x_n$,
are in like manner arbitrary. It does not hence follow, that
these arbitrary values are not connected with each other by necessary conditions dependent upon the data. The investigation
of such conditions would, however, properly fall under the methods of Chap.~\textsc{xix}.

If, reverting to the final logical equation, we seek the interpretation of $c$, we obtain but a restatement of the original problem. For since $C$ and $D$ together include all possible constituents of $t_1, t_2, \dotsc t_m$, we have
\[
  C + D = 1;
\]
and since $D$ is affected by the coefficient $\frac{1}{0}$, it is evident that on
substituting therein for $t_1, t_2, \dotsc t_m$, their expressions in terms of
$x_1, x_2, \dotsc x_n$, we should have $D = 0$. Hence the same substitution
would give $C = 1$. Now by the rule, $c$ is the probability that if
the event denoted by $C$ take place, the event $x_1$ will take place.
Hence $C$ being equal to $1$, and, therefore, embracing all possible
contingencies, $c$ must be interpreted as the absolute probability of
the occurrence of the event $x_1$.

It may be interesting to determine in a particular case the
actual form of the final logical equation. Suppose, then, that the
elements from which the data are derived are the records of
events distinct and mutually exclusive. For instance, let the
numerical data $a_1, a_2, \dotsc a_m$, be the respective probabilities of
distinct and definite majorities. Then the logical functions
$X_1, X_2, \dotsc X_m$ being mutually exclusive, must satisfy the conditions
\[
  X_1 X_2 = 0,\dotsc X_1 X_m = 0,\quad X_2 X_m = 0, \text{ \&c.}
\]
Whence we have,
\[
  t_1 t_2 = 0,\quad  t_1 t_m = 0, \text{ \&c.}
\]
%-----------------------File: 189.png----------------------------
Under these circumstances it may easily be shown, that the
developed logical value of $x_1$ will be
\begin{equation*}\begin{split}
x_1&=\frac{0}{0}(\bar{t}_1 \bar{t}_2 \dotsc \bar{t}_m + t_1 \bar{t}_2 \dotsc \bar{t}_m \dotsc + t_m \bar{t}_1\dotsc \bar{t}_{m-1}) \\
& + \text{ constitutents whose coefficients are }\frac{1}{0}.
\end{split}\end{equation*}
In the above equation $\bar{t}_1$ stands for $1-t_1$, \&c.

These investigations are equally applicable to the case in
which the probabilities of the verdicts of a jury, so far as
agreement and disagreement of opinion are concerned, form the data
of a problem. Let the logical symbol $w$ denote that event or
state of things which consists in the guilt of the accused person.
Then the functions $X_1$, $X_2$ \dots $X_m$ of the present problem are
such, that no change would therein ensue from simultaneously
converting $w, x_1, x_2 \dotsc x_n$ into $\bar{w}, \bar{x}_1, \bar{x}_2,\dotsc \bar{x}_n$ respectively.
Hence the final logical value of $w$, as well as those of $x_1, x_2, \dotsc x_n$
will be exhibited under the same form (3), and a like general
conclusion thence deduced.

It is therefore established, that from mere statistical
documents nothing can be inferred respecting either the individual
correctness of opinion of a judge or counsellor, the guilt of an
individual, or the merits of a disputed question. If the determination
of such elements as the above can be reduced within
the province of science at all, it must be by virtue either of
some assumed criterion of truth furnishing us with new data, or
of some hypothesis relative to the connexion or the independence
of individual judgments, which may warrant a new form of the
investigation. In the examination of the results of different
hypotheses, the following general Proposition will be of importance.

\begin{center}
\textsc{Proposition} II.
\end{center}

8. \emph{Given the probabilities of the $n$ simple events $x_1, x_2,\dotsc x_n$,
viz.:---}
\begin{equation}\tag{1}
\operatorname{Prob. } x_1 = c_1, \quad \operatorname{Prob. } x_2=c_2, \dotsc \operatorname{Prob. }x_n=c_n;
\end{equation}
\emph{also the probabilities of the $m-1$ compound events $X_1, X_2, \dotsc X_{m-1}$,
viz.:---}
\begin{equation}\tag{2}
\operatorname{Prob. } X_1=a_1, \quad \operatorname{Prob. } X_2 = a_2, \dotsc \operatorname{Prob. }X_{m-1}=a_{m-1};
\end{equation}
%-----------------------File: 190.png----------------------------
%*
\emph{the latter events $X_1 \dotsc X_{m-1}$ being distinct and mutually exclusive;
required the probability of any other compound event $X$.}

In this proposition it is supposed, that $X_1, X_2,\dotsc X_{m-1}$, as
well as $X$, are functions of the symbols $x_1, x_2, \dotsc x_n$ alone.
Moreover, the events $X_1, X_2,\dotsc X_{m-1}$ being mutually exclusive,
we have
\begin{equation}\tag{3}
X_1 X_2 = 0, \dotsc X_1 X_{m-1}=0, \quad X_2 X_3 = 0, \text{ \&c.;}
\end{equation}
the product of any two members of the system vanishing. Now
assume
\begin{equation}\tag{4}
X_1=t_1, \quad X_{m-1}=t_{m-1}, \quad X=t.
\end{equation}
Then $t$ must be determined as a logical function of $x_1, \dotsc x_n,
t_1, \dotsc t_{m-1}$.

Now by (3),
\begin{equation}\tag{5}
t_1 t_2=0, \quad t_1 t_{m-1}=0, \quad t_2 t_3=0, \text{ \&c.;}
\end{equation}
all binary products of $t_1, \dotsc \bar{t}_{m-1}$, vanishing. The developed
expression for $t$ can, therefore, only involve in the list of constituents
which have 1, 0, or $\displaystyle \frac{0}{0}$ for their coefficients, such as contain
some one of the following factors, viz.:---
\begin{equation}\tag{6}
\bar{t}_1 \bar{t}_2 \dotsc \bar{t}_{m-1}, \quad t_1 \bar{t}_2 \dotsc \bar{t}_{m-1},\dotsc
\bar{t}_1 \dotsc \bar{t}_{m-2} t_{m-1};
\end{equation}
$\bar{t}_1$ standing for $1-t_1$, \&c. It remains to assign that portion of
each constituent which involves the symbols $x_1 \dotsc x_n$; together
with the corresponding coefficients.

Since $X_i = t_i$ ($i$ being any integer between 1 and $m-1$
inclusive), it is evident that
\begin{equation*}
X_i \bar{t}_1 \dotsc \bar{t}_{m-1}=0,
\end{equation*}
from the very constitution of the functions. Any constituent
included in the first member of the above equation would, therefore,
have $\displaystyle \frac{1}{0}$ for its coefficient.

Now let
\begin{equation}\tag{7}
X_m = 1 - X_1 \dotsc - X_{m-1};
\end{equation}
and it is evident that such constituents as involve $\bar{t}_1 \dotsc \bar{t}_{m-1}$, as
a factor, and yet have coefficients of the form 1, 0, or $\displaystyle \frac{0}{0}$, must be
%-----------------------File: 191.png----------------------------
included in the expression
\[
  X_m \bar{t}_1 \dotsc \bar{t}_{m-1}.
\]
Now $X_m$ may be resolved into two portions, viz., $X\,X_m$ and
$(1-X)X_m$, the former being the sum of those constituents of
$X_m$ which are found in $X$, the latter of those which are not found
in $X$. It is evident that in the developed expression of $t$, which
is equivalent to $X$, the coefficients of the constituents in the
former portion $X\,X_m$ will be $1$, while those of the latter portion
$(1-X)X_m$ will be $0$. Hence the elements we have now considered will contribute to the development of $t$ the terms
\[
  X X_m     \bar{t}_1 \dotsc \bar{t}_{m-1}
+ 0(1-X)X_m \bar{t}_1 \dotsc \bar{t}_{m-1} .
\]
Again, since $X_1 = t_1$, while $X_2 t_1 = t_2 t_1 = 0$, \&c., it is evident
that the only constituents involving
$t_1 \bar{t}_2 \dotsc \bar{t}_{m-1}$, as a factor which
have coefficients of the form $1$, $0$, or $\frac{0}{0}$, will be included in the expression
\[
  X_1 t_1 \bar{t}_2 \dotsc \bar{t}_{m-1};
\]
and reasoning as before, we see that this will contribute to the development of $t$ the terms
\[
  X X_1 t_1 \bar{t}_2 \dotsc \bar{t}_{m-1}
  + 0(1-X)X_1 t_1 \bar{t}_2 \dotsc \bar{t}_{m-1} .
\]

Proceeding thus with the remaining terms of (6), we deduce
for the final expression of $t$,
\begin{gather*}
  t = X X_m \bar{t}_1 \dotsc \bar{t}_{m-1}
    + X X_1 t_1 \bar{t}_2 \dotsc \bar{t}_{m-1} \dotsc
    + X X_{m-1} \bar{t}_1 \dotsc \bar{t}_{m-2} t_{m-1}
\\
  + 0(1-X)X_m \bar{t}_1 \dotsc \bar{t}_{m-1}
    + 0(1-X)X_1 t_1 \bar{t}_2 \dotsc \bar{t}_{m-1}  + \text{ \&c.}   \tag{8}
\\
  + \text{terms whose coefficients are }\frac{1}{0}.
\end{gather*}

In this expression it is to be noted that $X\,X_m$ denotes the sum
of those constituents which are common to $X$ and $X_m$, that sum
being actually given by multiply ing $X$ and $X_m$ together, according
to the rules of the calculus of Logic.

In passing from Logic to Algebra, we shall represent by
($X\,X_m$) what the above product becomes, when, after effecting
the multiplication, or selecting the common constituents, we
give to the symbols $x_1, \dotsc x_n$, a quantitative meaning.
%-----------------------File: 192.png----------------------------
With this understanding we shall have, by the general Rule
(XVII.~17),
\begin{gather*}
  \text{Prob. }t
\\
= \frac{ (X X_m)    \bar{t}_1\dotsc \bar{t}_{m-1}
       + (X X_1)t_1 \bar{t}_2\dotsc \bar{t}_{m-1}
       + (X X_{m-1})\bar{t}_1\dotsc \bar{t}_{m-2} t_{m-1}
       }{V},                                               \tag{9}
\\
  V = X_m     \bar{t}_1\dotsc \bar{t}_{m-1}
    + X_1 t_1 \bar{t}_2\dotsc \bar{t}_{m-1} \dotsc
    + X_{m-1} \bar{t}_1\dotsc \bar{t}_{m-2} t_{m-1}       \tag{10}
\end{gather*}
whence the relations determining
$x_1, \dotsc x_n, t_1, \dotsc t_{m-1}$ will be of
the following type ($i$ varying from $1$ to $n$),
\begin{gather*}
  \frac{ (x_i X_m)    \bar{t}_1\dotsc \bar{t}_{m-1}
       + (x_i X_1)t_1 \bar{t}_2\dotsc \bar{t}_{m-1}
       + (x_i X_{m-1})\bar{t}_1\dotsc \bar{t}_{m-2} t_{m-1}
       }{c_i}
\\
= \frac{ X_1 t_1 \bar{t}_2\dotsc \bar{t}_{m-1} }{a_1} \dotso
= \frac{ X_{m-1} \bar{t}_1\dotsc \bar{t}_{m-2} t_{m-1}}{a_{m-1}}
= V.         \tag{11}
\end{gather*}

From the above system we shall next eliminate the symbols
$t_1, \dotsc t_{m-1}$.

We have
\[
  t_1 \bar{t}_2\dotsc \bar{t}_{m-1} = \frac{a_1V}{X_1},\quad
  \bar{t}_1\dotsc \bar{t}_{m-2} t_{m-1} = \frac{a_{m-1}V}{X_{m-1}}.
\tag{12}
\]

Substituting these values in (10), we find
\[
  V = X_m \bar{t}_1 \dotsc \bar{t}_{m-1} + a_1 V \dotsc + a_{m-1} V.
\]
Hence,
\[
  \bar{t}_1 \dotsc \bar{t}_{m-1}
= \frac{(1 - a_1 \dotsc - a_{m-1})V}{X_m}.
\]

Now let
\[
  a_m = 1 - a_1 \dotsc - a_{m-1},    \tag{13}
\]
then we have
\[
  \bar{t}_1 \dotsc \bar{t}_{m-1} = \frac{a_m V}{X_m}.   \tag{14}
\]
Now reducing, by means of (12) and (14), the equation~(9),
and the equation formed by equating the first line of (11) to the
symbol $V$; writing also Prob. $X$ for Prob. $t$, we have
\begin{gather*}
  \text{Prob. }X
= \frac{a_1(X X_1)}{X_1}
+ \frac{a_2(X X_2)}{X_2} \dotsc
+ \frac{a_m(X X_m)}{X_m},             \tag{15}
\\
  \frac{a_1(x_i X_1)}{X_1}
+ \frac{a_2(x_i X_2)}{X_2} \dotsc
+ \frac{a_m(x_i X_m)}{X_m} = c_i;    \tag{16}
\end{gather*}
wherein $X_m$ and $a_m$ are given by (7) and (13).
%-----------------------File: 193.png----------------------------
These equations involve the direct solution of the problem
under consideration. In (16) we have the type of $n$ equations
(formed by giving to $i$ the values $1, 2,\dotsc n$ successively), from
which the values of $x_1, x_2,\dotsc x_n$, will be found, and those values
substituted in (15) give the value of $\operatorname{Prob. } X$ as a function of
the constants $a_1$, $c_1$, \&c.

One conclusion deserving of notice, which is deducible from
the above solution, is, that if the probabilities of the compound
events $X_1,\dotsc X_{m-1}$, are the same as they would be were the
events $x_1, \dotsc x_n$ entirely independent, and with given probabilities
$c_1,\dotsc c_n$, then the probability of the event $X$ will be the
same as if calculated upon the same hypothesis of the absolute
independence of the events $x_1, \dotsc x_n$. For upon the hypothesis
supposed, the assumption $x_1 = c_1$, $x_n = c_n$, in the quantitative
system would give $X_1 = a_1$, $X_m = a_m$, whence (15) and (16)
would give
\begin{equation}\tag{17}
\operatorname{Prob. } X = (X X_1) + (X X_2)\dotsc + (X X_m),
\end{equation}
\begin{equation}\tag{18}
(x_i X_1)+(x_i X_2)\dotsc + (x_i X_m) = c_i.
\end{equation}

But since $X_1 + X_2\dotsc + X_m=1$, it is evident that the second
member of (17) will be formed by taking all the constituents that
are contained in $X$, and giving them an algebraic significance.
And a similar remark applies to (18). Whence those equations
respectively give
\begin{gather*}
\operatorname{Prob. } X \text{ (logical)} = X \text{ (algebraic)}, \\
x_i = c_i.
\end{gather*}

Wherefore, if $X = \phi(x_1, x_2,\dotsc x_n)$, we have
\begin{equation*}
\operatorname{Prob. } X = \phi(c_1, c_2, \dotsc c_n),
\end{equation*}
which is the result in question.

Hence too it would follow, that if the quantities $c_1,\dotsc c_n$
were indeterminate, and no hypothesis were made as to the
possession of a mean common value, the system (15) (16) would
be satisfied by giving to those quantities any such values,
$x_1, x_2, \dotsc x_n$, as would satisfy the equations
\begin{equation*}
X_1 = a_1 \dotsc X_{m-1} = a_{m-1}, \quad X=a,
\end{equation*}
supposing the value of the element $a$, like the values of
$a_1, \dotsc a_{m-1}$,
to be given by experience.
%-----------------------File: 194.png----------------------------
9. Before applying the general solution (15) (16), to the
question of the probability of judgments, it will be convenient to
make the following transformation. Let the data be
\begin{align*}
x_1 = c_1 & \dots x_n=c_n, \\
\operatorname{Prob. } X_1 = a_1 & \dots
\operatorname{Prob. } X_{m-2} = a_{m-2};
\end{align*}
and let it be required to determine $\operatorname{Prob. } X_{m-1}$, the unknown
value of which we will represent by $a_{m-1}$. Then in (15) and (16)
we must change \\
\begin{center}
\begin{tabular}{ll}
$X$ into $X_{m-1}$, & $\operatorname{Prob. } X$ into $a_{m-1}$, \\
$X_{m-1}$ into $X_{m-2}$, & $\quad \quad a_{m-1}$ into $a_{m-2}$, \\
$X_m$ into $X_{m-1} + X_m$, & $\quad \quad a_m$ into $a_{m-1} + a_m$;
\end{tabular}\end{center}
with these transformations, and observing that $(X_{m-1} X_r)=0$,
except when $r = m-1$, and that it is then equal to $X_{m-1}$, the
equations (15) (16) give
\begin{equation}\tag{19}
a_{m-1}=\frac{(a_{m-1} + a_m)X_{m-1}}{X_{m-1}+X_m},
\end{equation}
\begin{equation}\tag{20}
  \frac{a_1(x_i X_1)}{X_1} \dotsc
+ \frac{a_{m-2}(x_i X_{m-2})}{X_{m-2}}
+ \frac{(a_{m-1} + a_m)(x_i X_{m-1} + x_i X_m}{X_{m-1} + X_m}.
\end{equation}
Now from (19) we find
\begin{equation*}
\frac{X_{m-1}}{a_{m-1}}=\frac{X_m}{a_m} = \frac{X_{m-1}+X_m}{a_{m-1}+a_m},
\end{equation*}
by virtue of which the last term of (20) may be reduced to the
form
\begin{equation*}
\frac{a_{m-1}(x_i X_{m-1})}{X_{m-1}}+\frac{a_m(x_i X_m)}{X_m}.
\end{equation*}
With these reductions the system (17) and (18) may be replaced
by the following symmetrical one, viz.:
\begin{equation}\tag{21}
\frac{X_{m-1}}{a_{m-1}}=\frac{X_m}{a_m},
\end{equation}
\begin{equation}\tag{22}
\frac{a_1(x_i X_1)}{X_1} +
\frac{a_2(x_i X_2)}{X_2}\dotsc +
\frac{a_m(x_i X_m)}{X_m} = c_i.
\end{equation}
These equations, in connexion with (7) and (13), enable us to
%-----------------------File: 195.png----------------------------
determine $a_{m-1}$, as a function of
$c_1 \dotsc c_n, a_1 \dotsc a_{m-2}$, the numerical
data supposed to be furnished by experience. We now proceed
to their application.

\begin{center}
\textsc{Proposition III}.
\end{center}

10. \emph{Given any system of probabilities drawn from recorded
instances of unanimity, or of assigned numerical majority in the
decisions of a deliberative assembly; required, upon a certain
determinate hypothesis, the mean probability of correct judgment for a
member of the assembly.}

In what way the probabilities of unanimous decision and of
specific numerical majorities may be determined from experience,
has been intimated in a former part of this chapter. Adopting
the notation of Prop.~\textsc{i}. we shall represent the events whose
probabilities are given by the functions $X_1,X_2, \dotsc X_{m-1}$. It has
appeared from the very nature of the case that these events are
mutually exclusive, and that the functions by which they are
represented are symmetrical with reference to the symbols $x_1, x_2, \dots x_n$.
Those symbols we continue to use in the same sense as in Prop.~\textsc{i}.,
viz., by $x_i$ we understand that event which consists in the
formation of a correct opinion by the $i^\text{th}$ member of the assembly.

Now the immediate data of experience are---
\begin{equation}\tag{1}
\operatorname{Prob. } X_1 = a_1, \quad
\operatorname{Prob. } X_2 = a_2\dotsc
\operatorname{Prob. } X_{m-2} = a_{m-2},
\end{equation}
\begin{equation}\tag{2}
\operatorname{Prob. } X_{m-1} = a_{m-1}.
\end{equation}
$X_1\dots X_{m-1}$ being functions of the logical symbols
$x_1, \dotsc x_n$ to the
probabilities of the events denoted by which, we shall assign the
indeterminate value $c$. Thus we shall have
\begin{equation}\tag{3}
\operatorname{Prob. } x_1 = \operatorname{Prob. } x_2 \dots = \operatorname{Prob. } x_n = c.
\end{equation}

Now it has been seen, Prop.~\textsc{i}., that the immediate data (1)
(2), unassisted by any hypothesis, merely conduct us to a
restatement of the problem. On the other hand, it is manifest that
if, adopting the methods of Laplace and Poisson, we employ the
system (3) alone as the data for the application of the method of
this work, finally comparing the results obtained with the
experimental system (1) (2), we are relying \emph{wholly} upon a doubtful
hypothesis,---the independence of individual judgments. But
%-----------------------File: 196.png----------------------------
though we ought not wholly to rely upon this hypothesis, we
cannot wholly dispense with it, or with some equivalent substitute. Let us then examine the consequences of a \emph{limited} independence of the individual judgments; the conditions of limitation
being furnished by the apparently superfluous data. From the
system (1) (3) let us, by the method of this work, determine
Prob. $X_{m-1}$, and, comparing the result with (2), determine $c$.
Even here an arbitrary power of selection is claimed. But it is
manifest from Prop. \textsc{i.} that something of this kind is unavoidable,
if we would obtain a definite solution at all. As to the principle
of selection, I apprehend that the equation (2) reserved for final
comparison should be that which, from the magnitude of its numerical element $a_{m-1}$, is esteemed the most important of the primary series furnished by experience.

Now, from the mutually exclusive character of the events
denoted by the functions $X_1, X_2,\dotsc X_{m-1}$, the concluding equations of the previous proposition become applicable. On account
of the symmetry of the same functions, and the reduction of the
system of values denoted by $c_i$, to a single value $c$, the equations
represented by (22) become identical, the values of
$x_1, x_2,\dotsc x_n$
become equal, and may be replaced by a single value $x$, and we
have simply,
\begin{gather*}
  \frac{X_{m-1}}{a_{m-1}} = \frac{X_m}{a_m},   \tag{4}
\\
  \frac{a_1(x X_1)}{X_1}
+ \frac{a_2(x X_2)}{X_2} \dotsc
+ \frac{a_m(x X_m)}{X_m} = c.                  \tag{5}
\end{gather*}
The following is the nature of the solution thus indicated:

The functions $X_1,\dotsc X_{m-1}$, and the values
$a_1,\dotsc a_{m-1}$, being
given in the data, we have first,
\begin{align*}
  X_m &= 1 - X_1 \dotsc - X_{m-1},   \\
  a_m &= 1 - a_1 \dotsc - a_{m-1}.
\end{align*}

From each of the functions $X_1, X_2,\dotsc X_m$ thus given or determined,
we must select those constituents which contain a particular symbol, as
$x_1$ for a factor. This will determine the functions ($x X_1$), ($x X_2$),
\&c., and then in all the functions we must
change $x_1, x_2,\dotsc x_n$ individually to $x$. Or we may regard any
%-----------------------File: 197.png----------------------------
algebraic function $X_i$ in the system (4) (5) as expressing the
probability of the event denoted by the logical function $X_i$, on
the supposition that the logical symbols $x_1, x_2,\dotsc x_n$ denote
independent events whose common probability is $x$. On the same
supposition ($xX_i$) would denote the probability of the concurrence
of any particular event of the series $x_1, x_2,\dotsc x_n$ with $X_i$.
The forms of $X_i$, ($xX_i$), \&c. being determined, the equation~(4)
gives the value of $x$, and this, substituted in (5), determines the
value of the element $c$ required. Of the two values which its solution
will offer, one being greater, and the other less, than $\tfrac{1}{2}$, the
greater one must be chosen, whensoever, upon general considerations, it is
thought more probable that a member of the assembly
will judge correctly, than that he will judge incorrectly.

Here then, upon the assumed principle that the largest of
the values $a_{m-1}$ shall be reserved for final comparison in the
equation~(2), we possess a definite solution of the problem proposed. And
the same form of solution remains applicable should
any other equation of the system, upon any other ground, as that
of superior accuracy, be similarly reserved in the place of (2).

11. Let us examine to what extent the above reservation has
influenced the final solution. It is evident that the equation~(5)
is quite independent of the choice in question. So is likewise
the second member of (4). Had we reserved the function $X_1$,
instead of $X_{m-1}$, the equation for the determination of $x$ would
have been
\[
  \frac{X_1}{a_1} = \frac{X_m}{a_m},   \tag{6}
\]
but the value of $x$ thence determined would still have to be substituted
in the same final equation~(5). We know that were
the events $x_1, x_2,\dotsc x_n$ really independent, the equations~(4),
(6), and all others of which they are types, would prove equivalent, and
that the value of $x$ furnished by any one of them
would be the true value of $c$. This affords a means of verifying
(5). For if that equation be correct, it ought, under the above
circumstances, to be satisfied by the assumption $c = x$. In other
words, the equation
\[
  \frac{a_1(x X_1)}{X_1}
+ \frac{a_2(x X_2)}{X_2} \dotsc
+ \frac{a_m(x X_m)}{X_m} = x     \tag{7}
\]
%-----------------------File: 198.png----------------------------
ought, on solution, to give the same value of $x$ as the equation
(4) or (6). Now this will be the case. For since, by hypothesis,
\[
  \frac{X_1}{a_1} = \frac{X_2}{a_2} \dotso = \frac{X_m}{a_m},
\]
we have, by a known theorem,
\[
  \frac{X_1}{a_1} = \frac{X_2}{a_2} \dotso = \frac{X_m}{a_m}
= \frac{X_1 + X_2 \dotsc + X_m}{a_1 + a_2 \dotsc + a_m}
= 1.
\]
Hence (7) becomes on substituting $a_1$ for $X_1$, \&c.
\[
  (xX_1) + (xX_2)\dotsc (xX_m) = x
\]
a mere identity.

Whenever, therefore, the events $x_1, x_2,\dotsc x_n$ are really
independent, the system (4) (5) is a correct one, and is independent
of the arbitrariness of the first step of the process by which it
was obtained. When the said events are not independent, the
final system of equations will possess, leaving in abeyance the
principle of selection above stated, an arbitrary element. But
from the persistent form of the equation (5) it may be inferred
that the solution is arbitrary in a less degree than the solutions
to which the hypothesis of the absolute independence of the individual
judgments would conduct us. The discussion of the
limits of the value of $c$, as dependent upon the limits of the value
of $x$, would determine such points.

These considerations suggest to us the question whether the
equation (7), which is symmetrical with reference to the functions $X_1, X_2,\dotsc X_m$, free from any arbitrary elements, and rigorously exact when the events $x_1, x_2,\dotsc x_n$ are really independent,
might not be accepted as a mean general solution of the problem.
The proper mode of determining this point would, I conceive, be
to ascertain whether the value of $x$ which it would afford would,
in general, fall within the limits of the value of $c$, as determined
by the systems of equations of which the system (4), (5), presents
the type. It seems probable that under ordinary circumstances
this would be the case. Independently of such considerations,
however, we may regard (7) as itself the expression of a certain
principle of solution, viz., that regarding
$X_1, X_2,\dotsc X_m$ as exclusive \emph{causes} of the event whose
probability is $x$, we accept the
%-----------------------File: 199.png----------------------------
probabilities of those causes $a_1, a_2,\dotsc a_m$ from experience, but form
the conditional probabilities of the event as dependent upon such
causes,
\[
  \frac{(x X_1)}{x_1},\quad \frac{(x X_2)}{X_2},
  \text{ \&c. (XVII. Prop~\textsc{i.})}
\]
on the hypothesis of the independence of individual judgments,
and so deduce the equation~(7). I conceive this, however, to be
a less rigorous, though possibly, in practice a more convenient
mode of procedure than that adopted in the general solution.

12. It now only remains to assign the particular forms which
the algebraic functions $X_i$, ($x X_i$), \&c. in the above equations assume when the logical function $X_i$ represents that event which
consists in $r$ members of the assembly voting one way, and $n-r$
members the other way. It is evident that in this case the algebraic function $X_i$ expresses what the probability of the supposed
event would be were the events $x_1, x_2,\dotsc x_n$ independent, and
their common probability measured by $x$. Hence we should
have, by Art.~3,
\[
  X_i = \frac{n(n-1)\dotsc (n-r+1)}{1\centerdot 2\dotsc r}
        \{ x^r + (1-x)^{n-r} \}.
\]
Under the same circumstances ($xX_i$) would represent the probability of the compound event, which consists in a particular
member of the assembly forming a correct judgment, conjointly
with the general state of voting recorded above. It would,
therefore, be the probability that a particular member votes correctly, while of the remaining $n-1$ members, $r-1$ vote correctly; or that the same member votes correctly, while of the
remaining $n-1$ members $r$ vote incorrectly. Hence
\[
  (xX_i)
= \frac{(n-1)(n-2)\dotsc (n-r+1)}{1\centerdot 2\dotsc r-1}x^r
+ \frac{(n-1)(n-2)\dotsc (n-r)}{1\centerdot 2\dotsc r}x^{n-r}.
\]

\begin{center}\textsc{Proposition~IV.}\end{center}

13. \emph{Given any system of probabilities drawn from recorded instances of unanimity, or of assigned numerical majority in the decisions of a criminal court of justice, required upon hypotheses
similar to those of the last proposition, the mean probability $c$ of
}%**[emphasized text continues]
%-----------------------File: 200.png----------------------------
%**[emphasised text continued from previous page]
\emph{correct judgment for a member of the court, and the general probability $k$ of guilt in an accused person.}

The solution of this problem differs in but a slight degree
from that of the last, and may be referred to the same general
formul{\ae}, (4) and (5), or (7). It is to be observed, that as there
are two elements, $c$ and $k$, to be determined, it is necessary to
reserve two of the functions $X_1, X_2,\dotsc X_{m-1}$, let us suppose $X_1$,
and $X_{m-1}$, for final comparison, employing either the remaining
$m-3$ functions in the expression of the data, or the two respective sets $X_2, X_3,\dotsc X_{m-1}$, and $X_1, X_2,\dotsc X_{m-2}$. In either case
it is supposed that there must be at least two original independent data. If the equation~(7) be alone employed, it would in
the present instance furnish two equations, which may thus be
written:
\begin{align*}
  \frac{a_1(xX_1)}{X_1}
+ \frac{a_2(xX_2)}{X_2} \dotsc
+ \frac{a_m(xX_m)}{X_m} &= x,     \tag{1}
\\
  \frac{a_1(kX_1)}{X_1}
+ \frac{a_2(kX_2)}{X_2} \dotsc
+ \frac{a_m(kX_m)}{X_m} &= k.     \tag{2}
\end{align*}
These equations are to be employed in the following manner:---
Let $x_1, x_2, \dotsc x_n$ represent those events which consist in the formation of a correct opinion by the members of the court respectively. Let also $w$ represent that event which consists in the
guilt of the accused member. By the aid of these symbols we
can logically express the functions $X_1, X_2,\dotsc X_{m-1}$, whose probabilities are given, as also the function $X_m$. Then from the function $X_1$ select those constituents which contain, as a factor, any
particular symbol of the set $x_1, x_2, \dotsc x_n$, and also those constituents which contain as a factor $w$. In both results change
$x_1, x_2, \dotsc x_n$ severally into $x$, and $w$ into $k$. The above results
will give ($xX_1$) and ($kX_1$). Effecting the same transformations
throughout, the system (1), (2) will, upon the particular hypothesis involved, determine $x$ and $k$.

14. We may collect from the above investigations the following facts and conclusions:

1st. That from the mere records of agreement and disagreement in the opinions of any body of men, no definite numerical
conclusions can be drawn respecting either the probability of
%-----------------------File: 201.png----------------------------
correct judgment in an individual member of the body, or the merit
of the questions submitted to its consideration.

2nd. That such conclusions may be drawn upon various distinct hypotheses, as---1st, Upon the usual hypothesis of the absolute independence of individual judgments; 2ndly, upon certain
definite modifications of that hypothesis warranted by the actual
data; 3rdly, upon a distinct principle of solution suggested by
the appearance of a common form in the solutions obtained by
the modifications above adverted to.

Lastly. That whatever of doubt may attach to the final results, rests not upon the imperfection of the method, which
adapts itself equally to all hypotheses, but upon the uncertainty
of the hypotheses themselves.

It seems, however, probable that with even the widest limits
of hypothesis, consistent with the taking into account of \emph{all} the
data of experience, the deviation of the results obtained would be
but slight, and that their mean values might be determined with
great confidence by the methods of Prop.~\textsc{iii}. Of those methods
I should be disposed to give the preference to the first. Such a
principle of mean solution having been agreed upon, other considerations seem to indicate that the values of $c$ and $k$ for tribunals
and assemblies possessing a definite constitution, and governed
in their deliberations by fixed rules, would remain nearly constant, subject, however, to a small secular variation, dependent
upon the progress of knowledge and of justice among mankind.
There exist at present few, if any, data proper for their determination.
%-----------------------File: 202.png----------------------------




%CHAPTER XXII.

\chapter[CONSTITUTION OF THE
INTELLECT]{\large ON THE NATURE OF SCIENCE, AND THE CONSTITUTION OF THE
INTELLECT.}

1. What I mean by the constitution of a system is the
aggregate of those causes and tendencies which produce
its observed character, when operating, without interference,
under those conditions to which the system is conceived to be
adapted. Our judgment of such adaptation must be founded
upon a study of the circumstances in which the system attains its
freest action, produces its most harmonious results, or fulfils in
some other way the apparent design of its construction. There
are cases in which we know distinctly the causes upon which the
operation of a system depends, as well as its conditions and its
end. This is the most perfect kind of knowledge relatively to
the subject under consideration. There are also cases in which
we know only imperfectly or partially the causes which are at
work, but are able, nevertheless, to determine to some extent
the laws of their action, and, beyond this, to discover general
tendencies, and to infer ulterior purpose. It has thus, I think
rightly, been concluded that there is a moral faculty in our nature,
not because we can understand the special instruments by
which it works, as we connect the organ with the faculty of sight,
nor upon the ground that men agree in the adoption of universal
rules of conduct; but because while, in some form or other, the
sentiment of moral approbation or disapprobation manifests itself
in all, it tends, wherever human progress is observable, wherever
society is not either stationary or hastening to decay, to attach
itself to certain classes of actions, consentaneously, and after a
manner indicative both of permanency and of law. Always and
everywhere the manifestation of Order affords a presumption, not
measurable indeed, but real (XX.~22), of the fulfilment of an end
or purpose, and the existence of a ground of orderly causation.

%-----------------------File: 203.png----------------------------
2. The particular question of the constitution of the intellect
has, it is almost needless to say, attracted the efforts of speculative
ingenuity in every age. For it not only addresses itself to that
desire of knowledge which the greatest masters of ancient thought
believed to be innate in our species, but it adds to the ordinary
strength of this motive the inducement of a human and personal
interest. A genuine devotion to truth is, indeed, seldom partial
in its aims, but while it prompts to expatiate over the fair fields of
outward observation, forbids to neglect the study of our own faculties.
Even in ages the most devoted to material interests,
some portion of the current of thought has been reflected inwards,
and the desire to comprehend that by which all else is
comprehended has only been baffled in order to be renewed.

It is probable that this \emph{pertinacity} of effort would not have
been maintained among sincere inquirers after truth, had the
conviction been general that such speculations are hopelessly
barren. We may conceive that it has been felt that if something
of error and uncertainty, always incidental to a state of partial
information, must ever be attached to the results of such inquiries,
a residue of positive knowledge may yet remain; that
the contradictions which are met with are more often verbal than
real; above all, that even \emph{probable} conclusions derive here an interest
and a value from their subject, which render them not
unworthy to claim regard beside the more definite and more
splendid results of physical science. Such considerations seem
to be perfectly legitimate. Insoluble as many of the problems
connected with the inquiry into the nature and constitution of
the mind must be presumed to be, there are not wanting others
upon which a limited but not doubtful knowledge, others upon
which the conclusions of a highly probable analogy, are attainable.
As the realms of day and night are not strictly conterminous,
but are separated by a crepuscular zone, through which the
light of the one fades gradually off into the darkness of the other,
so it may be said that every region of positive knowledge lies surrounded
by a debateable and speculative territory, over which it
in some degree extends its influence and its light. Thus there
may be questions relating to the constitution of the intellect
which, though they do not admit, in the present state of
%-----------------------File: 204.png----------------------------
knowledge, of an absolute decision, may receive so much of reflected
information as to render their probable solution not difficult; and
there may also be questions relating to the nature of science, and
even to particular truths and doctrines of science, upon which
they who accept the general principles of this work cannot but be
led to entertain positive opinions, differing, it may be, from those
which are usually received in the present day.%
\footnote{The following illustration may suffice:--

It is maintained by some of the highest modern authorities in grammar that
conjunctions connect propositions only. Now, without inquiring directly whether this opinion is sound or not, it is obvious that it cannot consistently be held
by any who admit the scientific principles of this treatise; for to such it would
seem to involve a denial, either, $1$st, of the possibility of \emph{performing}, or $2$ndly, of
the possibility of \emph{expressinq}, a mental operation, the laws of which, viewed in
both these relations, have been investigated and applied in the present work---
(Latham on the English Language; Sir John Stoddart's Universal Grammar, \&c.)}%[endfootnote]
\ In what follows I shall recapitulate some of the more definite conclusions
established in the former parts of this treatise, and shall then
indicate one or two trains of thought, connected with the general objects above adverted to, which they seem to me calculated
to suggest.

3. Among those conclusions, relating to the intellectual constitution, which may be considered as belonging to the realm of
positive knowledge, we may reckon the scientific laws of thought
and reasoning, which have formed the basis of the general methods of this treatise, together with the principles, Chap,~\textsc{v.}, by
which their application has been determined. The resolution of
the domain of thought into two spheres, distinct but coexistent
(IV.~XI.); the subjection of the intellectual operations within
those spheres to a common system of laws (XI.); the general
mathematical character of those laws, and their actual expression
(II.~III.); the extent of their affinity with the laws of thought in
the domain of number, and the point of their divergence therefrom; the dominant character of the two limiting conceptions of
universe and eternity among all the subjects of thought with
which Logic is concerned; the relation of those conceptions to
the fundamental conception of unity in the science of number,---
these, with many similar results, are not to be ranked as merely
%-----------------------File: 205.png----------------------------
probable or analogical conclusions, but are entitled to be regarded as truths of science. Whether they be termed metaphysical or not, is a matter of indifference. The nature of the
evidence upon which they rest, though in kind distinct, is not
inferior in value to any which can be adduced in support of the
general truths of physical science.

Again, it is agreed that there is a certain order observable in the progress of all the exacter forms of knowledge.
The study of every department of physical science begins with
observation, it advances by the collation of facts to a presumptive acquaintance with their connecting law, the validity of
such presumption it tests by new experiments so devised as to
augment, if the presumption be well founded, its probability indefinitely; and finally, the law of the ph{\ae}nomenon having been
with sufficient confidence determined, the investigation of causes,
conducted by the due mixture of hypothesis and deduction,
crowns the inquiry. In this advancing order of knowledge, the
particular faculties and laws whose nature has been considered
in this work bear their part. It is evident, therefore, that if we
would impartially investigate either the nature of science, or
the intellectual constitution in its relation to science, no part of
the two series above presented ought to be regarded as isolated.
More especially ought those truths which stand in any kind of
\emph{supplemental} relation to each other to be considered in their mutual bearing and connexion.

4. Thus the necessity of an experimental basis for all positive
knowledge, viewed in connexion with the existence and the
peculiar character of that system of mental laws, and principles,
and operations, to which attention has been directed, tends to
throw light upon some important questions by which the world
of speculative thought is still in a great measure divided. How,
from the particular facts which experience presents, do we arrive
at the general propositions of science? What is the nature of
these propositions? Are they solely the collections of experience, or does the mind supply some connecting principle of its
own? In a word, what is the nature of scientific truth, and
what are the grounds of that confidence with which it claims to
be received?
%-----------------------File: 206.png----------------------------

That to such questions as the above, no single and general
answer can be given, must be evident. There are cases in which
they do not even need discussion. Instances are familiar, in
which general propositions merely express \textit{per enumerationem
simplicem}, a fact established by actual observation in all the
cases to which the proposition applies. The astronomer asserts upon this ground, that all the known planets move from
west to east round the sun. But there are also cases in which
general propositions are assumed from observation of their truth
in particular instances, and extension of that truth to instances
unobserved. No principle of merely deductive reasoning can
warrant such a procedure. When from a large number of observations on the planet Mars, Kepler inferred that it revolved
in an ellipse, the conclusion was larger than his premises, or indeed than any premises which mere observation could give.
What other element, then, is necessary to give even a prospective
validity to such generalizations as this? It is the ability inherent in our nature to appreciate Order, and the concurrent presumption, however founded, that the ph{\ae}nomena of Nature are
connected by a principle of Order. Without these, the general
truths of physical science could never have been ascertained.
Grant that the procedure thus established can only conduct us
to probable or to approximate results; it only follows, that the
larger number of the generalizations of physical science possess
but a probable or approximate truth. The security of the tenure
of knowledge consists in this, that wheresoever such conclusions
do truly represent the constitution of Nature, our confidence in
their truth receives indefinite confirmation, and soon becomes
undistinguishable from certainty. The existence of that principle above represented as the basis of inductive reasoning
enables us to solve the much disputed question as to the necessity of general propositions in reasoning. The logician affirms,
that it is impossible to deduce any conclusion from particular
premises. Modern writers of high repute have contended, that
all reasoning is from particular to particular truths. They instance, that in concluding from the possession of a property by
certain members of a class, its possession by some other member,
it is not necessary to establish the intermediate \emph{general}
%-----------------------File: 207.png----------------------------
conclusion which affirms its possession by \emph{all} the members of the class
in common. Now whether it is so or not, that principle of
order or analogy upon which the reasoning is conducted must
either be stated or apprehended as a general truth, to give validity to the final conclusion. In this form, at least, the necessity
of general propositions as the basis of inference is confirmed,---a
necessity which, however, I conceive to be involved in the very
existence, and still more in the peculiar \emph{nature}, of those faculties
whose laws have been investigated in this work. For if the process of reasoning be carefully analyzed, it will appear that abstraction is made of all peculiarities of the individual to which
the conclusion refers, and the attention confined to those properties by which its membership of the class is defined.

5. But besides the general propositions which are derived by
induction from the collated facts of experience, there exist others
belonging to the domain of what is termed \emph{necessary} truth. Such
are the general propositions of Arithmetic, as well as the propositions expressing the laws of thought upon which the general
methods of this treatise are founded; and these propositions
are not only capable of being rigorously verified in particular
instances, but are made manifest in all their generality from the
study of particular instances. Again, there exist general propositions expressive of necessary truths, but incapable, from the
imperfection of the senses, of being exactly verified. Some, if
not all, of the propositions of Geometry are of this nature; but
it is not in the region of Geometry alone that such propositions
are found. The question concerning their nature and origin
is a very ancient one, and as it is more intimately connected
with the inquiry into the constitution of the intellect than any
other to which allusion has been made, it will not be irrelevant
to consider it here. Among the opinions which have most
widely prevailed upon the subject are the following. It has
been maintained, that propositions of the class referred to exist
in the mind independently of experience, and that those conceptions which are the subjects of them are the imprints of eternal
archetypes. With such archetypes, conceived, however, to possess a reality of which all the objects of sense are but a faint
shadow or dim suggestion, Plato furnished his ideal world. It
%-----------------------File: 208.png----------------------------
has, on the other hand, been variously contended, that the
subjects of such propositions are copies of individual objects of
experience; that they are mere names; that they are individual
objects of experience themselves; and that the propositions which
relate to them are, on account of the imperfection of those objects,
but partially true; lastly, that they are intellectual products
formed by abstraction from the sensible perceptions of individual
things, but so formed as to become, what the individual things
never can be, subjects of science, i.e. subjects concerning which
exact and general propositions may be affirmed. And there exist, perhaps, yet other views, in some of which the sensible, in
others the intellectual or ideal, element predominates.

Now if the last of the views above adverted to be taken (for
it is not proposed to consider either the purely ideal or the
purely nominalist view) and if it be inquired what, in the
sense above stated, are the proper objects of science, objects in
relation to which its propositions are true without any mixture
of error, it is conceived that but one answer can be given. It
is, that neither do individual objects of experience, nor with all
probability do the mental images which they suggest, possess
any strict claim to this title. It seems to be certain, that neither
in nature nor in art do we meet with anything absolutely agreeing
with the geometrical definition of a straight line, or of a triangle,
or of a circle, though the deviation therefrom may be inappreciable by sense; and it may be conceived as at least doubtful,
whether we can form a perfect mental image, or conception, with
which the agreement shall be more exact. But it is not doubtful
that such conceptions, however imperfect, do point to something
beyond themselves, in the gradual approach towards which all
imperfection tends to disappear. Although the perfect triangle,
or square, or circle, exists not in nature, eludes all our powers of
\emph{representative} conception, and is presented to us in thought
only, as the limit of an indefinite process of abstraction, yet, by
a wonderful faculty of the understanding, it may be made the
subject of propositions which are \emph{absolutely} true. The domain of
reason is thus revealed to us as larger than that of imagination.
Should any, indeed, think that we are able to picture to ourselves,
with rigid accuracy, the scientific elements of form, direction,
%-----------------------File: 209.png----------------------------
magnitude, \&c., these things, as actually conceived, will, in the view
of such persons, be the proper objects of science. But if, as
seems to me the more just opinion, an incurable imperfection
attaches to all our attempts to realize with precision these elements, then we can only affirm, that the more external objects
do approach in reality, or the conceptions of fancy by abstraction,
to certain \emph{limiting} states, never, it may be, actually attained, the
more do the general propositions of science concerning those
things or conceptions approach to absolute truth, the actual deviation therefrom tending to disappear. To some extent, the same
observations are applicable also to the physical sciences. What
have been termed the \lq\lq fundamental ideas\rq\rq\ of those sciences as
force, polarity, crystallization, \&c.,%
\footnote{Whewell's Philosophy of the Inductive Sciences, pp. 71, 77, 213.
}%endfootnote
 are neither, as I conceive,
intellectual products independent of experience, nor mere copies
of external things; but while, on the one hand, they have a necessary
antecedent in experience, on the other hand they require
for their formation the exercise of the power of abstraction, in
obedience to some general faculty or disposition of our nature,
which ever prompts us to the research, and qualifies us for the
appreciation, of order.%
\footnote{Of the idea of order it has been profoundly said, that it carries within itself
its own justification or its own control, the very trustworthiness of our faculties
being judged by the conformity of their results to an order which satisfies the
reason. \lq\lq L'id\'ee de l'ordre a cela de singulier et d'eminent, qu'elle porte en elle
m\^eme sa justification ou son contr\^ole. Pour trouver si nos autres facult\'es nous
trompent ou nous ne trompent pas, nous examinons si les notions qu'elles nous
donnent s'encha\^inent on ne s'encha\^inent pas suivant un ordre qui satisfasse la
raison.''---\textit{Cournot, Essai sur les fondements de nos Connaissances}.
Admitting this principle as the guide of those powers of abstraction which we undoubtedly
possess, it seems unphilosophical to assume that the fundamental ideas of the
sciences are not derivable from experience. Doubtless the capacities which
have been given to us for the comprehension of the actual world would avail us
in a differently constituted scene, if in some form or other the dominion of
order was still maintained. It is conceivable that in such a new theatre
of speculation, the laws of the intellectual procedure remaining the same,
the fundamental ideas of the sciences might be wholly different from those
with which we are at present acquainted.
}%endfootnote
 Thus we study approximately the effects
of gravitation on the motions of the heavenly bodies, by a reference
to the \emph{limiting} supposition, that the planets are perfect
%-----------------------File: 210.png----------------------------
spheres or spheroids. We determine approximately the path
of a ray of light through the atmosphere, by a process in which
abstraction is made of all disturbing influences of temperature.
And such is the order of procedure in all the higher walks of
human knowledge. Now what is remarkable in connexion with
these processes of the intellect is the disposition, and the corresponding ability, to ascend from the imperfect representations
of sense and the diversities of individual experience, to the perception of general, and it may be of immutable truths. Whereever this disposition and this ability unite, each series of connected facts in nature may furnish the intimations of an order
more exact than that which it directly manifests. For it may
serve as ground and occasion for the exercise of those powers,
whose office it is to apprehend the general truths which are indeed exemplified, but never with perfect fidelity, in a world of
changeful ph{\ae}nomena.

6. The truth that the ultimate laws of thought are mathematical in their form, viewed in connexion with the fact of the
possibility of error, establishes a ground for some remarkable conclusions. If we directed our attention to the scientific truth
alone, we might be led to infer an almost exact parallelism between the intellectual operations and the movements of external
nature. Suppose any one conversant with physical science, but
unaccustomed to reflect upon the nature of his own faculties, to
have been informed, that it had been proved, that the laws of
those faculties were mathematical; it is probable that after the
first feelings of incredulity had subsided, the impression would
arise, that the order of thought must, \emph{therefore}, be as necessary as that of the material universe. We know that in the
realm of natural science, the absolute connexion between the
initial and final elements of a problem, exhibited in the mathematical form, fitly symbolizes that physical necessity which binds
together effect and cause. The necessary sequence of states and
conditions in the inorganic world, and the necessary connexion
of premises and conclusion in the processes of exact demonstration thereto applied, seem to be co-ordinate. It may possibly be
a question, to which of the two series the primary application of
the term \lq\lq necessary\rq\rq\ is due; whether to the observed constancy of
%-----------------------File: 211.png----------------------------
Nature, or to the indissoluble connexion of propositions in all valid
reasoning upon her works. Historically we should perhaps give
the preference to the former, philosophically to the latter view.
But the fact of the connexion is indisputable, and the analogy to
which it points is obvious.

Were, then, the laws of valid reasoning uniformly obeyed, a
very close parallelism would exist between the operations of the
intellect and those of external Nature. Subjection to laws mathematical
in their form and expression, even the subjection of
an absolute obedience, would stamp upon the two series one
common character. The reign of necessity over the intellectual
and the physical world would be alike complete and universal.

But while the observation of external Nature testifies with
ever-strengthening evidence to the fact, that uniformity of
operation and unvarying obedience to appointed laws prevail
throughout her entire domain, the slightest attention to the processes
of the intellectual world reveals to us another state of
things. The mathematical laws of reasoning are, properly speaking,
the laws of \emph{right} reasoning only, and their actual transgression
is a perpetually recurring phenomenon. Error, which has
no place in the material system, occupies a large one here. We
must accept this as one of those ultimate facts, the origin of which
it lies beyond the province of science to determine. We must
admit that there exist laws which even the rigour of their mathematical
forms does not preserve from violation. We must
ascribe to them an authority the essence of which does not consist
in power, a supremacy which the analogy of the inviolable
order of the natural world in no way assists us to comprehend.

As the distinction thus pointed out is \emph{real}, it remains unaffected
by any peculiarity in our views respecting other portions
of the mental constitution. If we regard the intellect as free,
and this is apparently the view most in accordance with the general
spirit of these speculations, its freedom must be viewed as
opposed to the dominion of necessity, not to the existence of a
certain just supremacy of truth. The laws of correct inference
may be violated, but they do not the less truly \emph{exist} on this account.
Equally do they remain unaffected in character and authority
if the hypothesis of necessity in its extreme form be
%-----------------------File: 212.png----------------------------
adopted. Let it be granted that the laws of valid reasoning,
such as they are determined to be in this work, or, to speak more
generally, such as they would finally appear in the conclusions of
an exhaustive analysis, form but a \emph{part} of the system of laws by
which the actual processes of reasoning, whether right or wrong,
are governed. Let it be granted that if that system were known
to us in its completeness, we should perceive that the whole intellectual
procedure was \emph{necessary}, even as the movements of the
inorganic world are necessary. And let it finally, as a consequence
of this hypothesis, be granted that the ph{\ae}nomena of incorrect
reasoning or error, wheresoever presented, are due to the
interference of other laws with those laws of which \emph{right} reasoning
is the product. Still it would remain that there exist among
the intellectual laws a number marked out from the rest by this
special character, viz., that every movement of the intellectual
system which is accomplished solely under their direction is
\emph{right}, that every interference therewith by other laws is not interference
only, but \emph{violation}. It cannot but be felt that this
circumstance would give to the laws in question a character of
distinction and of predominance. They would but the more
evidently seem to indicate a final purpose which is not always
fulfilled, to possess an authority inherent and just, but not
always commanding obedience.

Now a little consideration will show that there is nothing
analogous to this in the government of the world by natural law.
The realm of inorganic Nature admits neither of preference nor
of distinctions. We cannot separate any portion of her laws
from the rest, and pronounce them alone worthy of obedience,---alone
charged with the fulfilment of her highest purpose. On
the contrary, all her laws seem to stand co-ordinate, and the
larger our acquaintance with them, the more necessary does their
united action seem to the harmony and, so far as we can comprehend
it, to the general \emph{design} of the system. How often the
most signal departures from apparent order in the inorganic
world, such as the perturbations of the planetary system, the interruption
of the process of crystallization by the intrusion of a
foreign force, and others of a like nature, either merge into the
conception of some more exalted scheme of order, or lose to a
%-----------------------File: 213.png----------------------------
more attentive and instructed gaze their abnormal aspect, it is
needless to remark. One explanation only of these facts can be
given, viz., that the distinction between \emph{true} and \emph{false}, between
\emph{correct} and \emph{incorrect}, exists in the processes of the intellect, but
not in the region of a physical necessity. As we advance from
the lower stages of organic being to the higher grade of conscious
intelligence, this contrast gradually dawns upon us. Wherever
the ph{\ae}nomena of life are manifested, the dominion of rigid law
in some degree yields to that mysterious principle of activity.
Thus, although the structure of the animal tribes is conformable
to certain general types, yet are those types sometimes, perhaps,
in relation to the highest standards of beauty and proportion,
always, imperfectly realized. The two alternatives, between
which Art in the present day fluctuates, are the exact imitation
of individual forms, and the endeavour, by abstraction from all
such, to arrive at the conception of an ideal grace and expression,
never, it may be, perfectly manifested in forms of earthly mould.
Again, those teleological adaptations by which, without the organic
type being sacrificed, species become fitted to new conditions
or abodes, are but slowly accomplished,---accomplished,
however, not, apparently, by the fateful power of external circumstances,
but by the calling forth of an energy from within.
Life in all its forms may thus be contrasted with the passive fixity
of inorganic nature. But inasmuch as the perfection of the types
in which it is corporeally manifested is in some measure of an
ideal character, inasmuch as we cannot precisely define the
highest \emph{suggested} excellency of form and of adaptation, the contrast
is less marked here than that which exists between the intellectual
processes and those of the purely material world. For
the definite and technical character of the mathematical laws by
which both are governed, places in stronger light the fundamental
difference between the kind of authority which, in their capacity
of government, they respectively exercise.

7. There is yet another instance connected with the general
objects of this chapter, in which the collation of truths or facts,
drawn from different sources, suggests an instructive train of reflection.
It consists in the comparison of the laws of thought, in
their scientific expression, with the actual forms which physical
%-----------------------File: 214.png----------------------------
speculation in early ages, and metaphysical speculation in all
ages, have tended to assume. There are two illustrations of this
remark, to which, in particular, I wish to direct attention here.

1st. It has been shown (III.~13) that there is a scientific
connexion between the conceptions of unity in Number, and the
universe in Logic. They occupy in their respective systems the
same relative place, and are subject to the same formal laws.
Now to the Greek mind, in that early stage of activity,---a stage
not less marked, perhaps not less \emph{necessary}, in the progression of
the human intellect, than the era of Bacon or of Newton,---when
the great problems of Nature began to unfold themselves, while
the means of observation were as yet wanting, and its necessity
not understood, the terms \lq\lq Universe\rq\rq\  and \lq\lq The One\rq\rq\  seem to
have been regarded as almost identical. To assign the nature of
that unity of which all existence was thought to be a manifestation, was the first aim of philosophy.%
\footnote{See various passages in Aristotle's Metaphysics, Book\textsc{i.}
}%endfootnote
\ Thales sought for this
fundamental unity in water. Anaximenes and Diogenes conceived it to be air. Hippasus of Metapontum, and Heraclitus
the Ephesian, pronounced that it was fire. Less definite or
less confident in his views, Parmenides simply declared that all
existing things were One; Melissus that the Universe was infinite, unsusceptible of change or motion, One, like to itself, and
that motion was not, but seemed to be.% **[Oy vey!]
% Yeah.. BTW greek is available only in math mode, which ignores spaces, and \' style of accenting is not available in math mode.
% The original seems to have dropped an accent which was added here.. "kai" has accent on i everywhere except on the second line.
% \omicron does not exist, the non greek plain o is used instead
% \iota has no dot, but in some places there is what appears a bar over a dotted i, in which case \bar{i} is used instead of the greek \bar{\iota}
\footnote{'$E\delta\acute{o}\kappa\varepsilon\iota\; \delta\grave{\varepsilon}\; \alpha\acute{\upsilon}\tau\tilde{\psi}\; \tau\grave{o}\; \pi\tilde{\alpha}\nu\; \tilde{\alpha}\pi\varepsilon\iota\rho{o}\nu\; \varepsilon\bar{i}\nu\alpha\iota,\; \kappa\alpha\grave{\iota}\; \acute{\alpha}\nu\alpha\lambda\lambda{o}\acute{\iota}\omega\tau{o}%
\nu,\; \kappa\alpha\grave{\iota}\; \acute{\alpha}\kappa\acute{\iota}\nu\eta\tau{o}\nu,\; \kappa\alpha\grave{\iota}$
%
 $\ddot{\varepsilon}\nu,\; \ddot{o}\mu{o}\iota{o}\nu\; \grave{\varepsilon}\alpha\upsilon\tau\tilde{\psi}\; \kappa\alpha\grave{\iota}\; \pi\lambda\tilde{\eta}\rho\varepsilon\varsigma.\; \kappa\iota\nu\eta\sigma\acute{\iota}\nu\; \tau\varepsilon\; \mu\grave{\eta}\; \varepsilon\bar{i}\nu\alpha\iota\;
 \delta{o}\kappa\varepsilon\tilde{\iota}\nu\; \delta\grave{\varepsilon}\; \varepsilon\bar{i}\nu\alpha\iota.$
---\textit{Diog. Laert.}~\textsc{ix.}
cap. 4.
% [DPStyleGreek: Edokei de autps to pan apeiron einai,
% kai analloioton, kai akinaeton, kai
% hen, homoion heutps kai plaeres. kinaesin te mae einai
% dokein de einai.]
}%endfootnote
\ In a spirit which, to the
reflective mind of Aristotle, appeared sober when contrasted
with the rashness of previous speculation, Anaxagoras of Clazomen{\ae}, following, perhaps, the steps of his fellow-citizen, Hermotimus, sought in Intelligence the cause of the world and of its
order.%
\footnote{$N{o}\tilde{\upsilon}\nu\; \delta\acute{\eta}\; \tau\iota\varsigma\; \varepsilon\acute{\iota}\pi\grave{\omega}\nu\; \grave{\varepsilon}\nu\varepsilon\tilde{\iota}\nu\alpha\iota,\; \kappa\alpha\theta\acute{\alpha}\pi\varepsilon\rho\; \grave{\varepsilon}\nu\; \tau{o}\tilde{\iota}\varsigma\; \zeta\acute{\psi}{o}\iota\varsigma,\; \kappa\alpha\grave{\iota}\; \grave{\varepsilon}\nu\; \tau\tilde{\eta}\; \phi\acute{\upsilon}\sigma\varepsilon\iota,\; \tau\grave{o}\nu$
%
 $\alpha\ddot{\iota}\tau\iota{o}\nu\; \tau{o}\tilde{\upsilon}\; \kappa\acute{o}\sigma\mu{o}\upsilon\; \kappa\alpha\grave{\iota}\; \tau\tilde{\eta}\varsigma\;
\tau\acute{\alpha}\xi\varepsilon\omega\varsigma\; \pi\acute{\alpha}\sigma\eta\varsigma\ {o}\bar{i}{o}\nu\; \nu\acute{\eta}\phi\omega\nu\; \grave{\varepsilon}\phi\acute{\alpha}\nu\eta\; \pi\alpha\rho\text{'}\; \varepsilon\acute{\iota}\kappa\tilde{\eta}\; \lambda\acute{\varepsilon}\gamma{o}\nu\tau\alpha\varsigma$
%
 $\tau{o}\grave{\upsilon}\varsigma\; \pi\rho\acute{o}\tau\underset{'}{\varepsilon}\rho{o}\nu.\; \Phi\alpha\nu\varepsilon\rho\tilde{\omega}\varsigma\; \mu\grave{\varepsilon}\nu\; {o}\tilde{\upsilon}\nu\; \text{'}A\nu\alpha\xi\alpha\gamma\acute{o}\rho\alpha\nu\; \acute{\iota}\sigma\mu\varepsilon\nu\; \grave{\alpha}\psi\acute{\alpha}\mu\varepsilon\nu{o}\nu\; \tau{o}\acute{\upsilon}\tau\omega\nu\; \tau\tilde{\omega}\nu\;$
%
$ \lambda\acute{o}\gamma\omega\nu,
 \alpha\acute{\iota}\tau\acute{\iota}\alpha\nu\; \delta\text{'}\; \ddot{\varepsilon}\chi\varepsilon\iota\; \pi\rho\acute{o}\tau\varepsilon\rho{o}\nu\; \text{'}E\rho\mu\acute{o}\tau\iota\mu{o}\varsigma\; \grave{o}\; K\lambda\alpha\zeta{o}\mu\acute{\varepsilon}\nu\iota{o}%
\varsigma\; \varepsilon\acute{\iota}\pi\varepsilon\tilde{\iota}\nu.$
---\textit{Arist. Met.}~\textsc{i.}~3.
%[DPStyleGreek: Noun dae tis eipon heneinai, kathaper hen tois
% zpsois, kai hen tae thusei, ton
% aition tou kosmou kai taes taxeos pasaes oion naephon ephanae
% par eikae legontas
% tous proteron. Phaneros men oun Anaxagoran ismen hapsamenon
% touton ton logon,
% aitian d' echei proteron Ermotimos o Klazomenios eipein.]
}%endfootnote
\ The pantheistic tendency which pervaded many of these
speculations is manifest in the language of Xenophanes, the
founder of the Eleatic school, who, \lq\lq surveying the expanse of
%-----------------------File: 215.png----------------------------
heaven, declared that the One was God.\rq\rq%
\footnote{$\Xi\varepsilon\nu{o}\phi\acute{\alpha}\nu\eta\varsigma\ \delta\grave{\varepsilon}\ \dotso\ \varepsilon\acute{\iota}\varsigma\ \tau\grave{o}\nu\ \ddot{o}\lambda{o}\nu\ {o}\acute{\upsilon}\rho\alpha\nu\grave{o}\nu\ \acute{\alpha}\pi{o}\beta\lambda\acute{\varepsilon}\psi\alpha\varsigma,\ \tau\grave{o}\ \ddot{\varepsilon}\nu\ \varepsilon\bar{i}\nu\alpha\iota\ \phi\eta\sigma\iota\ \tau\grave{o}\nu\
\theta\varepsilon\acute{o}\nu.$
---\textit{Ib.}
%[DPStyleGreek: Xenophanaes de ... eis ton holon ouranon apoblepsas,
% to hen einai phaesi ton
% theon.]
}%endfootnote
\ Perhaps there are few,
if any, of the forms in which unity can be conceived, in the abstract as numerical or rational, in the concrete as a passive substance, or a central and living principle, of which we do not
meet with applications in these ancient doctrines. The writings
of Aristotle, to which I have chiefly referred, abound with allusions of this nature, though of the larger number of those who
once addicted themselves to such speculations, it is probable that
the very names have perished. Strange, but suggestive truth,
that while Nature in all but the aspect of the heavens must have
appeared as little else than a scene of unexplained disorder, while
the popular belief was distracted amid the multiplicity of its gods,
---the conception of a primal unity, if only in a rude, material form,
should have struck deepest root; surviving in many a thoughtful breast the chills of a lifelong disappointment, and an endless
search!%
\footnote{The following lines, preserved by Sextus Empiricus, and ascribed to Timon
the Sillograph, are not devoid of pathos:---
\begin{verse}
$\acute{\omega}\varsigma\ \kappa\alpha\grave{\iota}\ \acute{\varepsilon}\gamma\grave{\omega}\nu\ \ddot{o}\phi\varepsilon\lambda{o}\nu\ \pi\upsilon\kappa\iota\nu{o}\tilde{\upsilon}\ \nu\acute{o}{o}\upsilon\ \acute{\alpha}\nu\tau\iota\beta{o}\lambda\tilde{\eta}\sigma\alpha\iota
$\\$
\acute{\alpha}\mu\phi{o}\tau\varepsilon\rho\acute{o}\beta\lambda%
\varepsilon\pi\tau{o}\varsigma\ (\delta{o}\lambda\acute{\iota}\eta\ \delta\acute{}\ \grave{o}\ddot{o}\tilde{\psi}\ \acute{\varepsilon}\xi\varepsilon\pi\alpha\tau\acute{\eta}\theta\eta\nu,
$\\$
\pi\rho\varepsilon\sigma\beta\upsilon\gamma\varepsilon\nu\grave{\eta}%
\varsigma\ \ddot{\varepsilon}\tau\ \grave{\varepsilon}\grave{\omega}\nu)\ \kappa\alpha\grave{\iota}\ \grave{\alpha}\nu\alpha\mu\phi\acute{\eta}\rho\iota\sigma\tau{o}%
\varsigma\ \grave{\alpha}\pi\acute{\alpha}\sigma\eta\varsigma
$\\$
\sigma\kappa\varepsilon\pi\tau{o}\sigma\acute{\upsilon}\nu\eta%
\varsigma^{\centerdot}\ \ddot{o}\pi\pi\eta\ \gamma\grave{\alpha}\rho\ \acute{\varepsilon}\mu\grave{o}\nu\ \nu\acute{o}{o}\nu\ \varepsilon\acute{\iota}\rho\acute{\upsilon}\sigma\alpha\iota\mu\iota,
$\\$
\varepsilon\acute{\iota}\varsigma\ \ddot{\varepsilon}\nu\ \tau\text{'}\ \alpha\acute{\upsilon}\tau\grave{o}\ \tau\varepsilon\ \pi\tilde{\alpha}\nu\ \acute{\alpha}\nu\acute{\varepsilon}\lambda\upsilon\varepsilon\tau{o}.$
\end{verse}
% [DPStyleGreek:
% os kai egon ophelon pukinou noon antibolaesai
% amphoterobleptos (doliae d' hoops exepataethaen,
% presbugenaes et heon) kai hanamphaeristos hapasaes
% skeptosunaes hoppae gar emon noon eirusaimi,
% eis hen t' auto te pan anelueto. ]
%
I quote them from Ritter, and venture to give the following version:---
\begin{verse}
Be mine, to partial views no more confin'd\\
Or sceptic doubts, the truth-illumin'd mind!\\
For, long deceiv'd, yet still on Truth intent,\\
Life's waning years in wand'rings wild are spent.\\
Still restless thought the same high quest essays,\\
And still the One, the All, eludes my gaze.
\end{verse}
}%endfootnote

2ndly. In equally intimate alliance with that law of thought
which is expressed by an equation of the second degree, and
which has been termed in this treatise the law of duality, stands
the tendency of ancient thought to those forms of philosophical
speculation which are known under the name of dualism. The
theory of Empedocles,\footnote{Arist. Met. \textsc{i}. 4. 6.}
which explained the apparent contradictions of nature by referring them
to the two opposing principles
%-----------------------File: 216.png----------------------------
of \lq\lq strife\rq\rq\ and \lq\lq friendship;\rq\rq\ and the theory of Leucippus,%
\footnote{Arist. Met. \textsc{i.} 4, 9.}%endfootnote
\ which resolved all existence into the two elements of a \emph{plenum}
and a vacuum, are of this nature. The famous comparison of the
universe to a lyre or a bow,%
\footnote{$ \pi\alpha\lambda\acute{\iota}\nu\tau\rho{o}\pi{o}\varsigma\ \grave{\alpha}\rho\mu{o}\nu\acute{\iota}\eta\  \ddot{o}\kappa\omega\varsigma\ \pi\varepsilon\rho\ \tau\acute{o}\xi{o}\upsilon\ \kappa\alpha\grave{\iota}\ \lambda\acute{\upsilon}\rho\eta\varsigma.
$---\textit{Heraclitus}, quoted in
\textit{Origenis Philosophumena}, \textsc{ix.}~9. Also \textit{Plutarch, De Iside et Osiride.}
%
%DPStyleGreek: palintropos harmoniae okos per toxou kai lyraes.
}%endfootnote
\ its \lq\lq recurrent harmony\rq\rq\ being the
product of opposite states of tension, betrays the same origin.
In the system of Pythagoras, which seems to have been a combination of dualism with other elements derived from the study of
numbers, and of their relations, ten fundamental antitheses are
recognised: finite and infinite, even and odd, unity and multitude,
right and left, male and female, rest and motion, straight and
curved, light and darkness, good and evil, the square and the
oblong. In that of Alcm{\ae}on the same fundamental dualism is
accepted, but without the definite and numerical limitation with
which it is connected in the Pythagorean system. The grand
development of this idea is, however, met with in that ancient
Manich{\ae}an doctrine, which not only formed the basis of the religious system of Persia, but spread widely through other regions
of the East, and became memorable in the history of the Christian
Church. The origin of dualism as a speculative opinion, not
yet connected with the personification of the Evil Principle, but
naturally succeeding those doctrines which had assumed the
primal unity of Nature, is thus stated by Aristotle:---\lq\lq Since
there manifestly existed in Nature things opposite to the good,
and not only order and beauty, but also disorder and deformity;
and since the evil things did manifestly preponderate in number
over the good, and the deformed over the beautiful, some one
else at length introduced strife and friendship as the respective
causes of these diverse ph{\ae}nornena.\rq\rq%
\footnote{'$E\pi\varepsilon\grave{\iota}\ \delta\grave{\varepsilon}\ \kappa\alpha\grave{\iota}\ \tau\grave{\alpha}\nu\alpha\nu\tau\acute{\iota}\alpha\ \tau{o}\tilde{\iota}\varsigma\ \acute{\alpha}\gamma\alpha\theta{o}\tilde{\iota}\varsigma\ \acute{\varepsilon}\nu\acute{o}\nu\tau\alpha\ \acute{\varepsilon}\phi\alpha\acute{\iota}\nu\varepsilon\tau{o}\ \acute{\varepsilon}\nu\ \tau\tilde{\eta}\ \phi\acute{\upsilon}\sigma\varepsilon\iota,\ \kappa\alpha\grave{\iota}\ {o}\acute{\upsilon}
$ $
\mu\acute{o}\nu{o}\nu\ \tau\acute{\alpha}\xi\iota\varsigma\ \kappa\alpha\grave{\iota}\ \tau\grave{o}\ \kappa\alpha\lambda\grave{o}\nu\ \acute{\alpha}\lambda\lambda\grave{\alpha}\ \kappa\alpha\grave{\iota}\ \acute{\alpha}\tau\alpha\xi\acute{\iota}\alpha\ \kappa\alpha\grave{\iota}\ \tau\grave{o}\ \alpha\acute{\iota}\sigma\chi\rho\acute{o}\nu,\ \kappa\alpha\grave{\iota}\ \pi\lambda\varepsilon\acute{\iota}\omega\ \tau\grave{\alpha}\ \kappa\alpha\kappa\grave{\alpha}\
$ $
\tau\tilde{\omega}\nu\ \acute{\alpha}\gamma\alpha\theta\tilde{\omega}\nu\ \kappa\alpha\grave{\iota}\ \tau\grave{\alpha}\ \phi\alpha\tilde{\upsilon}\lambda\alpha\ \tau\tilde{\omega}\nu\ \kappa\alpha\lambda\tilde{\omega}\nu,\ {o}\ddot{\upsilon}\tau\omega\varsigma\ \ddot{\alpha}\lambda\lambda{o}\varsigma\ \tau\iota\varsigma\ \phi\iota\lambda\acute{\iota}\alpha\nu\ \varepsilon\acute{\iota}\sigma\acute{\eta}\nu\varepsilon\gamma\kappa%
\varepsilon\ \kappa\alpha\grave{\iota}\ \nu\varepsilon\tilde{\iota}\kappa{o}\varsigma,
$ $
\grave{\varepsilon}\kappa\acute{\alpha}\tau\varepsilon\rho{o}\nu\ \grave{\varepsilon}\kappa\alpha\tau\acute{\varepsilon}\rho\omega\nu\ \alpha\ddot{\iota}\tau\iota{o}\nu\ \tau{o}\acute{\upsilon}\tau\omega\nu.
$---\textit{Arist. Metaphysica}, \textsc{i.}~4.
%DPStyleGreek:
% Epei de kai tanantia tois agathois enouta ephaineto hen tae
% physei, kai ou
% monon taxis kai to kalon alla kai ataxia kai to aischron, kai
% pleio to kaka
% ton agathon kai ta phaula ton kalon, outos allos tis philian
% eisaenegke kai veikos,
% ekateron ekateron aition touton.
%
}%endfootnote
And in Greece, indeed,
it seems to have been chiefly as a philosophical opinion, or as an
adjunct to philosophical speculation, that the dualistic theory obtained ground.%
\footnote{Witness Aristotle's well-known derivation of the elements from the
qualities ''warm,'' and ''dry,'' and their contraries. It is characteristic that Plato
connects their generation with mathematical
principles.--\emph{Tim\ae us}, cap.~xi.}%endfootnote
The moral application of the doctrine most in
%-----------------------File: 217.png----------------------------
accordance with the Greek mind is preserved in the great Platonic
antithesis of ''being and non-being,''---the connexion of the
former with whatsoever is good and true, with the eternal ideas,
and the archetypal world: of the latter with evil, with error,
with the perishable ph\ae nomena of the present scene. The two
forms of speculation which we have considered were here blended
together; nor was it during the youth and maturity of Greek
philosophy alone that the tendencies of thought above described
were manifested. Ages of imitation caught up and adopted as
their own the same spirit. Especially wherever the genius of
Plato exercised sway was this influence felt. The unity of all
real being, its identity with truth and goodness considered
as to their essence; the illusion, the profound unreality, of all
merely ph\ae nomenal existence; such were the views,---such the
dispositions of thought, which it chiefly tended to foster. Hence
that strong tendency to mysticism which, when the days of renown,
whether on the field of intellectual or on that of social enterprise,
had ended in Greece, became prevalent in her schools
of philosophy, and reached their culminating point among the
Alexandrian Platonists. The supposititious treatises of Dionysius
the Areopagite served to convey the same influence, much modified
by its contact with Aristotelian doctrines, to the scholastic
disputants of the middle ages. It can furnish no just ground of
controversy to say, that the tone of thought thus encouraged was
as little consistent with genuine devotion as with a sober philosophy.
That kindly influence of human affections, that homely
intercourse with the common things of life, which form so large
a part of the true, because intended, discipline of our nature,
would be ill replaced by the contemplation even of the highest
object of thought, viewed by an excessive abstraction as something
concerning which not a single intelligible proposition could
either be affirmed or denied.%
\footnote{%\Alpha\upsilon\tau\omicron\varsigma\ \kappa\alpha\iota\ '\upsilon\pi\varepsilon\rho\ \theta\varepsilon\sigma\iota\nu\ \varepsilon\sigma\tau\iota\ \kappa\alpha\iota\ \alpha\phi\alpha\iota\rho\varepsilon\sigma\iota\nu.---\textit{Dion.\ Areop. De Divinis Nominibus}, cap.~\textsc{ii}.
%**DPstyle
\textgreek{Autos kai hyper thesin esti kai aphairesin.}
}%endfootnote
 I would but slightly allude to
those connected speculations on the Divine Nature which ascribed
%-----------------------File: 218.png----------------------------
to it the perfect union of opposite qualities,%
\footnote{See especially the lofty strain of Hildebert beginning ''Alpha et $\Omega$ magne
Deus.'' (Trench's Sacred Latin Poetry.) The principle upon which all these
speculations rest is thus stated in the treatise referred to in the last note.
%$O\upsilon\delta\varepsilon\nu\  o\upsilon\nu\ \alpha\tau o\pi o\nu, \varepsilon\xi\ \alpha\mu\upsilon\delta\rho\omega\nu\ \varepsilon\iota\kappa o\nu\omega\nu\ \varepsilon\pi\iota\ \tau o\ \pi\alpha\nu\tau\omega\nu\ \alpha\iota\tau\iota o\nu\ \alpha\nu\alpha\beta\alpha\nu\tau\alpha\varsigma, \upsilon\pi\varepsilon\rho\kappa o\sigma\mu\iota o\iota\varsigma
%
% o\phi\theta\alpha\lambda\mu o\iota\sigma\ \theta\varepsilon\omega\rho\eta\sigma\alpha\iota\ \pi\alpha\nu\tau\alpha\ \varepsilon\nu\ \tau\psi\ \pi\alpha\nu\tau\omega\nu\ \alpha\iota\tau\iota\psi, \kappa\alpha\iota\ \tau\alpha\ \alpha\lambda\lambda\eta\lambda o\iota\varsigma\ \varepsilon\nu\alpha\nu\tau\iota\alpha
%
%\upsilon o\nu o\varepsilon\iota\delta\omega\varsigma\ \kappa\alpha\iota\ \eta\nu\omega\mu\varepsilon\nu\omega\varsigma
%$
\textgreek{ Ouden oun hatopon, ex amydron eikonon epi to panton
aition anabantas, hyperkosmiois
 ophthalmois theoraesai panta en tps panton aitips, kai ta allaelois enantia
 monoeidos kai haeomenos}
.---\textit{De Divinis Nominibus}, cap.~\textsc{v}. And the kind of
knowledge which it is thus sought to attain is described as a ''darkness beyond
light,''
$\upsilon\pi\varepsilon\rho\phi\omega\tau o\varsigma\ \gamma\nu o\phi o\varsigma$.
%
%[DPStyle Greek: hyperphotos gnophos].
%
(\textit{De Mystica Theologia}, cap.~\textsc{i}.) Milton has a similar
thought---
\begin{quote}
{\centering ''Dark with excessive bright Thy skirts appear.''}\\
\hfill \textit{Par.~Lost}, Book~\textsc{iii}.
\end{quote}
Contrast with these the nobler simplicity of I~John, i.~5.
}%endfootnote
 or to the remarkable
treatises of Anselm, designed to establish a theory of the universe
upon the analogies of thought and being.%
\footnote{Monologium, Prosologium, and De Veritate.}
 The primal unity is
there represented as having its abode in the one eternal Truth.
The conformity of Nature to her laws, the obedience of moral
agents to the dictates of rectitude, are the same Truth seen in
action; the world itself being but an expression of the self-reflecting
thought of its Author.%
\footnote{''Idcirco cum ipse summus spiritus dicit seipsum dicit omnia qu\ae\ facta
sunt.''---\textit{Monolog}.~cap.~\textsc{xxiii}.
}%endfootnote
 Still more marked was the revival
of the older forms of speculation during the sixteenth and seventeenth
centuries. The friends and associates of Lorenzo the
Magnificent, the recluses known in England as the Cambridge
Platonists, together with many meditative spirits scattered
through Europe, devoted themselves anew, either to the task of
solving the ancient problem, De Uno, Vero, Bono, or to that of
proving that all such inquiries are futile and vain.%
\footnote{See dissertations in Spinoza, Picus of Mirandula, H.~More, \&c. Modern
discussions of this nature are chiefly in connexion with aesthetics, the ground of
the application being contained in the formula of Augustine: ''Omnis porro
pulchritudinis forma, unitas est.''}%endfootnote
 The logical
elements which underlie all these speculations, and from which
they appear to borrow at least their form, it would be easy to
trace in the outlines of more modern systems,--more especially
in that association of the doctrine of the absolute unity with the
distinction of the \emph{ego} and the \emph{non-ego} as the type of Nature,
which forms the basis of the philosophy of Hegel. The attempts
of speculative minds to ascend to some high pinnacle of truth,
from which they might survey the entire framework and
%-----------------------File: 219.png----------------------------
connexion of things \emph{in the order of deductive thought}, have differed
less in the forms of theory which they have produced, than
through the nature of the interpretations which have been assigned
to those forms.%
\footnote{For instance, the learned mysticism of Gioberti, widely as it differs in its
spirit and its conclusions from the pantheism of Hegel (both being, perhaps,
equally remote from truth), resembles it in applying both to \emph{thought} and
to \emph{being} the principles of unity and duality. It is asked:---''Or non \`{e} egli
chiaro che ogni discorso si riduce in fine in fine alle idee di Dio, del mondo, e
della creazione, l'ultima delle quali \`{e} il legame delle due prime?''
And this question
being affirmatively answered in the formula, ''l'Ente crea le esistenze,'' it
is said of that formula,---''Essa abbraccia la realt\`{a} universale nella dualit\`{a} del
necessario e del contingente, esprime il vincolo di questi due ordini, e collocandolo
nella creazion sostanziale, riduce la dualit\`{a} reale a un principio unico, all
unit\`{a} primordiale dell'Ente non astratto, complessivo, e generico, ma concreto,
individuato, assoluto, e creatore.''---\textit{Del Bello e del Buono}, pp.~30, 31.}%endfootnote
 And herein lies the real question as to
the influence of philosophical systems upon the disposition and
the life. For though it is of slight moment that men should
agree in tracing back all the forms and conditions of being to a
primal unity, it is otherwise as concerns their conceptions of
what that unity is, and what are the kinds of relation, beside
that of mere causality, which it sustains to themselves. Herein
too may be felt the powerlessness of mere Logic, the insufficiency
of the profoundest knowledge of the laws of the understanding,
to resolve those problems which lie nearer to our hearts, as progressive
years strip away from our life the illusions of its golden
dawn.

8. If the extremely arbitrary character of human opinion be
considered, it will not be expected, nor is it here maintained, that
the above are the only forms in which speculative men have
shaped their conjectural solutions of the problem of existence.
Under particular influences other forms of doctrine have arisen,
not unfrequently, however, masking those portrayed above.%
\footnote{Evidence in support of this statement will be found in the remarkable
treatise recently published under the title (the correctness of which seems doubtful)
of \textit{Origenis Philosophumena}. The early corruptions of Christianity of which
it contains the record, though many of them, as is evident from their Ophite
character, derived from the very dregs of paganism, manifest certain persistent
forms of philosophical speculation. For the most part they either belong to the
dualistic scheme, or recognise three principles, primary or derived, between two
of which the dualistic relation may be traced---\textit{Orig.~Phil.}, pp.~135, 139, 150,
235, 253, 264.}%endfootnote

%-----------------------File: 220.png----------------------------
But the wide prevalence of the particular theories which we have
considered, together with their manifest analogy with the expressed
laws of thought, may justly be conceived to indicate a
connexion between the two systems. As all other mental acts
and procedures are beset by their peculiar fallacies, so the operation
of that law of thought termed in this work the law of duality
may have its own peculiar tendency to error, exalting mere want
of agreement into contrariety, and thus form a world which we
necessarily view as formed of parts supplemental to each other,
framing the conception of a world fundamentally divided by opposing
powers. Such, with some large but hasty inductions from
ph{\ae}nomena, may have been the origin of dualism,---independently
of the question whether dualism is in any form a true
theory or not. Here, however, it is of more importance to consider
in detail the bearing of these ancient forms of speculation,
as revived in the present day, upon the progress of real knowledge;
and upon this point I desire, in pursuance of what has
been said in the previous section, to add the following remarks:

1st. All sound philosophy gives its verdict against such speculations,
if regarded as a means of determining the actual constitution
of things. It may be that the progress of natural
knowledge tends towards the recognition of some central Unity
in Nature. Of such unity as consists in the mutual relation of
the parts of a system there can be little doubt, and able men
have speculated, not without grounds, on a more intimate correlation
of physical forces than the mere idea of a system would
lead us to conjecture. Further, it may be that in the bosom of
that supposed unity are involved some general principles of division
and re-union, the sources, under the Supreme Will, of much
of the \emph{related} variety of Nature. The instances of sex and polarity
have been adduced in support of such a view. As a supposition,
I will venture to add, that it is not very improbable
that, in some such way as this, the constitution of things without
may correspond to that of the mind within. But such correspondence,
if it shall ever be proved to exist, will appear as the
last induction from human knowledge, not as the first principle
of scientific inquiry. The natural order of discovery is from the
particular to the universal, and it may confidently be affirmed
%-----------------------File: 221.png----------------------------
that we have not yet advanced sufficiently far on this track to
enable us to determine what are the ultimate forms into which all
the special differences of Nature shall merge, and from which
they shall receive their explanation.

2ndly. Were this correspondence between the forms of thought
and the actual constitution of Nature proved to exist, whatsoever connexion or relation it might be supposed to establish between the two systems, it would in no degree affect the question
of their mutual independence. It would in no sense lead to the
consequence that the one system is the mere \emph{product} of the other.
A too great addiction to metaphysical speculations seems, in
some instances, to have produced a tendency toward this species
of illusion. Thus, among the many attempts which have been
made to explain the existence of evil, it has been sought to assign
to the fact a merely \emph{relative} character,---to found it upon a species
of logical opposition to the equally relative element of good. It
suffices to say, that the assumption is purely gratuitous. What
evil may be in the eyes of Infinite wisdom and purity, we can at
the best but dimly conjecture; but to us, in all its forms, whether of pain or defect, or moral transgression, or retributory wo,
it can wear but one aspect,---that of a sad and stern reality,
against which, upon somewhat more than the highest order of
prudential considerations, the whole preventive force of our
nature may be exerted. Now what has been said upon the
particular question just considered, is equally applicable to many
other of the debated points of philosophy; such, for instance,
as the external reality of space and time. We have no warrant for resolving these into mere forms of the understanding,
though they unquestionably determine the present sphere of
our knowledge. And, to speak more generally, there is no warrant for the extremely \emph{subjective} tendency of much modern speculation. Whenever, in the view of the intellect, different
hypotheses are equally consistent with an observed fact, the
instinctive testimony of consciousness as to their relative value
must be allowed to possess \emph{authority}.

3rdly. If the study of the laws of thought avails us neither
to determine the actual constitution of things, nor to explain the
facts involved in that constitution which have perplexed the wise
%-----------------------File: 222.png----------------------------
and saddened the thoughtful in all ages,---still less does it enable
us to rise above the present conditions of our being, or lend its
sanction to the doctrine which affirms the possibility of an \emph{intuitive}
knowledge of the infinite, and the unconditioned,---whether
such knowledge be sought for in the realm of Nature, or
above that realm. We can never be said to \emph{comprehend} that
which is represented to thought as the limit of an indefinite
process of abstraction. A progression \textit{ad infinitum} is impossible
to finite powers. But though we cannot comprehend the
infinite, there may be even scientific grounds for believing that
human nature is constituted in some relation to the infinite. We
cannot perfectly express the laws of thought, or establish in the
most general sense the methods of which they form the basis, without
at least the implication of elements which ordinary language
expresses by the terms ``Universe'' and ``Eternity.'' As in the
pure abstractions of Geometry, so in the domain of Logic it is
seen, that the empire of Truth is, in a certain sense, larger than
that of Imagination. And as there are many special departments
of knowledge which can only be completely surveyed from an external
point, so the theory of the intellectual processes, as applied
only to finite objects, seems to involve the recognition of a
sphere of thought from which all limits are withdrawn. If then,
on the one hand, we cannot discover in the laws of thought and
their analogies a sufficient basis of proof for the conclusions of
a too daring mysticism; on the other hand we should err in regarding
them as wholly unsuggestive. As parts of our intellectual
nature, it seems not improbable that they should manifest
their presence otherwise than by merely prescribing the conditions
of formal inference. Whatever grounds we have for connecting
them with the peculiar tendencies of physical speculation
among the Ionian and Italic philosophers, the same grounds
exist for associating them with a disposition of thought at once
more common and more legitimate. To no casual influences, at
least, ought we to attribute that meditative spirit which then
most delights to commune with the external magnificence of
Nature, when most impressed with the consciousness of sempiternal
verities,---which reads in the nocturnal heavens a bright
manifestation of order; or feels in some wild scene among the
%-----------------------File: 223.png----------------------------
hills, the intimations of more than that abstract eternity which
had rolled away ere yet their dark foundations were laid.%
\footnote{Psalm xc. 2}

9. Refraining from the further prosecution of a train of thought
which to some may appear to be of too speculative a character,
let us briefly review the positive results to which we have been led.
It has appeared that there exist in our nature faculties which
enable us to ascend from the particular facts of experience to the
general propositions which form the basis of Science; as well as
faculties whose office it is to deduce from general propositions
accepted as true the particular conclusions which they involve.
It has been seen, that those faculties are subject in their operations
to laws capable of precise scientific expression, but invested
with an authority which, as contrasted with the authority of the
laws of nature, is distinct, \textit{sui generis}, and underived. Further,
there has appeared to be a manifest fitness between the intellectual
procedure thus made known to us, and the conditions of
that system of things by which we are surrounded,---such conditions,
I mean, as the existence of species connected by general
resemblances, of facts associated under general laws; together
with that union of permanency with order, which while it gives
stability to acquired knowledge, lays a foundation for the hope
of indefinite progression. Human nature, quite independently
of its observed or manifested tendencies, is seen to be \emph{constituted}
in a certain relation to Truth; and this relation, considered as a
subject of speculative knowledge, is as capable of being studied
in its details, is, moreover, as worthy of being so studied, as are
the several departments of physical science, considered in the same
aspect. I would especially direct attention to that view of the
constitution of the intellect which represents it as subject to laws
determinate in their character, but not operating by the power of
necessity; which exhibits it as redeemed from the dominion of
fate, without being abandoned to the lawlessness of chance. We
cannot embrace this view without accepting at least as \emph{probable}
the intimations which, upon the principle of analogy, it seems to
furnish respecting another and a higher aspect of our nature,---its
subjection in the sphere of duty as well as in that of knowledge to
%-----------------------File: 224.png----------------------------
fixed laws whose authority does not consist in power,---its constitution with reference to an ideal standard and a final purpose.
It has been thought, indeed, that scientific pursuits foster a disposition either to overlook the specific differences between the
moral and the material world, or to regard the former as in no proper sense a subject for exact knowledge. Doubtless all exclusive
pursuits tend to produce partial views, and it may be, that a mind
long and deeply immersed in the contemplation of scenes over
which the dominion of a physical necessity is unquestioned and supreme, may admit with difficulty the possibility of another order of
things. But it is because of the \emph{exclusiveness} of this devotion to a
particular sphere of knowledge, that the prejudice in question
takes possession, if at all, of the mind. The application of
scientific methods to the study of the intellectual ph{\ae}nomena,
conducted in an impartial spirit of inquiry, and without overlooking those elements of error and disturbance which must be
accepted as \emph{facts}, though they cannot be regarded as \emph{laws}, in
the constitution of our nature, seems to furnish the materials of
a juster analogy.

10. If it be asked to what \emph{practical} end such inquiries as the
above point, it may be replied, that there exist various objects,
in relation to which the courses of men's actions are mainly determined by their speculative views of human nature. Education, considered in its largest sense, is one of those objects. The
ultimate ground of all inquiry into its nature and its methods
must be laid in some previous theory of what man is, what are
the ends for which his several faculties were designed, what
are the motives which have power to influence them to sustained
action, and to elicit their most perfect and most stable results.
It may be doubted, whether these questions have ever been
considered fully, and at the same time impartially, in the relations here suggested. The highest cultivation of taste by the
study of the pure models of antiquity, the largest acquaintance
with the facts and theories of modern physical science, viewed
from this larger aspect of our nature, can only appear as parts of
a perfect intellectual discipline. Looking from the same point
of view upon the means to be employed, we might be led to inquire, whether that all but exclusive appeal which is made in
%-----------------------File: 225.png----------------------------
the present day to the spirit of emulation or cupidity, does not
tend to weaken the influence of those more enduring motives
which seem to have been implanted in our nature for the immediate
end in view. Upon these, and upon many other questions,
the just limits of authority, the reconciliation of freedom of
thought with discipline of feelings, habits, manners, and upon
the whole \emph{moral} aspect of the question,---what unfixedness of
opinion, what diversity of practice, do we meet with! Yet, in
the sober view of reason, there is no object within the compass
of human endeavours which is of more weight and moment than
this, considered, as I have said, in its largest meaning. Now,
whatsoever tends to make more exact and definite our view of
human nature, in any of its real aspects, tends, in the same proportion,
to reduce these questions into narrower compass, and
restrict the limits of their possible solution. Thus may even
speculative inquiries prove fruitful of the most important principles
of action.

11. Perhaps the most obviously legitimate bearing of such
speculations would be upon the question of the place of Mathematics
in the system of human knowledge, and the nature
and office of mathematical studies, as a means of intellectual
discipline. No one who has attended to the course of recent
discussions can think this question an unimportant one. Those
who have maintained that the position of Mathematics is in
both respects a fundamental one, have drawn one of their strongest
arguments from the actual constitution of things. The material
frame is subject in all its parts to the relations of number.
All dynamical, chemical, electrical, thermal, actions, seem not
only to be measurable in themselves, but to be connected with
each other, even to the extent of mutual convertibility, by numerical
relations of a perfectly definite kind. But the opinion
in question seems to me to rest upon a deeper basis than this.
The laws of thought, in all its processes of conception and of
reasoning, in all those operations of which language is the expression
or the instrument, are of the same kind as are the laws
of the acknowledged processes of Mathematics. It is not contended
that it is necessary for us to acquaint ourselves with those
laws in order to think coherently, or, in the ordinary sense of
%-----------------------File: 226.png----------------------------
the terms, to reason well. Men draw inferences without any
consciousness of those elements upon which the entire procedure
depends. Still less is it desired to exalt the reasoning faculty
over the faculties of observation, of reflection, and of judgment.
But upon the very ground that human thought, traced to its
ultimate elements, reveals itself in mathematical forms, we have
a presumption that the mathematical sciences occupy, by the
constitution of our nature, a fundamental place in human knowledge, and that no system of mental culture can be complete or
fundamental, which altogether neglects them.

But the very same class of considerations shows with equal
force the error of those who regard the study of Mathematics,
and of their applications, as a sufficient basis either of knowledge
or of discipline. If the constitution of the material frame is mathematical, it is not merely so. If the mind, in its capacity of
formal reasoning, obeys, whether consciously or unconsciously,
mathematical laws, it claims through its other capacities of sentiment and action, through its perceptions of beauty and of
moral fitness, through its deep springs of emotion and affection,
to hold relation to a different order of things. There is, moreover, a breadth of intellectual vision, a power of sympathy with
truth in all its forms and manifestations, which is not measured
by the force and subtlety of the dialectic faculty. Even the
revelation of the material universe in its boundless magnitude,
and pervading order, and constancy of law, is not necessarily the
most fully apprehended by him who has traced with minutest
accuracy the steps of the great demonstration. And if we embrace in our survey the interests and duties of life, how little do
any processes of mere ratiocination enable us to comprehend the
weightier questions which they present! As truly, therefore, as
the cultivation of the mathematical or deductive faculty is a part
of intellectual discipline, so truly is it only a part. The prejudice which would either banish or make supreme any one
department of knowledge or faculty of mind, betrays not only
error of judgment, but a defect of that intellectual modesty
which is inseparable from a pure devotion to truth. It assumes
the office of criticising a constitution of things which no human
appointment has established, or can annul. It sets aside the
%-----------------------File: 227.png----------------------------
ancient and just conception of truth as one though manifold.
Much of this error, as actually existent among us, seems due
to the special and isolated character of scientific teaching---which character it, in its turn, tends to foster. The study of
philosophy, notwithstanding a few marked instances of exception,
has failed to keep pace with the advance of the several departments
of knowledge, whose mutual relations it is its province to
determine. It is impossible, however, not to contemplate the
particular evil in question as part of a larger system, and connect
it with the too prevalent view of knowledge as a merely secular
thing, and with the undue predominance, already adverted to, of
those motives, legitimate within their proper limits, which are
founded upon a regard to its secular advantages. In the extreme
case it is not difficult to see that the continued operation of
such motives, uncontrolled by any higher principles of action,
uncorrected by the personal influence of superior minds, must
tend to lower the standard of thought in reference to the objects
of knowledge, and to render void and ineffectual whatsoever elements
of a noble faith may still survive. And ever in proportion
as these conditions are realized must the same effects follow.
Hence, perhaps, it is that we sometimes find juster conceptions
of the unity, the vital connexion, and the subordination to a
moral purpose, of the different parts of Truth, among those who
acknowledge nothing higher than the changing aspect of collective
humanity, than among those who profess an intellectual
allegiance to the Father of Lights. But these are questions
which cannot further be pursued here. To some they will appear
foreign to the professed design of this work. But the
consideration of them has arisen naturally, either out of the
speculations which that design involved, or in the course of
reading and reflection which seemed necessary to its accomplishment.

\centerline{\textsc{the end.}}
%-----------------------File: 228.png----------------------------
\newpage
\chapter{PROJECT GUTENBERG ``SMALL PRINT''}
\small \pagenumbering{gobble}
\begin{verbatim}
End of the Project Gutenberg EBook of An Investigation of the Laws of Thought, by 
George Boole

*** END OF THIS PROJECT GUTENBERG EBOOK LAWS OF THOUGHT ***

***** This file should be named 15114-pdf.pdf or 15114-pdf.zip *****
This and all associated files of various formats will be found in:
        http://www.gutenberg.org/1/5/1/1/15114/

Produced by David Starner, Joshua Hutchinson, David Bowden
and the Online Distributed Proofreading Team.


Updated editions will replace the previous one--the old editions will
be renamed.

Creating the works from print editions not protected by U.S. copyright
law means that no one owns a United States copyright in these works,
so the Foundation (and you!) can copy and distribute it in the United
States without permission and without paying copyright
royalties. Special rules, set forth in the General Terms of Use part
of this license, apply to copying and distributing Project
Gutenberg-tm electronic works to protect the PROJECT GUTENBERG-tm
concept and trademark. Project Gutenberg is a registered trademark,
and may not be used if you charge for the eBooks, unless you receive
specific permission. If you do not charge anything for copies of this
eBook, complying with the rules is very easy. You may use this eBook
for nearly any purpose such as creation of derivative works, reports,
performances and research. They may be modified and printed and given
away--you may do practically ANYTHING in the United States with eBooks
not protected by U.S. copyright law. Redistribution is subject to the
trademark license, especially commercial redistribution.

START: FULL LICENSE

THE FULL PROJECT GUTENBERG LICENSE
PLEASE READ THIS BEFORE YOU DISTRIBUTE OR USE THIS WORK

To protect the Project Gutenberg-tm mission of promoting the free
distribution of electronic works, by using or distributing this work
(or any other work associated in any way with the phrase "Project
Gutenberg"), you agree to comply with all the terms of the Full
Project Gutenberg-tm License available with this file or online at
www.gutenberg.org/license.

Section 1. General Terms of Use and Redistributing Project
Gutenberg-tm electronic works

1.A. By reading or using any part of this Project Gutenberg-tm
electronic work, you indicate that you have read, understand, agree to
and accept all the terms of this license and intellectual property
(trademark/copyright) agreement. If you do not agree to abide by all
the terms of this agreement, you must cease using and return or
destroy all copies of Project Gutenberg-tm electronic works in your
possession. If you paid a fee for obtaining a copy of or access to a
Project Gutenberg-tm electronic work and you do not agree to be bound
by the terms of this agreement, you may obtain a refund from the
person or entity to whom you paid the fee as set forth in paragraph
1.E.8.

1.B. "Project Gutenberg" is a registered trademark. It may only be
used on or associated in any way with an electronic work by people who
agree to be bound by the terms of this agreement. There are a few
things that you can do with most Project Gutenberg-tm electronic works
even without complying with the full terms of this agreement. See
paragraph 1.C below. There are a lot of things you can do with Project
Gutenberg-tm electronic works if you follow the terms of this
agreement and help preserve free future access to Project Gutenberg-tm
electronic works. See paragraph 1.E below.

1.C. The Project Gutenberg Literary Archive Foundation ("the
Foundation" or PGLAF), owns a compilation copyright in the collection
of Project Gutenberg-tm electronic works. Nearly all the individual
works in the collection are in the public domain in the United
States. If an individual work is unprotected by copyright law in the
United States and you are located in the United States, we do not
claim a right to prevent you from copying, distributing, performing,
displaying or creating derivative works based on the work as long as
all references to Project Gutenberg are removed. Of course, we hope
that you will support the Project Gutenberg-tm mission of promoting
free access to electronic works by freely sharing Project Gutenberg-tm
works in compliance with the terms of this agreement for keeping the
Project Gutenberg-tm name associated with the work. You can easily
comply with the terms of this agreement by keeping this work in the
same format with its attached full Project Gutenberg-tm License when
you share it without charge with others.

1.D. The copyright laws of the place where you are located also govern
what you can do with this work. Copyright laws in most countries are
in a constant state of change. If you are outside the United States,
check the laws of your country in addition to the terms of this
agreement before downloading, copying, displaying, performing,
distributing or creating derivative works based on this work or any
other Project Gutenberg-tm work. The Foundation makes no
representations concerning the copyright status of any work in any
country outside the United States.

1.E. Unless you have removed all references to Project Gutenberg:

1.E.1. The following sentence, with active links to, or other
immediate access to, the full Project Gutenberg-tm License must appear
prominently whenever any copy of a Project Gutenberg-tm work (any work
on which the phrase "Project Gutenberg" appears, or with which the
phrase "Project Gutenberg" is associated) is accessed, displayed,
performed, viewed, copied or distributed:

  This eBook is for the use of anyone anywhere in the United States and
  most other parts of the world at no cost and with almost no
  restrictions whatsoever. You may copy it, give it away or re-use it
  under the terms of the Project Gutenberg License included with this
  eBook or online at www.gutenberg.org. If you are not located in the
  United States, you'll have to check the laws of the country where you
  are located before using this ebook.

1.E.2. If an individual Project Gutenberg-tm electronic work is
derived from texts not protected by U.S. copyright law (does not
contain a notice indicating that it is posted with permission of the
copyright holder), the work can be copied and distributed to anyone in
the United States without paying any fees or charges. If you are
redistributing or providing access to a work with the phrase "Project
Gutenberg" associated with or appearing on the work, you must comply
either with the requirements of paragraphs 1.E.1 through 1.E.7 or
obtain permission for the use of the work and the Project Gutenberg-tm
trademark as set forth in paragraphs 1.E.8 or 1.E.9.

1.E.3. If an individual Project Gutenberg-tm electronic work is posted
with the permission of the copyright holder, your use and distribution
must comply with both paragraphs 1.E.1 through 1.E.7 and any
additional terms imposed by the copyright holder. Additional terms
will be linked to the Project Gutenberg-tm License for all works
posted with the permission of the copyright holder found at the
beginning of this work.

1.E.4. Do not unlink or detach or remove the full Project Gutenberg-tm
License terms from this work, or any files containing a part of this
work or any other work associated with Project Gutenberg-tm.

1.E.5. Do not copy, display, perform, distribute or redistribute this
electronic work, or any part of this electronic work, without
prominently displaying the sentence set forth in paragraph 1.E.1 with
active links or immediate access to the full terms of the Project
Gutenberg-tm License.

1.E.6. You may convert to and distribute this work in any binary,
compressed, marked up, nonproprietary or proprietary form, including
any word processing or hypertext form. However, if you provide access
to or distribute copies of a Project Gutenberg-tm work in a format
other than "Plain Vanilla ASCII" or other format used in the official
version posted on the official Project Gutenberg-tm web site
(www.gutenberg.org), you must, at no additional cost, fee or expense
to the user, provide a copy, a means of exporting a copy, or a means
of obtaining a copy upon request, of the work in its original "Plain
Vanilla ASCII" or other form. Any alternate format must include the
full Project Gutenberg-tm License as specified in paragraph 1.E.1.

1.E.7. Do not charge a fee for access to, viewing, displaying,
performing, copying or distributing any Project Gutenberg-tm works
unless you comply with paragraph 1.E.8 or 1.E.9.

1.E.8. You may charge a reasonable fee for copies of or providing
access to or distributing Project Gutenberg-tm electronic works
provided that

* You pay a royalty fee of 20% of the gross profits you derive from
  the use of Project Gutenberg-tm works calculated using the method
  you already use to calculate your applicable taxes. The fee is owed
  to the owner of the Project Gutenberg-tm trademark, but he has
  agreed to donate royalties under this paragraph to the Project
  Gutenberg Literary Archive Foundation. Royalty payments must be paid
  within 60 days following each date on which you prepare (or are
  legally required to prepare) your periodic tax returns. Royalty
  payments should be clearly marked as such and sent to the Project
  Gutenberg Literary Archive Foundation at the address specified in
  Section 4, "Information about donations to the Project Gutenberg
  Literary Archive Foundation."

* You provide a full refund of any money paid by a user who notifies
  you in writing (or by e-mail) within 30 days of receipt that s/he
  does not agree to the terms of the full Project Gutenberg-tm
  License. You must require such a user to return or destroy all
  copies of the works possessed in a physical medium and discontinue
  all use of and all access to other copies of Project Gutenberg-tm
  works.

* You provide, in accordance with paragraph 1.F.3, a full refund of
  any money paid for a work or a replacement copy, if a defect in the
  electronic work is discovered and reported to you within 90 days of
  receipt of the work.

* You comply with all other terms of this agreement for free
  distribution of Project Gutenberg-tm works.

1.E.9. If you wish to charge a fee or distribute a Project
Gutenberg-tm electronic work or group of works on different terms than
are set forth in this agreement, you must obtain permission in writing
from both the Project Gutenberg Literary Archive Foundation and The
Project Gutenberg Trademark LLC, the owner of the Project Gutenberg-tm
trademark. Contact the Foundation as set forth in Section 3 below.

1.F.

1.F.1. Project Gutenberg volunteers and employees expend considerable
effort to identify, do copyright research on, transcribe and proofread
works not protected by U.S. copyright law in creating the Project
Gutenberg-tm collection. Despite these efforts, Project Gutenberg-tm
electronic works, and the medium on which they may be stored, may
contain "Defects," such as, but not limited to, incomplete, inaccurate
or corrupt data, transcription errors, a copyright or other
intellectual property infringement, a defective or damaged disk or
other medium, a computer virus, or computer codes that damage or
cannot be read by your equipment.

1.F.2. LIMITED WARRANTY, DISCLAIMER OF DAMAGES - Except for the "Right
of Replacement or Refund" described in paragraph 1.F.3, the Project
Gutenberg Literary Archive Foundation, the owner of the Project
Gutenberg-tm trademark, and any other party distributing a Project
Gutenberg-tm electronic work under this agreement, disclaim all
liability to you for damages, costs and expenses, including legal
fees. YOU AGREE THAT YOU HAVE NO REMEDIES FOR NEGLIGENCE, STRICT
LIABILITY, BREACH OF WARRANTY OR BREACH OF CONTRACT EXCEPT THOSE
PROVIDED IN PARAGRAPH 1.F.3. YOU AGREE THAT THE FOUNDATION, THE
TRADEMARK OWNER, AND ANY DISTRIBUTOR UNDER THIS AGREEMENT WILL NOT BE
LIABLE TO YOU FOR ACTUAL, DIRECT, INDIRECT, CONSEQUENTIAL, PUNITIVE OR
INCIDENTAL DAMAGES EVEN IF YOU GIVE NOTICE OF THE POSSIBILITY OF SUCH
DAMAGE.

1.F.3. LIMITED RIGHT OF REPLACEMENT OR REFUND - If you discover a
defect in this electronic work within 90 days of receiving it, you can
receive a refund of the money (if any) you paid for it by sending a
written explanation to the person you received the work from. If you
received the work on a physical medium, you must return the medium
with your written explanation. The person or entity that provided you
with the defective work may elect to provide a replacement copy in
lieu of a refund. If you received the work electronically, the person
or entity providing it to you may choose to give you a second
opportunity to receive the work electronically in lieu of a refund. If
the second copy is also defective, you may demand a refund in writing
without further opportunities to fix the problem.

1.F.4. Except for the limited right of replacement or refund set forth
in paragraph 1.F.3, this work is provided to you 'AS-IS', WITH NO
OTHER WARRANTIES OF ANY KIND, EXPRESS OR IMPLIED, INCLUDING BUT NOT
LIMITED TO WARRANTIES OF MERCHANTABILITY OR FITNESS FOR ANY PURPOSE.

1.F.5. Some states do not allow disclaimers of certain implied
warranties or the exclusion or limitation of certain types of
damages. If any disclaimer or limitation set forth in this agreement
violates the law of the state applicable to this agreement, the
agreement shall be interpreted to make the maximum disclaimer or
limitation permitted by the applicable state law. The invalidity or
unenforceability of any provision of this agreement shall not void the
remaining provisions.

1.F.6. INDEMNITY - You agree to indemnify and hold the Foundation, the
trademark owner, any agent or employee of the Foundation, anyone
providing copies of Project Gutenberg-tm electronic works in
accordance with this agreement, and any volunteers associated with the
production, promotion and distribution of Project Gutenberg-tm
electronic works, harmless from all liability, costs and expenses,
including legal fees, that arise directly or indirectly from any of
the following which you do or cause to occur: (a) distribution of this
or any Project Gutenberg-tm work, (b) alteration, modification, or
additions or deletions to any Project Gutenberg-tm work, and (c) any
Defect you cause.

Section 2. Information about the Mission of Project Gutenberg-tm

Project Gutenberg-tm is synonymous with the free distribution of
electronic works in formats readable by the widest variety of
computers including obsolete, old, middle-aged and new computers. It
exists because of the efforts of hundreds of volunteers and donations
from people in all walks of life.

Volunteers and financial support to provide volunteers with the
assistance they need are critical to reaching Project Gutenberg-tm's
goals and ensuring that the Project Gutenberg-tm collection will
remain freely available for generations to come. In 2001, the Project
Gutenberg Literary Archive Foundation was created to provide a secure
and permanent future for Project Gutenberg-tm and future
generations. To learn more about the Project Gutenberg Literary
Archive Foundation and how your efforts and donations can help, see
Sections 3 and 4 and the Foundation information page at
www.gutenberg.org Section 3. Information about the Project Gutenberg
Literary Archive Foundation

The Project Gutenberg Literary Archive Foundation is a non profit
501(c)(3) educational corporation organized under the laws of the
state of Mississippi and granted tax exempt status by the Internal
Revenue Service. The Foundation's EIN or federal tax identification
number is 64-6221541. Contributions to the Project Gutenberg Literary
Archive Foundation are tax deductible to the full extent permitted by
U.S. federal laws and your state's laws.

The Foundation's principal office is in Fairbanks, Alaska, with the
mailing address: PO Box 750175, Fairbanks, AK 99775, but its
volunteers and employees are scattered throughout numerous
locations. Its business office is located at 809 North 1500 West, Salt
Lake City, UT 84116, (801) 596-1887. Email contact links and up to
date contact information can be found at the Foundation's web site and
official page at www.gutenberg.org/contact

For additional contact information:

    Dr. Gregory B. Newby
    Chief Executive and Director
    gbnewby@pglaf.org

Section 4. Information about Donations to the Project Gutenberg
Literary Archive Foundation

Project Gutenberg-tm depends upon and cannot survive without wide
spread public support and donations to carry out its mission of
increasing the number of public domain and licensed works that can be
freely distributed in machine readable form accessible by the widest
array of equipment including outdated equipment. Many small donations
($1 to $5,000) are particularly important to maintaining tax exempt
status with the IRS.

The Foundation is committed to complying with the laws regulating
charities and charitable donations in all 50 states of the United
States. Compliance requirements are not uniform and it takes a
considerable effort, much paperwork and many fees to meet and keep up
with these requirements. We do not solicit donations in locations
where we have not received written confirmation of compliance. To SEND
DONATIONS or determine the status of compliance for any particular
state visit www.gutenberg.org/donate

While we cannot and do not solicit contributions from states where we
have not met the solicitation requirements, we know of no prohibition
against accepting unsolicited donations from donors in such states who
approach us with offers to donate.

International donations are gratefully accepted, but we cannot make
any statements concerning tax treatment of donations received from
outside the United States. U.S. laws alone swamp our small staff.

Please check the Project Gutenberg Web pages for current donation
methods and addresses. Donations are accepted in a number of other
ways including checks, online payments and credit card donations. To
donate, please visit: www.gutenberg.org/donate

Section 5. General Information About Project Gutenberg-tm electronic works.

Professor Michael S. Hart was the originator of the Project
Gutenberg-tm concept of a library of electronic works that could be
freely shared with anyone. For forty years, he produced and
distributed Project Gutenberg-tm eBooks with only a loose network of
volunteer support.

Project Gutenberg-tm eBooks are often created from several printed
editions, all of which are confirmed as not protected by copyright in
the U.S. unless a copyright notice is included. Thus, we do not
necessarily keep eBooks in compliance with any particular paper
edition.

Most people start at our Web site which has the main PG search
facility: www.gutenberg.org

This Web site includes information about Project Gutenberg-tm,
including how to make donations to the Project Gutenberg Literary
Archive Foundation, how to help produce our new eBooks, and how to
subscribe to our email newsletter to hear about new eBooks.
\end{verbatim}

% %%%%%%%%%%%%%%%%%%%%%%%%%%%%%%%%%%%%%%%%%%%%%%%%%%%%%%%%%%%%%%%%%%%%%%% %
%                                                                         %
% End of the Project Gutenberg EBook of An Investigation of the Laws of Thought, by 
% George Boole                                                            %
%                                                                         %
% *** END OF THIS PROJECT GUTENBERG EBOOK LAWS OF THOUGHT ***             %
%                                                                         %
% ***** This file should be named 15114-t.tex or 15114-t.zip *****        %
% This and all associated files of various formats will be found in:      %
%         http://www.gutenberg.org/1/5/1/1/15114/                         %
%                                                                         %
% %%%%%%%%%%%%%%%%%%%%%%%%%%%%%%%%%%%%%%%%%%%%%%%%%%%%%%%%%%%%%%%%%%%%%%% %

\end{document}
This is pdfTeX, Version 3.1415926-2.5-1.40.14 (TeX Live 2013/Debian) (format=pdflatex 2015.9.16)  19 JUL 2017 06:12
entering extended mode
 %&-line parsing enabled.
**15114-t.tex
(./15114-t.tex
LaTeX2e <2011/06/27>
Babel <3.9h> and hyphenation patterns for 78 languages loaded.
(/usr/share/texlive/texmf-dist/tex/latex/base/book.cls
Document Class: book 2007/10/19 v1.4h Standard LaTeX document class
(/usr/share/texlive/texmf-dist/tex/latex/base/bk10.clo
File: bk10.clo 2007/10/19 v1.4h Standard LaTeX file (size option)
)
\c@part=\count79
\c@chapter=\count80
\c@section=\count81
\c@subsection=\count82
\c@subsubsection=\count83
\c@paragraph=\count84
\c@subparagraph=\count85
\c@figure=\count86
\c@table=\count87
\abovecaptionskip=\skip41
\belowcaptionskip=\skip42
\bibindent=\dimen102
) (/usr/share/texlive/texmf-dist/tex/generic/babel/babel.sty
Package: babel 2013/12/03 3.9h The Babel package
(/usr/share/texlive/texmf-dist/tex/generic/babel-greek/greek.ldf
Language: greek 2013/12/03 v1.8a Greek support for the babel system
(/usr/share/texlive/texmf-dist/tex/generic/babel/babel.def
File: babel.def 2013/12/03 3.9h Babel common definitions
\babel@savecnt=\count88
\U@D=\dimen103
) (/usr/share/texlive/texmf-dist/tex/latex/greek-fontenc/lgrenc.def
File: lgrenc.def 2013/07/16 v0.9 LGR Greek font encoding definitions
(/usr/share/texlive/texmf-dist/tex/latex/greek-fontenc/greek-fontenc.def
File: greek-fontenc.def 2013/11/28 v0.11 Common Greek font encoding definitions

))) (/usr/share/texlive/texmf-dist/tex/generic/babel-english/english.ldf
Language: english 2012/08/20 v3.3p English support from the babel system
\l@canadian = a dialect from \language\l@american 
\l@australian = a dialect from \language\l@british 
\l@newzealand = a dialect from \language\l@british 
)) (/usr/share/texlive/texmf-dist/tex/latex/amsfonts/amssymb.sty
Package: amssymb 2013/01/14 v3.01 AMS font symbols
(/usr/share/texlive/texmf-dist/tex/latex/amsfonts/amsfonts.sty
Package: amsfonts 2013/01/14 v3.01 Basic AMSFonts support
\@emptytoks=\toks14
\symAMSa=\mathgroup4
\symAMSb=\mathgroup5
LaTeX Font Info:    Overwriting math alphabet `\mathfrak' in version `bold'
(Font)                  U/euf/m/n --> U/euf/b/n on input line 106.
)) (/usr/share/texlive/texmf-dist/tex/latex/amsmath/amsmath.sty
Package: amsmath 2013/01/14 v2.14 AMS math features
\@mathmargin=\skip43
For additional information on amsmath, use the `?' option.
(/usr/share/texlive/texmf-dist/tex/latex/amsmath/amstext.sty
Package: amstext 2000/06/29 v2.01
(/usr/share/texlive/texmf-dist/tex/latex/amsmath/amsgen.sty
File: amsgen.sty 1999/11/30 v2.0
\@emptytoks=\toks15
\ex@=\dimen104
)) (/usr/share/texlive/texmf-dist/tex/latex/amsmath/amsbsy.sty
Package: amsbsy 1999/11/29 v1.2d
\pmbraise@=\dimen105
) (/usr/share/texlive/texmf-dist/tex/latex/amsmath/amsopn.sty
Package: amsopn 1999/12/14 v2.01 operator names
)
\inf@bad=\count89
LaTeX Info: Redefining \frac on input line 210.
\uproot@=\count90
\leftroot@=\count91
LaTeX Info: Redefining \overline on input line 306.
\classnum@=\count92
\DOTSCASE@=\count93
LaTeX Info: Redefining \ldots on input line 378.
LaTeX Info: Redefining \dots on input line 381.
LaTeX Info: Redefining \cdots on input line 466.
\Mathstrutbox@=\box26
\strutbox@=\box27
\big@size=\dimen106
LaTeX Font Info:    Redeclaring font encoding OML on input line 566.
LaTeX Font Info:    Redeclaring font encoding OMS on input line 567.
\macc@depth=\count94
\c@MaxMatrixCols=\count95
\dotsspace@=\muskip10
\c@parentequation=\count96
\dspbrk@lvl=\count97
\tag@help=\toks16
\row@=\count98
\column@=\count99
\maxfields@=\count100
\andhelp@=\toks17
\eqnshift@=\dimen107
\alignsep@=\dimen108
\tagshift@=\dimen109
\tagwidth@=\dimen110
\totwidth@=\dimen111
\lineht@=\dimen112
\@envbody=\toks18
\multlinegap=\skip44
\multlinetaggap=\skip45
\mathdisplay@stack=\toks19
LaTeX Info: Redefining \[ on input line 2665.
LaTeX Info: Redefining \] on input line 2666.
)
No file 15114-t.aux.
\openout1 = `15114-t.aux'.

LaTeX Font Info:    Checking defaults for OML/cmm/m/it on input line 74.
LaTeX Font Info:    ... okay on input line 74.
LaTeX Font Info:    Checking defaults for T1/cmr/m/n on input line 74.
LaTeX Font Info:    ... okay on input line 74.
LaTeX Font Info:    Checking defaults for OT1/cmr/m/n on input line 74.
LaTeX Font Info:    ... okay on input line 74.
LaTeX Font Info:    Checking defaults for OMS/cmsy/m/n on input line 74.
LaTeX Font Info:    ... okay on input line 74.
LaTeX Font Info:    Checking defaults for OMX/cmex/m/n on input line 74.
LaTeX Font Info:    ... okay on input line 74.
LaTeX Font Info:    Checking defaults for U/cmr/m/n on input line 74.
LaTeX Font Info:    ... okay on input line 74.
LaTeX Font Info:    Checking defaults for LGR/cmr/m/n on input line 74.
LaTeX Font Info:    Try loading font information for LGR+cmr on input line 74.
(/usr/share/texlive/texmf-dist/tex/latex/cbfonts-fd/lgrcmr.fd
File: lgrcmr.fd 2013/09/01 v1.0 Greek European Computer Regular
)
LaTeX Font Info:    ... okay on input line 74.

Overfull \hbox (14.09583pt too wide) in paragraph at lines 102--102
[]\OT1/cmtt/m/n/9 Project Gutenberg's An Investigation of the Laws of Thought, 
by George Boole[] 
 []


Overfull \hbox (4.64594pt too wide) in paragraph at lines 102--102
[]\OT1/cmtt/m/n/9 This eBook is for the use of anyone anywhere in the United St
ates and most[] 
 []


Overfull \hbox (4.64594pt too wide) in paragraph at lines 102--102
[]\OT1/cmtt/m/n/9 whatsoever.  You may copy it, give it away or re-use it under
 the terms of[] 
 []


Overfull \hbox (14.09583pt too wide) in paragraph at lines 102--102
[]\OT1/cmtt/m/n/9 www.gutenberg.org.  If you are not located in the United Stat
es, you'll have[] 
 []


Overfull \hbox (28.27066pt too wide) in paragraph at lines 102--102
[]\OT1/cmtt/m/n/9 to check the laws of the country where you are located before
 using this ebook.[] 
 []

[1

{/var/lib/texmf/fonts/map/pdftex/updmap/pdftex.map}] [1


] [2]
LaTeX Font Info:    Try loading font information for U+msa on input line 155.
(/usr/share/texlive/texmf-dist/tex/latex/amsfonts/umsa.fd
File: umsa.fd 2013/01/14 v3.01 AMS symbols A
)
LaTeX Font Info:    Try loading font information for U+msb on input line 155.
(/usr/share/texlive/texmf-dist/tex/latex/amsfonts/umsb.fd
File: umsb.fd 2013/01/14 v3.01 AMS symbols B
) [3

] [4] [5

] [6] [7

]
Chapter \textlatin {I}.
[1


] [2] [3] [4] [5] [6] [7] [8] [9] [10] [11] [12] [13] [14] [15] [16]
Chapter \textlatin {II}.
[17

] [18] [19] [20] [21] [22] [23] [24] [25] [26] [27]
Chapter \textlatin {III}.
[28

] [29] [30] [31] [32] [33] [34] [35] [36]
Chapter \textlatin {IV}.
[37

] [38] [39] [40] [41] [42] [43] [44] [45] [46] [47]
Chapter \textlatin {V}.
[48

] [49] [50] [51] [52] [53] [54] [55] [56] [57] [58]
Chapter \textlatin {VI}.
[59

] [60] [61]
Overfull \hbox (21.48987pt too wide) in alignment at lines 3916--3919
 [][][] []
 []

[62] [63] [64] [65]

LaTeX Warning: Command \& invalid in math mode on input line 4220.

[66] [67]

LaTeX Warning: Command \& invalid in math mode on input line 4345.

[68] [69]
Underfull \hbox (badness 10000) in paragraph at lines 4430--4437

 []

[70] [71] [72] [73]
Chapter \textlatin {VII}.
[74

] [75] [76] [77] [78] [79]
Underfull \hbox (badness 10000) in paragraph at lines 4987--4994

 []

[80] [81] [82] [83] [84] [85] [86]
Chapter \textlatin {VIII}.
[87

] [88] [89]
Overfull \hbox (3.22737pt too wide) detected at line 5570
\OML/cmm/m/it/10 xy \OMS/cmsy/m/n/10 ^^@ \OML/cmm/m/it/10 v\OT1/cmr/m/n/10 (\OM
L/cmm/m/it/10 w\OT1/cmr/m/n/10 (1 \OMS/cmsy/m/n/10 ^^@ \OML/cmm/m/it/10 z\OT1/c
mr/m/n/10 ) + \OML/cmm/m/it/10 z\OT1/cmr/m/n/10 (1 \OMS/cmsy/m/n/10 ^^@ \OML/cm
m/m/it/10 w\OT1/cmr/m/n/10 )); \OML/cmm/m/it/10 yz \OT1/cmr/m/n/10 = \OML/cmm/m
/it/10 v\OT1/cmr/m/n/10 (\OML/cmm/m/it/10 xw \OT1/cmr/m/n/10 + (1 \OMS/cmsy/m/n
/10 ^^@ \OML/cmm/m/it/10 x\OT1/cmr/m/n/10 )(1 \OMS/cmsy/m/n/10 ^^@ \OML/cmm/m/i
t/10 w\OT1/cmr/m/n/10 )); (1 \OMS/cmsy/m/n/10 ^^@ \OML/cmm/m/it/10 x\OT1/cmr/m/
n/10 )(1 \OMS/cmsy/m/n/10 ^^@ \OML/cmm/m/it/10 y\OT1/cmr/m/n/10 ) = (1 \OMS/cms
y/m/n/10 ^^@ \OML/cmm/m/it/10 z\OT1/cmr/m/n/10 )(1 \OMS/cmsy/m/n/10 ^^@ \OML/cm
m/m/it/10 w\OT1/cmr/m/n/10 )\OML/cmm/m/it/10 :
 []


LaTeX Warning: Command \& invalid in math mode on input line 5598.

[90] [91]

LaTeX Warning: Command \& invalid in math mode on input line 5690.


LaTeX Warning: Command \& invalid in math mode on input line 5719.


LaTeX Warning: Command \& invalid in math mode on input line 5722.


LaTeX Warning: Command \& invalid in math mode on input line 5730.


LaTeX Warning: Command \& invalid in math mode on input line 5733.


LaTeX Warning: Command \& invalid in math mode on input line 5733.


LaTeX Warning: Command \& invalid in math mode on input line 5736.


LaTeX Warning: Command \& invalid in math mode on input line 5740.

[92]

LaTeX Warning: Command \& invalid in math mode on input line 5749.


LaTeX Warning: Command \& invalid in math mode on input line 5752.

[93] [94] [95] [96] [97] [98] [99]
Chapter \textlatin {IX}.
[100

] [101] [102] [103] [104] [105] [106] [107] [108]

LaTeX Warning: Command \& invalid in math mode on input line 6711.

[109] [110]
Underfull \hbox (badness 10000) in paragraph at lines 6854--6856

 []


Underfull \hbox (badness 10000) in paragraph at lines 6857--6865

 []

[111] [112] [113] [114] [115] [116]
Chapter \textlatin {X}.
[117

] [118]

LaTeX Warning: Command \& invalid in math mode on input line 7257.


LaTeX Warning: Command \& invalid in math mode on input line 7265.


LaTeX Warning: Command \& invalid in math mode on input line 7278.

[119] [120] [121] [122] [123]
Chapter \textlatin {XI}.
[124

] [125] [126] [127] [128] [129] [130] [131] [132] [133] [134] [135] [136]
Chapter \textlatin {XII}.
[137

] [138] [139] [140] [141] [142]
Chapter \textlatin {XIII}.
[143

] [144] [145]
Underfull \hbox (badness 10000) in paragraph at lines 8933--8934

 []

[146] [147] [148]
Overfull \hbox (2.88245pt too wide) in alignment at lines 9112--9119
 [][][] []
 []

[149] [150] [151]
Overfull \hbox (37.52142pt too wide) in alignment at lines 9290--9298
 [][][] []
 []

[152] [153] [154] [155]
Overfull \hbox (0.93797pt too wide) in alignment at lines 9553--9561
 [][][] []
 []

[156] [157] [158] [159] [160] [161] [162] [163] [164] [165] [166] [167] [168]
Chapter \textlatin {XIV}.

Overfull \hbox (9.9288pt too wide) in alignment at lines 10355--10363
 [][][] []
 []

[169

] [170] [171] [172] [173]
Chapter \textlatin {XV}.
[174

] [175] [176] [177] [178] [179] [180] [181] [182] [183] [184] [185] [186]
Chapter \textlatin {XVI}.
[187

] [188] [189] [190] [191] [192] [193]
Chapter \textlatin {XVII}.
[194

] [195] [196] [197]
Underfull \hbox (badness 10000) in paragraph at lines 11990--11994

 []


Underfull \hbox (badness 10000) in paragraph at lines 11990--11994

 []

[198] [199] [200] [201] [202] [203] [204] [205] [206] [207] [208] [209] [210]
Chapter \textlatin {XVIII}.
[211

] [212] [213] [214] [215] [216] [217] [218] [219] [220] [221] [222] [223] [224]
[225] [226]
Chapter \textlatin {XIX}.
[227

] [228] [229] [230] [231]
Underfull \hbox (badness 10000) in paragraph at lines 13984--13985

 []

[232] [233] [234] [235] [236] [237] [238] [239] [240] [241] [242] [243] [244] [
245] [246]
Chapter \textlatin {XX}.
[247

] [248] [249] [250] [251] [252] [253] [254] [255] [256] [257] [258] [259] [260]
[261] [262] [263] [264] [265] [266] [267] [268] [269] [270] [271] [272] [273] [
274]
Underfull \hbox (badness 1072) in paragraph at lines 16647--16650
[]\OT1/cmr/m/n/10 Probability that if some one
 []

[275] [276] [277]
Overfull \hbox (18.56445pt too wide) in paragraph at lines 16866--16875
[]\OT1/cmr/m/sc/10 Problem \OT1/cmr/m/n/10 X.---\OT1/cmr/m/it/10 The prob-a-bil
-ity of the oc-cur-rence of a cer-tain nat-u-ral ph^^Znomenon
 []

[278] [279] [280] [281] [282] [283] [284] [285] [286] [287] [288] [289] [290] [
291] [292]
Chapter \textlatin {XXI}.
[293

] [294] [295] [296] [297] [298] [299] [300] [301] [302] [303]
Underfull \hbox (badness 10000) in paragraph at lines 18370--18374

 []

[304] [305] [306] [307] [308] [309] [310]
Chapter \textlatin {XXII}.
[311

] [312] [313] [314] [315] [316] [317] [318] [319]
Underfull \hbox (badness 1308) in paragraph at lines 19327--19327
[][]\OT1/cmr/m/n/8 '$\OML/cmm/m/it/8 E^^Y"[] ^^N[] ^^T[] ^^\[]^^W^^W^^\[] ^^
\o[]& []^^M^^Ro[]& []^^W[]^^W^^\ []^^^[]^^W"^^\o []^^W ^^\[] ^^^[]^^["^^S;  
^^T[] o[]$
 []

[320] [321] [322] [323] [324] [325] [326] [327] [328]
Chapter \textlatin {XXIII}.

Overfull \hbox (37.72055pt too wide) in paragraph at lines 20144--20144
[]\OT1/cmtt/m/n/9 End of the Project Gutenberg EBook of An Investigation of the
 Laws of Thought, by[] 
 []

[1

] [2] [3] [4] [5] [6] [7]
Overfull \hbox (9.37088pt too wide) in paragraph at lines 20144--20144
[]\OT1/cmtt/m/n/9 Section 5. General Information About Project Gutenberg-tm ele
ctronic works.[] 
 []

[8] (./15114-t.aux)

 *File List*
    book.cls    2007/10/19 v1.4h Standard LaTeX document class
    bk10.clo    2007/10/19 v1.4h Standard LaTeX file (size option)
   babel.sty    2013/12/03 3.9h The Babel package
   greek.ldf    2013/12/03 v1.8a Greek support for the babel system
  lgrenc.def    2013/07/16 v0.9 LGR Greek font encoding definitions
greek-fontenc.def    2013/11/28 v0.11 Common Greek font encoding definitions
 english.ldf    2012/08/20 v3.3p English support from the babel system
 amssymb.sty    2013/01/14 v3.01 AMS font symbols
amsfonts.sty    2013/01/14 v3.01 Basic AMSFonts support
 amsmath.sty    2013/01/14 v2.14 AMS math features
 amstext.sty    2000/06/29 v2.01
  amsgen.sty    1999/11/30 v2.0
  amsbsy.sty    1999/11/29 v1.2d
  amsopn.sty    1999/12/14 v2.01 operator names
  lgrcmr.fd    2013/09/01 v1.0 Greek European Computer Regular
    umsa.fd    2013/01/14 v3.01 AMS symbols A
    umsb.fd    2013/01/14 v3.01 AMS symbols B
 ***********

 ) 
Here is how much of TeX's memory you used:
 2373 strings out of 493304
 33132 string characters out of 6139871
 114759 words of memory out of 5000000
 5835 multiletter control sequences out of 15000+600000
 16949 words of font info for 60 fonts, out of 8000000 for 9000
 957 hyphenation exceptions out of 8191
 27i,23n,26p,515b,414s stack positions out of 5000i,500n,10000p,200000b,80000s
</usr/share/texlive/texmf-dist/fonts/type1/public/amsfonts/cm/cmbx10.pfb></us
r/share/texlive/texmf-dist/fonts/type1/public/amsfonts/cm/cmbx12.pfb></usr/shar
e/texlive/texmf-dist/fonts/type1/public/amsfonts/cm/cmcsc10.pfb></usr/share/tex
live/texmf-dist/fonts/type1/public/amsfonts/cm/cmex10.pfb></usr/share/texlive/t
exmf-dist/fonts/type1/public/amsfonts/cmextra/cmex8.pfb></usr/share/texlive/tex
mf-dist/fonts/type1/public/amsfonts/cm/cmmi10.pfb></usr/share/texlive/texmf-dis
t/fonts/type1/public/amsfonts/cm/cmmi12.pfb></usr/share/texlive/texmf-dist/font
s/type1/public/amsfonts/cm/cmmi5.pfb></usr/share/texlive/texmf-dist/fonts/type1
/public/amsfonts/cm/cmmi7.pfb></usr/share/texlive/texmf-dist/fonts/type1/public
/amsfonts/cm/cmmi8.pfb></usr/share/texlive/texmf-dist/fonts/type1/public/amsfon
ts/cm/cmr10.pfb></usr/share/texlive/texmf-dist/fonts/type1/public/amsfonts/cm/c
mr12.pfb></usr/share/texlive/texmf-dist/fonts/type1/public/amsfonts/cm/cmr17.pf
b></usr/share/texlive/texmf-dist/fonts/type1/public/amsfonts/cm/cmr5.pfb></usr/
share/texlive/texmf-dist/fonts/type1/public/amsfonts/cm/cmr6.pfb></usr/share/te
xlive/texmf-dist/fonts/type1/public/amsfonts/cm/cmr7.pfb></usr/share/texlive/te
xmf-dist/fonts/type1/public/amsfonts/cm/cmr8.pfb></usr/share/texlive/texmf-dist
/fonts/type1/public/amsfonts/cm/cmsl10.pfb></usr/share/texlive/texmf-dist/fonts
/type1/public/amsfonts/cm/cmsy10.pfb></usr/share/texlive/texmf-dist/fonts/type1
/public/amsfonts/cm/cmsy5.pfb></usr/share/texlive/texmf-dist/fonts/type1/public
/amsfonts/cm/cmsy6.pfb></usr/share/texlive/texmf-dist/fonts/type1/public/amsfon
ts/cm/cmsy7.pfb></usr/share/texlive/texmf-dist/fonts/type1/public/amsfonts/cm/c
msy8.pfb></usr/share/texlive/texmf-dist/fonts/type1/public/amsfonts/cm/cmti10.p
fb></usr/share/texlive/texmf-dist/fonts/type1/public/amsfonts/cm/cmti7.pfb></us
r/share/texlive/texmf-dist/fonts/type1/public/amsfonts/cm/cmti8.pfb></usr/share
/texlive/texmf-dist/fonts/type1/public/amsfonts/cm/cmtt9.pfb></usr/share/texliv
e/texmf-dist/fonts/type1/public/cbfonts/grmn0800.pfb></usr/share/texlive/texmf-
dist/fonts/type1/public/cbfonts/grmn1000.pfb></usr/share/texlive/texmf-dist/fon
ts/type1/public/amsfonts/symbols/msam10.pfb></usr/share/texlive/texmf-dist/font
s/type1/public/amsfonts/symbols/msam7.pfb>
Output written on 15114-t.pdf (344 pages, 1360320 bytes).
PDF statistics:
 1239 PDF objects out of 1440 (max. 8388607)
 853 compressed objects within 9 object streams
 0 named destinations out of 1000 (max. 500000)
 1 words of extra memory for PDF output out of 10000 (max. 10000000)

